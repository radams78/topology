\documentclass{report}

\usepackage{amsmath}
\usepackage{amssymb}
\usepackage{amsthm}
\usepackage{hyperref}
\let\proof\relax
\let\endproof\relax
\let\qed\relax
\usepackage{pf2}
\usepackage[all]{xy}

\title{Topology}
\author{Robin Adams}

\newtheorem{lm}{Lemma}[section]
\newtheorem{prop}[lm]{Proposition}
\newtheorem{thm}[lm]{Theorem}
\newtheorem{cor}{Corollary}[lm]
\newtheorem{ax}[lm]{Axiom}
\theoremstyle{definition}
\newtheorem{df}[lm]{Definition}
\newtheorem{ex}[lm]{Example}

\pfshortindent

\newcommand{\id}[1]{\ensuremath{\mathrm{id}_{#1}}}
\newcommand{\inv}[1]{\ensuremath{{#1}^{-1}}}
\newcommand{\dom}{\ensuremath{\operatorname{dom}}}
\newcommand{\im}{\ensuremath{\operatorname{im}}}
\newcommand{\supp}{\ensuremath{\operatorname{supp}}}
\newcommand{\Int}{\ensuremath{\operatorname{Int}}}
\newcommand{\Cl}{\ensuremath{\operatorname{Cl}}}
\newcommand{\Bd}{\ensuremath{\operatorname{Bd}}}
\newcommand{\diam}{\ensuremath{\operatorname{diam}}}

\DeclareMathSymbol{\magic}{\mathord}{operators}{"3C}

\begin{document}
  \maketitle
  \tableofcontents

\chapter{Set Theory}

\section{Primitive Notions}

Let there be \emph{sets}.

Given sets $A$ and $B$, let there be \emph{functions} from $A$ to $B$. We
write $f : A \rightarrow B$ iff $f$ is a function from $A$ to $B$.

Given functions $f : A \rightarrow B$ and $g : B \rightarrow C$, let there be
a function $g \circ f : A \rightarrow C$, the \emph{composite} of $f$ and $g$.

\begin{df}[Injective]
  A function $f : A \rightarrow B$ is \emph{injective}, $f : A
  \rightarrowtail B$, iff, for every set $X$ and functions $g, h : X
  \rightarrow
  A$, if $f \circ g = f \circ h$ then $g = h$.
\end{df}

\begin{df}[Surjective]
  Let $f : A \rightarrow B$. Then $f$ is \emph{surjective}, $f : A
\twoheadrightarrow B$, iff, for any set
  $X$ and functions $g, h : B \rightarrow X$, if $g \circ f = h \circ f$ then
  $g
  = h$.
\end{df}

\section{Axioms}

\subsection{Axioms for a Category}

\begin{ax}[Associativity]
  Let $f : A \rightarrow B$, $g : B \rightarrow C$ and $h : C \rightarrow D$.
  Then $h \circ (g \circ f) = (h \circ g) \circ f : A \rightarrow D$.
\end{ax}

From now on we write $h \circ g \circ f$ for the composite of $f$, $g$ and
$h$, and similarly for more than three functions.

\begin{lm}
 Let $f : A \rightarrow B$ and $g : B \rightarrow C$. If $f$ and $g$ are
injective then $g \circ f$ is injective.
\end{lm}

\begin{proof}
 \pf
 \step{<1>1}{\assume{$f$ and $g$ are injective.}}
 \step{<1>2}{\pflet{$X$ be a set and $x, y : X \rightarrow A$}}
 \step{<1>3}{\assume{$g \circ f \circ x = g \circ f \circ y$}}
 \step{<1>4}{$f \circ x = f \circ y$}
 \begin{proof}
   \pf\ $g$ is injective (\stepref{<1>1})
 \end{proof}
 \step{<1>5}{$x = y$}
 \begin{proof}
   \pf\ $f$ is injective (\stepref{<1>1})
 \end{proof}
 \qed
\end{proof}

\begin{lm}
 Let $f : A \rightarrow B$ and $g : B \rightarrow C$. If $f$ and $g$ are
surjective then $g \circ f$ is surjective.
\end{lm}

\begin{proof}
 \pf\ Dual. \qed
\end{proof}

\begin{lm}
 Let $f : A \rightarrow B$ and $g : B \rightarrow C$. If $g \circ f$ is
injective then $f$ is injective.
\end{lm}

\begin{proof}
 \pf
 \step{<1>1}{\assume{$g \circ f$ is injective.}}
 \step{<1>2}{\pflet{$X$ be any set and $x, y : X \rightarrow A$}}
 \step{<1>3}{\assume{$f \circ x = f \circ y$}}
 \step{<1>4}{$g \circ f \circ x = g \circ f \circ y$}
 \step{<1>5}{$x = y$}
 \begin{proof}
   \pf\ Using \stepref{<1>1}.
 \end{proof}
 \qed
\end{proof}

\begin{lm}
 Let $f : A \rightarrow B$ and $g : B \rightarrow C$. If $g \circ f$ is
surjective then $g$ is surjective.
\end{lm}

\begin{proof}
 \pf\ Dual. \qed
\end{proof}

\begin{ax}[Identity Function]
  For any set $A$, there exists a function $\id{A} : A \rightarrow A$, the
  \emph{identity} function on $A$, such that:
  \begin{description}
    \item[Left Unit Law] for every set $B$ and function $f : B \rightarrow A$
    we have $\id{A}      \circ f = f : B \rightarrow A$;
    \item[Right Unit Law] for every set $B$ and function $f : A \rightarrow
    B$
    we have $f      \circ \id{A} = f : A \rightarrow B$.
  \end{description}
\end{ax}

\begin{prop}
  The identity function on a set is unique.
\end{prop}

\begin{proof}
  \pf If $i, j : A \rightarrow A$ are both identity functions, then
  \begin{align*}
    i & = i \circ j & (\text{Right Unit Law for } j) \\
    & = j & (\text{Left Unit Law for } i) \\
    & : A \rightarrow A & \qed
  \end{align*}
\end{proof}

\begin{prop}
 Every identity function is injective.
\end{prop}

\begin{proof}
  \pf\ If $\id{B} \circ x = \id{B} \circ y$ then $x = y$ by the Left Unit
Law. \qed
\end{proof}

\begin{prop}
Every identity function is surjective.
\end{prop}

\begin{proof}
\pf\ Dual. \qed
\end{proof}

\begin{df}[Retraction, Section]
  Let $r : A \rightarrow B$ and $s : B \rightarrow A$. Then $r$ is a
  \emph{retraction} of $s$, and $s$ is a \emph{section} of $r$, iff $r \circ
  s = \id{B}$.
\end{df}

\begin{prop}
\label{prop:sets:retraction:comp}
If $r_1 : A \rightarrow B$ is a retraction of $s_1 : B \rightarrow A$ and $r_2 : B \rightarrow C$ is a retraction of $s_2 : C \rightarrow B$ then $r_2 \circ r_1$ is a retraction of $s_1 \circ s_2$.
\end{prop}

\begin{proof}
\pf
\begin{align*}
r_2 \circ r_1 \circ s_1 \circ s_2 & = r_2 \circ \id{B} \circ s_2 & (\text{$r_1$ is a retraction of $s_1$}) \\
& = r_2 \circ s_2 & (\text{Unit Laws}) \\
& = \id{C} & (\text{$r_2$ is a retraction of $s_2$})
\end{align*}
\qed
\end{proof}

\begin{prop}
\label{prop:sets:retraction:surjective}
Every retraction is surjective.
\end{prop}

\begin{proof}
\pf
\step{<1>1}{\pflet{$r : A \rightarrow B$ be a retraction of $s$}}
\step{<1>2}{\pflet{$X$ be a set and $x, y : B \rightarrow X$ with $x \circ r = y \circ r$}}
\step{<1>3}{$x \circ r \circ s = y \circ r \circ s$}
\step{<1>4}{$x = y$}
\qed
\end{proof}

\begin{prop}
\label{prop:sets:section:injective}
Every section is injective.
\end{prop}

\begin{proof}
\pf\ Dual. \qed
\end{proof}

\begin{prop}
\label{prop:sets:retraction:id}
Every identity function is a retraction of itself.
\end{prop}

\begin{proof}
\pf\ Immediate from the Unit Laws. \qed
\end{proof}

\begin{prop}
\label{prop:sets:retraction:ret_sect_unique}
If $r : B \rightarrow A$ is a retraction of $f : A \rightarrow B$ and $s$ is a section of $f$  then $ r = s$.
\end{prop}

\begin{proof}
\pf
\begin{align*}
r & = r \circ \id{B} & (\text{Right Unit Law}) \\
& = r \circ f \circ s & (\text{$s$ is a section of $f$}) \\
& = \id{A} \circ s & (\text{$r$ is a retraction of $f$}) \\
& = s & (\text{Left Unit Law})
\end{align*}
\end{proof}

\subsection{Isomorphisms}

\begin{df}[Isomorphism]
Let $A$ and $B$ be sets. A function $i : A \rightarrow B$  is an
\emph{isomorphism} between $A$ and $B$, $i : A \cong B$, iff there exists a
function $\inv{i} : B \rightarrow A$, the \emph{inverse} to $i$, that is a
section and a retraction of $i$.
\end{df}

\begin{prop}
The inverse of an isomorphism is unique.
\end{prop}

\begin{proof}
\pf\ Immediate from Proposition \ref{prop:sets:retraction:ret_sect_unique}. \qed
\end{proof}

\begin{prop}
Every isomorphism is injective.
\end{prop}

\begin{proof}
\pf\ Immediate from Proposition \ref{prop:sets:section:injective}. \qed
\end{proof}

\begin{prop}
Every isomorphism is surjective.
\end{prop}

\begin{proof}
\pf\ Immediate from Proposition \ref{prop:sets:retraction:surjective}. \qed
\end{proof}

\begin{prop}
\label{prop:sets:isomorphism:identity}
Every identity function is an isomorphism and is its own inverse.
\end{prop}

\begin{proof}
\pf\ Immediate from Proposition \ref{prop:sets:retraction:id}. \qed
\end{proof}

\begin{prop}
\label{prop:sets:isomorphism:inverse}
If $i : A \cong B$ is an isomorphism then $\inv{i} : B \cong A$ is an
isomorphism and $\inv{(\inv{i})} = i$.
\end{prop}

\begin{proof}
\pf\ Immediate from the definition of isomorphism. \qed
\end{proof}

\begin{prop}
  \label{prop:sets:isomorphism:comp}
If $i : A \cong B$ and $j : B \cong C$ then $j \circ i : A \cong C$ and
$\inv{(j \circ i)} = \inv{i} \circ \inv{j}$.
\end{prop}

\begin{proof}
\pf\ Immediate from Proposition \ref{prop:sets:retraction:comp}. \qed
\end{proof}


\subsection{Parts of a Set}

\begin{df}[Part]
  A \emph{part} $S$ of a set $A$ consists of:
  \begin{itemize}
   \item a set $\dom S$;
   \item an injective function $i : S \hookrightarrow A$
  \end{itemize}
\end{df}

\begin{df}
  Two parts $i : S \hookrightarrow A$, $j : T \hookrightarrow A$ are
  \emph{equivalent}, $i \equiv_A j$, iff there exists an isomorphism $\phi :
S \cong T$ such that $i = j \circ \phi$.
\end{df}

\begin{prop}
 Any part of a set is equivalent to itself.
\end{prop}

\begin{proof}
 \pf
 \step{<1>1}{\pflet{$i : S \hookrightarrow A$ be a part of $A$.} \prove{$i
     \equiv_A i$}}
 \step{<1>2}{$\id{S} : S \cong S$}
 \begin{proof}
   \pf\ By Proposition \ref{prop:sets:isomorphism:identity}
 \end{proof}
 \step{<1>3}{$i = i \circ \id{S}$}
 \begin{proof}
   \pf\ By the Right Unit Law.
 \end{proof}
 \qed
\end{proof}

\begin{prop}
 If $i \equiv_A j$ then $j \equiv_A i$.
\end{prop}

\begin{proof}
 \pf
 \step{<1>1}{\pflet{$i : S \hookrightarrow A$ and $j : T \hookrightarrow A$}}
 \step{<1>2}{\assume{$i \equiv_A j$}}
 \step{<1>3}{\pick\ an isomorphism $\phi : S \cong T$ such that $i = j \circ
   \phi$}
 \step{<1>4}{$\inv{\phi} : T \cong S$}
 \begin{proof}
   \pf\ By Proposition \ref{prop:sets:isomorphism:inverse}.
 \end{proof}
 \step{<1>5}{$j = i \circ \inv{\phi}$}
 \begin{proof}
   \pf\ Compose both sides of \stepref{<1>3} with $\inv{\phi}$.
 \end{proof}
 \qed
\end{proof}

\begin{prop}
  If $i \equiv_A j$ and $j \equiv_A k$ then $i \equiv_A k$.
\end{prop}

\begin{proof}
  \pf
  \step{<1>1}{\pflet{$i : R \hookrightarrow A$, $j : S \hookrightarrow A$ and $k : T \rightarrow A$}}
  \step{<1>2}{\pick\ isomorphisms $\phi : R \cong S$ and $\psi : S \cong T$ such that $i = j \circ \phi$ and $j = k \circ \psi$}
  \step{<1>3}{$\psi \circ \phi : R \cong T$}
  \begin{proof}
    \pf\ By Proposition \ref{prop:sets:isomorphism:comp}.
  \end{proof}
  \step{<1>4}{$i = k \circ \psi \circ \phi$}
\qed
\end{proof}

\begin{df}[Inclusion]
 Let $i : U \hookrightarrow A$ and $j : V \hookrightarrow A$ be parts of $A$.
 Then $i$ is \emph{included} in $j$, $i \subseteq_A j$, iff there exists a
 function $\phi : U \rightarrow V$ such that $i = j \circ \phi$.
\end{df}

\begin{prop}
  If $i \equiv_A i'$ and $j \equiv_A j'$ and $i \subseteq_A j$ then $i' \subseteq_A j'$.
\end{prop}

\begin{proof}
  \pf
  \step{<1>1}{\pflet{$i : S \hookrightarrow A$, $i' : S' \hookrightarrow A$, $j : T \hookrightarrow A$, $j' : T' \hookrightarrow A$}}
  \step{<1>2}{\pick\ $\phi : S \cong S'$, $\psi : T \cong T'$ and $\chi : S \rightarrow T$ such that $i = i' \circ \phi$, $j = j' \circ \psi$ and $i = j \circ \chi$}
  \step{<1>3}{$\psi \circ \chi \circ \inv{\phi} : S' \rightarrow T'$}
  \step{<1>4}{$i' = j' \circ \psi \circ \chi \circ \inv{\phi}$}
  \qed
\end{proof}

\begin{prop}
  For any part $i$ of $A$ we have $i \subseteq_A i$.
\end{prop}

\begin{proof}
  \pf
  \step{<1>1}{\pflet{$i : S \hookrightarrow A$}}
  \step{<1>2}{$\id{S} : S \rightarrow S$}
  \step{<1>3}{$i = i \circ \id{S}$}
  \qed
\end{proof}

\begin{prop}
  If $i \subseteq_A j$ and $j \subseteq_A k$ then $i \subseteq_A k$.
\end{prop}

\begin{proof}
  \pf
  \step{<1>1}{\pflet{$i : R \hookrightarrow A$, $j : S \hookrightarrow A$ and $k : T \hookrightarrow A$}}
  \step{<1>2}{\pick\ $\phi : R \rightarrow S$ and $\psi : S \rightarrow T$ such that $i = j \circ \phi$ and $j = k \circ \psi$}
  \step{<1>3}{$\psi \circ \phi : R \rightarrow T$}
  \step{<1>4}{$i = k \circ \psi \circ \phi$}
  \qed
\end{proof}

\begin{prop}
  If $i \subseteq_A j$ and $j \subseteq_A i$ then $i \equiv_A j$.
\end{prop}

\begin{proof}
  \pf
  \step{<1>1}{\pflet{$i : R \hookrightarrow A$, $j : S \hookrightarrow A$}}
  \step{<1>2}{\pick\ $\phi : R \rightarrow S$ and $\inv{\phi} : S \rightarrow R$ such that $i = j \circ \phi$ and $j = i \circ \inv{\phi}
  $}
  \step{<1>3}{$\phi \circ \inv{\phi} = \id{S}$}
  \begin{proof}
    \step{<2>1}{$j \circ \phi \circ \inv{\phi} = j$}
    \qedstep
    \begin{proof}
      \pf\ The result follows because $j$ is injective.
    \end{proof}
  \end{proof}
  \step{<1>4}{$\inv{\phi} \circ \phi = \id{T}$}
  \begin{proof}
    \pf\ Similar.
  \end{proof}
  \qed
\end{proof}

\subsection{The Empty Set}

\begin{ax}[Empty Set]
  There exists a set $\emptyset$, the \emph{empty set}, such that, for every
  set $X$, there exists a unique function $\magic_X : \emptyset \rightarrow
  X$.
\end{ax}

\begin{prop}[Uniqueness of Empty Set]
  Let $E$ be any set. Then $E$ is empty if and only if there exists an isomorphism $E \cong \emptyset$, in which case the isomorphism is unique.
\end{prop}

\begin{proof}
  \pf
  \step{<1>1}{If $E$ is empty then $E \cong \emptyset$}
  \begin{proof}
    \step{<2>1}{\assume{$E$ is empty}}
    \step{<2>2}{\pflet{$\phi$ be the unique function $E \rightarrow \emptyset$}}
    \step{<2>3}{$\magic_E \circ \phi = \id{E}$}
    \begin{proof}
      \pf\ There is only one function $E \rightarrow E$.
    \end{proof}
    \step{<2>4}{$\phi \circ \magic_E = \id{\emptyset}$}
    \begin{proof}
      \pf\ There is only one function $\emptyset \rightarrow \emptyset$.
    \end{proof}
  \end{proof}
  \step{<1>2}{If $E \cong \emptyset$ then $E$ is empty}
  \begin{proof}
    \step{<2>1}{\pflet{$\phi : E \cong \emptyset$}}
    \step{<2>2}{\pflet{$X$ be a set} \prove{There is a unique function $E \rightarrow X$}}
    \step{<2>3}{$\magic_X \circ \phi : E \rightarrow X$}
    \step{<2>4}{If $f : E \rightarrow X$ then $f = \magic_X \circ \phi$}
    \begin{proof}
      \step{<3>1}{\pflet{$f : E \rightarrow X$}}
      \step{<3>2}{$f \circ \inv{\phi} : \emptyset \rightarrow X$}
      \step{<3>3}{$f \circ \inv{\phi} = \magic_X$}
      \begin{proof}
        \pf\ Uniqueness of $\magic_X$.
      \end{proof}
      \qedstep
    \end{proof}
  \end{proof}
  \step{<1>3}{There is at most one isomorphism $E \cong \emptyset$}
  \begin{proof}
    \pf\ This holds because there is at most one function $E \rightarrow \emptyset$.
  \end{proof}
  \qed
\end{proof}

\begin{prop}
  \[ \magic_\emptyset = \id{\emptyset} \]
\end{prop}

\begin{proof}
  \pf\ By the uniqueness of $\magic_\emptyset$. \qed
\end{proof}

\subsection{The Terminal Set}

\begin{ax}[Terminal Set]
  There exists a set $1$, the \emph{terminal set}, such that, for every set
  $X$, there exists a unique function $!_X : X \rightarrow 1$.
\end{ax}

\begin{prop}[The Terminal Set is Unique up to Unique Canonical Isomorphism]
  If $T$ and $T'$ are terminal sets with unique functions $b_X : X \rightarrow T$ and $b_X' : X \rightarrow T'$ for all $X$,
  then there exists a unique isomorphism $\phi : T \cong T'$ such that, for all $X$, we have $\phi \circ b_X = b_X'$.
\end{prop}

\begin{proof}
  \pf\ Dual to Proposition \ref{prop:sets:empty:unique}.
\end{proof}

\begin{prop}
  \[ !_1 = \id{1} \]
\end{prop}

\begin{proof}
  \pf\ From the uniqueness of $!_1$. \qed
\end{proof}

\begin{df}[Element]
  An \emph{element} of a set $A$ is a function $1 \rightarrow A$. We write $a
  \in A$ for $a : 1 \rightarrow A$. We write $f(a)$ for $f \circ a$ when $f :
  A \rightarrow B$ and $a \in A$.
\end{df}

\begin{ax}[Extensionality]
 Let $A$ and $B$ be sets and $f, g : A \rightarrow B$ be functions. If, for
all $a \in A$, we have $f(a) = g(a) \in B$, then $f = g$.
\end{ax}

\begin{prop}
  \label{prop:sets:injective:elements}
  Let $f : A \rightarrow B$. Then $f$ is injective if and only if, for all $x, y \in A$, if $f(x) = f(y) \in B$ then $x = y \in A$.
\end{prop}

\begin{proof}
  \pf
  \step{<1>1}{If $f$ is injective and $f(x) = f(y) \in B$ then $x = y \in A$}
  \begin{proof}
    \pf\ Immediate from the definition of injective.
  \end{proof}
  \step{<1>2}{If, for all $x, y \in A$, if $f(x) = f(y) \in B$ then $x = y \in A$}
  \begin{proof}
    \step{<2>1}{\assume{For all $x, y \in A$, if $f(x) = f(y)$, then $x = y$}}
    \step{<2>2}{\pflet{$X$ be any set and $g,h : X \rightarrow A$ with $f \circ g = f \circ h$} \prove{$g = h$}}
    \step{<2>3}{\pflet{$x \in X$} \prove{$g(x) = h(x)$}}
    \step{<2>4}{$f(g(x)) = f(h(x))$}
    \begin{proof}
      \pf\ From \stepref{<2>2}.
    \end{proof}
    \step{<2>5}{$g(x) = h(x)$}
    \begin{proof}
      \pf\ By \stepref{<2>1}
    \end{proof}
  \end{proof}
  \qed
\end{proof}

\begin{prop}
  Any element $e \in X$ is a section of the unique function $!_X : X \rightarrow 1$.
\end{prop}

\begin{proof}
  \pf
  $!_X \circ e = \id{1}$ because there is only one function $1 \rightarrow 1$.
  \qed
\end{proof}

\begin{ax}[Non-degeneracy]
 The empty set $\emptyset$ has no elements.
\end{ax}

\begin{prop}
  For any set $X$, the function $\magic_X : \emptyset \rightarrow X$ is injective.
\end{prop}

\begin{proof}
  \pf\ From Proposition \ref{prop:sets:injective:elements}. \qed
\end{proof}

\begin{df}[Empty Part]
  For any set $X$, the \emph{empty part} of $X$ is $\emptyset = \magic_X : \emptyset \hookrightarrow X$.
\end{df}

  \begin{df}[Constant Function]
  A function $f : A \rightarrow B$ is \emph{constant} iff there exists $b \in
  B$ such that $f = b \circ !_A$.
\end{df}

\begin{df}[Membership]
 Let $i : U \hookrightarrow A$ be a part of $A$ and $a \in A$. Then $a$ is a
 \emph{member} of $i$, $a \in_A i$, iff there exists $\overline{a} \in U$ such that $i(\overline{a}) = a$.
\end{df}

\begin{prop}
 Let $A$ be a set. Let $i$, $j$ be parts of $A$ and $a \in A$. If $a \in_A i$
 and $i \subseteq_A j$ then $a \in_A j$.
\end{prop}

\begin{proof}
 \pf
 \step{<1>1}{\pick\ $\overline{a} \in \dom i$ such that $a =
i(\overline{a})$.}
 \step{<1>2}{\pick\ $\phi : \dom i \rightarrow \dom j$ such that $i = j \circ
   \phi$}
 \step{<1>3}{$a = j(\phi(\overline{a}))$}
 \qed
\end{proof}

\subsection{Products}

\begin{ax}[Products]
  For any sets $A$ and $B$, there exists a set $A \times B$, the
  \emph{product} of $A$ and $B$, and functions $\pi_1 : A \times B
  \rightarrow A$, $\pi_2 : A \times B \rightarrow B$, the \emph{projections},
  such that, for any set $C$ and functions $f : C \rightarrow A$, $g : C
  \rightarrow B$, there exists a unique function $\langle f, g \rangle : C
  \rightarrow A \times B$ such that
  \[ \pi_1 \circ \langle f, g \rangle = f; \qquad \pi_2 \circ \langle f,g
  \rangle
  = g \enspace . \]
\end{ax}

\begin{df}
  Given functions $f : A \rightarrow B$ and $g : C \rightarrow D$, define $f
  \times g : A \times C \rightarrow B \times D$ by
  \[ f \times g = \langle f \circ \pi_1, g \circ \pi_2 \rangle \]
\end{df}

\subsection{Coproducts}

\begin{ax}[Coproducts]
  For any sets $A$ and $B$, there exists a set $A \uplus B$, the
  \emph{coproduct} or \emph{sum} of $A$ and $B$, and functions $\kappa_1 : A
\rightarrow A
  \uplus B$, $\kappa_2 : B \rightarrow A \uplus B$, the \emph{injections},
  such that, for any set $C$ and functions $f : A \rightarrow C$, $g : B
  \rightarrow C$, there exists a unique function $[f, g] : A \uplus B
  \rightarrow
  C$ such that
  \[ [f,g] \circ \kappa_1 = f; \qquad [f,g] \circ \kappa_2 = g \enspace . \]
\end{ax}

  \begin{df}[Complement]
 Let $i : I \hookrightarrow J$ and $i' : I' \hookrightarrow J$ be parts of
 $J$. Then $i'$ is the \emph{complement} of $i$ iff $J$ is the sum of $I$ and
 $I'$ with injections $i$ and $i'$.
\end{df}


\subsection{Equalizers}

\begin{ax}[Equalizers]
  For any sets $A$ and $B$ and functions $f, g : A \rightarrow B$, there
  exists a set $E$ and function $e : E \rightarrow A$, the \emph{equalizer}
  of
  $A$ and $B$, such that:
  \begin{itemize}
    \item $f \circ e = g \circ e : E \rightarrow B$;
    \item For any set $C$ and function $h : C \rightarrow A$ such that $f
    \circ h
    = g \circ h$, there exists a unique function $\overline{h} : C
    \rightarrow
    E$
    such that $h = e \circ \overline{h}$.
  \end{itemize}
\end{ax}

\begin{prop}
  \label{prop:set_theory:equalizer:injective}
 All equalizers are injective.
\end{prop}

\begin{proof}
 \pf
 \step{<1>1}{\pflet{$e : E \rightarrow A$ be the equalizer of $f, g : A
     \rightarrow B$}}
 \step{<1>2}{\pflet{$x, y : X \rightarrow E$ with $e \circ x = e \circ y$}}
 \step{<1>3}{$f \circ e \circ x = g \circ e \circ x$}
 \begin{proof}
   \pf\ $f \circ e = g \circ e$ by \stepref{<1>1}1.
 \end{proof}
 \step{<1>4}{$x = y$}
 \begin{proof}
   \pf\ $x$ and $y$ are both the unique $z : X \rightarrow E$ such that $e
   \circ z = e \circ x$.
 \end{proof}
 \qed
\end{proof}

\subsection{Coequalizers}

\begin{ax}[Coequalizers]
  For any sets $A$ and $B$ and functions $f, g : A \rightarrow B$, there
  exists a set $C$ and function $c : B \rightarrow C$, the \emph{coequalizer}
  of $f$ and $g$, such that:
  \begin{itemize}
    \item $c \circ f = c \circ g : A \rightarrow C$
    \item For any set $X$ and function $h : B \rightarrow X$ such that $h
    \circ f
    = h \circ g$, there exists a unique function $\overline{h} : C
    \rightarrow
    X$
    such that $\overline{h} \circ c = h$.
  \end{itemize}
\end{ax}

\subsection{Pullbacks}

  \begin{df}[Pullback]
  The diagram below is a \emph{pullback diagram} iff:
  \begin{itemize}
    \item $f \circ p = g \circ q$
    \item for every set $X$ and
    functions $x : X \rightarrow B$ and $y : X \rightarrow C$ such that $f
    \circ x
    = g \circ y$, there exists a unique function $\langle x, y \rangle : X
    \rightarrow A$ such that $p \circ \langle x,y \rangle = x$ and $q \circ
    \langle
    x, y \rangle = y$.
  \end{itemize}
\[      \xymatrix{ A \ar[r]^p \ar[d]_q & B \ar[d]^f \\
      C \ar[r]_g & D } \]
\end{df}

   \begin{prop}
     \label{prop:sets:axioms:pullback}
 Let $f : A \rightarrow C$ and $g : B \rightarrow C$. Then $f$ and $g$ have a
 pullback.
\end{prop}

   \[ \xymatrix{
  E \ar[dr]^e \\
  & A \times B \ar[r]^{\pi_1} \ar[d]_{\pi_2} & A \ar[d]^f \\
  & B \ar[r]_g & C}
 \]

\begin{proof}
 \pf
 \step{<1>1}{Construct the product $\pi_1 : A \times B \rightarrow A$, $\pi_2
:
   A \times B \rightarrow B$.}
 \step{<1>2}{Construct the equalizer $e : E \rightarrow A$ of $f \circ \pi_1$
   and $g \circ \pi_2$.
   \prove{$\pi_1 \circ e$ and $\pi_2 \circ e$ form a pullback of $f$ and $g$}}
 \step{<1>3}{$f \circ \pi_1 \circ e = g \circ \pi_2 \circ e$}
 \step{<1>4}{\pflet{$X$ be a set and $x : X \rightarrow A$, $y : X
\rightarrow
     B$ satisfy $f \circ x = g \circ y$}}
 \step{<1>5}{$f \circ \pi_1 \circ \langle x, y \rangle = g \circ \pi_2 \circ
   \langle x, y \rangle$}
 \step{<1>6}{\pflet{$m : X \rightarrow E$ be the function such that $e \circ
     m = \langle x, y \rangle$}}
 \step{<1>7}{$\pi_1 \circ e \circ m = x$ and $\pi_2
   \circ e \circ m = y$}
 \step{<1>8}{$m$ is unique.}
 \begin{proof}
   \pf
   \step{<2>1}{\pflet{$n : X \rightarrow E$ be such that $\pi_1 \circ e \circ n
       = x$ and $\pi_2 \circ e \circ n = y$}}
   \step{<2>2}{$e \circ n = \langle x, y \rangle$}
   \step{<2>3}{$n = m$}
   \begin{proof}
     \pf\ By \stepref{<1>6}
   \end{proof}
 \end{proof}
 \qed
\end{proof}

\begin{prop}
 Pullbbacks are unique up to isomorphism.

 That is, let $P$ be a pullback of $f : A \rightarrow C$ and $g : B
\rightarrow C$ with projections $p : P \rightarrow A$ and $q : P \rightarrow
B$. Let $Q$ be a set and $p' : Q \rightarrow A$, $q' : Q \rightarrow B$. Then
$Q$ is a pullback of $f$ and $g$ with projections $p'$ and $q'$ if and only if
there exists a bijection $\phi : Q \cong P$ such that $p \circ \phi = p'$ and
$q \circ \phi = q'$, in which case $\phi$ is unique.

 \[ \xymatrix{
   Q \ar@/^/[rrd]^{p'} \ar@/_/[ddr]_{q'} \ar[dr]^{\phi} \\
  & P \ar[r]_p \ar[d]^q & A \ar[d]^{f} \\
  & B \ar[r]_g & C
}
 \]

\end{prop}

\begin{proof}
 \pf
 \step{<1>1}{If $Q$ is a pullback then there exists a bijection $\phi : Q
\cong
   P$ such that $p \circ \phi = p'$ and $q \circ \phi = q'$}
 \begin{proof}
   \step{<2>1}{\assume{$Q$ is a pullback with projections $p'$ and $q'$}}
   \step{<2>2}{\pflet{$\phi : Q \rightarrow P$ be the unique function such that
       $p \circ \phi = p'$ and $q \circ \phi = q'$}}
   \begin{proof}
     \pf\ Such a $\phi$ exists because $f \circ p' = g \circ q'$.
   \end{proof}
   \step{<2>3}{\pflet{$\inv{\phi} : P \rightarrow Q$ be the unique function such
       that $p' \circ \inv{\phi} = p$ and $q' \circ \inv{\phi} = q$}}
   \begin{proof}
     \pf\ Such a function exists because $f \circ p = g \circ q$.
   \end{proof}
   \step{<2>4}{$\phi \circ \inv{\phi} = \id{P}$}
   \begin{proof}
     \pf\ Each is the unique function $x$ such that $p \circ x = p$ and $q
     \circ x = q$.
   \end{proof}
   \step{<2>5}{$\inv{\phi} \circ \phi = \id{Q}$}
   \begin{proof}
     \pf\ Similar.
   \end{proof}
 \end{proof}
 \step{<1>2}{If $\phi : Q \cong P$ is a bijection then $Q$ is a pullback with
   projections $p \circ \phi$ and $q \circ \phi$}
 \begin{proof}
   \step{<2>1}{$f \circ p \circ \phi = g \circ q \circ \phi$}
   \begin{proof}
     \pf\ This holds because $f \circ p = g \circ q$
   \end{proof}
   \step{<2>2}{For any set $X$ and functions $x : X \rightarrow A$, $y : X
     \rightarrow B$ such that $f \circ x = g \circ y$, there exists a unique
     function $m : X \rightarrow Q$ such that $p \circ \phi \circ m = x$ and
$q \circ
     \phi \circ m = y$}
   \begin{proof}
     \pf\
     \begin{align*}
       & p \circ \phi \circ m = x \text{ and } q \circ \phi \circ m = y \\
      \Leftrightarrow & \phi \circ m = \langle x, y \rangle \\
      \Leftrightarrow & m = \inv{\phi} \circ \langle x, y \rangle
     \end{align*}
   \end{proof}
 \end{proof}
 \step{<1>3}{If $\phi, \phi' : P \cong Q$ are bijections such that $p \circ \phi
   = p \circ \phi'$ and $q \circ \phi = q \circ \phi'$}
 \begin{proof}
   \pf\ This follows from the definition of pullback.
 \end{proof}
 \qed
\end{proof}

\begin{prop}
  \label{prop:sets:pullback:injective}
 The pullback of an injective function is injective.

 That is, if the diagram below is a pullback diagram and $f$ is injective
then $q$ is injective.
\[      \xymatrix{ A \ar[r]^p \ar@{>->}[d]_q & B \ar@{>->}[d]^f \\
     C \ar[r]_g & D } \]
\end{prop}

\begin{proof}
 \pf
 \step{<1>1}{\pflet{$X$ be a set and $x, y : X \rightarrow A$ with $q \circ x =
     q \circ y$}}
 \step{<1>2}{$f \circ p \circ x = g \circ q \circ x$}
 \step{<1>3}{\pflet\ $z : X \rightarrow A$ be the function such that $p \circ z
   =      p \circ x$ and $q \circ z = q \circ x$}
 \step{<1>4}{$z = x$}
 \step{<1>5}{$z = y$}
 \begin{proof}
   \step{<2>1}{$q \circ x = q \circ y$}
   \begin{proof}
     \pf\ By \stepref{<1>1}.
   \end{proof}
   \step{<2>2}{$f \circ p \circ x = f \circ p \circ y$}
   \begin{proof}
     \pf
     \begin{align*}
       f \circ p \circ x & = g \circ q \circ x & (\text{\stepref{<1>2}}) \\
       & = g \circ q \circ y & (\text{\stepref{<1>1}}) \\
       & = f \circ p \circ y & (\text{the diagram is a pullback})
     \end{align*}
   \end{proof}
   \step{<2>3}{$p \circ x = p \circ y$}
   \begin{proof}
     \pf\ $f$ is injective.
   \end{proof}
 \end{proof}
 \qed
\end{proof}

\subsection{Function Sets}

\begin{ax}[Function Sets]
  For any sets $A$ and $B$, there exists a set $A^B$ and a function $\epsilon
  : A^B \times B \rightarrow A$, the \emph{evaluation} function, such that,
  for any set $C$ and function $f : C \times B \rightarrow A$, there exists a
  unique function $\lambda f : C \rightarrow A^B$ such that
  \[ \epsilon \circ (\lambda f \times \id{B}) = f \enspace . \]
\end{ax}

\subsection{The Subset Classifier}

\begin{df}
  The set $2$ is $1 + 1$. We write $\top$ (\emph{truth}) for $\kappa_1 : 1
  \rightarrow 2$, and $\bot$ (\emph{falsehood}) for $\kappa_2 : 1 \rightarrow
  2$.
\end{df}

\begin{ax}[Subset Classifier]
  For every injective function $m : A \rightarrowtail B$, there exists a
  unique function $\chi_m : B \rightarrow 2$, the \emph{characteristic
    function} of $m$, such that the following diagram is a pullback diagram:

  \[
  \xymatrix{
    A \ar[d]_m \ar[r]^{!} & 1 \ar[d]^{\top} \\
    B \ar[r]_{\chi_m} & 2
  }
  \]
\end{ax}

\begin{prop}
 Every function $\phi : A \rightarrow 2$ is the characteristic function of a
 part of $A$.
\end{prop}

\begin{proof}
 \pf
 \step{<1>1}{Construct a pullback
   \[ \xymatrix{
     I \ar[r] \ar[d]^{q} & 1 \ar[d]^{\top} \\
     A \ar[r]_{\phi} & 2 }
   \]}
 \begin{proof}
   \pf\ By Proposition \ref{prop:sets:axioms:pullback}.
 \end{proof}
 \step{<1>2}{$q$ is injective}
 \begin{proof}
   \pf\ By Proposition \ref{prop:sets:pullback:injective}.
 \end{proof}
 \qed
\end{proof}

\begin{ax}[Boolean]
  For any $p \in 2$ we have $p = \top$ or $p = \bot$.
\end{ax}

 \begin{prop}
Let $i : U \hookrightarrow A$ and $j : V \hookrightarrow A$ be parts of $A$.
Then the following are equivalent:
\begin{enumerate}
\item $i \subseteq_A j$ and $j \subseteq_A i$
\item There exist $h : U \rightarrow V$ and $k : V \rightarrow U$ such that $i
= j \circ h$, $j = i \circ k$, $k \circ h = \id{U}$ and $h \circ k = \id{V}$.
\item The characteristic function of $i$ is the characteristic function of $j$.
\end{enumerate}
\end{prop}

\begin{proof}
\pf
\step{<1>1}{$1 \Rightarrow 2$}
\begin{proof}
  \step{<2>1}{\assume{$i \subseteq_A j$ and $j \subseteq_A i$}}
  \step{<2>2}{\pflet{$h : U \rightarrow V$ be such that $i = j \circ h$}}
  \step{<2>3}{\pflet{$k : V \rightarrow U$ be such that $j = i \circ k$}}
  \step{<2>4}{$k \circ h = \id{U}$}
  \begin{proof}
    \step{<3>1}{$i \circ k \circ h = i$}
    \begin{proof}
      \pf\ From \stepref{<2>2} and \stepref{<2>3}.
    \end{proof}
    \qedstep
    \begin{proof}
      \pf\ Since $i$ is injective.
    \end{proof}
  \end{proof}
  \step{<2>5}{$h \circ k = \id{V}$}
  \begin{proof}
    \pf\ Similar.
  \end{proof}
\end{proof}
\step{<1>2}{$2 \Rightarrow 1$}
\begin{proof}
  \pf\ Trivial.
\end{proof}
\step{<1>3}{$2 \Rightarrow 3$}
\begin{proof}
  \step{<2>1}{\assume{2}}
  \step{<2>2}{\pflet{$\phi : A \rightarrow 2$ be the characteristic function
of
      $i$} \prove{$\phi$ is the characteristic function of $j$}
    \[ \xymatrix{
      X \ar[rrrdd] \ar[dddrr] \ar[dr] \\
      & V \ar[rrd] \ar[ddr] \ar[dr] \\
      & & U \ar[r] \ar[d] & 1 \ar[d]^{\top} \\
      & & A \ar[r]_{\phi} & 2
    }
    \]
  }
  \begin{proof}
    \pf\ By the Subset Classifier Axiom.
  \end{proof}
  \step{<2>3}{\pflet{$X$ be a set and $x : X \rightarrow 1$, $y : X
\rightarrow
      A$ satisfy $\phi \circ y = \top \circ x$}}
  \step{<2>4}{\pflet{$\langle x,y \rangle : X \rightarrow U$ be the unique
function such that $! \circ \langle x,y \rangle = x$ and $i \circ \langle x,y
\rangle = y$}}
\begin{proof}
\pf\ By \stepref{<2>2}.
\end{proof}
\step{<2>5}{$h \circ \langle x,y \rangle$ is the unique function $X \rightarrow
V$
such that $! \circ h \circ \langle x,y \rangle = x$ and $j \circ h \circ
\langle x,y \rangle = y$}
\begin{proof}
\step{<3>1}{$! \circ h \circ \langle x,y \rangle = x$}
\begin{proof}
  \pf\ Since 1 is terminal.
\end{proof}
\step{<3>2}{$j \circ h \circ \langle x,y \rangle = y$}
\begin{proof}
  \pf\ From \stepref{<2>1} and \stepref{<2>4}.
\end{proof}
\step{<3>3}{If $! \circ f = x$ and $j \circ f = y$ then $f = h  \circ \langle
x,y
  \rangle$}
\begin{proof}
  \step{<4>1}{\pflet{$f : X \rightarrow V$ satisfy $! \circ f = x$ and $j
\circ
      f = y$}}
  \step{<4>2}{$! \circ k \circ f = x$}
  \begin{proof}
    \pf\ As 1 is terminal.
  \end{proof}
  \step{<4>3}{$i \circ k \circ f = y$}
  \begin{proof}
    \pf\ From \stepref{<2>1} and \stepref{<4>1}.
  \end{proof}
  \step{<4>4}{$k \circ f = \langle x, y \rangle$}
  \begin{proof}
    \pf\ From \stepref{<2>4}, \stepref{<4>2} and \stepref{<4>3}.
  \end{proof}
  \step{<4>5}{$f = h \circ \langle x,y \rangle$}
  \begin{proof}
    \pf\ From \stepref{<2>1} and \stepref{<4>4}.
  \end{proof}
\end{proof}
\end{proof}
%TODO Extract lemma
\end{proof}
\step{<1>4}{$3 \Rightarrow 2$
  \[ \xymatrix{
    U \ar[dr] \ar[rrd] \ar[ddr]_{i} \\
    &  V \ar[r] \ar[d]_{j} \ar[ul] & 1 \ar[d]^{\top} \\
   &  A \ar[r]_{\phi} & 2
  } \]
}
\begin{proof}
  \step{<2>1}{\assume{3}}
  \step{<2>2}{\pflet{$\phi$ be the characteristic function of $i$ and $j$}}
  \step{<2>3}{\pflet{$h : U \rightarrow V$ be the unique function such that
$!
      \circ h = !$ and $j \circ h = i$}}
  \begin{proof}
    \step{<3>1}{$\top \circ ! = \phi \circ i$}
    \begin{proof}
      \pf\ This holds because $\phi$ is the characteristic function of $i$.
    \end{proof}
    \qedstep
    \begin{proof}
      \pf\ Since $\phi$ is the characteristic function of $j$.
    \end{proof}
  \end{proof}
  \step{<2>4}{\pflet{$k : V \rightarrow U$ be the unique function such that
$!
      \circ k = !$ and $i \circ k = j$}}
  \begin{proof}
    \pf\ Similar.
  \end{proof}
  \step{<2>5}{$k \circ h = \id{U}$}
  \begin{proof}
    \pf\ Each is the unique function $f$ such that $! \circ f = !$ and $i
\circ f = i$
\end{proof}
  \step{<2>6}{$h \circ k = \id{V}$}
  \begin{proof}
    \pf\ Each is the unique function $f$ such that $! \circ f = !$ and $j
\circ f = j$
\end{proof}
\end{proof}
\qed
\end{proof}

\section{The Basics}


\begin{lm}
  \label{lm:set_theory:union_of_subsets}
  Let $X$ be a set, $\mathcal{B} \subseteq \mathcal{P} X$ and $U \subseteq
  X$. Then the following are equivalent:
  \begin{enumerate}
    \item For all $x \in U$ there exists $B \in \mathcal{B}$ such that $x \in
    B \subseteq U$.
    \item There exists $\mathcal{B}_0 \subseteq \mathcal{B}$ such that $U =
    \bigcup \mathcal{B}_0$.
  \end{enumerate}
\end{lm}

\begin{proof}
  \pf
  \step{<1>1}{$1 \Rightarrow 2$}
  \begin{proof}
    \pf\ If 1 is true then $U = \bigcup \{ B \in \mathcal{B} : B \subseteq U
    \}$.
  \end{proof}
  \step{<1>2}{$2 \Rightarrow 1$}
  \begin{proof}
    \pf\ Trivial.
  \end{proof}
  \qed
\end{proof}

\begin{df}[Fixed Point]
  Let $X$ be a set, $f : X \rightarrow X$, and $x \in X$. Then $x$ is a
  \emph{fixed point} of $f$ iff $f(x) = x$.
\end{df}

\begin{df}[Saturated]
  Let $X$, $Y$ be sets and $p : X \twoheadrightarrow Y$ be a surjective
  function. Let $C \subseteq X$. Then $C$ is \emph{saturated} with respect to
  $p$ iff, for all $x, x' \in X$, if $x \in C$ and $p(x) = p(x')$ then $x'
  \in
  C$.
\end{df}

\begin{df}[Cover]
  Let $A$ be a set and $\mathcal{C} \subseteq \mathcal{P}
  A$. Then $\mathcal{C}$ \emph{covers} $A$ iff $\bigcup \mathcal{C} = A$.
\end{df}

\begin{df}[Finite Intersection Property]
  Let $X$ be a set and $\mathcal{C} \subseteq \mathcal{P} X$. Then
  $\mathcal{C}$ has the \emph{finite intersection property} if and only if
  every finite nonempty subset of $\mathcal{C}$ has nonempty intersection.
\end{df}

\begin{lm}[AC]
  \label{lm:sets:finite_intersection_property:maximal}
  Let $X$ be a set and $\mathcal{A} \subseteq \mathcal{P} X$ have the finite intersection property. Then there exists
  a maximal $\mathcal{D} \subseteq \mathcal{P} X$ that has the finite intersection property and includes $\mathcal{A}$.
\end{lm}

\begin{proof}
  \pf\ A straightforward application of Zorn's lemma, since the union of a chain of sets that has the finite intersection property has the finite intersection property. \qed
\end{proof}

\begin{lm}
  \label{lm:sets:finite_intersection_property:finite_intersection}
  Let $X$ be a set and $\mathcal{D} \subseteq \mathcal{P} X$ be maximal with respect to the finite intersection property. Then any finite intersection of elements of $\mathcal{D}$ is an element of $\mathcal{D}$.
\end{lm}

\begin{proof}
  \pf
  \step{<1>1}{\pflet{$A$ be a finite intersection of elements of $\mathcal{D}$}}
  \step{<1>2}{$\mathcal{D} \cup \{ A \}$ has the finite intersection property.}
  \step{<1>3}{$\mathcal{D} \cup \{ A \} = \mathcal{D}$}
  \qed
\end{proof}

\begin{lm}
  \label{lm:sets:finite_intersection_property:intersect_all}
  Let $X$ be a set and $\mathcal{D} \subseteq \mathcal{P} X$ be maximal with respect to the finite intersection property. If $A \subseteq X$ intersects every element of $\mathcal{D}$ then $A \in \mathcal{D}$.
\end{lm}

\begin{proof}
  \pf\ This holds because $\mathcal{D} \cup \{ A \}$ satisfies the finite intersection property. \qed
\end{proof}

\begin{df}[Graph]
  Let $f : A \rightarrow B$. The \emph{graph} of $f$ is the set $\{ (x, f(x))
  : x \in A \} \subseteq A \times B$.
\end{df}

 \begin{df}[Point-Finite]
 Let $X$ be a set and $\{ A_\alpha \}_{\alpha \in J}$ be a family of subsets
 of $X$. Then $\{ A_\alpha \}_{\alpha \in J}$ is \emph{point-finite} iff, for
all $x \in X$, there are only finitely many $\alpha \in J$ such that $x \in
A_\alpha$.
\end{df}

\begin{df}[Countable Intersection Property]
  A family of parts of a set $X$ has the \emph{countable intersection property} iff every countable subfamily has nonempty intersection.
\end{df}

\section{Order Theory}

  \begin{df}[Cofinal]
  Let $J$ be a poset and $K \subseteq J$. Then $K$ is \emph{cofinal} iff, for
all $x \in J$, there exists $y \in K$ such that $x \leq y$.
\end{df}

  \begin{df}[Directed Set]
  A \emph{directed set} is a poset $J$ such that, for all $x, y \in J$, there
exists $z \in J$ such that $x \leq z$ and $y \leq z$.
\end{df}

\begin{df}[Linear Order]
  Let $X$ be a set. A \emph{linear order} on $X$ is a relation $\leq
  \subseteq X^2$ such that:
  \begin{itemize}
    \item For all $x \in X$, $x \leq x$
    \item For all $x, y, z \in X$, if $x \leq y$ and $y \leq z$ then $x \leq
    z$
    \item For all $x, y \in X$, if $x \leq y$ and $y \leq x$ then $x = y$
    \item For all $x, y \in X$, we have $x \leq y$ or $y \leq x$
  \end{itemize}
  We write $x < y$ iff $x \leq y$ and $x \neq y$.

  A \emph{linearly ordered set} consists of a set and a linear order on the
  set.
\end{df}

\begin{df}[Convex]
  Let $L$ be a linearly ordered set and $A \subseteq L$. Then $A$ is
  \emph{convex} iff, for all $x, y \in A$ and $z \in L$, if $x < z < y$ then
  $z \in A$.
\end{df}

\begin{df}[Least Upper Bound Property]
  A linearly ordered set $L$ has the \emph{least upper bound property} iff
  every subset of $L$ bounded above has a least upper bound.
\end{df}

\begin{df}[Linear Continuum]
  A \emph{linear continuum} is a linearly ordered set $L$ such that:
  \begin{itemize}
    \item $L$ has the least upper bound property.
    \item For all $x, y \in L$ with $x < y$, there exists $z \in L$ such that
    $x < z < y$.
  \end{itemize}
\end{df}

\begin{prop}
  If $L$ is a linear continuum then every convex subset of $L$ is a linear
  continuum.
\end{prop}

\begin{proof}
  \pf
  \step{<1>1}{\pflet{$L$ be a linear continuum and $C \subseteq L$ be convex}}
  \step{<1>2}{$C$ satisfies the least upper bound property.}
  \begin{proof}
    \step{<2>1}{\pflet{$S \subseteq C$ be nonempty and bounded above by $u$
        in
        $C$.}}
    \step{<2>2}{\pflet{$s$ be the supremum of $S$ in $L$}}
    \step{<2>3}{\pick\ $x \in S$}
    \step{<2>4}{$x \leq s \leq u$}
    \step{<2>5}{$s \in C$}
    \begin{proof}
      \pf\ $C$ is convex.
    \end{proof}
    \step{<2>6}{$s$ is the supremum of $S$ in $C$}
  \end{proof}
  \step{<1>3}{$C$ is dense.}
  \begin{proof}
    \pf
    \step{<2>1}{\pflet{$x, y \in C$ satisfy $x < y$}}
    \step{<2>2}{\pick\ $z \in L$ such that $x < z < y$}
    \step{<2>3}{$z \in C$}
    \begin{proof}
      \pf\ $C$ is convex.
    \end{proof}
  \end{proof}
  \qed
\end{proof}

\begin{lm}
  \label{lm:order:half_open}
  For any real numbers $a$, $b$ with $a < b$ we have $[a, b) \cong [0, 1)$.
\end{lm}

\begin{proof}
  \pf\ The map $\phi : [a, b) \cong [0, 1)$ where $\phi(x) = (x - a) / (b -
  a)$ is an order isomorphism. \qed
\end{proof}

\begin{prop}
  \label{prop:order:zero_one_twice}
  Let $X$ be a linearly ordered set. Let $a, b, c \in X$ with $a < c < b$.
  Then $[a, b) \cong [0, 1)$ if and only if $[a, c) \cong [c, b) \cong [0,
  1)$.
\end{prop}

\begin{proof}
  \pf
  \step{<1>1}{If $[a, b) \cong [0, 1)$ then $[a, c) \cong [c, b) \cong [0,
    1)$.}
  \begin{proof}
    \step{<2>1}{\assume{$\phi : [a, b) \cong [0, 1)$ is an order
        isomorphism.}}
    \step{<2>2}{$[a, c) \cong [0, 1)$}
    \begin{proof}
      \pf
      \begin{align*}
        [a ,c) & \cong [0, \phi(c)) & (\text{under } \phi) \\
        & \cong [0, 1) & (\text{Lemma \ref{lm:order:half_open} })
      \end{align*}
    \end{proof}
    \step{<2>3}{$[c, b) \cong [0, 1)$}
    \begin{proof}
      \pf\ Similar.
    \end{proof}
  \end{proof}
  \step{<1>2}{If $[a, c) \cong [c, b) \cong [0, 1)$ then $[a, b) \cong [0,
    1)$.}
  \begin{proof}
    \step{<2>1}{\assume{$[a, c) \cong [c, b) \cong [0, 1)$}}
    \step{<2>2}{\pflet{$\phi : [a, c) \cong [0, 1/2)$ and $\psi : [c, b)
        \cong
        [1/2, 1)$}}
    \step{<2>3}{\pflet{$\chi : [a, b) \rightarrow [0, 1)$ be given by
        $\chi(x)
        =
        \begin{cases} \phi(x) & \text{if } x < c \\ \psi(x) & \text{if } x
          \geq c \end{cases}$}}
    \step{<2>4}{$\chi : [a, b) \cong [0, 1)$}
    \begin{proof}
      \pf\ Easy to check. % TODO Extract lemma
    \end{proof}
  \end{proof}
  \qed
\end{proof}


\begin{prop}[CC]
  \label{prop:order:zero_one_countable}
  Let $X$ be a linearly ordered set. Let $\{ x_n \}_{n \geq 0}$ be an
  increasing sequence of points of $X$. Suppose $b$ is the supremum of $\{
  x_n : n \geq 0 \}$. Then $[x_0, b) \cong [0, 1)$ if and only if $[x_i,
  x_{i+1}) \cong [0,1)$ for all $i$.
\end{prop}

\begin{proof}
  \pf
  \step{<1>1}{If $[x_0, b) \cong [0, 1)$ then for all $i$ $[x_i, x_{i+1})
    \cong
    [0, 1)$.}
  \begin{proof}
    \pf\ If $\phi : [x_0, b) \cong [0, 1)$ then $[x_i, x_{i+1}) \cong
    [\phi(x_i), \phi(x_{i+1})) \cong [0, 1)$ by Lemma
    \ref{lm:order:half_open}.
  \end{proof}
  \step{<1>2}{If for all $i$ $[x_i, x_{i+1}) \cong [0, 1)$ then $[x_0, b)
    \cong
    [0, 1)$.}
  \begin{proof}
    \pf
    \step{<2>1}{\pflet{$\phi_i : [x_i, x_{i+1}) \cong [0, 1)$ for all $i$}}
    \step{<2>2}{Define $\phi : [x_0, b) \cong [0, 1)$ by: $\phi(y) =
      \phi_i(y) \qquad (x_0 \leq y < b)$ where $i$ is least such that $y <
      i_{i+1}$}
    \begin{proof}
      \pf\ There exists such an $i$ because $y$ is not an upper bound for
      $\{x_n : n \geq 0 \}$.
    \end{proof}
    \step{<2>3}{$\phi$ is an order isomorphism.}
    \begin{proof}
      \pf\ Easy to check.
    \end{proof}
  \end{proof}
  \qed
\end{proof}

\begin{prop}[CC]
  \label{prop:order:long_line_zero_one}
  For all $0 < \alpha < \Omega$, the interval $[(0, 0), (\alpha, 0))$ in
  $S_\Omega \times [0, 1)$ is order isomorphic to $[0, 1)$ in $\mathbb{R}$.
\end{prop}

\begin{proof}
  \pf
  \step{<1>1}{If $[(0, 0), (\alpha, 0)) \cong [0, 1)$ then $[(0, 0), (\alpha
    +
    1,
    0)) \cong [0, 1)$}
  \begin{proof}
    \pf\ By Proposition \ref{prop:order:zero_one_twice}.
  \end{proof}
  \step{<1>2}{Let $\lambda$ be a limit ordinal, $0 < \lambda < \Omega$. If,
    for
    all $\alpha$ with $0 < \alpha < \lambda$, we have $[(0, 0), (\alpha, 0))
    \cong [0, 1)$, then $[(0, 0), (\lambda, 0)) \cong [0, 1)$.}
  \begin{proof}
    \pf\ By Propositoin \ref{prop:order:zero_one_countable}.
  \end{proof}
  \qedstep
  \begin{proof}
    \pf\ By transfinite induction.
  \end{proof}
  \qed
\end{proof}


  \chapter{Real Analysis}

  \begin{df}[Cantor Set]
    Define a sequence of sets $A_n \subseteq [0, 1]$ by:
    \begin{align*}
      A_0 & = [0,1] \\
      A_{n+1} & = A_n \setminus \bigcup_{k=0}^{3^{n-1} - 1}
      \left( \frac{3k+1}{3^n},     \frac{3k+2}{3^n} \right)
    \end{align*}
    The \emph{Cantor set} is $\bigcap_{n=0}^\infty A_n$.
  \end{df}


  \chapter{Topological Spaces}

  \section{Topologies}

  \begin{df}[Topology]
    A \emph{topology} on a set $X$ is a set $\mathcal{T} \subseteq \mathcal{P}
    X$ such that:
    \begin{enumerate}
      \item $X \in \mathcal{T}$;
      \item for all $U, V \in \mathcal{T}$, we have $U \cap V \in \mathcal{T}$;
      \item For all $\mathcal{A} \subseteq \mathcal{T}$, we have $\bigcup
      \mathcal{A} \in \mathcal{T}$.
    \end{enumerate}
    A \emph{topological space} $X$ consists of a set $X$ and a topology on $X$.
    The elements of $X$ are called \emph{points} and the elements of
    $\mathcal{T}$ are called \emph{open sets}.
  \end{df}

  \begin{prop}
    \label{prop:topology:topological_space:emptyset}
    In any topological space, the empty set is open.
  \end{prop}

  \begin{proof}
    \pf\ Immediate from axiom 3. \qed
  \end{proof}

  \begin{df}[Discrete Topology]
    The \emph{discrete} topology on a set $X$ is $\mathcal{P} X$.
  \end{df}

  \begin{df}[Indiscrete Topology]
    The \emph{indiscrete} topology on a set $X$ is $\{ \emptyset, X \}$.
  \end{df}

  \begin{df}[Open Cover] % TODO Move?
    Let $X$ be a topological space. A cover $\mathcal{C} \subseteq \mathcal{P}
    X$ of $X$ is an \emph{open cover} iff every member of $\mathcal{C}$ is open.
  \end{df}

  \begin{df}[Finer, Coarser]
    Let $\mathcal{T}$, $\mathcal{T}'$ be topologies on a set $X$. Then
    $\mathcal{T}$ is \emph{finer} than $\mathcal{T}'$, and $\mathcal{T}'$ is
    \emph{coarser} than $\mathcal{T}$, iff $\mathcal{T}' \subseteq \mathcal{T}$.

    The topology $\mathcal{T}$ is \emph{strictly} finer than $\mathcal{T}'$,
    and
    $\mathcal{T}'$ is \emph{strictly} coarser than $\mathcal{T}$, iff
    $\mathcal{T} \subset \mathcal{T}'$.

    The topologies $\mathcal{T}$ and $\mathcal{T}'$ are \emph{comparable} iff
    $\mathcal{T} \subseteq \mathcal{T}'$ or $\mathcal{T}' \subseteq
    \mathcal{T}$.
  \end{df}

  \begin{df}[Finite Complement Topology]
    The \emph{finite complement topology} on a set $X$ is $\{ U : X \setminus U
    \text{ is finite} \} \cup \{ X \}$.
  \end{df}

  \begin{df}[Isolated Point]
    Let $X$ be a topological space and $a \in X$. Then $a$ is an \emph{isolated
      point} iff $\{a\}$ is open.
  \end{df}

  \section{Neighbourhoods}

  \begin{df}[Neighbourhood]
    Let $X$ be a topological space and $A \subseteq X$. A \emph{neighbourhood}
    of $A$ is an set that includes an open set that includes $A$.

    A \emph{neighbourhood} of a point $a$ is a neighbourhood of $\{a\}$.
  \end{df}

  \begin{prop}
    If $N$ is a neighbourhood of $A$ and $B \subseteq A$ then $N$ is a
    neighbourhood of $B$.
  \end{prop}

  \begin{proof}
    \pf\ Immediate from definitions. \qed
  \end{proof}

  \begin{prop}
    \label{prop:topology:neighbourhood:open}
    A set $U$ is open if and only if it is a neighbourhood of each of its
    points.
  \end{prop}

  \begin{proof}
    \pf
    \step{<1>1}{\pflet{$X$ be a topological space and $A \subseteq X$}}
    \step{<1>2}{If $U$ is a neighbourhood of each of its points then $A$
      is
      open.}
    \begin{proof}
      \step{<2>1}{\assume{$U$ includes a neighbourhood of each of its points}
        \prove{$U = \bigcup \{ V \subseteq U : V \text{ is open} \}$}}
      \step{<2>2}{$\bigcup \{ V \subseteq U : V \text{ is open} \} \subseteq U$}
      \begin{proof}
        \pf\ Set theory.
      \end{proof}
      \step{<2>3}{$U \subseteq \bigcup \{ V \subseteq U : V \text{ is open} \}$}
      \begin{proof}
        \pf\ Immediate from \stepref{<2>1}.
      \end{proof}
    \end{proof}
    \step{<1>3}{If $U$ is open then $U$ is a neighbourhood of each of
      its points.}
    \begin{proof}
      \pf\ Immediate from definitions.
    \end{proof}
    \qed
  \end{proof}

  \begin{prop}
    \label{prop:topology:neighbourhood:monotone}
    If $M$ is a neighbourhood of $A$ and $M \subseteq N$ then $N$ is a
    neighbourhood of $A$.
  \end{prop}

  \begin{proof}
    \pf\ Immediate from definitions. \qed
  \end{proof}

  \begin{prop}
    \label{prop:topology:neighbourhood:intersection}
    If $M$ and $N$ are neighbourhoods of $A$ then $M \cap N$ is a neighbourhood
    of $A$.
  \end{prop}

  \begin{proof}
    \pf\ Pick open sets $U$ and $V$ such that $A \subseteq U \subseteq M$
    and $A \subseteq N \subseteq V$. Then $A \subseteq U \cap V \subseteq M
    \cap N$.
  \end{proof}

  \begin{prop}
    If $N$ is a neighbourhood of $x$ then $x \in N$.
  \end{prop}

  \begin{proof}
    \pf\ Immediate from definitions. \qed
  \end{proof}

  \begin{prop}
    If $N$ is a neighbourhood of $x$ then there exists a neighbourhood $U$ of
    $x$ such that, for all $y \in U$, $M$ is a neighbourhood of $y$.
  \end{prop}

  \begin{proof}
    \pf\ Pick an open set $U$ such that $x \in U \subseteq N$. \qed
  \end{proof}

  \begin{thm}
    Let $X$ be a set and $\rhd \subseteq \mathcal{P} X \times X$ a relation
    such that:
    \begin{enumerate}
      \item If $M \rhd x$ and $M \subseteq N$ then $N \rhd x$
      \item $X \rhd x$ for all $x \in X$
      \item If $M \rhd x$ and $N \rhd x$ then $M \cap N \rhd x$
      \item If $N \rhd x$ then $x \in N$
      \item If $M \rhd x$ then there exists $N \rhd x$ such that, for all $y
      \in N$, $M \rhd y$.
    \end{enumerate}
    Then there exists a unique topology $\mathcal{T}$ such that $N \rhd x$ iff
    $N$ is a neighbourhood of $x$.
  \end{thm}

  \begin{proof}
    \pf
    \step{<1>1}{\pflet{$\rhd$ be a relation satisfying 1--3}}
    \step{<1>2}{\pflet{$\mathcal{T} = \{ U \in \mathcal{P} X : \forall x \in U.
        U
        \rhd x \}$}}
    \step{<1>3}{$\mathcal{T}$ is a topology.}
    \begin{proof}
      \step{<2>1}{$X \in \mathcal{T}$}
      \begin{proof}
        \pf\ By axiom 2
      \end{proof}
      \step{<2>2}{For all $U, V \in \mathcal{T}$ we have $U \cap V \in
        \mathcal{T}$}
      \begin{proof}
        \pf\ By axiom 3
      \end{proof}
      \step{<2>3}{For all $\mathcal{A} \subseteq \mathcal{T}$ we have $\bigcup
        \mathcal{A} \in \mathcal{T}$}
      \begin{proof}
        \step{<3>1}{\pflet{$x \in \bigcup \mathcal{A}$}}
        \step{<3>2}{\pick\ $U \in \mathcal{A}$ such that $x \in U$}
        \step{<3>3}{$U \rhd x$}
        \step{<3>4}{$\bigcup \mathcal{A} \rhd x$}
        \begin{proof}
          \pf\ By axiom 1
        \end{proof}
      \end{proof}
    \end{proof}
    \step{<1>4}{In $\mathcal{T}$, $N \rhd x$ iff $N$ is a neighbourhood of $x$.}
    \begin{proof}
      \step{<2>1}{If $N \rhd x$ then $N$ is a neighbourhood of $x$}
      \begin{proof}
        \step{<3>1}{\assume{$N \rhd x$}}
        \step{<3>2}{$x \in N$}
        \begin{proof}
          \pf\ By axiom 4
        \end{proof}
        \step{<3>3}{\pflet{$U = \{ y \in N : N \rhd y \}$}}
        \step{<3>4}{$U$ is open}
        \begin{proof}
          \step{<4>1}{\pflet{$y \in U$} \prove{$U \rhd y$}}
          \step{<4>2}{$N \rhd y$}
          \step{<4>3}{\pick\ $W \rhd y$ such that, for all $z \in W$, $N \rhd
            z$}
          \begin{proof}
            \pf\ By axiom 5
          \end{proof}
          \step{<4>4}{$W \subseteq U$}
          \step{<4>5}{$U \rhd y$}
          \begin{proof}
            \pf\ By axiom 1
          \end{proof}
        \end{proof}
        \step{<3>5}{$x \in U \subseteq N$}
      \end{proof}
      \step{<2>2}{If $N$ is a neighbourhood of $x$ then $N \rhd x$}
      \begin{proof}
        \step{<3>1}{\pflet{$N$ be a neighbourhood of $x$}}
        \step{<3>2}{\pick\ $U$ open such that $x \in U \subseteq N$}
        \step{<3>3}{$U \rhd x$}
        \begin{proof}
          \pf\ By \stepref{<1>2}
        \end{proof}
        \step{<3>4}{$N \rhd x$}
        \begin{proof}
          \pf\ By axiom 1
        \end{proof}
      \end{proof}
    \end{proof}
    \step{<1>5}{$\mathcal{T}$ is unique.}
    \begin{proof}
      \pf\ By Proposition \ref{prop:topology:neighbourhood:open}.
    \end{proof}
    \qed
  \end{proof}

  \begin{df}[Sufficiently Close]
    Let $X$ be a topological space, $a \in X$, and $P$ be a property of points
    of $X$. We write ``For all $x$ sufficiently close to $a$, $P(x)$'' to mean
    ``There exists a neighbourhood $N$ of $a$ such that, for all $x \in N$,
    $P(x)$.''
  \end{df}

  \section{Local Bases}

  \begin{df}[Local Basis]
    Let $X$ be a topological space and $x \in X$. A \emph{local basis} at $x$
    is a set $\mathcal{B}$ of open neighbourhoods of $x$ such that every
    neighbourhood of $x$ includes a member of $\mathcal{B}$. We call the
    elements of $\mathcal{B}$ \emph{basic open neighbourhoods}.
  \end{df}

  \begin{prop}
    Let $\mathcal{B}$ be a local basis at $x$ and $M, N \in \mathcal{B}$. Then
    there exists $P \in \mathcal{B}$ such that $P \subseteq M \cap N$.
  \end{prop}

  \begin{proof}
    \pf\ This holds because $M \cap N$ is a neighbourhood of $x$ (Proposition
    \ref{prop:topology:neighbourhood:intersection}). \qed
  \end{proof}

  \begin{prop}
    \label{prop:topology:local_basis:characterisation}
    Let $X$ be a topological space, $x \in X$ and $\mathcal{B} \subseteq
    \mathcal{P} X$. Then $\mathcal{B}$ is a local basis at $x$ iff
    $\mathcal{B}$ is a set of open neighbourhoods of $x$ such that every open
    neighbourhood of $x$ includes a member of $\mathcal{B}$.
  \end{prop}

  \begin{proof}
    \pf
    \step{<1>1}{If $\mathcal{B}$ is a local basis at $x$ then
      $\mathcal{B}$ is a set of open neighbourhoods of $x$ such that every open
      neighbourhood of $x$ includes a member of $\mathcal{B}$}
    \begin{proof}
      \pf\ Trivial.
    \end{proof}
    \step{<1>2}{If $\mathcal{B}$ is a set of open neighbourhoods of $x$ such
      that
      every open neighbourhood of $x$ includes a member of $\mathcal{B}$ then
      $\mathcal{B}$ is a local basis at $x$.}
    \begin{proof}
      \pf\ Every neighbourhood of $x$ includes an open neighbourhood of $x$,
      which therefore includes an element of $\mathcal{B}$.
    \end{proof}
    \qed
  \end{proof}

  \section{Bases}

  \begin{df}[Basis for a Topology]
    Let $(X, \mathcal{T})$ be a topological space. A \emph{basis} for the
    topology on $X$ is a
    set of open sets $\mathcal{B}$ such that every open set is a union of
    members of $\mathcal{B}$. The members of $\mathcal{B}$ are called
    \emph{basic open sets}, and $\mathcal{T}$ is called the topology
    \emph{generated} by $\mathcal{B}$.
  \end{df}

  \begin{prop}
    \label{prop:topology:basis:open}
    Let $(X, \mathcal{T})$ be a topological space and $\mathcal{B} \subseteq
    \mathcal{P} X$. Then the following are equivalent:
    \begin{enumerate}
      \item $\mathcal{B}$ is a basis for $\mathcal{T}$.
      \item A set $U$ is open if and only if, for all $x \in U$, there exists
      $B \in \mathcal{B}$ such that $x \in B \subseteq U$.
      \item $\mathcal{T}$ is the set of all unions of subsets of $\mathcal{B}$.
      \item Every member of $\mathcal{B}$ is open and, for all $x \in X$
      and every open neighbourhood $U$ of $x$, there exists $B \in \mathcal{B}$
      such that $x \in B \subseteq U$.
      \item For all $x \in X$, the set $\{ B \in \mathcal{B} : x \in B \}$ is a
      local basis at $x$.
    \end{enumerate}
  \end{prop}

  \begin{proof}
    \pf
    \step{<1>1}{$1 \Rightarrow 2$}
    \begin{proof}
      \step{<2>1}{\assume{$\mathcal{B}$ is a basis for the topology
          $\mathcal{T}$.}}
      \step{<2>2}{For all $U \in \mathcal{T}$ and $x \in U$, there exists $B
        \in
        \mathcal{B}$ such that $x \in B$}
      \begin{proof}
        \pf\ Immediate from the definition of basis (\stepref{<2>1}).
      \end{proof}
      \step{<2>3}{For all $U \subseteq X$, if $\forall x \in U. \exists B \in
        \mathcal{B}. x \in B \subseteq U$ then $U \in \mathcal{T}$}
      \begin{proof}
        \pf\ By Proposition \ref{prop:topology:neighbourhood:open}.
      \end{proof}
    \end{proof}
    \step{<1>2}{$2 \Leftrightarrow 3$}
    \begin{proof}
      \pf\ From Lemma \ref{lm:set_theory:union_of_subsets}.
    \end{proof}
    \step{<1>3}{$3 \Rightarrow 1$}
    \begin{proof}
      \pf\ Trivial.
    \end{proof}
    \step{<1>4}{$2 \Rightarrow 4$}
    \begin{proof}
      \pf\ Trivial.
    \end{proof}
    \step{<1>5}{$4 \Rightarrow 2$}
    \begin{proof}
      \pf
      \step{<2>1}{\assume{4}}
      \step{<2>2}{If $U$ is open then, for all $x \in U$, there exists $B \in
        \mathcal{B}$ such that $x \in B \subseteq U$}
      \begin{proof}
        \pf\ Immediate from \stepref{<2>1}.
      \end{proof}
      \step{<2>3}{If, for all $x \in U$, there exists $B \in \mathcal{B}$ such
        that $x \in B \subseteq U$, then $U$ is open.}
      \begin{proof}
        \pf\ By Proposition \ref{prop:topology:neighbourhood:open} using the
        fact that every member of $\mathcal{B}$ is open (\stepref{<2>1}).
      \end{proof}
    \end{proof}
    \step{<1>6}{$4 \Leftrightarrow 5$}
    \begin{proof}
      \pf\ From Proposition \ref{prop:topology:local_basis:characterisation}.
    \end{proof}
    \qed
  \end{proof}

  \begin{cor}
    \label{cor:topology:basis:coarsest}
    If $\mathcal{B}$ is a basis for the topology $\mathcal{T}$, then
    $\mathcal{T}$ is the coarsest topology in which every element of
    $\mathcal{B}$ is open.
  \end{cor}

  \begin{lm}
    \label{lm:topology:basis:generate}
    Let $X$ be a set and $\mathcal{B} \subseteq \mathcal{P} X$. Then
    $\mathcal{B}$ is a basis for a topology $\mathcal{T}$ on $X$ if and only if:
    \begin{enumerate}
      \item
      $\bigcup \mathcal{B} = X$
      \item
      for all $B_1, B_2 \in \mathcal{B}$ and $x \in B_1 \cap B_2$, there exists
      $B_3
      \in \mathcal{B}$ such that $x \in B_3 \subseteq B_1 \cap B_2$.
    \end{enumerate}
    In this case, $\mathcal{T}$ is unique.
  \end{lm}

  \begin{proof}
    \pf
    \step{<1>1}{If $\mathcal{B}$ is a basis for a topology then $\bigcup
      \mathcal{B}
      = X$}
    \begin{proof}
      \step{<2>1}{\assume{$\mathcal{B}$ is a basis for the topology
          $\mathcal{T}$}}
      \step{<2>2}{\pflet{$x \in X$}}
      \step{<2>3}{There exists $B \in \mathcal{B}$ such that $x \in B$}
      \begin{proof}
        \pf\ From the definition of basis, since $X \in \mathcal{T}$.
        (\stepref{<2>1}, \stepref{<2>2}).
      \end{proof}
    \end{proof}
    \step{<1>2}{If $\mathcal{B}$ is a basis for a topology then it satisfies
      condition 2}
    \begin{proof}
      \step{<2>1}{\assume{$\mathcal{B}$ is a basis for the topology
          $\mathcal{T}$}}
      \step{<2>2}{\pflet{$B_1, B_2 \in \mathcal{B}$}}
      \step{<2>3}{$B_1, B_2 \in \mathcal{T}$}
      \begin{proof}
        \pf\ From the definition of basis (\stepref{<2>1}, \stepref{<2>2}).
      \end{proof}
      \step{<2>4}{$B_1 \cap B_2 \in \mathcal{T}$}
      \begin{proof}
        \pf\ By the definition of topology, the open sets in $\mathcal{T}$ are
        closed under binary intersection (\stepref{<2>1}, \stepref{<2>3})
      \end{proof}
      \step{<2>5}{For all $x \in B_1 \cap B_2$, there exists $B_3 \in
        \mathcal{B}$
        such that $x \in B_3 \subseteq B_1 \cap B_2$}
      \begin{proof}
        \pf\ From the definition of basis (\stepref{<2>1}, \stepref{<2>4})
      \end{proof}
    \end{proof}
    \step{<1>3}{If $\mathcal{B}$ satisfies conditions 1 and 2 then $\mathcal{T}
      =
      \{
      U \subseteq X : \forall x \in U. \exists B \in \mathcal{B}. x \in B
      \subseteq U \}$ is a topology and $\mathcal{B}$ is a basis for
      $\mathcal{T}$.}
    \begin{proof}
      \step{<2>1}{\assume{$\mathcal{B}$ satisfies conditions 1 and 2}}
      \step{<2>2}{$X \in \mathcal{T}$}
      \begin{proof}
        \pf\ For all $x \in X$, there exists $B \in \mathcal{B}$ such that $x
        \in
        B \subseteq X$ by condition 1 (\stepref{<2>1}).
      \end{proof}
      \step{<2>3}{For all $\mathcal{A} \subseteq \mathcal{T}$, we have $\bigcup
        \mathcal{A} \in \mathcal{T}$}
      \begin{proof}
        \step{<3>1}{\pflet{$\mathcal{A} \subseteq \mathcal{T}$}}
        \step{<3>2}{\pflet{$x \in \bigcup \mathcal{A}$}}
        \step{<3>3}{\pick\ $U \in \mathcal{A}$ such that $x \in U$}
        \begin{proof}
          \pf\ From \stepref{<3>2}.
        \end{proof}
        \step{<3>4}{\pick\ $B \in \mathcal{B}$ such that $x \in B \subseteq U$}
        \begin{proof}
          \pf\ Since $U \in \mathcal{T}$, using the definition of $\mathcal{T}$
          (\stepref{<3>1}, \stepref{<3>3})
        \end{proof}
        \step{<3>5}{$x \in B \subseteq \bigcup \mathcal{A}$}
        \begin{proof}
          \pf\ From \stepref{<3>3} and \stepref{<3>4}.
        \end{proof}
      \end{proof}
      \step{<2>4}{For all $U, V \in \mathcal{T}$, we have $U \cap V \in
        \mathcal{T}$}
      \begin{proof}
        \step{<3>1}{\pflet{$U, V \in \mathcal{T}$}}
        \step{<3>2}{\pflet{$x \in U \cap V$}}
        \step{<3>3}{\pick\ $B_1, B_2 \in \mathcal{B}$ such that $x \in B_1
          \subseteq U$ and $x \in B_2 \subseteq V$}
        \begin{proof}
          \pf\ From \stepref{<3>1}, \stepref{<3>2} and the definition of
          $\mathcal{T}$.
        \end{proof}
        \step{<3>4}{\pick\ $B_3 \in \mathcal{B}$ such that $x \in B_3 \subseteq
          B_1
          \cap B_2$}
        \begin{proof}
          \pf\ Using condition 2 (\stepref{<2>1}, \stepref{<3>3}).
        \end{proof}
        \step{<3>5}{$x \in B_3 \subseteq U \cap V$}
        \begin{proof}
          \pf\ From \stepref{<3>3} and \stepref{<3>4}.
        \end{proof}
      \end{proof}
      \step{<2>5}{$\bigcup \mathcal{B} = X$}
      \begin{proof}
        \pf\ This is condition 1 (\stepref{<2>1}).
      \end{proof}
      \step{<2>6}{For all $U \in \mathcal{T}$ and $x \in U$, there exists $B
        \in
        \mathcal{B}$ such that $x \in B \subseteq U$}
      \begin{proof}
        \pf\ Immediate from the definition of $\mathcal{T}$.
      \end{proof}
    \end{proof}
    \step{<1>4}{$\mathcal{T}$ is unique.}
    \begin{proof}
      \pf\ From Proposition \ref{prop:topology:basis:open}.
    \end{proof}
    \qed
  \end{proof}

  \begin{cor}
    \label{cor:topology:basis:generate}
    Let $X$ be a set and $\mathcal{B} \subseteq \mathcal{P} X$ be such that
    $\bigcup \mathcal{B} = X$ and $\mathcal{B}$ is closed under binary
    intersection. Then $\mathcal{B}$ is a basis for a unique topology on $X$.
  \end{cor}

  \begin{lm}
    \label{lm:topology:basis:finer}
    Let $\mathcal{B}$ and $\mathcal{B}'$ be bases for the topologies
    $\mathcal{T}$ and $\mathcal{T}'$ on $X$ respectively. Then $\mathcal{T}
    \subseteq \mathcal{T}'$ if and only if, for all $B \in \mathcal{B}$ and $x
    \in B$, there exists $B' \in \mathcal{B}'$ such that $x \in B' \subseteq B$.
  \end{lm}

  \begin{proof}
    \pf
    \step{<1>1}{If $\mathcal{T} \subseteq \mathcal{T}'$ then, for all $B \in
      \mathcal{B}$ and $x
      \in B$, there exists $B' \in \mathcal{B}'$ such that $x \in B' \subseteq
      B$.}
    \begin{proof}
      \step{<2>1}{\assume{$\mathcal{T} \subseteq \mathcal{T}'$}}
      \step{<2>2}{\pflet{$B \in \mathcal{B}$ and $x \in B$}}
      \step{<2>3}{$B \in \mathcal{T}$}
      \begin{proof}
        \pf\ This holds because $\mathcal{B} \subseteq \mathcal{T}$ by the
        definition of basis. (\stepref{<2>2})
      \end{proof}
      \step{<2>4}{$B \in \mathcal{T}'$}
      \begin{proof}
        \pf\ From \stepref{<2>1} and \stepref{<2>3}.
      \end{proof}
      \step{<2>5}{There exists $B' \in \mathcal{B}'$ such that $x \in B'
        \subseteq
        B$.}
    \end{proof}
    \step{<1>2}{If, for all $B \in \mathcal{B}$ and $x
      \in B$, there exists $B' \in \mathcal{B}'$ such that $x \in B' \subseteq
      B$,   then $\mathcal{T} \subseteq \mathcal{T}'$.}
    \begin{proof}
      \step{<2>1}{\assume{For all $B \in \mathcal{B}$ and $x
          \in B$, there exists $B' \in \mathcal{B}'$ such that $x \in B'
          \subseteq B$}}
      \step{<2>2}{\pflet{$U \in \mathcal{T}$} \prove{$U \in \mathcal{T}'$}}
      \step{<2>3}{\pflet{$x \in U$}}
      \step{<2>4}{\pick\ $B \in \mathcal{B}$ such that $x \in B \subseteq U$}
      \begin{proof}
        \pf\ Since $\mathcal{B}$ is a basis for $\mathcal{T}$ (\stepref{<2>2},
        \stepref{<2>3}).
      \end{proof}
      \step{<2>5}{\pick\ $B' \in \mathcal{B}'$ such that $x \in B' \subseteq B$}
      \begin{proof}
        \pf\ From \stepref{<2>1} and \stepref{<2>4}.
      \end{proof}
      \step{<2>6}{$x \in B' \subseteq U$}
      \begin{proof}
        \pf\ From \stepref{<2>4} and \stepref{<2>5}.
      \end{proof}
      \qedstep
      \begin{proof}
        \pf\ By Proposition \ref{prop:topology:basis:open}.
      \end{proof}
    \end{proof}
    \qed
  \end{proof}

  \begin{df}[Lower Limit Topology]
    The \emph{lower limit topology} on $\mathbb{R}$ is the one generated by the
    set of all half-open intervals of the form $[a,b)$. We write $\mathbb{R}_l$
    for the topological space consisting of $\mathbb{R}$ under this topology.

    We prove this is a topology.
  \end{df}

  \begin{proof}
    \pf
    \step{<1>1}{\pflet{$\mathcal{B}$ be the set of all half-open intervals of
        the
        form $[a,b)$.}}
    \step{<1>2}{$\bigcup \mathcal{B} = \mathbb{R}$}
    \begin{proof}
      \pf\ For all $x \in \mathbb{R}$, we have $x \in [x, x+1) \in \mathcal{B}$.
    \end{proof}
    \step{<1>3}{For all $B_1, B_2 \in \mathcal{B}$ and $x \in B_1 \cap B_2$,
      there
      exists $B_3 \in \mathcal{B}$ such that $x \in B_3 \subseteq B_1 \cap
      B_2$.}
    \begin{proof}
      \pf\ If $x \in [a,b) \cap [c,d)$ then $x \in [\max(a,c), \min(b,d))
      \subseteq [a,b) \cap [c,d)$.
    \end{proof}
    \qedstep
    \begin{proof}
      \pf\ By Lemma \ref{lm:topology:basis:generate}.
    \end{proof}
    \qed
  \end{proof}

  \begin{df}[$K$-topology]
    The \emph{$K$-topology} on $\mathbb{R}$ is the one generated by the set of
    all
    open intervals $(a,b)$ and all sets of the form $(a,b) \setminus K$, where
    $K =
    \{ 1/n : n \in \mathbb{Z}^+ \}$. We write $\mathbb{R}_K$ for the
    topological
    space consisting of $\mathbb{R}$ under this topology.

    We prove this is a topology.
  \end{df}

  \begin{proof}
    \pf
    \step{<1>1}{\pflet{$\mathcal{B} = \{ (a,b) : a, b \in \mathbb{R}, a < b \}
        \cup
        \{ (a,b) \setminus K : a, b \in \mathbb{R}, a < b \}$}}
    \step{<1>2}{$\bigcup \mathcal{B} = \mathbb{R}$}
    \begin{proof}
      \pf\ For all $x \in \mathbb{R}$, we have $x \in (x-1, x+1) \in
      \mathcal{B}$.
    \end{proof}
    \step{<1>3}{For all $B_1, B_2 \in \mathcal{B}$ and $x \in B_1 \cap B_2$,
      there
      exists $B_3 \in \mathcal{B}$ such that $x \in B_3 \subseteq B_1 \cap
      B_2$.}
    \begin{proof}
      \step{<2>1}{\pflet{$B_1, B_2 \in \mathcal{B}$ and $x \in B_1 \cap B_2$}
        \prove{There exists $B_3 \in \mathcal{B}$ such that $x \in B_3
          \subseteq
          B_1 \cap B_2$}}
      \step{<2>2}{\case{$B_1 = (a, b)$, $B_2 = (c, d)$}}
      \begin{proof}
        \pf\ Take $B_3 = (\max(a,c), \min(b,d))$
      \end{proof}
      \step{<2>3}{\case{$B_1 = (a, b)$, $B_2 = (c, d) \setminus K$}}
      \begin{proof}
        \pf\ Take $B_3 = (\max(a,c), \min(b,d)) \setminus K$
      \end{proof}
      \step{<2>4}{\case{$B_1 = (a, b) \setminus K$, $B_2 = (c, d)$}}
      \begin{proof}
        \pf\ Take $B_3 = (\max(a,c), \min(b,d)) \setminus K$
      \end{proof}
      \step{<2>5}{\case{$B_1 = (a, b) \setminus K$, $B_2 = (c, d) \setminus K$}}
      \begin{proof}
        \pf\ Take $B_3 = (\max(a,c), \min(b,d)) \setminus K$
      \end{proof}
    \end{proof}
    \qedstep
    \begin{proof}
      \pf\ By Lemma \ref{lm:topology:basis:generate}.
    \end{proof}
    \qed
  \end{proof}

  \begin{lm}
    The lower limit topology and the $K$-topology are incomparable.
  \end{lm}

  \begin{proof}
    \pf\ $[0, 1)$ is not open in the $K$-topology. $(-1, 1) \setminus K$ is not
    open in the lower limit topology, because there is no half-open interval
    $[a,
    b)$ such that $0 \in [a,b) \subseteq (-1, 1) \setminus K$. \qed
  \end{proof}

  \begin{prop}
    The set of all singletons is a basis for any discrete space.
  \end{prop}

  \begin{proof}
    \pf\ Easy. \qed
  \end{proof}

   \begin{df}[Line with Two Origins]
   The \emph{line with two origins} is the set $\mathbb{R} \setminus \{ 0 \}
   \cup \{ p,q \}$ under the topology generated by the basis consisting of:
   \begin{itemize}
     \item all open intervals in $\mathbb{R}$ that do not contain $0$;
     \item all sets of the form $(-a, 0) \cup \{ p \} \cup (0, a)$ where $a
> 0$;
\item all sets of the form $(-a, 0) \cup \{ q \} \cup (0, a)$ where $a > 0$
   \end{itemize}
 \end{df}

  \section{Closed Sets}

  \begin{df}[Closed]
    Let $X$ be a topological space and $A \subseteq X$. Then $A$ is
    \emph{closed}
    iff $X \setminus A$ is open.
  \end{df}

  \begin{prop}
    \label{prop:topology:closed:empty}
    In any topological space $X$, the empty set $\emptyset$ is closed.
  \end{prop}

  \begin{proof}
    \pf\ This holds because $X \setminus \emptyset = X$ is open. \qed
  \end{proof}

  \begin{prop}
    \label{prop:topology:closed:whole_set}
    In any topological space $X$, the set $X$ is closed.
  \end{prop}

  \begin{proof}
    \pf\ This holds because $X \setminus X = \emptyset$ is open. \qed
  \end{proof}

  \begin{prop}
    \label{prop:topology:closed:union}
    The union of two closed sets is closed.
  \end{prop}

  \begin{proof}
    \pf\ If $C$ and $D$ are closed then $X \setminus (C \cup D) = (X \setminus
    C) \cup (X \setminus D)$ is open. \qed
  \end{proof}

  \begin{prop}
    \label{prop:topology:closed:intersection}
    In any topological space, the intersection of a nonempty set of closed sets
    is closed.
  \end{prop}

  \begin{proof}
    \pf\ Let $\mathcal{C}$ be a nonempty set of closed sets. Then $X \setminus
    \bigcap \mathcal{C} = \bigcup \{ X \setminus C : C \in \mathcal{C} \}$ is
    open. \qed
  \end{proof}

  \begin{prop}
    \label{prop:topology:closed:open}
    Let $X$ be a topological space and $U \subseteq X$. Then $U$ is open if and
    only if $X \setminus U$ is closed.
  \end{prop}

  \begin{proof}
    \pf\ Immediate from definitions.
  \end{proof}

  \begin{thm}
    \label{thm:topology:closed}
    Let $X$ be a set and $\mathcal{C} \subseteq \mathcal{P} X$. Suppose:
    \begin{enumerate}
      \item $\emptyset, X \in \mathcal{C}$;
      \item for all nonempty $\mathcal{A} \subseteq \mathcal{C}$, we have
      $\bigcap
      \mathcal{A} \in \mathcal{C}$;
      \item for all $C, D \in \mathcal{C}$, we have $C \cup D \in \mathcal{C}$.
    \end{enumerate}
    Then there exists a unique topology on $X$ under which $\mathcal{C}$ is the
    set of all closed sets, namely
    \[ \mathcal{T} = \{ U \subseteq X : X \setminus U \in \mathcal{C} \} \]
  \end{thm}

  \begin{proof}
    \pf
    \step{<1>1}{\pflet{$\mathcal{C}$ be a set satisfying 1--3}}
    \step{<1>2}{\pflet{$\mathcal{T} = \{ X \setminus C : C \in \mathcal{C}
        \}$}}
    \step{<1>3}{$\mathcal{T}$ is a topology}
    \begin{proof}
      \step{<2>1}{$X \in \mathcal{T}$}
      \begin{proof}
        \pf\ $X \setminus X = \emptyset \in \mathcal{C}$ by condition 1.
      \end{proof}
      \step{<2>2}{For all $\mathcal{A} \subseteq \mathcal{T}$ we have
        $\bigcup
        \mathcal{A} \in \mathcal{T}$.}
      \begin{proof}
        \step{<3>1}{\pflet{$\mathcal{A} \subseteq \mathcal{T}$}}
        \step{<3>2}{\case{$\mathcal{A} = \emptyset$}}
        \begin{proof}
          \pf\ In this case, $X \setminus \bigcup \mathcal{A} = X \in
          \mathcal{C}$ by condition 1.
        \end{proof}
        \step{<3>3}{\case{$\mathcal{A}$ is nonempty}}
        \begin{proof}
          \pf\ In this case, we have $X \setminus \bigcup \mathcal{A} =
          \bigcap
          \{ X \setminus U : U \in \mathcal{A} \} \in \mathcal{C}$ by
          condition
          2.
        \end{proof}
      \end{proof}
      \step{<2>3}{For all $U, V \in \mathcal{T}$ we have $U \cap V \in
        \mathcal{T}$}
      \begin{proof}
        \pf\ $X \setminus (U \cap V) = (X \setminus U) \cup (X \setminus V)
        \in
        \mathcal{C}$ by condition 3.
      \end{proof}
    \end{proof}
    \step{<1>4}{$\mathcal{C}$ is the set of closed sets.}
    \begin{proof}
      \pf
      \begin{align*}
        C \text{ is closed} & \Leftrightarrow X \setminus C \in \mathcal{T} \\
        & \Leftrightarrow X \setminus (X \setminus C) \in \mathcal{C} \\
        & \Leftrightarrow C \in \mathcal{C}
      \end{align*}
    \end{proof}
    \step{<1>5}{$\mathcal{T}$ is unique.}
    \begin{proof}
      \pf\ By Proposition \ref{prop:topology:closed:open}.
    \end{proof}
    \qed
  \end{proof}

  \begin{df}[Closed Covering]
    A \emph{closed covering} of a topological space is a covering in which
    every member is a closed set.
  \end{df}

  \section{Locally Finite Families}

  \begin{df}[Locally Finite]
    Let $X$ be a topological space and $\{ A_i \}_{i \in I}$ a family of
    subsets of $X$. Then $\{ A_i \}_{i \in I}$ is \emph{locally finite} iff,
    for all $x \in X$, there exists a neighbourhood $N$ of $x$ such that there
    are only finitely many $i \in I$ such that $N$ intersects $A_i$.
  \end{df}

  \begin{prop}
    If $\{ A_i \}_{i \in I}$ is locally finite and $B_i \subseteq A_i$ for all
    $i$ then $\{ B_i \}_{i \in I}$ is locally finite.
  \end{prop}

  \begin{proof}
    \pf\ Immediate from definitions. \qed
  \end{proof}

  \begin{prop}
    Every finite family of open sets is locally finite.
  \end{prop}

  \begin{proof}
    \pf\ Trivial. \qed
  \end{proof}

  \section{Closure of a Set}

  \begin{df}[Closure]
    Let $X$ be a topological space and $A \subseteq X$. The \emph{closure} of
    $A$, $\Cl A$ or $\overline{A}$, is the intersection of all closed sets that
    include $A$.
  \end{df}

  \begin{proof}
    \pf\ This intersection always exists because $X$ is a closed set that
    includes $A$. \qed
  \end{proof}

  \begin{prop}
    \label{prop:topology:closure:A_sub_Abar}
    Let $X$ be a topological space and $A \subseteq X$. Then $A \subseteq
    \overline{A}$.
  \end{prop}

  \begin{proof}
    \pf\ Immediate from definitions. \qed
  \end{proof}

  \begin{prop}
    \label{prop:topology:closure:closed}
    Let $X$ be a topological space and $A \subseteq X$. Then $\overline{A}$ is
    closed.
  \end{prop}

  \begin{proof}
    \pf\ This follows from Proposition \ref{prop:topology:closed:intersection}.
    \qed
  \end{proof}

  \begin{prop}
    \label{prop:topology:closure:minimal}
    Let $X$ be a topological space and $A, C \subseteq X$. If $A \subseteq C$
    and $C$ is closed then $\overline{A} \subseteq C$.
  \end{prop}

  \begin{proof}
    \pf\ Immediate from definitions. \qed
  \end{proof}

  \begin{prop}
    \label{prop:topology:closure:monotone}
    Let $X$ be a topological space and $A, B \subseteq X$. If $A \subseteq B$
    then $\overline{A} \subseteq \overline{B}$.
  \end{prop}

  \begin{proof}
    \pf
    \step{<1>1}{\assume{$A \subseteq B$}}
    \step{<1>2}{$A \subseteq \overline{B}$}
    \begin{proof}
      \pf\ Proposition \ref{prop:topology:closure:A_sub_Abar}.
    \end{proof}
    \step{<1>3}{$\overline{A} \subseteq \overline{B}$}
    \begin{proof}
      \pf\ Propositions \ref{prop:topology:closure:closed},
      \ref{prop:topology:closure:minimal}.
    \end{proof}
    \qed
  \end{proof}

  \begin{prop}
    \label{prop:topology:closure:closed2}
    Let $X$ be a set and $A \subseteq X$. Then $A$ is closed if and only if $A
    = \overline{A}$.
  \end{prop}

  \begin{proof}
    \pf
    \step{<1>1}{If $A$ is closed then $A = \overline{A}$}
    \begin{proof}
      \step{<2>1}{\assume{$A$ is closed}}
      \step{<2>2}{$A \subseteq \overline{A}$}
      \begin{proof}
        \pf\ By Proposition \ref{prop:topology:closure:A_sub_Abar}.
      \end{proof}
      \step{<2>3}{$\overline{A} \subseteq A$}
      \begin{proof}
        \pf\ By Proposition \ref{prop:topology:closure:minimal} since $A
        \subseteq A$.
      \end{proof}
    \end{proof}
    \step{<1>2}{If $A = \overline{A}$ then $A$ is closed.}
    \begin{proof}
      \pf\ By Proposition \ref{prop:topology:closure:closed}.
    \end{proof}
    \qed
  \end{proof}

  \begin{cor}
    \[ \overline{\emptyset} = \emptyset \]
  \end{cor}

  \begin{thm}[Kuratowski Closure Axioms]
    Let $X$ be a set and $(\overline{\ }) : \mathcal{P} X \rightarrow
    \mathcal{P} X$ be a function such that:
    \begin{enumerate}
      \item $\overline{\emptyset} = \emptyset$
      \item For all $A \subseteq X$, $A \subseteq \overline{A}$
      \item For all $A \subseteq X$, $\overline{A} = \overline{\overline{A}}$
      \item For all $A, B \subseteq X$, $\overline{A \cup B} = \overline{A}
      \cup
      \overline{B}$
    \end{enumerate}
    Then there exists a unique topology $\mathcal{T}$ on $X$ such that
    $\overline{A}$ is the closure of $A$ for all $A \in \mathcal{P} X$.
  \end{thm}

  \begin{proof}
    \pf
    \step{<1>1}{For all $C, D \subseteq X$, if $C \subseteq D$ then
      $\overline{C} \subseteq \overline{D}$}
    \begin{proof}
      \step{<2>1}{\assume{$C \subseteq D$}}
      \step{<2>2}{$\overline{C} = \overline{D}$}
      \begin{proof}
        \pf
        \begin{align*}
          \overline{D} & = \overline{C \cup D} & (\text{\stepref{<2>1}}) \\
          & = \overline{C} \cup \overline{D} & (\text{axiom 4})
        \end{align*}
      \end{proof}
    \end{proof}

    \step{<1>2}{\pflet{$\mathcal{T}$ be the topology in which a set $C$ is
        closed iff $\overline{C} = C$.}}
    \begin{proof}
      \step{<2>1}{$\overline{\emptyset} = \emptyset$}
      \begin{proof}
        \pf\ This is axiom 1.
      \end{proof}
      \step{<2>2}{$\overline{X} = X$}
      \begin{proof}
        \pf\ By axiom 2.
      \end{proof}
      \step{<2>3}{For any set $\mathcal{A}$ of sets $C$ such that $\overline{C}
        =
        C$, we have $\overline{\bigcap \mathcal{A}} = \bigcap \mathcal{A}$}
      \begin{proof}
        \step{<3>1}{$\overline{\bigcap \mathcal{A}} \subseteq \bigcap
          \mathcal{A}$}
        \begin{proof}
          \step{<4>1}{\pflet{$C \in \mathcal{A}$}}
          \step{<4>2}{$\overline{\bigcap \mathcal{A}} \subseteq C$}
          \begin{proof}
            \pf
            \begin{align*}
              \overline{\bigcap \mathcal{A}} & \subseteq \overline{C} &
              (\text{\stepref{<1>1}}) \\
              & = C & (\text{\stepref{<4>1}})
            \end{align*}
          \end{proof}
        \end{proof}
        \qedstep
      \end{proof}
      \step{<2>4}{If $\overline{C} = C$ and $\overline{D} = D$ then
        $\overline{C
          \cup D} = C \cup D$}
      \begin{proof}
        \pf\ By axiom 4.
      \end{proof}
      \qedstep
      \begin{proof}
        \pf\ By Theorem \ref{thm:topology:closed}.
      \end{proof}
    \end{proof}
    \step{<1>3}{For all $A \subseteq X$, the closure of $A$ in $\mathcal{T}$ is
      $\overline{A}$}
    \begin{proof}
      \step{<2>1}{$\overline{A}$ is closed}
      \begin{proof}
        \pf\ From axiom 3.
      \end{proof}
      \step{<2>2}{If $A \subseteq C$ and $C$ is closed then $\overline{A}
        \subseteq C$}
      \begin{proof}
        \pf
        \begin{align*}
          C & = \overline{C} & (C \text{ is closed}) \\
          & = \overline{A \cup C} & (A \subseteq C) \\
          & = \overline{A} \cup \overline{C} & (\text{axiom 4})
        \end{align*}

      \end{proof}
    \end{proof}
    \qed
  \end{proof}

  \begin{thm}
    \label{thm:topology:closure:basis}
    Let $A$ be a subset of the topological space $X$ and $\mathcal{B}$ a basis
    for $X$. Then $x \in \overline{A}$ if and only if, for all $B \in
    \mathcal{B}$, if $x \in B$ then $B$ intersects $A$.
  \end{thm}

  \begin{proof}
    \pf
    \step{<1>1}{If $x \in \overline{A}$ then, for all $B \in \mathcal{B}$, if
      $x
      \in
      B$ then $B$ intersects $A$.}
    \begin{proof}
      \pf\ Immediate from Theorem \ref{thm:topology:closure:neighbourhoods}.
    \end{proof}
    \step{<1>2}{If, for all $B \in \mathcal{B}$, if $x \in B$ then $B$
      intersects
      $A$, then $x \in \overline{A}$.}
    \begin{proof}
      \step{<2>1}{\assume{for all $B \in \mathcal{B}$, if $x \in B$ then $B$
          intersects $A$.}}
      \step{<2>2}{\pflet{$U$ be a neighbourhood of $x$}}
      \step{<2>3}{\pick\ $B \in \mathcal{B}$ such that $x \in B \subseteq U$}
      \begin{proof}
        \pf\ $\mathcal{B}$ is a basis.
      \end{proof}
      \step{<2>4}{$B$ intersects $A$.}
      \begin{proof}
        \pf\ By \stepref{<2>1}.
      \end{proof}
      \step{<2>5}{$U$ intersects $A$.}
      \qedstep
      \begin{proof}
        \pf\ By Theorem \ref{thm:topology:closure:neighbourhoods}.
      \end{proof}
    \end{proof}
    \qed
  \end{proof}

  \section{Interior of a Set}

  \begin{df}[Interior]
    Let $X$ be a topological space and $A \subseteq X$.
    The \emph{interior} of $A$, $\Int A$, is the union of all open sets
    included
    in $A$.
  \end{df}

  \begin{lm}
    If $A \subseteq B$ then $\overline{A} \subseteq \overline{B}$.
  \end{lm}

  \begin{proof}
    \pf\ $\overline{B}$ is a closed set that includes $B$, hence includes $A$.
    \qed
  \end{proof}

  \begin{thm}
    \label{thm:topology:closure:neighbourhoods}
    Let $A$ be a subset of the topological space $X$ and $x \in X$. Then $x \in
    \overline{A}$ if and only if every neighbourhood of $x$ intersects $A$.
  \end{thm}

  \begin{proof}
    \pf
    \begin{align*}
      x \notin \overline{A} & \Leftrightarrow \exists C \text{ closed } (A
      \subseteq C \wedge x \notin C) \\
      & \Leftrightarrow \exists U \text{ open } (A \subseteq X \setminus U
      \wedge x
      \in U) \\
      & \Leftrightarrow \exists U \text{ open } (A \text{ intersects } U \wedge
      x
      \in U) & \qed
    \end{align*}
  \end{proof}

  \begin{lm}
    \label{lm:topology:closure_interior:complementary}
    \[ X \setminus \Int A = \overline{X \setminus A} \]
  \end{lm}

  \begin{proof}
    \pf
    \step{<1>1}{$X \setminus \Int A \subseteq \overline{X \setminus A}$}
    \begin{proof}
      \step{<2>1}{$X \setminus A \subseteq \overline{X \setminus A}$}
      \step{<2>2}{$X \setminus \overline{X \setminus A} \subseteq A$}
      \step{<2>3}{$X \setminus \overline{X \setminus A} \subseteq \Int A$}
    \end{proof}
    \step{<1>2}{$\overline{X \setminus A} \subseteq X \setminus \Int A$}
    \begin{proof}
      \step{<2>1}{$\Int A \subseteq A$}
      \step{<2>2}{$X \setminus A \subseteq X \setminus \Int A$}
      \step{<2>3}{$\overline{X \setminus A} \subseteq X \setminus \Int A$}
    \end{proof}
    \qed
  \end{proof}

  \section{Boundary}

  \begin{df}[Boundary]
    Let $X$ be a topological space and $A \subseteq X$. The \emph{boundary} of
    $A$, $\Bd A$, is $\overline{A} \cap \overline{X \setminus A}$.
  \end{df}

  \begin{lm}
    \label{lm:topology:boundary:difference}
    \[ \Bd A = \overline{A} \setminus \Int A \]
  \end{lm}

  \begin{proof}
    \pf\ From Lemma \ref{lm:topology:closure_interior:complementary}. \qed
  \end{proof}

  \begin{lm}
    \label{lm:topology:boundary:interior_disjoint}
    $\overline{A} = \Int A \cup \Bd A$
  \end{lm}

  \begin{proof}
    \pf
    \begin{align*}
      \Int A \cup \Bd A & = \Int A \cup (\overline{A} \cap (X \setminus \Int
      A))
      \\
      & = \Int A \cup \overline{A} \\
      & = \overline{A} & \qed
    \end{align*}
  \end{proof}

  \begin{cor}
    $\Bd A = \emptyset$ iff $A$ is open and closed.
  \end{cor}

  \begin{lm}
    For any set $U$, the following are equivalent:
    \begin{enumerate}
      \item $U$ is open.
      \item $\Bd U \cap U = \emptyset$
      \item $\Bd U = \overline{U} \setminus U$
    \end{enumerate}
  \end{lm}

  \begin{proof}
    \pf
    \step{<1>1}{$1 \Rightarrow 3$}
    \begin{proof}
      \pf\ From Lemma \ref{lm:topology:boundary:difference}.
    \end{proof}
    \step{<1>2}{$3 \Rightarrow 2$}
    \begin{proof}
      \pf\ Set theory.
    \end{proof}
    \step{<1>3}{$2 \Rightarrow 1$}
    \begin{proof}
      \pf
      \begin{align*}
        U & \subseteq \overline{U} \\
        & = \Int U \cup \Bd U & (\text{Lemma
          \ref{lm:topology:boundary:interior_disjoint}}) \\
        \therefore U & \subseteq \Int U
      \end{align*}
    \end{proof}
    \qed
  \end{proof}

  \section{Limit Points}

  \begin{df}[Limit Point]
    Let $X$ be a topological space, $A \subseteq X$, and $x \in X$. Then $x$ is
    a
    \emph{limit point}, \emph{cluster point} or \emph{point of accumulation} of
    $A$ iff every neighbourhood of $x$ intersects $A$ in a point other than $x$.
  \end{df}

  \begin{lm}
    \label{lm:topology:limit_point:subset}
    If $A \subseteq B$ then every limit point of $A$ is a limit point of $B$.
  \end{lm}

  \begin{proof}
    \pf\ Immediate from the definition. \qed
  \end{proof}

  \begin{thm}
    Let $A$ be a subset of the topological space $X$. Let $A'$ be the set of
    all
    limit points of $A$. Then $\overline{A} = A \cup A'$.
  \end{thm}

  \begin{proof}
    \pf
    \step{<1>1}{If $x \in \overline{A}$ and $x \notin A$ then $x \in A'$}
    \begin{proof}
      \pf\ in this case, every neighbourhood of $x$ intersects $A$ in a point
      other than $x$.
    \end{proof}
    \step{<1>2}{$A \subseteq \overline{A}$}
    \begin{proof}
      \pf\ From the definition of $\overline{A}$.
    \end{proof}
    \step{<1>3}{$A' \subseteq \overline{A}$}
    \begin{proof}
      \pf\ By Theorem \ref{thm:topology:closure:neighbourhoods}.
    \end{proof}
    \qed
  \end{proof}

  \begin{cor}
    \label{cor:topology:limit_point:closed}
    A set is closed if and only if it contains all its limit points.
  \end{cor}

  \section{Subbases}

  \begin{df}[Subbasis]
    Let $X$ be a topological space. A \emph{subbasis} for the topology on $X$
    is
    a set $\mathcal{S} \subseteq \mathcal{P} X$ such that, for every open set
    $U$
    and $x \in U$, there exist $S_1, \ldots, S_n \in \mathcal{S}$ such that $x
    \in S_1 \cap \cdots \cap S_n \subseteq U$. We say the topology is
    \emph{generated} by $\mathcal{S}$.
  \end{df}

  \begin{lm}
    \label{lm:topology:subbasis:generate}
    Let $\mathcal{T}$ be a topology on $X$ and $\mathcal{S}
    \subseteq \mathcal{P} X$.   Then the following are equivalent:
    \begin{enumerate}
      \item $\mathcal{S}$ is a subbasis for $\mathcal{T}$.
      \item The set of all finite intersections of members of $\mathcal{S}$ is
      a
      basis for $\mathcal{T}$
      \item $\mathcal{T}$ is the set of all unions of finite intersections of
      members of $\mathcal{S}$.
    \end{enumerate}
  \end{lm}

  \begin{proof}
    \pf\ $1 \Leftrightarrow 2$ holds immediately from the definitions. $2
    \Leftrightarrow 3$ holds by Proposition \ref{prop:topology:basis:open}. \qed
  \end{proof}

  \begin{cor}
    \label{cor:topology:subbasis:coarsest}
    If $\mathcal{S}$ is a subbasis for the topology $\mathcal{T}$, then
    $\mathcal{T}$ is the coarsest topology in which every element of
    $\mathcal{S}$ is open.
  \end{cor}

  \begin{lm}
    Let $X$ be a set and $\mathcal{S} \subseteq \mathcal{P} X$. Then
    $\mathcal{S}$ is a subbasis for a topology on $X$ if and only if $\bigcup
    \mathcal{S} = X$.
  \end{lm}

  \begin{proof}
    \pf
    \step{<1>1}{If $\mathcal{S}$ is a subbasis for a topology on $X$ then
      $\bigcup
      \mathcal{S} = X$}
    \begin{proof}
      \step{<2>1}{\assume{$\mathcal{S}$ is a subbasis for a topology
          $\mathcal{T}$
          on $X$.}}
      \step{<2>2}{\pflet{$x \in X$}}
      \step{<2>3}{\pick\ $S_1, \ldots, S_n \in \mathcal{S}$ such that $x \in
        S_1
        \cap \cdots \cap S_n \subseteq X$}
      \begin{proof}
        \pf\ From the definition of subbasis (\stepref{<2>1}, \stepref{<2>2}).
      \end{proof}
      \step{<2>4}{$x \in \bigcup \mathcal{S}$}
      \begin{proof}
        \pf\ Immediate from \stepref{<2>3}.
      \end{proof}
    \end{proof}
    \step{<1>2}{If $\bigcup \mathcal{S} = X$ then $\mathcal{S}$ is a subbasis
      for
      a
      topology on $X$}
    \begin{proof}
      \step{<2>1}{\assume{$\bigcup \mathcal{S} = X$}
        \prove{The set of all finite intersections of elements of $\mathcal{S}$
          is a basis for a topology on $X$.}}
      \step{<2>2}{\pflet{$\mathcal{B}$ be the set of all finite intersections
          of
          elements of $\mathcal{S}$.}}
      \step{<2>3}{$\bigcup \mathcal{B} = X$}
      \begin{proof}
        \pf\ From \stepref{<2>1} and \stepref{<2>2}.
      \end{proof}
      \step{<2>4}{For all $B_1, B_2 \in \mathcal{B}$ and $x \in B_1 \cap B_2$,
        there exists $B_3 \in       \mathcal{B}$ such that $x \in B_3 \subseteq
        B_1  \cap B_2$}
      \begin{proof}
        \pf\ Take $B_3 = B_1 \cap B_2$ (\stepref{<2>2}).
      \end{proof}
      \step{<2>5}{$\mathcal{B}$ is a basis for a topology on $X$.}
      \begin{proof}
        \pf\ By Lemma \ref{lm:topology:basis:generate}.
      \end{proof}
      \qedstep
      \begin{proof}
        \pf\ By Lemma \ref{lm:topology:subbasis:generate}.
      \end{proof}
    \end{proof}
    \qed
  \end{proof}

  \section{Convergence}

      \begin{df}[Net]
    Let $X$ be a topological space. A \emph{net} $(x_\alpha)_{\alpha \in J}$ in
$X$ consists of a directed set $J$ and a function $x : J \rightarrow X$.
  \end{df}

    \begin{df}[Convergence]
    Let $(x_\alpha)_{\alpha \in J}$ be a net in the topological space $X$, and
    $l \in X$. Then the net \emph{converges} to $l$, $x_\alpha \rightarrow l$,
if and only if, for every neighbourhood $U$ of $l$, there exists $\alpha \in J$
such that, for all $\beta \geq \alpha$, we have $x_\beta \in U$.
  \end{df}

    \begin{thm}[AC]
      \label{thm:topology:convergence:closure}
    Let $X$ be a topological space and $A \subseteq X$. Then $x \in
\overline{A}$ if and only if there exists a net of points of $A$ converging to
$x$.
  \end{thm}

  \begin{proof}
   \pf
   \step{<1>1}{If $x \in \overline{A}$ then there exists a net of points of $A$
     converging to $x$.}
   \begin{proof}
     \step{<2>1}{\pflet{$x \in \overline{A}$}}
     \step{<2>2}{\pflet{$J$ be the poset of neighbourhoods of $x$ under
         $\supseteq$.}}
     \step{<2>3}{For $U \in J$ \pick\ a point $x_U \in U \cap A$}
     \begin{proof}
       \pf\ By Theorem \ref{thm:topology:closure:neighbourhoods}
     \end{proof}
     \step{<2>4}{$(x_U)_{U \in J}$ is a net}
     \begin{proof}
       \pf\ Given $U, V \in J$ we have $U \cap V \in J$ and $U \supseteq U \cup
V$, $V \supseteq U \cup V$.
     \end{proof}
     \step{<2>5}{$x_U \rightarrow x$}
     \begin{proof}
       \pf\ For any neighbourhood $U$ of $x$ we have $U \in J$ and if $U
\supseteq V$ then $x_V \in U$.
     \end{proof}
   \end{proof}
   \step{<1>2}{If there exists a net of points of $A$ converging to $x$ then $x
     \in \overline{A}$.}
   \begin{proof}
     \step{<2>1}{\pflet{$(x_\alpha)_{\alpha \in J}$ be a net of points in $A$ that
         converges to          $x$.}}
     \step{<2>2}{\pflet{$U$ be a neighbourhood of $x$}}
     \step{<2>3}{\pick\ $\alpha \in J$ such that, for all $\beta \geq \alpha$, we
       have $x_\beta \in U$}
     \step{<2>4}{$x_\alpha \in U \cap A$}
     \qedstep
     \begin{proof}
       \pf\ By Theorem \ref{thm:topology:closure:neighbourhoods}
     \end{proof}
   \end{proof}
   \qed
  \end{proof}

    \begin{thm}
      \label{thm:topology:convergence:continuous}
   Let $X$ and $Y$ be topological spaces and $f : X \rightarrow Y$. Then $f$ is
   continuous if and only if, for every net $(x_\alpha)_{\alpha \in J}$ in $X$,
if $x_\alpha \rightarrow x$ then $f(x_\alpha) \rightarrow f(x)$.
  \end{thm}

  \begin{proof}
   \pf
   \step{<1>1}{If $f$ is continuous and $x_\alpha \rightarrow x$ then
$f(x_\alpha)
     \rightarrow f(x)$}
   \begin{proof}
     \step{<2>1}{\assume{$f$ is continuous.}}
     \step{<2>2}{\assume{$x_\alpha \rightarrow x$}}
     \step{<2>3}{\pflet{$V$ be a neighbourhood of $f(x)$}}
     \step{<2>4}{$\inv{f}(V)$ is a neighbourhood of $x$}
     \step{<2>5}{\pick\ $\alpha$ such that, for all $\beta \geq \alpha$, we
have
       $x_\beta \in \inv{f}(V)$}
     \step{<2>6}{For all $\beta \geq \alpha$ we have $f(x_\beta) \in V$}
   \end{proof}
   \step{<1>2}{If, for every net $(x_\alpha)$ in $X$, if $x_\alpha \rightarrow
x$
     then $f(x_\alpha) \rightarrow f(x)$, then $f$ is continous.}
   \begin{proof}
     \step{<2>1}{\assume{for every net $(x_\alpha)$ in $X$, if $x_\alpha
         \rightarrow x$ then $f(x_\alpha) \rightarrow f(x)$}}
     \step{<2>2}{\pflet{$A \subseteq X$} \prove{$f(\overline{A}) \subseteq
         \overline{f(A)}$}}
     \step{<2>3}{\pflet{$x \in \overline{A}$}}
     \step{<2>4}{\pick\ a net $(x_\alpha)$ in $A$ such that $x_\alpha
\rightarrow
       x$}
     \begin{proof}
       \pf\ Theorem \ref{thm:topology:convergence:closure}
     \end{proof}
     \step{<2>5}{$f(x_\alpha) \rightarrow f(x)$}
     \begin{proof}
       \pf\ By \stepref{<2>1}
     \end{proof}
     \step{<2>6}{$f(x) \in \overline{f(A)}$}
     \begin{proof}
       \pf\ Theorem \ref{thm:topology:convergence:closure}
     \end{proof}
     \qedstep
     \begin{proof}
       \pf\ By Theorem \ref{thm:topology:continuous:characterisation}.
     \end{proof}
   \end{proof}
   \qed
  \end{proof}

    \begin{df}[Subnet]
    Let $(x_\alpha)_{\alpha \in J}$ be a net in $X$. Let $K$ be a directed set
and $g : K \rightarrow J$ be a monotone function such that $g(K)$ is cofinal in
$J$. Then the net $(x_{g(\beta)})_{\beta \in K}$ is called a \emph{subnet} of
$(x_\alpha)$.
  \end{df}

  \section{Accumulation Points}

    \begin{df}[Accumulation Point]
    Let $X$ be a topological space, and $(x_\alpha)_{\alpha \in J}$ a net in $X$,
and $a \in X$.    Then $a$ is an \emph{accumulation point} of $(x_\alpha)$ iff,
for every    neighbourhood $U$ of $x$, the set $\{ \alpha \in J : x_\alpha \in
U \}$ is cofinal in $J$.
  \end{df}

    \begin{lm}
      \label{lm:topology:accumulation_point:subnet}
    Let $X$ be a topological space, $(x_\alpha)_{\alpha \in J}$ be a
nonempty net in $X$
and $a \in X$. Then $a$ is an accumulation point of $(x_\alpha)$ if and only if
there exists a subnet of $(x_\alpha)$ that converges to $a$.
  \end{lm}

  \begin{proof}
   \pf
   \step{<1>1}{If $a$ is an accumulation point of $(x_\alpha)$ then there exists
a
     subnet of $(x_\alpha)$ that converges to $a$.}
   \begin{proof}
     \step{<2>1}{\assume{$a$ is an accumulation point of $(x_\alpha)$.}}
     \step{<2>2}{\pflet{$K$ be the poset $\{ (\alpha, U) : \alpha \in J, U
\text{
           is a neighbourhood of } a, x_\alpha \in U \}$ under: $(\alpha, U)
         \leq (\beta, V)$ iff $\alpha \leq \beta$ and $U \subseteq V$.}}
     \step{<2>3}{$(x_\alpha)_{(\alpha, U) \in K}$ is a subnet of
       $(x_\alpha)_{\alpha \in J}$}
     \begin{proof}
       \step{<3>1}{$K$ is directed.}
       \begin{proof}
         \step{<4>1}{\pflet{$(\alpha, U), (\beta, V) \in K$}}
         \step{<4>2}{\pick\ $\gamma \in J$ such that $\alpha \leq \gamma$ and
           $\beta \leq \gamma$.}
         \step{<4>3}{\pick\ $\delta \in J$ such that $\gamma \leq \delta$ and
           $x_\delta \in U \cap V$}
         \begin{proof}
           \pf\ By \stepref{<2>1}.
         \end{proof}
         \step{<4>4}{$(\delta, U \cap V) \in K$ and $(\alpha, U) \leq (\delta,
U
           \cap V)$, $(\beta, V) \leq (\delta, U \cap V)$}
       \end{proof}
       \step{<3>2}{If $(\alpha, U) \leq (\beta, V)$ then $\alpha \leq \beta$}
       \begin{proof}
         \pf\ From \stepref{<2>2}.
       \end{proof}
       \step{<3>3}{$\{ \alpha : \exists U. (\alpha, U) \in K \}$ is cofinal in
         $J$}
       \begin{proof}
         \pf\ For $\alpha \in J$ we have $(\alpha, X) \in K$, so in fact $\{
         \alpha : \exists U. (\alpha, U) \in K \} = J$.
       \end{proof}
     \end{proof}
     \step{<2>4}{The subnet converges to $a$.}
     \begin{proof}
       \step{<3>1}{\pflet{$U$ be a neighbourhood of $a$.}}
       \step{<3>2}{\pick\ $\alpha \in J$}
       \step{<3>3}{\pick\ $\beta \in J$ such that $\alpha \leq \beta$ and
$x_\beta
         \in U$}
       \begin{proof}
         \pf\ By \stepref{<2>1}.
       \end{proof}
       \step{<3>4}{For all $(\gamma, V) \geq (\beta, U)$ we have $x_\gamma \in
         U$}
       \begin{proof}
         \pf\ $x_\gamma \in V \subseteq U$.
       \end{proof}
     \end{proof}
   \end{proof}
   \step{<1>2}{If there exists a subnet of $(x_\alpha)$ that converges to $a$
then
     $a$ is an accumulation point of $(x_\alpha)$.}
   \begin{proof}
     \step{<2>1}{\assume{$(x_{g(\beta)})_{\beta \in K}$ converges to $a$}}
     \step{<2>2}{\pflet{$U$ be a neighbourhoof of $a$}}
     \step{<2>3}{\pflet{$\alpha \in J$} \prove{There exists $\gamma \geq
\alpha$
         such that $x_\gamma \in U$}}
     \step{<2>4}{\pick\ $\beta \in K$ such that, for all $\beta' \geq \beta$,
we
       have $x_{g(\beta')} \in U$}
     \begin{proof}
       \pf\ By \stepref{<2>1}.
     \end{proof}
     \step{<2>5}{\pick\ $\beta' \in K$ such that $g(\beta') \geq \alpha$}
     \begin{proof}
       \pf\ Since $g(K)$ is cofinal in $J$.
     \end{proof}
     \step{<2>6}{\pick\ $\beta'' \in K$ such that $\beta \leq \beta''$ and
$\beta'
       \leq \beta''$}
     \begin{proof}
       \pf\ $K$ is directed.
     \end{proof}
     \step{<2>7}{$g(\beta'') \geq \alpha$ and $x_{g(\beta'')} \in U$}
   \end{proof}
   \qed
  \end{proof}

  \section{Dense Sets}

    \begin{df}[Dense]
    Let $X$ be a topological space and $A \subseteq X$. Then $A$ is
    \emph{dense} in $X$ iff $\overline{A} = X$.
  \end{df}

  \section{$G_\delta$ Sets}

    \begin{df}[$G_\delta$ Set]
    A \emph{$G_\delta$ set} is the intersection of a countable set of open sets.
  \end{df}

  \section{Separated Sets}

    \begin{df}[Separated Sets]
   Let $X$ be a topological space and $A, B \subseteq X$. Then $A$ and $B$ are
   \emph{separated} iff $\overline{A} \cap B = \emptyset$ and $A \cap
   \overline{B} = \emptyset$.
  \end{df}

  \section{Coherent Topology}

   \begin{df}[Coherent Topology]
  Let $X_1 \subseteq X_2 \subseteq \cdots$ be a sequence of topological spaces
  such that each $X_n$ is a closed subspace of $X_{n+1}$. Let $X =
  \bigcup_{n=1}^\infty X_n$. Then the topology on $X$ \emph{coherent} with the
  subspaces $X_n$ is the topology defined by: $U \subseteq X$ is open iff $U
  \cap  X_n$ is open in $X_n$ for all $n$.
 \end{df}

  \chapter{Constructions of Topological Spaces}

  \section{The Order Topology}

  \begin{df}[Order Topology]
    Let $X$ be a linearly ordered set with more than one element. The
    \emph{order
      topology} on $X$ is the topology generated by the basis consisting of:
    \begin{itemize}
      \item all open intervals $(a, b)$
      \item all half-open intervals $(a, \top]$ where $\top$ is the greatest
      element of $X$, if there is one;
      \item all half-open intervals $[\bot, a)$ where $\bot$ is the least
      element of
      $X$, if there is one.
    \end{itemize}

    We prove this is a basis for a topology.
  \end{df}

  \begin{proof}
    \pf
    \step{<1>1}{\pflet{$\mathcal{B}$ be the set of all sets of these three
        forms.}}
    \step{<1>2}{$\bigcup \mathcal{B} = X$}
    \begin{proof}
      \step{<2>1}{\pflet{$x \in X$} \prove{There exists $B \in \mathcal{B}$
          such
          that $x \in B$}}
      \step{<2>2}{\case{$x$ is least in $X$}}
      \begin{proof}
        \step{<3>1}{\pick\ $a \in X$ such that $a > x$}
        \begin{proof}
          \pf\ $X$ has more than one element.
        \end{proof}
        \step{<3>2}{$x \in [x, a) \in \mathcal{B}$}
      \end{proof}
      \step{<2>3}{\case{$x$ is greatest in $X$}}
      \begin{proof}
        \step{<3>1}{\pick\ $a \in X$ such that $a < x$}
        \begin{proof}
          \pf\ $X$ has more than one element.
        \end{proof}
        \step{<3>2}{$x \in (a, x] \in \mathcal{B}$}
      \end{proof}
      \step{<2>4}{\case{$x$ is neither least nor greatest in $X$}}
      \begin{proof}
        \step{<3>1}{\pick\ $a, b \in X$ such that $a < x < b$}
        \step{<3>2}{$x \in (a, b) \in \mathcal{B}$}
      \end{proof}
    \end{proof}
    \step{<1>3}{For all $B_1, B_2 \in \mathcal{B}$ and $x \in B_1 \cap B_2$,
      there
      exists $B_3 \in \mathcal{B}$ such that $x \in B_3 \subseteq B_1 \cap B_2$}
    \begin{proof}
      \step{<2>1}{\pflet{$B_1, B_2 \in \mathcal{B}$ and $x \in B_1 \cap B_2$}}
      \step{<2>2}{\case{$B_1 = (a, b), B_2 = (c, d)$}}
      \begin{proof}
        \pf\ Take $B_3 = (\max(a, c), \min(b, d))$.
      \end{proof}
      \step{<2>3}{\case{$B_1 = (a, b), B_2 = (c, \top]$}}
      \begin{proof}
        \pf\ Take $B_3 = (\max(a, c), b)$.
      \end{proof}
      \step{<2>4}{\case{$B_1 = (a, b), B_2 = [\bot, d)$}}
      \begin{proof}
        \pf\ Take $B_3 = (a, \min(b, d))$.
      \end{proof}
      \step{<2>5}{\case{$B_1 = (a, \top], B_2 = (c, d)$}}
      \begin{proof}
        \pf\ Similar to \stepref{<2>3}.
      \end{proof}
      \step{<2>6}{\case{$B_1 = (a, \top], B_2 = (c, \top]$}}
      \begin{proof}
        \pf\ Take $B_3 = (\max(a, c), \top]$.
      \end{proof}
      \step{<2>7}{\case{$B_1 = (a, \top], B_2 = [\bot, d)$}}
      \begin{proof}
        \pf\ Take $B_3 = (a, d)$.
      \end{proof}
      \step{<2>8}{\case{$B_1 = [\bot, b), B_2 = (c, d)$}}
      \begin{proof}
        \pf\ Similar to \stepref{<2>4}.
      \end{proof}
      \step{<2>9}{\case{$B_1 = [\bot, b), B_2 = (c, \top]$}}
      \begin{proof}
        \pf\ Simlar to \stepref{<2>7}.
      \end{proof}
      \step{<2>10}{\case{$B_1 = [\bot, b), B_2 = [\bot, d)$}}
      \begin{proof}
        \pf\ Take $B_3 = [\bot, \min(b, d))$.
      \end{proof}
    \end{proof}
    \qedstep
    \begin{proof}
      \pf\ By Lemma \ref{lm:topology:basis:generate}.
    \end{proof}
    \qed
  \end{proof}

      \begin{lm}
    \label{lm:topology:order:open}
    Let $X$ be a linearly ordered set, $U \subseteq X$ be open, and $a \in U$.
    Then:
    \begin{enumerate}
      \item Either $a$ is greatest in $X$, or there exists $a' > a$ such that
      $[a,
      a') \subseteq U$
      \item Either $a$ is least in $X$, or there exists $a' < a$ such that
      $(a',
      a]
      \subseteq U$.
    \end{enumerate}
  \end{lm}

  \begin{proof}
    \pf
    \step{<1>1}{Either $a$ is greatest in $X$, or there exists $a' > a$ such
      that
      $[a, a') \subseteq U$}
    \begin{proof}
      \step{<2>1}{\assume{$a$ is not greatest in $X$}}
      \step{<2>2}{\pick\ a basic open set $B$ such that $a \in B \subseteq U$}
      \step{<2>3}{\case{$B = (a'', a')$}}
      \begin{proof}
        \pf\ $a < a'$ and $[a, a') \subseteq B \subseteq U$
      \end{proof}
      \step{<2>4}{\case{$B = [\bot, a')$}}
      \begin{proof}
        \pf\ $a < a'$ and $[a, a') \subseteq B \subseteq U$
      \end{proof}
      \step{<2>5}{\case{$B = (a'', \top]$}}
      \begin{proof}
        \pf\ Pick any $a' > a$ (one exists by \stepref{<2>1}). Then $[a, a')
        \subseteq B \subseteq U$.S
      \end{proof}
    \end{proof}
    \step{<1>2}{Either $a$ is least in $X$, or there exists $a' < a$ such that
      $(a',
      a] \subseteq U$.}
    \begin{proof}
      \pf\ Similar.
    \end{proof}
    \qed
  \end{proof}

  \begin{lm}
    \label{lm:topology:order:subbasis}
    The open rays form a subbasis for the order topology.
  \end{lm}

  \begin{proof}
    \step{<1>1}{\pflet{$X$ be a linearly ordered set with more than one
        element.}}
    \step{<1>2}{The open rays form a subbasis for a topology.}
    \begin{proof}
      \step{<2>1}{\pflet{$x \in X$} \prove{$x$ is an element of an open ray.}}
      \step{<2>2}{\case{$x$ is greatest in $X$}}
      \begin{proof}
        \step{<3>1}{\pick\ $a \in X$ such that $a < x$}
        \begin{proof}
          \pf\ $X$ has more than one element (\stepref{<1>1}).
        \end{proof}
        \step{<3>2}{$x \in (a, +\infty)$}
      \end{proof}
      \step{<2>3}{\case{$x$ is not greatest in $X$}}
      \begin{proof}
        \step{<3>1}{\pick\ $a \in X$ such that $x < a$}
        \step{<3>2}{$x \in (-\infty, a)$}
      \end{proof}
      \qedstep
      \begin{proof}
        \pf\ By Lemma \ref{lm:topology:subbasis:generate}.
      \end{proof}
    \end{proof}
    \step{<1>3}{\pflet{$\mathcal{T}_o$ be the order topology and
        $\mathcal{T}_S$
        be
        the topology generated by the open rays.}}
    \step{<1>4}{$\mathcal{T}_o \subseteq \mathcal{T}_S$}
    \begin{proof}
      \step{<2>1}{Every open interval $(a, b)$ is open in $\mathcal{T}_S$}
      \begin{proof}
        \pf\ $(a, b) = (a, + \infty) \cap (-\infty, b)$.
      \end{proof}
      \step{<2>2}{If $\top$ is greatest then $(a, \top]$ is open in
        $\mathcal{T}_S$}
      \begin{proof}
        \pf\ $(a, \top] = (a, + \infty)$.
      \end{proof}
      \step{<2>3}{If $\bot$ is least then $[\bot, b)$ is open in
        $\mathcal{T}_S$}
      \begin{proof}
        \pf\ $[\bot, b) = [\bot, + \infty)$.
      \end{proof}
      \qedstep
      \begin{proof}
        \pf\ By Corollary \ref{cor:topology:basis:coarsest}.
      \end{proof}
    \end{proof}
    \step{<1>5}{$\mathcal{T}_S \subseteq \mathcal{T}_o$}
    \begin{proof}
      \step{<2>1}{For all $a \in X$, we have $(a, +\infty)$ is open in
        $\mathcal{T}_o$}
      \begin{proof}
        \step{<3>1}{\pflet{$x \in (a, +\infty)$} \prove{There exists a basis
            element
            $B$ such that $x \in B \subseteq (a, +\infty)$}}
        \step{<3>2}{\case{$x$ is greatest}}
        \begin{proof}
          \pf\ Take $B = (a, x]$
        \end{proof}
        \step{<3>3}{\case{$x$ is not greatest}}
        \begin{proof}
          \step{<4>1}{\pick\ $b > x$}
          \step{<4>2}{$x \in (a, b) \subseteq (a, +\infty)$}
        \end{proof}
      \end{proof}
      \step{<2>2}{For all $a \in X$, we have $(- \infty, a)$ is open in
        $\mathcal{T}_o$}
      \begin{proof}
        \pf\ Similar.
      \end{proof}
      \qedstep
      \begin{proof}
        \pf\ By Corollary \ref{cor:topology:subbasis:coarsest}.
      \end{proof}
    \end{proof}
    \qed
  \end{proof}

  \begin{lm}
    In a linearly ordered set $X$ under the order topology, the closed
    intervals
    and closed rays are closed.
  \end{lm}

  \begin{proof}
    \pf
    \begin{align*}
      X \setminus [a, b] & = (-\infty, a) \cup (b, +\infty) \\
      X \setminus (-\infty, a] & = (a, +\infty) \\
      X \setminus {[}a, +\infty) & = (-\infty, a) & \qed
    \end{align*}
  \end{proof}

  \begin{df}[Standard Topology on $\mathbb{R}$]
    The \emph{standard topology} on $\mathbb{R}$ is the order topology.
  \end{df}

  \begin{lm}
    The standard topology is strictly coarser than the lower limit topology.
  \end{lm}

  \begin{proof}
    \pf
    \step{<1>1}{The standard topology is coarser than the lower limit topology.}
    \begin{proof}
      \step{<2>1}{For every open interval $(a, b)$ and $x \in (a, b)$, there
        exists
        a half-open interval $[c, d)$ such that $x \in [c,d) \subseteq (a, b)$}
      \begin{proof}
        \pf\ Take $[c,d) = [x, b)$.
      \end{proof}
      \qedstep
      \begin{proof}
        \pf\ By Lemma \ref{lm:topology:basis:finer}.
      \end{proof}
    \end{proof}
    \step{<1>2}{There exists a set $U$ open in the lower limit topology that is
      not
      open in the standard topology.}
    \begin{proof}
      \pf\ Take $U = [0,1)$.
    \end{proof}
    \qed
  \end{proof}

  \begin{lm}
    The standard topology is strictly coarser than the $K$-topology.
  \end{lm}

  \begin{proof}
    \pf
    \step{<1>1}{The standard topology is coarser than the $K$-topology.}
    \begin{proof}
      \pf\ Every open interval is open in the $K$-topology.
    \end{proof}
    \step{<1>2}{There exists a set $U$ open in the $K$-topology that is
      not
      open in the standard topology.}
    \begin{proof}
      \pf\ Take $U = (-1, 1) \setminus K$. Then $0 \in U$ but there is no open
      interval $(a, b)$ such that $0 \in (a, b) \subseteq U$.
    \end{proof}
    \qed
  \end{proof}

  \begin{df}[Ordered Square]
    The \emph{ordered square} $I_o^2$ is the topological space $[0,1]^2$ under
    the order topology induced by the lexicographic order.
  \end{df}

  \begin{lm}
    \label{lm:topology:continuum:closed}
    Let $L$ be a linear continuum with a greatest element. Then every
    non-empty closed set in $L$ has a greatest element.
  \end{lm}

  \begin{proof}
    \pf
    \step{<1>1}{\pflet{$C$ be a non-empty closed set in $L$}}
    \step{<1>2}{\pflet{$u$ be the supremum of $C$}}
    \step{<1>3}{$u \in C$}
    \begin{proof}
      \step{<2>1}{\assume{w.l.o.g~$u$ is not least in $L$}}
      \begin{proof}
        \pf\ If $u$ is least then $C = \{ u \}$.
      \end{proof}
      \step{<2>2}{\pflet{$U$ be any open neighbourhood of $u$}}
      \step{<2>3}{\pick\ $v < u$ such that $(v, u] \subseteq U$}
      \begin{proof}
        \pf\ By Lemma \ref{lm:topology:order:open}.
      \end{proof}
      \step{<2>4}{\pick\ $x \in C$ such that $v < x$}
      \begin{proof}
        \pf\ $v$ is not an upper bound for $C$ (\stepref{<1>2}).
      \end{proof}
      \step{<2>5}{$U$ intersects $C$ in $v$}
      \qedstep
      \begin{proof}
        \pf\ By Theorem \ref{thm:topology:closure:neighbourhoods}.
      \end{proof}
    \end{proof}
    \qed
  \end{proof}

  \begin{df}[Long Line]
    The \emph{long line} is $(S_\Omega \times [0, 1)) \setminus \{(0, 0)\}$
    under the dictionary
    order, where $S_\Omega$ is the first uncountable ordinal under the order
    topology.
  \end{df}

  \section{The Product Topology}

  \begin{df}[Product Topology]
    Let $\{ X_\alpha \}_{\alpha \in J}$ be a family of topological spaces. The
    \emph{product topology} on $\prod_{\alpha \in J} X_\alpha$ is the topology
    generated by the subbasis consisting of all sets of the form
    $\pi_\alpha^{-1}(U)$ where $\alpha \in J$ and $U$ is open in $X_\alpha$.
    The \emph{product space} of $\{ X_\alpha \}_{\alpha \in J}$ is
    $\prod_{\alpha \in J} X_\alpha$ under the product topology.
  \end{df}

  \begin{lm}
    Let $\{ X_\alpha \}_{\alpha \in J}$ be a family of topological spaces and
    $A_\alpha$ be closed in $X_\alpha$ for all $\alpha$. Then $\prod_{\alpha
      \in J} A_\alpha$ is closed in $\prod_{\alpha \in J} X_\alpha$.
  \end{lm}

  \begin{proof}
    \pf\ This holds because $\prod_{\alpha \in J} X_\alpha \setminus
    \prod_{\alpha \in J} A_\alpha = \bigcup_{\alpha \in J}
    \pi_\alpha^{-1}(X_\alpha \setminus A_\alpha)$. \qed
  \end{proof}

  \begin{thm}
    \label{thm:topology:product:basis}
    Let $\{ X_\alpha \}_{\alpha \in J}$ be a family of topological spaces.
    The set of all sets of the form $\prod_{\alpha \in J} U_\alpha$ where each
    $U_\alpha$ is open in $X_\alpha$, and $U_\alpha = X_\alpha$ for all but
    finitely many $\alpha$, is a basis for the product topology on
    $\prod_{\alpha
      \in J} X_\alpha$.
  \end{thm}

  \begin{proof}
    \pf\ By Lemma \ref{lm:topology:subbasis:generate}. \qed
  \end{proof}

  \begin{thm}
    Let $\{X_\alpha\}_{\alpha \in J}$ be a family of topological spaces, and
    let $\mathcal{B}_\alpha$ be a basis for the topology on $X_\alpha$ for each
    $\alpha$. Then
    \begin{align*}
      \mathcal{B} & = \{ \prod_{\alpha \in J} U_\alpha : \text{for finitely
        many
      } \alpha \in J, U_\alpha \in \mathcal{B}_\alpha,  \\
      &  \text{ and } U_\alpha =
      X_\alpha \text{ for all other values of } \alpha \}
    \end{align*}
    is a basis for the product topology on $\prod_{\alpha \in J} X_\alpha$.
  \end{thm}

  \begin{proof}
    \pf
    \step{<1>1}{Every member of $\mathcal{B}$ is open in the product topology.}
    \begin{proof}
      \pf\ Immediate from definitions.
    \end{proof}
    \step{<1>2}{For every open set  $U$ and $\{x_\alpha\}_{\alpha \in J} \in
      U$,
      there exists $B \in \mathcal{B}$ such that $\{ x_\alpha \}_{\alpha \in J}
      \in B \subseteq U$.}
    \begin{proof}
      \step{<2>1}{\pflet{$U$ be open and $\{x_\alpha\}_{\alpha \in J} \in U$}}
      \step{<2>2}{\pick\ $U_\alpha$ open in $X_\alpha$ for each $\alpha$ such
        that
        $\{ x_\alpha \}_{\alpha \in J} \in \prod_{\alpha \in J} U_\alpha
        \subseteq U$ and $U_\alpha = X_\alpha$ for all $\alpha$ except
        $\alpha_1$, \ldots, $\alpha_n$.}
      \begin{proof}
        \pf\ By Theorem \ref{thm:topology:product:basis}.
      \end{proof}
      \step{<2>3}{\pick\ $B_{\alpha_i} \in \mathcal{B}_{\alpha_i}$ such that
        $x_\alpha \in
        B_{\alpha_i} \subseteq U_{\alpha_i}$ for $i=1, \ldots, n$}
      \step{<2>4}{$\{ x_\alpha \}_{\alpha \in J} \in \prod_{\alpha \in J}
        V_\alpha
        \subseteq U$ where $V_{\alpha_i} = B_{\alpha_i}$ for $i = 1, \ldots,
        n$,
        and
        $V_\alpha = X_\alpha$ for all other $\alpha$.}
    \end{proof}
    \qed
  \end{proof}

  \begin{thm}[AC]
    \label{thm:topology:product:closure}
    Let $\{ X_\alpha \}_{\alpha \in J}$ be a family of topological spaces and
    $A_\alpha \subseteq X_\alpha$ for all $\alpha$. If $\prod_{\alpha \in J}
    X_\alpha$ is given the product topology, then
    \[ \prod_{\alpha \in J} \overline{A_\alpha} = \overline{\prod_{\alpha \in
        J}
      A_\alpha} \enspace . \]
  \end{thm}

  \begin{proof}
    \pf
    \step{<1>1}{$\prod_{\alpha \in J} \overline{A_\alpha} \subseteq
      \overline{\prod_{\alpha \in J} A_\alpha}$}
    \begin{proof}
      \step{<2>1}{\pflet{$\{ x_\alpha \}_{\alpha \in J} \in \prod_{\alpha \in
            J}
          \overline{A_\alpha}$}}
      \step{<2>2}{\pflet{$\prod_{\alpha \in J} U_\alpha$ be a basic
          neighbourhood
          of $\{ x_\alpha \}_{\alpha \in J}$, where each $U_\alpha$ is open in
          $X_\alpha$, and $U_\alpha = X_\alpha$ except for $\alpha = \alpha_1,
          \ldots,  \alpha_n$.}}
      \step{<2>3}{For $\alpha \in J$, \pick\ $a_\alpha \in A_\alpha
        \cap         U_\alpha$.}
      \begin{proof}
        \pf\ By Theorem \ref{thm:topology:closure:neighbourhoods}, using the
        Axiom of Choice.
      \end{proof}
      \step{<2>4}{$\{ a_\alpha \}_{\alpha \in J} \in \prod_{\alpha \in J}
        A_\alpha \cap \prod_{\alpha \in J} U_\alpha$}
      \qedstep
      \begin{proof}
        \pf\ By Theorem \ref{thm:topology:closure:neighbourhoods}.
      \end{proof}
    \end{proof}
    \step{<1>2}{$\overline{\prod_{\alpha \in
          J} A_\alpha} \subseteq \prod_{\alpha \in J} \overline{A_\alpha}$}
    \begin{proof}
      \step{<2>1}{\pflet{$\{ x_\alpha \}_{\alpha \in J} \in
          \overline{\prod_{\alpha \in J} A_\alpha}$}}
      \step{<2>2}{\pflet{$\alpha \in J$} \prove{$x_\alpha \in
          \overline{A_\alpha}$}}
      \step{<2>3}{\pflet{$U$ be a neighbourhood of $x_\alpha$ in $X_\alpha$}}
      \step{<2>4}{$\pi_\alpha^{-1}(U)$ is a neighbourhood of $\{ x_\alpha
        \}_{\alpha \in J}$}
      \step{<2>5}{\pick\ $\{ a_\alpha \}_{\alpha \in J} \in \pi_\alpha^{-1}(U)
        \cap \prod_{\alpha \in J} A_\alpha$}
      \begin{proof}
        \pf\ By Theorem \ref{thm:topology:closure:neighbourhoods}.
      \end{proof}
      \step{<2>6}{$a_\alpha \in U \cap A_\alpha$}
      \qedstep
      \begin{proof}
        \pf\ By Theorem \ref{thm:topology:closure:neighbourhoods}.
      \end{proof}
    \end{proof}
    \qed
  \end{proof}

  \begin{df}[Standard Topology on $\mathbb{R}^J$]
    For $J$ a set, the \emph{standard topology} on $\mathbb{R}^J$ is the
    product topology where $\mathbb{R}$ is given the standard topology.
  \end{df}

  \begin{df}[Closed Unit Ball]
    The \emph{closed unit ball} $B^2$ is $\{ (x, y) \in \mathbb{R}^2 : x^2 +
    y^2 \leq 1 \}$ as a subset of $\mathbb{R}^2$.
  \end{df}

   \begin{df}[Sorgenfrey Plane]
   The \emph{Sorgenfrey plane} is $\mathbb{R}_l^2$.
 \end{df}

  \section{The Subspace Topology}

  \begin{df}[Subspace Topology]
    Let $X$ be a topological space and $Y \subseteq X$. The \emph{subspace
      topology} on $Y$ is $\{ Y \cap U : U \text{ open in } X \}$. With this
    topology, $Y$ is a \emph{subspace} of $X$.

    We prove this is a topology.
  \end{df}

  \begin{proof}
    \pf
    \step{<1>1}{\pflet{$\mathcal{T} = \{ Y \cap U : U \text{ open in } X \}$}}
    \step{<1>2}{$Y \in \mathcal{T}$}
    \begin{proof}
      \pf\ $Y = Y \cap X$
    \end{proof}
    \step{<1>3}{$\mathcal{T}$ is closed under union.}
    \begin{proof}
      \step{<2>1}{\pflet{$\mathcal{A} \subseteq \mathcal{T}$} \prove{$\bigcup
          \mathcal{A} = Y \cap \bigcup \{ U \text{ open in } X : Y \cap U \in
          \mathcal{A} \}$}}
      \step{<2>2}{$\bigcup
        \mathcal{A} \subseteq Y \cap \bigcup \{ U \text{ open in } X : Y \cap U
        \in \mathcal{A} \}$}
      \begin{proof}
        \step{<3>1}{\pflet{$x \in \bigcup \mathcal{A}$}}
        \step{<3>2}{\pick\ $V \in \mathcal{A}$ such that $x \in V$}
        \step{<3>3}{\pick\ $U$ open in $X$ such that $V = Y \cap U$}
        \begin{proof}
          \pf\ By the definition of $\mathcal{T}$ (\stepref{<1>1},
          \stepref{<2>1},
          \stepref{<3>2})
        \end{proof}
        \step{<3>4}{$x \in Y \cap \bigcup \{ U \text{ open in } X : Y \cap U
          \in
          \mathcal{A} \}$}
      \end{proof}
      \step{<2>3}{$Y \cap \bigcup \{ U \text{ open in } X : Y \cap U \in
        \mathcal{A} \} \subseteq \bigcup \mathcal{A}$}
      \begin{proof}
        \pf\ Set theory.
      \end{proof}
    \end{proof}
    \step{<1>4}{$\mathcal{T}$ is closed under binary intersection.}
    \begin{proof}
      \pf\ This holds because $(Y \cap U) \cap (Y \cap V) = Y \cap (U \cap V)$.
    \end{proof}
    \qed
  \end{proof}

  \begin{lm}
    Let $X$ be a topological space, $Y \subseteq X$, and $A \subseteq Y$. Then
    the
    topology $A$ inherits as a subspace of $X$ is the same as the topology $A$
    inherits as a subspace of $Y$.
  \end{lm}

  \begin{proof}
    \pf
    \begin{align*}
      & \text{topology as a subspace of } Y \\
      = & \{ V \cap A : V \text{ open in } Y \} \\
      = & \{ V \cap A : \exists U \text{ open in } X. V = U \cap Y \} \\
      = & \{ U \cap Y \cap A : U \text{ open in } X \} \\
      = & \{ U \cap A : U \text{ open in } X \} \\
      = & \text{topology as a subspace of } X \qed
    \end{align*}
  \end{proof}

  \begin{lm}
    \label{lm:topology:subspace:open}
    Let $Y$ be a subspace of $X$. If $U$ is open in $Y$ and $Y$ is open in $X$
    then $U$ is open in $X$.
  \end{lm}

  \begin{proof}
    \pf
    \step{<1>1}{\pick\ $V$ open in $X$ such that $U = Y \cap V$}
    \step{<1>2}{$U$ is open in $X$}
    \begin{proof}
      \pf\ The open sets in $X$ are closed under binary intersection.
    \end{proof}
    \qed
  \end{proof}

  \begin{thm}
    \label{thm:topology:subspace:closure}
    Let $Y$ be a subspace of $X$. Let $A \subseteq Y$. Let $\overline{A}$ be
    the
    closure of $A$ in $X$. Then the closure of $A$ in $Y$ is $\overline{A} \cap
    Y$.
  \end{thm}

  \begin{proof}
    \pf
    \step{<1>1}{$\overline{A} \cap Y$ is a closed set in $Y$ that includes $A$.}
    \begin{proof}
      \step{<2>1}{$\overline{A} \cap Y$ is closed in $Y$.}
      \begin{proof}
        \pf\ By Lemma \ref{cor:topology:subspace:closed}.
      \end{proof}
      \step{<2>2}{$A \subseteq \overline{A} \cap Y$.}
    \end{proof}
    \step{<1>2}{If $C$ is any closed set in $Y$ that includes $A$ then
      $\overline{A}
      \cap Y \subseteq C$.}
    \begin{proof}
      \step{<2>1}{\pflet{$C$ be a closed set in $Y$ that includes $A$.}}
      \step{<2>2}{\pick\ $D$ closed in $X$ such that $C = D \cap Y$.}
      \begin{proof}
        \pf\ By Lemma \ref{cor:topology:subspace:closed}.
      \end{proof}
      \step{<2>3}{$\overline{A} \subseteq D$}
      \step{<2>4}{$\overline{A} \subseteq C$}
    \end{proof}
    \qed
  \end{proof}

  \begin{cor}
    \label{cor:topology:subspace:closed}
    Let $Y$ be a subspace of $X$. Then a set $A \subseteq Y$ is closed in $Y$
    if
    and only if it is the intersection of a closed set in $X$ with $Y$.
  \end{cor}

  \begin{cor}
    \label{cor:topology:subspace:closed2}
    Let $Y$ be a subspace of $X$. If $A$ is closed in $Y$ and $Y$ is closed in
    $X$
    then $A$ is closed in $X$.
  \end{cor}

  \begin{lm}
    \label{lm:topology:subspace:basis}
    Let $X$ be a topological space and $Y \subseteq X$. If $\mathcal{B}$ is a
    basis for the topology on $X$ then $\{ B \cap Y : B \in \mathcal{B} \}$ is
    a
    basis for the subspace topology on $Y$.
  \end{lm}

  \begin{proof}
    \pf
    \step{<1>1}{For all $B \in \mathcal{B}$, we have $B \cap Y$ is open in $Y$.}
    \begin{proof}
      \pf\ Immediate from definitions.
    \end{proof}
    \step{<1>2}{For every $V$ open in $Y$ and $y \in V$, there exists $B \in
      \mathcal{B}$ such that $y \in B \cap Y \subseteq V$.}
    \begin{proof}
      \step{<2>1}{\pflet{$V$ be open in $Y$ and $y \in V$}}
      \step{<2>2}{\pick\ $U$ open in $X$ such that $V = Y \cap U$}
      \step{<2>3}{\pick\ $B \in \mathcal{B}$ such that $y \in B \subseteq U$}
      \step{<2>4}{$y \in B \cap Y \subseteq V$}
    \end{proof}
    \qed
  \end{proof}

  \begin{lm}
    \label{lm:topology:subspace:subbasis}
    Let $X$ be a topological space and $Y \subseteq X$. If $\mathcal{S}$ is a
    subbasis for the topology on $X$ then $\{ S \cap Y : S \in \mathcal{S} \}$
    is
    a subbasis for the subspace topology on $Y$.
  \end{lm}

  \begin{proof}
    \pf
    \step{<1>1}{For all $S \in \mathcal{S}$, we have $S \cap Y$ is open in $Y$.}
    \begin{proof}
      \pf\ Immediate from definitions.
    \end{proof}
    \step{<1>2}{For every $V$ open in $Y$ and $y \in V$, there exist $S_1,
      \ldots,
      S_n \in \mathcal{S}$ such that $y \in (S_1 \cap Y) \cap \cdots \cap (S_n
      \cap Y) \subseteq V$}
    \begin{proof}
      \step{<2>1}{\pflet{$V$ be open in $Y$ and $y \in V$}}
      \step{<2>2}{\pick\ $U$ open in $X$ such that $V = U \cap Y$}
      \step{<2>3}{\pick\ $S_1, \ldots, S_n \in \mathcal{S}$ such that $y \in
        S_1
        \cap \cdots \cap S_n \subseteq U$}
      \step{<2>4}{$y \in (S_1 \cap Y) \cap \cdots \cap (S_n \cap Y) \subseteq
        V$}
    \end{proof}
    \qed
  \end{proof}

  \begin{thm}
    Let $X$ be a linearly ordered set in the order topology. Let $Y \subseteq
    X$
    be convex. Then the order topology on $Y$ is the same as the subspace
    topology.
  \end{thm}

  \begin{proof}
    \pf
    \step{<1>1}{\pflet{$\mathcal{T}_o$ be the order topology and
        $\mathcal{T}_s$
        be
        the subspace topology.}}
    \step{<1>2}{$\mathcal{T}_o \subseteq \mathcal{T}_s$}
    \begin{proof}
      \step{<2>1}{For all $a \in Y$, we have $\{ y \in Y : a < y \} \in
        \mathcal{T}_s$}
      \begin{proof}
        \pf\ $\{ y \in Y : a < y \} = \{ x \in X : a < x \} \cap Y$
      \end{proof}
      \step{<2>2}{For all $a \in Y$, we have $\{ y \in Y : y < a \} \in
        \mathcal{T}_s$}
      \begin{proof}
        \pf\ Similar.
      \end{proof}
      \qedstep
      \begin{proof}
        \pf\ Lemma \ref{lm:topology:order:subbasis} and Corollary
        \ref{cor:topology:subbasis:coarsest}.
      \end{proof}
    \end{proof}
    \step{<1>3}{$\mathcal{T}_s \subseteq \mathcal{T}_o$}
    \begin{proof}
      \step{<2>1}{The sets $(a, +\infty) \cap Y$ and $(-\infty, a) \cap Y$ for
        $a
        \in X$ form a subbasis for $\mathcal{T}_s$}
      \begin{proof}
        \pf\ Lemma \ref{lm:topology:subspace:subbasis}, Lemma
        \ref{lm:topology:order:subbasis}.
      \end{proof}
      \step{<2>2}{For all $a \in X$, we have $(a, +\infty) \cap Y \in
        \mathcal{T}_o$}
      \begin{proof}
        \step{<3>1}{\pflet{$a \in X$}}
        \step{<3>2}{\case{$a \in Y$}}
        \begin{proof}
          \pf\ In this case, $(a, +\infty) \cap Y$ is an open ray in $Y$.
        \end{proof}
        \step{<3>3}{\case{For all $y \in Y$ we have $a < y$}}
        \begin{proof}
          \pf\ In this case, $(a, +\infty) \cap Y = Y$.
        \end{proof}
        \step{<3>4}{\case{For all $y \in Y$ we have $y < a$}}
        \begin{proof}
          \pf\ In this case, $(a, +\infty) \cap Y = \emptyset$.
        \end{proof}
        \qedstep
        \begin{proof}
          \pf\ These are the only cases because $Y$ is convex.
        \end{proof}
      \end{proof}
      \step{<2>3}{For all $a \in X$, we have $(-\infty, a) \cap Y \in
        \mathcal{T}_o$}
      \begin{proof}
        \pf\ Similar.
      \end{proof}
      \qedstep
      \begin{proof}
        \pf\ Corollary \ref{cor:topology:subbasis:coarsest}.
      \end{proof}
    \end{proof}
    \qed
  \end{proof}

  \begin{thm}
    Let $\{X_\alpha\}_{\alpha \in J}$ be a family of topological spaces, and
    let $A_\alpha$ be a subspace of $X_\alpha$ for all $\alpha$. Then the
    product
    topology on $\prod_{\alpha \in J} A_\alpha$ is the same as the topology it
    inherits as a subspace of $\prod_{\alpha \in J} X_\alpha$.
  \end{thm}

  \begin{proof}
    \pf\ Each is the topology generated by the subbasis consisting of
    $\pi_\alpha^{-1}(U) \cap \prod_{\alpha \in J} A_\alpha = \pi_\alpha^{-1}(U
    \cap A_\alpha)$ where $\alpha \in J$ and $U$ is open in $X_\alpha$, using
    Lemma
    \ref{lm:topology:subspace:subbasis}. \qed
  \end{proof}

  \begin{df}[Unit Circle]
    The \emph{unit circle} $S^1$ is $\{ (x,y) \in \mathbb{R}^2 : x^2 + y^2 = 1
    \}$ as a subspace of $\mathbb{R}^2$.
  \end{df}

  \begin{prop}
    \label{prop:topology:subspace:limit_point}
    Let $Y$ be a subspace of $X$, $A \subseteq Y$, and $a \in Y$. Then $a$ is a
    limit point of $A$ in the subspace topology on $Y$ if and only if $a$ is a
    limit point of $A$ is the topology of $X$.
  \end{prop}

  \begin{proof}
    \pf
    \begin{align*}
      & a \text{ is a limit point of } A \text{ in } Y \\
      \Leftrightarrow & \forall U \text{ open in } Y( a \in U \Rightarrow U
      \text{ intersects } A \text{ outside } a) \\
      \Leftrightarrow & \forall V \text{ open in } X( a \in V \cap Y
      \Rightarrow
      V \cap Y \text{ intersects } A \text{ outside } a) \\
      \Leftrightarrow & \forall V \text{ open in } X( a \in V \Rightarrow V
      \text{ intersects } A \text{ outside } a) \\
      & \qquad (a \in Y, A \subseteq Y) \\
      \Leftrightarrow & a \text{ is a limit point of } A \text{ in } X & \qed
    \end{align*}
  \end{proof}

  \section{The Box Topology}

  \begin{df}[Box Topology]
    Let $\{ X_\alpha \}_{\alpha \in J}$ be a family of topological spaces. The
    \emph{box topology} on $\prod_{\alpha \in J} X_\alpha$ is the topology
    generated by the basis consisting of all sets of the form $\prod_{\alpha
      \in J} U_\alpha$, where each $U_\alpha$ is open in $X_\alpha$.

    We prove this is a basis.
  \end{df}

  \begin{proof}
    \pf
    \step{<1>1}{\pflet{$\mathcal{B}$ be the set of all sets of the form
        $\prod_{\alpha \in J} U_\alpha$, where each $U_\alpha$ is open in
        $X_\alpha$.}}
    \step{<1>2}{$\bigcup \mathcal{B} = \prod_{\alpha \in J} X_\alpha$}
    \begin{proof}
      \pf\ This holds because $\prod_{\alpha \in J} X_\alpha \in \mathcal{B}$.
    \end{proof}
    \step{<1>3}{$\mathcal{B}$ is closed under binary intersection.}
    \begin{proof}
      \pf\ $\prod_{\alpha \in J} U_\alpha \cap \prod_{\alpha \in J} V_\alpha =
      \prod_{\alpha \in J} (U_\alpha \cap V_\alpha)$.
    \end{proof}
    \qedstep
    \begin{proof}
      \pf\ Corollary \ref{cor:topology:basis:generate}.
    \end{proof}
  \end{proof}

  \begin{thm}[AC]
    Let $\{X_\alpha\}_{\alpha \in J}$ be a family of topological spaces, and
    let $\mathcal{B}_\alpha$ be a basis for the topology on $X_\alpha$ for each
    $\alpha$. Then
    \[ \mathcal{B} = \{ \prod_{\alpha \in J} B_\alpha : \forall \alpha \in J.
    B_\alpha \in \mathcal{B}_\alpha \} \]
    is a basis for the box topology on $\prod_{\alpha \in J} X_\alpha$.
  \end{thm}

  \begin{proof}
    \pf
    \step{<1>1}{Every member of $\mathcal{B}$ is open in the box topology.}
    \begin{proof}
      \pf\ Immediate from definitions.
    \end{proof}
    \step{<1>2}{For every open set $U$ and $\{x_\alpha\}_{\alpha \in J} \in U$,
      there exists $B \in \mathcal{B}$ such that $\{ x_\alpha \}_{\alpha \in J}
      \in B \subseteq U$.}
    \begin{proof}
      \step{<2>1}{\pflet{$U$ be open and $\{x_\alpha\}_{\alpha \in J} \in U$}}
      \step{<2>2}{\pick\ $U_\alpha$ open in $X_\alpha$ for each $\alpha$ such
        that
        $\{ x_\alpha \}_{\alpha \in J} \in \prod_{\alpha \in J} U_\alpha
        \subseteq U$.}
      \step{<2>3}{\pick\ $B_\alpha \in \mathcal{B}_\alpha$ such that $x_\alpha
        \in
        B_\alpha \subseteq U_\alpha$ for each $\alpha$}
      \begin{proof}
        \pf\ Using the Axiom of Choice.
      \end{proof}
      \step{<2>4}{$\{ x_\alpha \}_{\alpha \in J} \in \prod_{\alpha \in J}
        B_\alpha
        \subseteq U$}
    \end{proof}
    \qed
  \end{proof}

  \begin{thm}
    Let $\{X_\alpha\}_{\alpha \in J}$ be a family of topological spaces, and
    let $A_\alpha$ be a subspace of $X_\alpha$ for all $\alpha$. Let
    $\prod_{\alpha \in J} X_\alpha$ be given the box topology. Then the box
    topology on $\prod_{\alpha \in J} A_\alpha$ is the same as the topology it
    inherits as a subspace of $\prod_{\alpha \in J} X_\alpha$.
  \end{thm}

  \begin{proof}
    \pf\ Each is the topology generated by the basis \\$\{ \prod_{\alpha \in J}
    (U_\alpha \cap A_\alpha) : U_\alpha \text{ is open in } X_\alpha \}$, using
    Lemma \ref{lm:topology:subspace:basis}. \qed
  \end{proof}

  \begin{thm}
    Let $\{ X_\alpha \}_{\alpha \in J}$ be a family of Hausdorff spaces. Then
    $\prod_{\alpha \in J} X_\alpha$ is Hausdorff under the box topology.
  \end{thm}

  \begin{proof}
    \pf
    \step{<1>1}{\pflet{$\{x_\alpha\}_{\alpha \in J}, \{y_\alpha\}_{\alpha \in
          J}
        \in \prod_{\alpha \in J} X_\alpha$ with $\{x_\alpha\}_{\alpha \in J}
        \neq \{y_\alpha\}_{\alpha \in J}$}}
    \step{<1>2}{\pick\ $\alpha \in J$ such that $x_\alpha \neq y_\alpha$}
    \step{<1>3}{\pick\ disjoint neighbourhoods $U$ of $x_\alpha$ and $V$ of
      $y_\alpha$.}
    \step{<1>4}{$\pi_\alpha^{-1}(U)$ and $\pi_\alpha^{-1}(V)$ are disjoint
      neighbourhoods of $\{x_\alpha\}_{\alpha \in J}$ and $\{y_\alpha\}_{\alpha
        \in
        J}$}
    \qed
  \end{proof}

  \begin{cor}
  	The space $\mathbb{R}^\omega$ under the box topology is Hausdorff.
  \end{cor}

  \begin{thm}[AC]
    Let $\{ X_\alpha \}_{\alpha \in J}$ be a family of topological spaces and
    $A_\alpha \subseteq X_\alpha$ for all $\alpha$. If $\prod_{\alpha \in J}
    X_\alpha$ is given the box topology, then
    \[ \prod_{\alpha \in J} \overline{A_\alpha} = \overline{\prod_{\alpha \in
        J}
      A_\alpha} \enspace . \]
  \end{thm}

  \begin{proof}
    \pf
    \step{<1>1}{$\prod_{\alpha \in J} \overline{A_\alpha} \subseteq
      \overline{\prod_{\alpha \in J} A_\alpha}$}
    \begin{proof}
      \step{<2>1}{\pflet{$\{ x_\alpha \}_{\alpha \in J} \in \prod_{\alpha \in
            J}
          \overline{A_\alpha}$}}
      \step{<2>2}{\pflet{$\prod_{\alpha \in J} U_\alpha$ be a basic
          neighbourhood
          of $\{ x_\alpha \}_{\alpha \in J}$, where each $U_\alpha$ is open in
          $X_\alpha$.}}
      \step{<2>3}{For $\alpha \in J$, \pick\ $a_\alpha \in A_\alpha
        \cap         U_\alpha$.}
      \begin{proof}
        \pf\ By Theorem \ref{thm:topology:closure:neighbourhoods}, using the
        Axiom of Choice.
      \end{proof}
      \step{<2>4}{$\{ a_\alpha \}_{\alpha \in J} \in \prod_{\alpha \in J}
        A_\alpha \cap \prod_{\alpha \in J} U_\alpha$}
      \qedstep
      \begin{proof}
        \pf\ By Theorem \ref{thm:topology:closure:neighbourhoods}.
      \end{proof}
    \end{proof}
    \step{<1>2}{$\overline{\prod_{\alpha \in
          J} A_\alpha} \subseteq \prod_{\alpha \in J} \overline{A_\alpha}$}
    \begin{proof}
      \step{<2>1}{\pflet{$\{ x_\alpha \}_{\alpha \in J} \in
          \overline{\prod_{\alpha \in J} A_\alpha}$}}
      \step{<2>2}{\pflet{$\alpha \in J$} \prove{$x_\alpha \in
          \overline{A_\alpha}$}}
      \step{<2>3}{\pflet{$U$ be a neighbourhood of $x_\alpha$ in $X_\alpha$}}
      \step{<2>4}{$\pi_\alpha^{-1}(U)$ is a neighbourhood of $\{ x_\alpha
        \}_{\alpha \in J}$}
      \step{<2>5}{\pick\ $\{ a_\alpha \}_{\alpha \in J} \in \pi_\alpha^{-1}(U)
        \cap \prod_{\alpha \in J} A_\alpha$}
      \begin{proof}
        \pf\ By Theorem \ref{thm:topology:closure:neighbourhoods}.
      \end{proof}
      \step{<2>6}{$a_\alpha \in U \cap A_\alpha$}
      \qedstep
      \begin{proof}
        \pf\ By Theorem \ref{thm:topology:closure:neighbourhoods}.
      \end{proof}
    \end{proof}
    \qed
  \end{proof}

  \section{The Quotient Topology}

  \begin{df}[Quotient Map]
    Let $X$ and $Y$ be topological spaces. Let $p : X \twoheadrightarrow Y$ be
    a
    surjective map. Then $p$ is a \emph{quotient map} iff, for all $U \subseteq
    Y$, we have $U$ is open in $Y$ iff $p^{-1}(U)$ is open in $X$.
  \end{df}

  \begin{lm}
    \label{lm:topology:quotient:saturated}
    Let $X$ and $Y$ be topological spaces and $p : X \twoheadrightarrow Y$ be
    surjective and continuous. Then the following are equivalent.
    \begin{enumerate}
      \item $p$ is a quotient map.
      \item $p$ maps saturated open sets to open sets.
      \item $p$ maps saturated closed sets to closed sets.
    \end{enumerate}
  \end{lm}

  \begin{proof}
    \pf
    \step{<1>1}{$1 \Rightarrow 2$}
    \begin{proof}
      \step{<2>1}{\assume{$p$ is a quotient map.}}
      \step{<2>2}{\pflet{$U \subseteq X$ be a saturated open set.}}
      \step{<2>3}{$U = p^{-1}(p(U))$}
      \begin{proof}
        \step{<3>1}{$U \subseteq p^{-1}(p(U))$}
        \begin{proof}
          \pf\ Set theory.
        \end{proof}
        \step{<3>2}{$p^{-1}(p(U)) \subseteq U$}
        \begin{proof}
          \step{<4>1}{\pflet{$x \in p^{-1}(p(U))$}}
          \step{<4>2}{\pick\ $y \in U$ such that $p(x) = p(y)$}
          \step{<4>3}{$x \in U$}
          \begin{proof}
            \pf\ \stepref{<2>2}, \stepref{<4>2}.
          \end{proof}
        \end{proof}
      \end{proof}
      \step{<2>4}{$p(U)$ is open}
      \begin{proof}
        \pf\ \stepref{<2>1}, \stepref{<2>3}.
      \end{proof}
    \end{proof}
    \step{<1>2}{$2 \Rightarrow 3$}
    \begin{proof}
      \step{<2>1}{\assume{$p$ maps saturated open sets to open sets}}
      \step{<2>2}{\pflet{$C \subseteq X$ be a saturated closed set.}}
      \step{<2>3}{$X  \setminus C$ is a saturated open set.}
      \begin{proof}
        \step{<3>1}{\pflet{$x \in X \setminus C$ and $x' \in X$ be such that
            $p(x)
            = p(x')$}}
        \step{<3>2}{$x' \notin C$}
        \begin{proof}
          \pf\ If $x' \in C$ then $x \in C$ since $C$ is saturated.
        \end{proof}
      \end{proof}
      \step{<2>4}{$p(X \setminus C)$ is open.}
      \begin{proof}
        \pf\ By \stepref{<2>1} and \stepref{<2>3}.
      \end{proof}
      \step{<2>5}{$p(X \setminus C) = Y \setminus p(C)$}
      \begin{proof}
        \step{<3>1}{$p(X \setminus C) \subseteq Y \setminus p(C)$}
        \begin{proof}
          \step{<4>1}{\pflet{$x \in X \setminus C$}}
          \step{<4>2}{\assume{for a contradiction $p(x) \in p(C)$}}
          \step{<4>3}{\pick\ $x' \in C$ such that $p(x) = p(x')$}
          \qedstep
          \begin{proof}
            \pf\ We have $x \notin C$, $x' \in C$ and $p(x) = p(x')$,
            contradicting \stepref{<2>2}.
          \end{proof}
        \end{proof}
        \step{<3>2}{$Y \setminus p(C) \subseteq p(X \setminus C)$}
        \begin{proof}
          \step{<4>1}{\pflet{$y \notin p(C)$}}
          \step{<4>2}{\pick\ $x \in X$ such that $p(x) = y$}
          \begin{proof}
            \pf\ $p$ is surjective.
          \end{proof}
          \step{<4>3}{$x \notin C$}
        \end{proof}
      \end{proof}
    \end{proof}
    \step{<1>3}{$3 \Rightarrow 1$}
    \begin{proof}
      \step{<2>1}{\assume{$p$ maps saturated closed sets to closed sets}}
      \step{<2>2}{\pflet{$C \subseteq Y$ be such that $p^{-1}(Y)$ is closed}}
      \step{<2>3}{$p^{-1}(C)$ is saturated} % TODO Extract lemma
      \begin{proof}
        \step{<3>1}{\pflet{$x \in p^{-1}(C)$, $x' \in X$ and $p(x) = p(x')$}}
        \step{<3>2}{$x' \in p^{-1}(C)$}
      \end{proof}
      \step{<2>4}{$p(p^{-1}(C))$ is closed}
      \begin{proof}
        \pf\ By \stepref{<2>1} and \stepref{<2>3}.
      \end{proof}
      \step{<2>5}{$C = p(p^{-1}(C))$}
      \begin{proof}
        \pf\ By set theory, since $p$ is surjective.
      \end{proof}
    \end{proof}
    \qed
  \end{proof}

  \begin{cor}
    If $p : X \twoheadrightarrow Y$ is a surjective continuous map that is
    either an open map or a closed map, then $p$ is a quotient map.
  \end{cor}

  \begin{df}[Quotient Topology]
    Let $X$ be a topological space, $A$ a set, and $p : X \twoheadrightarrow A$
    a surjective map. Then the \emph{quotient topology} on $A$ induced by $p$ is
    \[ \{ U \subseteq A : p^{-1}(U) \text{ is open in } X \} \enspace . \]

    It is easy to check this is a topology.
  \end{df}

  \begin{lm}
    Let $X$ be a topological space, $A$ a set, and $p : X \twoheadrightarrow A$
    a surjective map. Then the quotient topology induced by $p$ is the
    unique topology on $A$ such that $p$ is a quotient map.
  \end{lm}

  \begin{proof}
    \pf\ Immediate from definitions. \qed
  \end{proof}

  \begin{df}[Quotient Space]
    Let $X$ be a topological space and $X^*$ a partition of $X$. Let $p : X
    \twoheadrightarrow X^*$ be the canonical map. Then $X^*$ under the quotient
    topology induced by $p$ is called a \emph{quotient space} of $X$.
  \end{df}

  \begin{prop}
    Let $p : X \twoheadrightarrow Y$ be a quotient map. Let $A \subseteq X$ be
    open and saturated. Then $p \restriction_A : A \twoheadrightarrow p(A)$ is
    a
    quotient map.
  \end{prop}

  \begin{proof}
    \pf
    \step{<1>1}{\pflet{$q = p \restriction_A : A \twoheadrightarrow p(A)$}}
    \step{<1>2}{For all $V \subseteq p(A)$, we have $q^{-1}(V) = p^{-1}(V)$}
    \begin{proof}
      \step{<2>1}{$q^{-1}(V) \subseteq p^{-1}(V)$}
      \begin{proof}
        \pf\ Trivial.
      \end{proof}
      \step{<2>2}{$p^{-1}(V) \subseteq q^{-1}(V)$}
      \begin{proof}
        \step{<3>1}{\pflet{$x \in p^{-1}(V)$}}
        \step{<3>2}{\pick\ $x' \in A$ such that $p(x') = p(x)$}
        \begin{proof}
          \pf\ One exists because $p(x) \in V \subseteq p(A)$.
        \end{proof}
        \step{<3>3}{$x \in A$}
        \begin{proof}
          \pf\ This holds because $A$ is saturated.
        \end{proof}
        \step{<3>4}{$x \in q^{-1}(V)$}
        \begin{proof}
          \pf\ From \stepref{<3>1} and \stepref{<3>3}.
        \end{proof}
      \end{proof}
    \end{proof}
    \step{<1>3}{For all $U \subseteq X$, we have $p(U \cap A) = p(U) \cap p(A)$}
    \step{<1>4}{\pflet{$V \subseteq p(A)$ be such that $q^{-1}(V)$ is open in
        $A$.} \prove{$V$ is open in $p(A)$.}}
    \step{<1>5}{$q^{-1}(V)$ is open in $X$}
    \step{<1>6}{$p^{-1}(V)$ is open in $X$}
    \step{<1>7}{$V$ is open in $Y$}
    \step{<1>8}{$V$ is open in $p(A)$}
    \qed
  \end{proof}

  \begin{prop}
    Let $p : X \twoheadrightarrow Y$ be a quotient map. Let $A \subseteq X$ be
    closed and saturated. Then $p \restriction_A : A \twoheadrightarrow p(A)$
    is
    a
    quotient map.
  \end{prop}

  \begin{proof}
    \pf\ Similar. \qed
  \end{proof}

  \begin{prop}
    Let $p : X \twoheadrightarrow Y$ be an open quotient map. Let $A \subseteq
    X$ be
    saturated. Then $p \restriction_A : A \twoheadrightarrow p(A)$ is a
    quotient map.
  \end{prop}

  \begin{proof}
    \pf
    \step{<1>1}{\pflet{$q = p \restriction_A : A \twoheadrightarrow p(A)$}}
    \step{<1>2}{For all $V \subseteq p(A)$, we have $q^{-1}(V) = p^{-1}(V)$}
    \begin{proof}
      \step{<2>1}{$q^{-1}(V) \subseteq p^{-1}(V)$}
      \begin{proof}
        \pf\ Trivial.
      \end{proof}
      \step{<2>2}{$p^{-1}(V) \subseteq q^{-1}(V)$}
      \begin{proof}
        \step{<3>1}{\pflet{$x \in p^{-1}(V)$}}
        \step{<3>2}{\pick\ $x' \in A$ such that $p(x') = p(x)$}
        \begin{proof}
          \pf\ One exists because $p(x) \in V \subseteq p(A)$.
        \end{proof}
        \step{<3>3}{$x \in A$}
        \begin{proof}
          \pf\ This holds because $A$ is saturated.
        \end{proof}
        \step{<3>4}{$x \in q^{-1}(V)$}
        \begin{proof}
          \pf\ From \stepref{<3>1} and \stepref{<3>3}.
        \end{proof}
      \end{proof}
    \end{proof}
    \step{<1>3}{For all $U \subseteq X$, we have $p(U \cap A) = p(U) \cap p(A)$}
    \begin{proof}
      \step{<2>1}{$p(U \cap A) \subseteq p(U) \cap p(A)$}
      \begin{proof}
        \pf\ Set theory.
      \end{proof}
      \step{<2>2}{$p(U) \cap p(A) \subseteq p(U \cap A)$}
      \begin{proof}
        \step{<3>1}{\pflet{$x \in U$, $y \in A$, $p(x) = p(y)$} \prove{$p(x)
            \in
            p(U \cap A)$}}
        \step{<3>2}{$x \in A$}
        \begin{proof}
          \pf\ $A$ is saturated.
        \end{proof}
        \step{<3>3}{$x \in U \cap A$}
      \end{proof}
    \end{proof}
    \step{<1>4}{\pflet{$V \subseteq p(A)$ be such that $q^{-1}(V)$ is open in
        $A$.} \prove{$V$ is open in $p(A)$.}}
    \step{<1>5}{$p^{-1}(V)$ is open in $A$}
    \begin{proof}
      \pf\ By \stepref{<1>2}
    \end{proof}
    \step{<1>6}{\pick\ $U$ open in $X$ such that $p^{-1}(V) = U \cap A$}
    \step{<1>7}{$V = p(U) \cap p(A)$}
    \begin{proof}
      \pf
      \begin{align*}
        V & = p(p^{-1}(V)) & (p \text{ is surjective}) \\
        & = p(U \cap A) & (\text{\stepref{<1>6}}) \\
        & = p(U) \cap p(A) & (\text{\stepref{<1>3}})
      \end{align*}
    \end{proof}
    \step{<1>8}{$p(U)$ is open in $Y$}
    \begin{proof}
      \pf\ \stepref{<1>6}, $p$ is an open map.
    \end{proof}
    \step{<1>9}{$V$ is open in $p(A)$}
    \begin{proof}
      \pf\ \stepref{<1>7}, \stepref{<1>8}
    \end{proof}
    \qed
  \end{proof}

  \begin{prop}
    Let $p : X \twoheadrightarrow Y$ be a closed quotient map. Let $A \subseteq
    X$ be
    saturated. Then $p \restriction_A : A \twoheadrightarrow p(A)$ is a
    quotient map.
  \end{prop}

  \begin{proof}
    \pf\ Similar. \qed
  \end{proof}

  \begin{prop}
    \label{prop:topology:quotient:composite}
    The composite of two quotient maps is a quotient map.
  \end{prop}

  \begin{proof}
    \pf\ From Proposition \ref{prop:topology:strongly_continuous:composite}.
    \qed
  \end{proof}

  \begin{prop}
    Let $X^*$ be a quotient space of $X$. If every element of $X^*$ is closed
    in
    $X$, then $X^*$ is $T_1$.
  \end{prop}

  \begin{proof}
    \pf
    \step{<1>1}{\pflet{$C \in X^*$}}
    \step{<1>2}{$p^{-1}(\{C\}) = C$}
    \begin{proof}
      \pf\ Definition of $p$.
    \end{proof}
    \step{<1>3}{$p^{-1}(\{C\})$ is closed in $X$}
    \begin{proof}
      \pf\ By hypothesis.
    \end{proof}
    \step{<1>4}{$\{C\}$ is closed in $X^*$.}
    \begin{proof}
      \pf\ By Proposition \ref{prop:topology:strongly_continuous:closed}.
    \end{proof}
    \qed
  \end{proof}

  \chapter{Functions Between Topological Spaces}

  \section{Open Maps}

  \begin{df}
    Let $X$ and $Y$ be topological spaces and $f : X \rightarrow Y$. Then $f$
    is
    an \emph{open map} iff, for all $U$ open in $X$, $f(U)$ is open in $Y$.
  \end{df}

  \begin{lm}
    \label{lm:topology:open_map:basis}
    Let $X$ and $Y$ be topological spaces and $f : X \rightarrow Y$. Let
    $\mathcal{B}$ be a basis for the topology on $X$. Then $f$ is an open map
    if
    and only if, for all $B \in \mathcal{B}$, $f(B)$ is open in $Y$.
  \end{lm}

  \begin{proof}
    \pf
    \step{<1>1}{If $f$ is an open map then, for all $B \in \mathcal{B}$, $f(B)$
      is
      open in $Y$.}
    \begin{proof}
      \pf\ Immediate from definitions.
    \end{proof}
    \step{<1>2}{If, for all $B \in \mathcal{B}$, $f(B)$ is open in $Y$, then
      $f$ is
      an open map.}
    \begin{proof}
      \step{<2>1}{\assume{For all $B \in \mathcal{B}$, $f(B)$ is open in $Y$.}}
      \step{<2>2}{\pflet{$U$ be open in $X$} \prove{$f(U)$ is open in $Y$}}
      \step{<2>3}{\pflet{$\mathcal{B}_0 \subseteq \mathcal{B}$ be such that $U
          =
          \bigcup \mathcal{B}_0$}}
      \step{<2>4}{$f(U) = \bigcup_{B \in \mathcal{B}_0} f(B)$}
      \begin{proof}
        \pf\ Set theory.
      \end{proof}
      \step{<2>5}{$f(U)$ is open in $Y$.}
      \begin{proof}
        \pf\ From \stepref{<2>1}, \stepref{<2>4} and the fact that the open
        sets
        are closed under union.
      \end{proof}
    \end{proof}
    \qed
  \end{proof}

  \begin{cor}
    \label{cor:topology:open_map:subbasis}
    Let $X$ and $Y$ be topological spaces and $f : X \rightarrow Y$. Let
    $\mathcal{S}$ be a subbasis for the topology on $X$. Then $f$ is an open
    map
    if
    and only if, for all $S \in \mathcal{S}$, $f(S)$ is open in $Y$.
  \end{cor}

  \begin{lm}[AC]
    Let $\{X_\alpha\}_{\alpha \in J}$ be a family of topological spaces. Then
    the projection $\pi_\alpha : \prod_{\alpha \in J} X_\alpha \rightarrow
    X_\alpha$ is an open map.
  \end{lm}

  \begin{proof}
    \pf
    \step{<1>1}{For $U$ open in $X_\alpha$, we have
      $\pi_\alpha(\pi_\alpha^{-1}(U))$ is open in $X_\alpha$}
    \begin{proof}
      \pf\ $\pi_\alpha(\pi_\alpha^{-1}(U)) = U$ if all the other $X_\alpha$ are
      nonempty, $\emptyset$ otherwise.
    \end{proof}
    \step{<1>2}{For $\beta \neq \alpha$ and $U$ open in $X_\beta$, we have
      $\pi_\alpha(\pi_\beta^{-1}(U))$ is open in $X_\alpha$}
    \begin{proof}
      \pf\ $\pi_\alpha(\pi_\beta^{-1}(U)) = X_\alpha$ if all the $X_\gamma$ are
      nonempty for $\gamma \neq \alpha$, $\emptyset$ otherwise.
    \end{proof}
    \qedstep
    \begin{proof}
      \pf\ By Corollary \ref{cor:topology:open_map:subbasis}.
    \end{proof}
  \end{proof}

  \section{Continuous Functions}

  \begin{df}[Continuous]
    Let $X$ and $Y$ be topological spaces and $f : X \rightarrow Y$ a function.
    Then $f$ is \emph{continuous} if and only if, for every open set $U$ in
    $Y$,
    the set $f^{-1}(U)$ is open in $X$.
  \end{df}

  \begin{thm}
    \label{thm:topology:continuous:characterisation}
    Let $X$ and $Y$ be topological spaces and $f : X \rightarrow Y$. Then the
    following are equivalent.
    \begin{enumerate}
      \item $f$ is continuous.
      \item For every closed set $C$ in $Y$, the set $f^{-1}(C)$ is closed in
      $X$.
      \item For every set $A \subseteq X$, we have $f(\overline{A}) \subseteq
      \overline{f(A)}$.
    \end{enumerate}
  \end{thm}

  \begin{proof}
    \pf
    \step{<1>1}{$1 \Rightarrow 3$}
    \begin{proof}
      \step{<2>1}{\assume{$f$ is continuous.}}
      \step{<2>2}{\pflet{$A \subseteq X$}}
      \step{<2>3}{\pflet{$x \in \overline{A}$} \prove{$f(x) \in
          \overline{f(A)}$}}
      \step{<2>4}{\pflet{$V$ be a neighbourhood of $f(x)$}}
      \step{<2>5}{$f^{-1}(V)$ is a neighbourhood of $x$}
      \begin{proof}
        \pf\ \stepref{<2>1}, \stepref{<2>3}, \stepref{<2>4}
      \end{proof}
      \step{<2>6}{$f^{-1}(V)$ intersects $A$ in $a$, say.}
      \begin{proof}
        \pf\ \stepref{<2>3}, \stepref{<2>5}, Theorem
        \ref{thm:topology:closure:neighbourhoods}.
      \end{proof}
      \step{<2>7}{$V$ intersects $f(A)$ in $f(a)$.}
      \qedstep
      \begin{proof}
        \pf\ Theorem \ref{thm:topology:closure:neighbourhoods}.
      \end{proof}
    \end{proof}
    \step{<1>2}{$3 \Rightarrow 2$}
    \begin{proof}
      \step{<2>1}{\assume{3}}
      \step{<2>2}{\pflet{$C$ be a closed set in $Y$}}
      \step{<2>3}{$\overline{f^{-1}(C)} = f^{-1}(C)$}
      \begin{proof}
        \pf
        \begin{align*}
          f(\overline{f^{-1}(C)}) & \subseteq \overline{f(f^{-1}(C))} &
          (\text{\stepref{<2>1}}) \\
          & \subseteq \overline{C}
        \end{align*}
      \end{proof}
    \end{proof}
    \step{<1>3}{$2 \Rightarrow 1$}
    \begin{proof}
      \step{<2>1}{\assume{2}}
      \step{<2>2}{\pflet{$V$ be open in $Y$}}
      \step{<2>3}{$f^{-1}(Y \setminus V)$ is closed in $X$}
      \begin{proof}
        \pf\ By \stepref{<2>1}.
      \end{proof}
      \step{<2>4}{$f^{-1}(V)$ is open in $X$.}
      \begin{proof}
        \pf\ $f^{-1}(V) = X \setminus f^{-1}(Y \setminus V)$.
      \end{proof}
    \end{proof}
    \qed
  \end{proof}

  \begin{lm}
    \label{lm:topology:continuous:constant}
    If $f : X \rightarrow Y$ maps all of $X$ to the single point $y_0$ of $Y$,
    then $f$ is continuous.
  \end{lm}

  \begin{proof}
    \pf\ For $V$ open in $Y$, the set $f^{-1}(V)$ is either $X$ (if $y_0 \in
    V$) or $\emptyset$ (if $y_0 \notin V$).
  \end{proof}

  \begin{df}[Continuity at a Point]
    Let $X$ and $Y$ be topological spaces, $f : X \rightarrow Y$ a function,
    and
    $x \in X$. Then $f$ is \emph{continuous at $x$} if and only if, for every
    neighbourhood $V$ of $f(x)$, there exists a neighbourhood $U$ of $x$ such
    that
    $f(U) \subseteq V$.
  \end{df}

  \begin{thm}
    \label{thm:topology:continuous:at_every_point}
    Let $X$ and $Y$ be topological spaces and $f : X \rightarrow Y$. Then $f$
    is
    continuous if and only if $f$ is continuous at every point of $X$.
  \end{thm}

  \begin{proof}
    \pf
    \step{<1>1}{If $f$ is continuous then $f$ is continuous at every point of
      $X$.}
    \begin{proof}
      \step{<2>1}{\assume{$f$ is continuous}}
      \step{<2>2}{\pflet{$x \in X$}}
      \step{<2>3}{\pflet{$V$ be a neighbourhood of $f(x)$}}
      \step{<2>4}{$f^{-1}(V)$ is a neighbourhood of $x$}
      \step{<2>5}{$f(f^{-1}(V)) \subseteq V$}
    \end{proof}
    \step{<1>2}{If $f$ is continuous at every point of $X$ then $f$ is
      continuous.}
    \begin{proof}
      \step{<2>1}{\assume{$f$ is continuous at every point of $X$.}}
      \step{<2>2}{\pflet{$V$ be open in $Y$} \prove{$f^{-1}(V)$ is open in
          $X$.}}
      \step{<2>3}{\pflet{$x \in f^{-1}(V)$}}
      \step{<2>4}{$V$ is a neighbourhood of $f(x)$}
      \step{<2>5}{\pick\ a neighbourhood $U$ of $x$ such that $f(U) \subseteq
        V$}
      \begin{proof}
        \pf\ By \stepref{<2>1}.
      \end{proof}
      \step{<2>6}{$x \in U \subseteq f^{-1}(V)$}
      \qedstep
      \begin{proof}
        \pf\ By Proposition \ref{prop:topology:neighbourhood:open}.
      \end{proof}
    \end{proof}
    \qed
  \end{proof}

  \begin{lm}
    \label{lm:topology:continuous:basis}
    Let $X$ and $Y$ be topological spaces and $f : X \rightarrow Y$. Let
    $\mathcal{B}$ be a basis for the topology on $Y$. Then $f$ is continuous if
    and only if, for all $B \in \mathcal{B}$, the set $f^{-1}(B)$ is open in
    $X$.
  \end{lm}

  \begin{proof}
    \pf
    \step{<1>1}{If $f$ is continuous then, for all $B \in \mathcal{B}$, the set
      $f^{-1}(B)$ is open in $X$.}
    \begin{proof}
      \pf\ Immediate from definitions.
    \end{proof}
    \step{<1>2}{If, for all $B \in \mathcal{B}$, the set $f^{-1}(B)$ is open in
      $X$,
      then $f$ is continuous.}
    \begin{proof}
      \step{<2>1}{\assume{For all $B \in \mathcal{B}$, the set $f^{-1}(B)$ is
          open
          in $X$.}}
      \step{<2>2}{\pflet{$x \in X$}}
      \step{<2>3}{\pflet{$V$ be a neighbourhood of $f(x)$}}
      \step{<2>4}{\pick\ $B \in \mathcal{B}$ such that $f(x) \in B \subseteq V$}
      \step{<2>5}{$f^{-1}(B)$ is a neighbourhood of $x$}
      \begin{proof}
        \pf\ By \stepref{<2>1}.
      \end{proof}
      \step{<2>6}{$f(f^{-1}(B)) \subseteq B$}
      \begin{proof}
        \pf\ Set theory.
      \end{proof}
      \qedstep
      \begin{proof}
        \pf\ Theorem \ref{thm:topology:continuous:at_every_point}.
      \end{proof}
    \end{proof}
    \qed
  \end{proof}

  \begin{lm}
    \label{lm:topology:continuous:projections}
    The projections $\pi_1 : X \times Y \rightarrow X$ and $\pi_2 : X \times Y
    \rightarrow Y$ are continuous.
  \end{lm}

  \begin{proof}
    \pf Immediate from definitions. \qed
  \end{proof}

  \begin{thm}
    If $A$ is a subspace of $X$, the inclusion function $j : A \rightarrow
    X$
    is
    continuous.
  \end{thm}

  \begin{proof}
    \pf\ For $V$ open in $X$, the set $j^{-1}(V) = V \cap A$ is open in $A$.
  \end{proof}

  \begin{thm}
    If $f : X \rightarrow Y$ and $g : Y \rightarrow Z$ are continuous,
    then
    the map
    $g \circ f : X \rightarrow Z$ is continuous.
  \end{thm}

  \begin{proof}
    \pf
    \step{<1>1}{\pflet{$V$ be open in $Z$}}
    \step{<1>2}{$g^{-1}(V)$ is open in $Y$}
    \step{<1>3}{$f^{-1}(g^{-1}(V))$ is open in $X$}
    \qed
  \end{proof}

  \begin{thm}
    If $f : X \rightarrow Y$ is continuous and if $A$ is a subspace of
    $X$,
    then
    the restricted function $f \restriction A : A \rightarrow Y$ is
    continuous.
  \end{thm}

  \begin{proof}
    \pf\ For $V$ open in $Y$, the set $(f \restriction A)^{-1}(V) = f^{-1}(V)
    \cap A$ is open in $A$. \qed
  \end{proof}

  \begin{thm}
    Let $f : X \rightarrow Y$ be continuous. If $Z$ is a subspace of $Y$
    that
    includes the range of $f$, then the function $g : X \rightarrow Z$
    obtained by
    restricting the codomain of $f$ is continuous. If $Z$ is a space having
    $Y$ as
    a subspace, then the function $h : X \rightarrow Z$ obtained by expanding
    the
    codomain of $f$ is continuous.
  \end{thm}

  \begin{proof}
    \pf
    \step{<1>1}{If $Z$ is a subspace of $Y$ that
      includes the range of $f$, then the function $g : X \rightarrow Z$
      obtained by
      restricting the codomain of $f$ is continuous.}
    \begin{proof}
      \step{<2>1}{\pflet{$V$ be open in $Z$}}
      \step{<2>2}{\pick\ $W$ open in $Y$ such that $V = W \cap Z$}
      \step{<2>3}{$f^{-1}(W)$ is open in $X$.}
      \step{<2>4}{$g^{-1}(V)$ is open in $X$.}
      \begin{proof}
        \pf\ $g^{-1}(V) = f^{-1}(W)$.
      \end{proof}
    \end{proof}
    \step{<1>2}{If $Z$ is a space having $Y$ as
      a subspace, then the function $h : X \rightarrow Z$ obtained by
      expanding the
      codomain of $f$ is continuous.}
    \begin{proof}
      \pf\ For $V$ open in $Z$, we have $h^{-1}(V) = f^{-1}(V \cap Y)$ is
      open
      in
      $X$.
    \end{proof}
    \qed
  \end{proof}

  \begin{thm}
    Let $X$ and $Y$ be topological spaces and $f : X \rightarrow Y$.
    If $x_n \rightarrow x$ as $n \rightarrow \infty$ in $X$ and $f$ is
    continuous
    at $x$, then $f(x_n)
    \rightarrow f(x)$ as $n \rightarrow \infty$ in $Y$.
  \end{thm}

  \begin{proof}
    \pf
    \step{<1>1}{\assume{$x_n \rightarrow x$ as $n \rightarrow \infty$}}
    \step{<1>2}{\assume{$f$ is continuous at $x$}}
    \step{<1>3}{\pflet{$V$ be a neighbourhood of $f(x)$}}
    \step{<1>4}{\pick\ a neighbourhood $U$ of $x$ such that $f(U) \subseteq V$}
    \begin{proof}
      \pf\ By \stepref{<1>2}.
    \end{proof}
    \step{<1>5}{\pick\ $N$ such that, for all $n \geq N$, $x_n \in U$}
    \begin{proof}
      \pf\ By \stepref{<1>1}
    \end{proof}
    \step{<1>6}{For $n \geq N$, $f(x_n) \in V$}
    \begin{proof}
      \pf\ By \stepref{<1>4}.
    \end{proof}
    \qed
  \end{proof}

  \begin{thm}
    \label{thm:topology:continuous:local}
    Let $X$, $Y$ and $Z$ be topological spaces.
    Let $f : X \rightarrow Y$. If there exists a set $\mathcal{A}$ of open
    sets in
    $X$ such that:
    \begin{itemize}
      \item $\bigcup \mathcal{A} = X$;
      \item for all $U \in \mathcal{A}$, the function $f \restriction U : U
      \rightarrow X$ is continuous;
    \end{itemize}
    then $f$ is continuous.
  \end{thm}

  \begin{proof}
    \pf
    \step{<1>1}{\pflet{$V$ be open in $Y$}}
    \step{<1>2}{For all $U \in \mathcal{A}$, the set $(f \restriction
      U)^{-1}(V)$
      is open in $X$.}
    \begin{proof}
      \step{<2>1}{\pflet{$U \in \mathcal{A}$}}
      \step{<2>2}{$(f \restriction U)^{-1}(V)$ is open in $U$}
      \begin{proof}
        \pf\ Since $f \restriction U : U \rightarrow X$ is continuous.
      \end{proof}
      \qedstep
      \begin{proof}
        \pf\ By Lemma \ref{lm:topology:subspace:open}.
      \end{proof}
    \end{proof}
    \qedstep
    \begin{proof}
      \pf\ Since $f^{-1}(V) = \bigcup_{U \in \mathcal{A}} (f \restriction
      U)^{-1}(V)$.
    \end{proof}
  \end{proof}

  \begin{thm}[The Pasting Lemma]
    Let $X = A \cup B$ where $A$ and $B$ are closed in $X$. Let $f : A
    \rightarrow
    Y$ and $g : B \rightarrow Y$ be continuous. If $f(x) = g(x)$ for every $x
    \in A
    \cap B$, then the function $h : X \rightarrow Y$ defined by
    \[ h(x) = \begin{cases}
      f(x) & \text{if } x \in A \\
      g(x) & \text{if } x \in B
    \end{cases} \]
    is continuous.
  \end{thm}

  \begin{proof}
    \pf
    \step{<1>1}{\pflet{$C$ be closed in $Y$}}
    \step{<1>2}{$f^{-1}(C)$ is closed in $A$}
    \begin{proof}
      \pf\ Theorem \ref{thm:topology:continuous:characterisation}.
    \end{proof}
    \step{<1>3}{$f^{-1}(C)$ is closed in $X$}
    \begin{proof}
      \pf\ Lemma \ref{cor:topology:subspace:closed}.
    \end{proof}
    \step{<1>4}{$g^{-1}(C)$ is closed in $B$}
    \begin{proof}
      \pf\ Theorem \ref{thm:topology:continuous:characterisation}.
    \end{proof}
    \step{<1>5}{$g^{-1}(C)$ is closed in $X$}
    \begin{proof}
      \pf\ Lemma \ref{cor:topology:subspace:closed}.
    \end{proof}
    \step{<1>6}{$h^{-1}(C)$ is closed in $X$}
    \begin{proof}
      \pf\ $h^{-1}(C) = f^{-1}(C) \cup g^{-1}(C)$
    \end{proof}
    \qedstep
    \begin{proof}
      \pf\ Theorem \ref{thm:topology:continuous:characterisation}.
    \end{proof}
    \qed
  \end{proof}

  \begin{thm}
    \label{thm:topology:continuous:product}
    Let $f : A \rightarrow \prod_{\alpha \in J} X_\alpha$ be given by the
    equation
    \[ f(a) = \{ f_\alpha(a) \}_{\alpha \in J} \enspace , \]
    where $f_\alpha : A \rightarrow X_\alpha$ for each $\alpha$. Let
    $\prod_{\alpha
      \in J} X_\alpha$ have the product topology. Then the function $f$ is
    continuous if and only if each function $f_\alpha$ is continuous.
  \end{thm}

  \begin{proof}
    \pf
    \step{<1>1}{If $f$ is continuous then each $f_\alpha$ is continuous.}
    \begin{proof}
      \pf\ This holds because $f_\alpha = \pi_\alpha \circ f$.
    \end{proof}
    \step{<1>2}{If every $f_\alpha$ is continuous then $f$ is continuous.}
    \begin{proof}
      \step{<2>1}{\assume{Every $f_\alpha$ is continuous.}}
      \step{<2>2}{\pflet{$\alpha \in J$ and $U$ be open in $X_\alpha$}}
      \step{<2>3}{$f^{-1}(\pi_\alpha^{-1}(U))$ is open in $A$}
      \begin{proof}
        \pf\ $f^{-1}(\pi_\alpha^{-1}(U)) = f_\alpha^{-1}(U)$.
      \end{proof}
    \end{proof}
    \qed
  \end{proof}



  \subsection{Homeomorphisms}

  \begin{df}[Homeomorphism]
    Let $X$ and $Y$ be topological spaces and $f : X \rightarrow Y$. Then $f$
    is
    a
    \emph{homeomorphism} between $X$ and $Y$ iff $f$ is a bijection, and $f$
    and
    $f^{-1}$ are both continuous.
  \end{df}

  \begin{df}[Topological Property]
    A property $P$ of topological spaces is a \emph{topological property} iff,
    for any spaces $X$ and $Y$, if $X$ is homeomorphic to $Y$ then $P$ holds of
    $X$
    if and only if $P$ holds of $Y$.
  \end{df}

  \begin{df}[(Topological) Imbedding]
    Let $X$ and $Y$ be topological spaces and $f : X \rightarrow Y$. Then $f$
    is
    a
    \emph{(topological) imbedding} iff $f$ is a homeomorphism between $X$ and
    $\im f$.
  \end{df}

  \begin{df}[Homogeneous]
    A topological space $X$ is \emph{homogeneous} iff, for all $x, y \in X$,
    there exists a homeomorphism $f : X \cong X$ such that $f(x) = y$.
  \end{df}


  \subsection{Strongly Continuous Functions}

  \begin{df}[Strongly Continuous]
    Let $X$ and $Y$ be topological spaces and $f : X \rightarrow Y$. Then $f$
    is
    \emph{strongly continuous} iff, for all $V \subseteq Y$, we have $V$ is
    open
    in $Y$ if and only if $f^{-1}(V)$ is open in $X$.
  \end{df}

  \begin{prop}
    \label{prop:topology:strongly_continuous:closed}
    Let $X$ and $Y$ be topological spaces and $f : X \rightarrow Y$. Then $f$
    is
    strongly continuous if and only if, for all $C \subseteq Y$, $C$ is closed
    in
    $Y$ if and only if $f^{-1}(C)$ is closed in $X$.
  \end{prop}

  \begin{proof}
    \pf
    \step{<1>1}{If $f$ is strongly continuous then, for all $C \subseteq Y$, we
      have $C$ is closed in $Y$ if and only if $f^{-1}(C)$ is closed in $X$.}
    \begin{proof}
      \pf
      \begin{align*}
        C \text{ is closed in } Y & \Leftrightarrow Y \setminus C \text{ is
          open in } Y \\
        & \Leftrightarrow f^{-1}(Y \setminus C) \text{ is open in } X \\
        & \Leftrightarrow X \setminus f^{-1}(C) \text{ is open in } X \\
        & \Leftrightarrow f^{-1}(C) \text{ is closed in } X
      \end{align*}
    \end{proof}
    \step{<1>2}{If, for all $C \subseteq Y$, we have $C$ is closed in $Y$ if
      and
      only if $f^{-1}(C)$ is closed in $X$, then $f$ is strongly continuous.}
    \begin{proof}
      \pf\ Similar.
    \end{proof}
    \qed
  \end{proof}

  \begin{prop}
    \label{prop:topology:strongly_continuous:composite}
    The composite of two strongly continuous functions is strongly continuous.
  \end{prop}

  \begin{proof}
    \pf
    \step{<1>1}{\pflet{$f : X \rightarrow Y$ and $g : Y \rightarrow Z$ be
        strongly
        continuous.}}
    \step{<1>2}{\pflet{$V \subseteq Z$}}
    \step{<1>3}{$V$ is open iff $f^{-1}(g^{-1}(V))$ is open}
    \begin{proof}
      \pf
      \begin{align*}
        V \text{ is open} & \Leftrightarrow g^{-1}(V) \text{ is open} &
        (\text{\stepref{<1>1}}) \\
        & \Leftrightarrow f^{-1}(g^{-1}(V)) \text{ is open} &
        (\text{\stepref{<1>1}})
      \end{align*}
    \end{proof}
    \qed
  \end{proof}

  \begin{prop}
    Let $X$, $Y$ and $Z$ be topological spaces.
    Let $f : X \rightarrow Y$ and $g : Y \rightarrow Z$. If $f$ is strongly
    continuous and $g \circ f$ is continuous, then $g$ is continuous.
  \end{prop}

  \begin{proof}
    \pf
    \step{<1>1}{\pflet{$V \subseteq Z$ be open in $Z$.}}
    \step{<1>2}{$f^{-1}(g^{-1}(V))$ is open in $X$.}
    \begin{proof}
      \pf\ $g \circ f$ is continuous.
    \end{proof}
    \step{<1>3}{$g^{-1}(V)$ is open in $Y$.}
    \begin{proof}
      \pf\ $f$ is strongly continuous.
    \end{proof}
    \qed
  \end{proof}

  \begin{prop}
    Let $X$, $Y$ and $Z$ be topological spaces. Let $f : X \rightarrow Y$ and
    $g
    : Y \rightarrow Z$. If $f$ and $g \circ f$ are strongly continuous, then
    $g$
    is
    strongly continuous.
  \end{prop}

  \begin{proof}
    \pf
    \step{<1>1}{\pflet{$U \subseteq Z$}}
    \step{<1>2}{$U$ is open in $Z$ iff $g^{-1}(U)$ is open in $Y$}
    \begin{proof}
      \pf
      \begin{align*}
        U \text{ is open in } Z & \Leftrightarrow f^{-1}(g^{-1}(U)) \text{ is
          open in } X & (g \circ f \text{ is strongly continuous}) \\
        & \Leftrightarrow g^{-1}(U) \text{ is open in } Y & (f \text{ is
          strongly continuous})
      \end{align*}
    \end{proof}
    \qed
  \end{proof}

  \section{Closed Maps}

  \begin{df}[Closed Map]
    Let $X$ and $Y$ be topological spaces and $f : X \rightarrow Y$. Then $f$
    is
    a \emph{closed map} iff, for every closed set $C \subseteq X$, the
    set $f(C)$ is closed in $Y$.
  \end{df}

  \section{Local Homeomorphism}

  \begin{df}[Locally Homeomorphic]
    Let $X$ and $Y$ be topological spaces. Then $X$ is \emph{locally
      homeomorphic} to $Y$ iff every point in $X$ has an open neighborhood that
    is homeomorphic with an open set in $Y$.
  \end{df}

  \begin{prop}
    The long line is locally homeomorphic with $\mathbb{R}$.
  \end{prop}

  \begin{proof}
    \pf
    \step{<1>1}{\pflet{$x \in L$}}
    \step{<1>2}{\pick\ an ordinal $\alpha$ such that $x < (\alpha, 0)$.}
    \step{<1>3}{$(- \infty, (\alpha, 0))$ is an open neighbourhood of $x$ that
      is
      homeomorphic to $(0, 1)$.}
    \qed
  \end{proof}

  \section{Retracts}

   \begin{df}[Retract]
  Let $Z$ be a topological space. If $Y$ is a subspace of $Z$, we say that $Y$
  is a \emph{retract} of $Z$ iff there exists a continuous function $r : Z
\rightarrow Y$ such that $r(y) = y$ for all $y \in Y$.
 \end{df}

  \chapter{Separation Axioms}

  \section{$T_1$ Spaces}

  \begin{df}[$T_1$ Space]
    A topological space $X$ is a \emph{$T_1$ space} iff every finite set is
    closed.
  \end{df}

  \begin{thm}
    \label{thm:topology:T1:limit_point}
    Let $X$ be a $T_1$ space and $A \subseteq X$. Then $x$ is a limit point of
    $A$
    if and only if every neighbourhood of $x$ contains infinitely many points
    of
    $A$.
  \end{thm}

  \begin{proof}
    \pf
    \step{<1>1}{If some neighbourhood of $x$ contains only finitely many points
      of
      $A$ then $x$ is not a limit point of $A$.}
    \begin{proof}
      \step{<2>1}{\assume{Some neighbourhood $U$ of $x$ contains only finite
          many
          points $a_1$, \ldots, $a_n$ of $A$.}}
      \step{<2>2}{$X \setminus \{ a_1, \ldots, a_n \}$ is open.}
      \begin{proof}
        \pf\ $X$ is $T_1$.
      \end{proof}
      \step{<2>3}{$U \setminus \{ a_1, \ldots, a_n \}$ is a neighbourhood of
        $x$
        that does not intersect $A$.}
    \end{proof}
    \step{<1>2}{If every neighbourhood of $x$ contains infinitely many points
      of $A$
      then $x$ is a limit point of $A$.}
    \begin{proof}
      \pf\ From the definition of limit point.
    \end{proof}
    \qed
  \end{proof}

    \begin{prop}
    \label{prop:topology:T1:subspace}
   A subspace of a $T_1$ space is $T_1$.
  \end{prop}

  \begin{proof}
   \pf
   \step{<1>1}{\pflet{$X$ be a $T_1$ space and $Y \subseteq X$}}
   \step{<1>2}{\pflet{$a \in Y$}}
   \step{<1>3}{$\{a\}$ is closed in $X$}
   \begin{proof}
     \pf\ By \stepref{<1>1}.
   \end{proof}
   \step{<1>4}{$\{a\}$ is closed in $Y$}
   \begin{proof}
     \pf\ By Corollary \ref{cor:topology:subspace:closed}.
   \end{proof}
   \qed
  \end{proof}

    \begin{df}[Separate Points from Closed Sets]
    Let $X$ be a space and $\{ f_\alpha \}_{\alpha \in J}$ be a family of
     continuous functions $f_\alpha : X \rightarrow \mathbb{R}$. Then $\{
     f_\alpha \}$ \emph{separates points from closed sets} in $X$ iff, for
every point $x_0 \in X$ and every neighbourhood $U$ of $x_0$, there exists
$\alpha \in J$ such that $f_\alpha$ is positive at $x_0$ and vanishes outside
$U$.
 \end{df}

   \begin{thm}[Imbedding Theorem]
   Let $X$ be a $T_1$ space and $\{ f_\alpha \}_{\alpha \in J}$ be a family of
   functions $X \rightarrow \mathbb{R}$ that separates points from closed sets.
   Then the function $F : X \rightarrow \mathbb{R}^J$ defined by
   \[ F(x)_\alpha = f_\alpha(x) \]
   is an imbedding. If each $f_\alpha$ maps $X$ into $[0,1]$ then $F$ is an
   imbedding $X \rightarrow [0,1]^J$.
 \end{thm}

 \begin{proof}
  \pf
  \step{<1>1}{$F$ is continuous}
  \begin{proof}
    \pf\ By Theorem \ref{thm:topology:continuous:product}.
  \end{proof}
  \step{<1>2}{$F$ is injective}
  \begin{proof}
    \step{<2>1}{\pflet{$x, y \in X$ with $x \neq y$}}
    \step{<2>2}{\pick\ a neighbourhood $U$ of $x$ such that $y \notin U$}
    \begin{proof}
      \pf\ $X$ is $T_1$
    \end{proof}
    \step{<2>3}{\pick\ $\alpha \in J$ such that $f_\alpha$ is positive at $x$
and
      vanishes outside $U$}
    \step{<2>4}{$f_\alpha(x) \neq f_\alpha(y)$}
    \step{<2>5}{$F(x) \neq F(y)$}
  \end{proof}
  \step{<1>3}{$F$ is open as a map $X \rightarrow F(U)$}
  \begin{proof}
    \step{<2>1}{\pflet{$U$ be open}}
    \step{<2>2}{\pflet{$z \in F(U)$}}
    \step{<2>3}{\pick\ $x \in U$ such that $F(x) = z$}
    \step{<2>4}{\pick\ $\alpha \in J$ such that $f_\alpha$ is positive at $x$
and
      vanishes outside $U$}
    \step{<2>5}{$z \in \inv{\pi_\alpha}((0, +\infty)) \cap F(U) \subseteq F(U)$}
  \end{proof}
  \qed
 \end{proof}


  \section{Hausdorff Spaces}

  \begin{df}[Hausdorff Space]
    A topological space $X$ is a \emph{Hausdorff space} iff, for any points $x,
    y
    \in X$ with $x \neq y$, there exist disjoint neighbourhoods $U$ of $x$ and
    $V$
    of $y$.
  \end{df}

  \begin{thm}
    Every Hausdorff space is $T_1$.
  \end{thm}

  \begin{proof}
    \pf
    \step{<1>1}{\pflet{$X$ be a Hausdorff space}}
    \step{<1>2}{\pflet{$a \in X$} \prove{$\{a\}$ is closed.}}
    \step{<1>3}{\pflet{$b \in X \setminus \{a\}$}}
    \step{<1>4}{\pick\ disjoint neighbourhoods $U$ of $a$ and $V$ of $b$}
    \step{<1>5}{$b \in V \subseteq X \setminus \{a\}$}
    \qedstep
    \begin{proof}
      \pf\ By Proposition \ref{prop:topology:neighbourhood:open}.
    \end{proof}
    \qed
  \end{proof}

  \begin{thm}
    In a Hausdorff space, a sequence has at most one limit.
  \end{thm}

  \begin{proof}
    \pf
    \step{<1>1}{\assume{for a contradiction $x_n \rightarrow l$ and $x_n
        \rightarrow
        m$ as $n \rightarrow \infty$, and $l \neq m$}}
    \step{<1>2}{\pick\ disjoint neighbourhoods $U$ of $l$ and $V$ of $m$}
    \step{<1>3}{\pick\ $N$ such that, for all $n \geq N$, $x_n \in U$ and $x_n
      \in V$}
    \step{<1>4}{$x_N \in U \cap V$}
    \qed
  \end{proof}

  \begin{thm}
    Every linearly ordered set is Hausdorff under the order topology.
  \end{thm}

  \begin{proof}
    \pf
    \step{<1>1}{\pflet{$X$ be a linearly ordered set under the order topology.}}
    \step{<1>2}{\pflet{$x, y \in X$ with $x \neq y$}}
    \step{<1>3}{\assume{w.l.o.g.~$x < y$} \prove{There exist disjoint
        neighbourhoods
        $U$ of $x$ and $V$ of $y$.}}
    \step{<1>4}{\case{There exists $z$ such that $x < z < y$}}
    \begin{proof}
      \pf\ In this case, take $U = (-\infty, z)$ and $V = (z, +\infty)$.
    \end{proof}
    \step{<1>5}{\case{There does not exist $z$ such that $x < z < y$}}
    \begin{proof}
      \pf\ In this case, take $U = (-\infty, y)$ and $V = (x, +\infty)$.
    \end{proof}
    \qed
  \end{proof}

  \begin{thm}
    \label{thm:topology:Hausdorff:product}
    Let $\{ X_\alpha \}_{\alpha \in J}$ be a family of Hausdorff spaces. Then
    $\prod_{\alpha \in J} X_\alpha$ is Hausdorff under the product topology.
  \end{thm}

  \begin{proof}
    \pf
    \step{<1>1}{\pflet{$\{x_\alpha\}_{\alpha \in J}, \{y_\alpha\}_{\alpha \in
          J}
        \in \prod_{\alpha \in J} X_\alpha$ with $\{x_\alpha\}_{\alpha \in J}
        \neq \{y_\alpha\}_{\alpha \in J}$}}
    \step{<1>2}{\pick\ $\alpha \in J$ such that $x_\alpha \neq y_\alpha$}
    \step{<1>3}{\pick\ disjoint neighbourhoods $U$ of $x_\alpha$ and $V$ of
      $y_\alpha$.}
    \step{<1>4}{$\pi_\alpha^{-1}(U)$ and $\pi_\alpha^{-1}(V)$ are disjoint
      neighbourhoods of $\{x_\alpha\}_{\alpha \in J}$ and $\{y_\alpha\}_{\alpha
        \in
        J}$}
    \qed
  \end{proof}

  \begin{cor}
   The Sorgenfrey plane is Hausdorff.
  \end{cor}

  \begin{cor}
  	For any set $I$,
    the space $\mathbb{R}^I$ is Hausdorff.
  \end{cor}

  \begin{prop}
    Let $X$ and $Y$ be topological spaces and $f : X \rightarrow Y$. If $f$ is
    continuous and injective and $Y$ is Hausdorff then $X$ is Hausdorff.
  \end{prop}

  \begin{proof}
    \pf
    \step{<1>1}{\pflet{$x, y \in X$ with $x \neq y$}}
    \step{<1>2}{$f(x) \neq f(y)$}
    \begin{proof}
      \pf\ $f$ is injective.
    \end{proof}
    \step{<1>3}{\pick\ disjoint neighbourhoods $U$, $V$ of $f(x)$ and $f(y)$}
    \begin{proof}
      \pf\ $Y$ is Hausdorff.
    \end{proof}
    \step{<1>4}{$f^{-1}(U)$ and $f^{-1}(V)$ are disjoint neighbourhoods of $x$
      and
      $y$.}
    \qed
  \end{proof}

  \begin{cor}
    \label{cor:topology:Hausdorff:subspace}
    A subspace of a Hausdorff space is Hausdorff.
  \end{cor}

  \begin{cor}
    Let $\{ X_\alpha \}_{\alpha \in J}$ be a family of nonempty spaces. If
    $\prod_{\alpha \in J} X_\alpha$ is Hausdorff then so is each $X_\alpha$.
  \end{cor}

  \begin{cor}
    Let $\mathcal{T}$ and $\mathcal{T}'$ be topologies on the same set $X$. If
    $\mathcal{T} \subseteq \mathcal{T}'$ and $X$ is Hausdorff under
    $\mathcal{T}$ then $X$ is Hausdorff under $\mathcal{T}'$.
  \end{cor}

  \begin{cor}
    The space $\mathbb{R}_K$ is Hausdorff.
  \end{cor}

   \begin{prop}
   $\mathbb{R}_l$ is Hausdorff.
 \end{prop}

 \begin{proof}
   \pf\ Let $a, b \in \mathbb{R}_l$ with $a < b$. Then $(- \infty, b)$ and $[b,
+\infty)$ are disjoint open sets containing $a$ and $b$ respectively. \qed
 \end{proof}

 \begin{prop}
   The continuous image of a Hausdorff space is not necessarily Hausdorff.
 \end{prop}

 \begin{proof}
   \pf\ The identity map from the discrete two-point space to the indiscrete two-point space is continuous. \qed
 \end{proof}

  \section{Regular Spaces}

  \begin{df}[Regular]
    A topological space $X$ is \emph{regular} iff, for every closed set $A$ and
    point $a \notin A$, there exist disjoint neighbourhoods $U$ of $A$ and $V$
    of
    $a$.
  \end{df}

    \begin{prop}
      \label{prop:topology:regular:closure}
   Let $X$ be a $T_1$ space. Then $X$ is regular if and only if, for every
point $x$ and neighbourhood $U$ of $x$, there exists a neighbourhood $V$ of $x$
such that $\overline{V} \subseteq U$.
  \end{prop}

  \begin{proof}
   \pf
   \step{<1>1}{If $X$ is regular then,  for every
     point $x$ and neighbourhood $N$ of $x$, there exists a neighbourhood $V$
     of $x$ such that $\overline{V} \subseteq N$.}
   \begin{proof}
     \step{<2>1}{\assume{$X$ is regular.}}
     \step{<2>2}{\pflet{$x \in X$ and $N$ be a neighbourhood of $x$}}
     \step{<2>3}{\pick\ an open set $U$ such that $x \in U \subseteq N$}
     \step{<2>4}{\pick\ disjoint open sets $V$, $W$ such that $x \in V$ and $X
       \setminus U \subseteq W$}
     \step{<2>5}{$\overline{V} \subseteq N$}
     \begin{proof}
       \pf
       \begin{align*}
         \overline{V} & \subseteq X \setminus W \\
         & \subseteq U \\
         & \subseteq N
       \end{align*}
     \end{proof}
   \end{proof}
   \step{<1>2}{If, for every
     point $x$ and neighbourhood $U$ of $x$, there exists a neighbourhood $V$
     of $x$ such that $\overline{V} \subseteq U$, then $X$ is regular.}
   \begin{proof}
     \step{<2>1}{\assume{For every point $x$ and neighbourhood $U$ of $x$,
there
         exists a neighbourhood $V$ of $x$ such that $\overline{V} \subseteq
         U$.}}
     \step{<2>2}{\pflet{$x \in X$ and $A$ be a closed set with $x \notin A$}}
     \step{<2>3}{\pick\ a neighbourhood $V$ of $x$ such that $\overline{V}
       \subseteq X \setminus A$}
     \step{<2>4}{$x \in V$ and $A \subseteq X \setminus \overline{V}$}
   \end{proof}
   \qed
  \end{proof}

   \begin{prop}
  Every linearly ordered set under the order topology is regular.
 \end{prop}

 \begin{proof}
  \pf
  \step{<1>1}{\pflet{$X$ be a linearly ordered set under the order topology.}}
  \step{<1>2}{\pflet{$x \in X$ and $U$ be a neighbourhood of $x$} \prove{There
      exists a neighbourhood $V$ of $x$ with $\overline{V} \subseteq U$}}
  \step{<1>3}{\case{$x$ is greatest and least in $X$}}
  \begin{proof}
    \pf\ Take $V = U = X = \{x\}$
  \end{proof}
  \step{<1>4}{\case{$x$ is greatest in $X$ and there exists $a < x$ such that
      $(a,x] \subseteq U$}}
  \begin{proof}
    \step{<2>1}{\case{There exists $b$ such that $a < b < x$}}
    \begin{proof}
      \pf\ Take $V = (b, x]$.
    \end{proof}
    \step{<2>2}{\case{There is no $b$ such that $a < b < x$}}
    \begin{proof}
      \step{<3>1}{\pflet{$V = U = \{x\}$}}
      \step{<3>2}{$\overline{V} = V$}
      \begin{proof}
        \pf\ For any $y \neq x$, we have $(- \infty, x)$ is a neighbourhood of
$y$ that does not intersect $V$.
      \end{proof}
    \end{proof}
  \end{proof}
  \step{<1>5}{\case{$x$ is least in $X$ and there exists $b > x$ such that
$[x,b)
      \subseteq U$}}
  \begin{proof}
    \pf\ Similar.
  \end{proof}
  \step{<1>6}{\case{There exist $a < x < b$ such that $(a,b) \subseteq U$}}
  \begin{proof}
    \step{<2>1}{\pick\ a point $c$ such that $a < c < x$ if there is one,
      otherwise \pflet{$c = a$}}
    \step{<2>2}{\pick\ a point $d$ such that $x < d < b$ if there is one,
      otherwise \pflet{$d = b$}}
    \step{<2>3}{\pflet{$V = (c, d)$}}
    \step{<2>4}{$\overline{V} \subseteq U$}
    \begin{proof}
      \pf
      \begin{align*}
        \overline{V} & \subseteq [c,d] \\
        & \subseteq (a, b) \\
        & \subseteq U
      \end{align*}
    \end{proof}
  \end{proof}
  \qedstep
  \begin{proof}
    \pf\ These cases are exhaustive by Lemma \ref{lm:topology:order:open}. They
prove $X$ is regular by Proposition \ref{prop:topology:regular:closure}.
  \end{proof}
  \qed
 \end{proof}

    \begin{prop}
      \label{prop:topology:regular:subspace}
   A subspace of a regular space is regular.
  \end{prop}

  \begin{proof}
   \pf
   \step{<1>1}{\pflet{$X$ be a regular space and $Y \subseteq X$}}
   \step{<1>2}{\pflet{$A \subseteq Y$ be closed in $Y$ and $a \in Y \setminus
A$}}
   \step{<1>3}{\pick\ $C$ closed in $X$ such that $A = C \cap Y$}
   \begin{proof}
     \pf\ By Corollary \ref{cor:topology:subspace:closed}.
   \end{proof}
   \step{<1>4}{\pick\ disjoint open sets $U$, $V$ in $X$ such that $C \subseteq
U$
     and $a \in V$}
   \step{<1>5}{$U \cap Y$ and $V \cap Y$ are disjoint open sets in $Y$ such
that
     $A \subseteq U \cap Y$ and $a \in V \cap Y$}
   \qed
  \end{proof}

  \begin{cor}
    Let $\{ X_\alpha \}_{\alpha \in J}$ be a family of nonempty spaces. If
    $\prod_{\alpha \in J} X_\alpha$ is regular then so is each $X_\alpha$.
  \end{cor}

  \begin{prop}[AC]
   The product of a family of regular spaces is regular.
  \end{prop}

  \begin{proof}
   \pf
   \step{<1>1}{\pflet{$\{X_\alpha\}_{\alpha \in J}$ be a family of regular
       spaces.}}
   \step{<1>2}{$\prod_{\alpha \in J} X_\alpha$ is $T_1$}
   \step{<1>3}{\pflet{$\vec{a} \in U$ where $U$ is open in $\prod_{\alpha \in
         J} X_\alpha$}}
   \step{<1>4}{\pick\ $\prod_{\alpha \in J} U_\alpha$ such that each $U_\alpha$
     is open in $X_\alpha$, $U_\alpha = X_\alpha$ except at $\alpha_1$, \ldots,
     $\alpha_n$, and $\vec{a} \in \prod_{\alpha \in J} U_\alpha \subseteq U$}
   \step{<1>5}{For $1 \leq i \leq n$, \pick\ $V_{\alpha_i}$ open in
     $X_{\alpha_i}$ such that $a_{\alpha_i} \in V_{\alpha_i}$ and
     $\overline{V_{\alpha_i}} \subseteq U_{\alpha_i}$}
   \step{<1>6}{For $\alpha \neq \alpha_1, \ldots, \alpha_n$, \pflet{$V_\alpha =
       X_\alpha$}}
   \step{<1>7}{$\vec{a} \in \prod_{\alpha \in J} V_\alpha$}
   \step{<1>8}{$\overline{\prod_{\alpha \in J} V_\alpha} \subseteq \prod_{\alpha
       \in J} U_\alpha$}
   \begin{proof}
     \pf\ By Theorem \ref{thm:topology:product:closure}.
   \end{proof}
   \qed
 \end{proof}

 \begin{cor}
  The Sorgenfrey plane is regular.
 \end{cor}

 \begin{cor}
   For any set $I$, the space $\mathbb{R}^I$ is regular.
 \end{cor}

 \begin{prop}
  The space $\mathbb{R}_K$ is not regular.
\end{prop}

\begin{proof}
 \pf There do not exist disjoint neighbourhoods of 0 and $K$. \qed
\end{proof}

\begin{prop}
  The continuous image of a regular space is not necessarily regular.
\end{prop}

\begin{proof}
  \pf\ The identity map from the discrete two-point space to the indiscrete two-point space is continuous. \qed
\end{proof}

 \section{Completely Regular Spaces}

   \begin{df}[Separated by a Continuous Function]
   Let $A$ and $B$ be subsets of a topological space $X$. Then $A$ and $B$ can
   be \emph{separated by a continuous function} iff there exists a continuous
   $f : X \rightarrow [0,1]$ such that $f(A) = \{ 0 \}$ and $f(B) = \{ 1 \}$.
  \end{df}

    \begin{df}[Completely Regular]
    A space $X$ is \emph{completely regular} iff $X$ is $T_1$ and, for every
    point $a$ and closed set $A$ not containing $a$, we have that $\{a\}$ and
    $A$ can be separated by a continuous function.
  \end{df}

   \begin{thm}
     \label{thm:topology:completely_regular:product}
  The product of a family of completely regular spaces is completely regular.
 \end{thm}

 \begin{proof}
  \pf
  \step{<1>1}{\pflet{$\{X_\alpha\}_{\alpha \in J}$ be a family of completely
      regular spaces.}}
  \step{<1>2}{\pflet{$a \in \prod_{\alpha \in J} X_\alpha$ and $A$ be closed in
      $\prod_{\alpha \in J} X_\alpha$ such that $a \notin A$}}
  \step{<1>3}{\pick\ a basic open neighbourhood $\prod_{\alpha \in J} U_\alpha
    \subseteq \prod_{\alpha \in J} X_\alpha \setminus A$ of     $a$ such that
$U_\alpha = X_\alpha$ except for $\alpha = \alpha_1, \ldots,     \alpha_n$}
\step{<1>4}{For $1 \leq i \leq n$, \pick\ a continuous $f_i : X_{\alpha_i}
  \rightarrow [0,1]$ that is $0$ at $a_{\alpha_i}$ and 1 on
  $X_{\alpha_i} \setminus U_{\alpha_i}$}
\step{<1>5}{\pflet{$f : \prod_{\alpha \in J} X_\alpha \rightarrow [0, 1]$ be
given
    by $f(x) = \prod_{i=1}^n f_i(x_{\alpha_i})$}}
\step{<1>6}{$f(a) = 0$}
\step{<1>7}{$f(x) = 1$ for $x \in A$}
\step{<1>8}{$f$ is continuous}
\qed
 \end{proof}

 \begin{cor}
  The Sorgenfrey plane is completely regular.
 \end{cor}

 \begin{cor}
   For any set $I$, the space $\mathbb{R}^I$ is completely regular.
 \end{cor}

  \begin{prop}
   For any set $J$, the space $\mathbb{R}^J$ in the box topology is completely
regular.
 \end{prop}

 \begin{proof}
  \pf
  \step{<1>1}{\pflet{$a \in \mathbb{R}^J$ and $A \subseteq \mathbb{R}^J$ be
closed
      with $a \notin A$} \prove{There exists $f : \mathbb{R}^J_{\mathrm{box}}
\rightarrow       [0,1]$ continuous such that $f(a) = 1$ and $f(A) = \{ 0 \}$}}
\step{<1>2}{\assume{w.l.o.g.~$A \cap (-1, 1)^J = \emptyset$ and $a = \vec{0}$}}
\begin{proof}
  \step{<2>1}{\pick\ a basic open set $\prod_{\alpha \in J} U_\alpha$ such that
$a
    \in \prod_{\alpha \in J} U_\alpha \subseteq \mathbb{R}^J \setminus A$}
  \step{<2>2}{For $\alpha \in J$, \pick\ $b_\alpha, c_\alpha$ such that
$a_\alpha
    \in (b_\alpha, c_\alpha) \subseteq U_\alpha$}
  \step{<2>3}{For $\alpha \in J$, \pick\ a homeomorphism $f_\alpha : \mathbb{R}
    \rightarrow \mathbb{R}$ that maps $b_\alpha$ to $-1$, $a_\alpha$ to $0$ and
    $c_\alpha$ to $1$}
  \step{<2>4}{$\prod_{\alpha \in J} f_\alpha$ is an automorphism
    $\mathbb{R}^J_{\mathrm{box}}$ that maps $a$ to $\vec{0}$ and $A$ to a
    closed set disjoint from $(-1, 1)^J$}
\end{proof}
\step{<1>3}{\pick\ a continuous function $f : \mathbb{R}^J_{\mathrm{uniform}}
  \rightarrow [0,1]$ such that $f(\vec{0}) = 1$ and $f(\mathbb{R}^J \setminus
(-1, 1)^J) = \{ 0 \}$}
\step{<1>4}{$f$ is continuous w.r.t.~the box topology}
\qed
 \end{proof}

  \begin{prop}
  Not every regular space is completely regular.
 \end{prop}

 \begin{proof}
  \pf
  \step{<1>1}{For $m \in \mathbb{Z}$, \pflet{$L_m = \{ m \} \times [-1, 0]$}}
  \step{<1>2}{For each odd integer $n$ and each integer $k \geq 2$,
    \pflet{$C_{nk} = (\{n + 1 - 1/k\} \/home/robin/fun/RogOMatic/src/actuatortimes [-1,0]) \cup (\{ n - 1 + 1/k \}
      \times [-1, 0]) \cup \{ (x,y) : (x-n)^2 + y^2 = (1 - 1/k)^2, y \geq 0
      \}$}}
  \step{<1>3}{For each odd integer $n$ and each integer $k \geq 2$,
\pflet{$p_{nk}
      = (n, 1 - 1/k)$}}
  \step{<1>4}{\pick\ two points $a$, $b$ not in any $L_m$ or $C_{nk}$}
  \step{<1>5}{\pflet{$X = \bigcup_{m \in \mathbb{Z}} L_m \cup \bigcup_{n, k}
      C_{nk} \cup \{ a, b \}$}}
  \step{<1>6}{\pflet{$\mathcal{B}$ be the set consisting of all subsets of
      $\mathbb{R}^2$ of the following forms:
      \begin{enumerate}
       \item The intersection of $X$ with a horizontal open line segment that
       contains none of the points $p_{nk}$
       \item A set formed from one of the sets $C_{nk}$ by deleting finitely
       many points.
       \item For each even integer $m$, the set $\{a\} \cup \{(x,y) \in X : x <
       m \}$
       \item For each even integer $m$, the set $\{b\} \cup \{(x,y) \in X : x >
       m \}$
     \end{enumerate}}}
   \step{<1>7}{$\mathcal{B}$ is a basis for a topology on $X$}
   \begin{proof}
     \step{<2>1}{For all $x \in X$, there exists $B \in \mathcal{B}$ such that
$x
       \in B$}
     \step{<2>2}{For all $B_1, B_2 \in \mathcal{B}$ and $x \in B_1 \cap B_2$,
       there exists $B_3 \in \mathcal{B}$ such that $x \in B_3 \subseteq B_1
       \cap B_2$}
     \begin{proof}
       \step{<3>1}{\case{$B_1$, $B_2$ are both of type 1}}
       \begin{proof}
         \pf\ Their intersection is of type 1.
       \end{proof}
       \step{<3>2}{\case{$B_1$ is of type 1 and $B_2$ is of type 2}}
       \begin{proof}
         \pf\ Their intersection is of type 2, since a horizontal line segment
         intersects $C_{nk}$ in at most two points.
       \end{proof}
       \step{<3>3}{\case{$B_1$ is of type 1 and $B_2$ is of type 3}}
       \begin{proof}
         \pf\ Their intersection is of type 1
       \end{proof}
       \step{<3>4}{\case{$B_1$ is of type 1 and $B_2$ is of type 4}}
       \begin{proof}
         \pf\ Their intersection is of type 1
       \end{proof}
       \step{<3>5}{\case{$B_1$ is of type 2 and $B_2$ is of type 2}}
       \begin{proof}
         \pf\ Their intersection is of type 2
       \end{proof}
       \step{<3>6}{\case{$B_1$ is of type 2 and $B_2$ is of type 3}}
       \begin{proof}
         \pf\ Their intersection is $B_1$
       \end{proof}
       \step{<3>7}{\case{$B_1$ is of type 2 and $B_2$ is of type 4}}
       \begin{proof}
         \pf\ Their intersection is $B_1$
       \end{proof}
       \step{<3>8}{\case{$B_1$ is of type 3 and $B_2$ is of type 3}}
       \begin{proof}
         \pf\ Their intersection is of type 3
       \end{proof}
       \step{<3>9}{\case{$B_1$ is of type 3 and $B_2$ is of type 4}}
       \begin{proof}
         \step{<4>1}{\pflet{$B_1 = \{ a \} \cup \{ (x,y) \in X : x < m \}$ and
             $B_2 = \{ b \} \cup \{ (x,y) \in X : x > n \}$}}
         \step{<4>2}{\case{$x = (s, 1-1/k)$ for some $s$ and integer $x \geq
2$}}
\begin{proof}
  \pf\ In this case, $x \in C_{nk}$ for some $n$ and $C_{nk} \subseteq B_1 \cap
B_2$.
\end{proof}
         \step{<4>3}{\case{$x = (s, t)$ and $t \neq 1 - 1/k$ for any integer $k
             \geq 2$}}
         \begin{proof}
           \pf\ In this case, $x \in ((n, m) \times \{ t \}) \cap X \subseteq
B_1 \cap B_2$
         \end{proof}
       \end{proof}
       \step{<3>10}{\case{$B_1$ is of type 4 and $B_2$ is of type 4}}
       \begin{proof}
         \pf\ Their intersection is of type 4
       \end{proof}
     \end{proof}
   \end{proof}
   \step{<1>8}{For any continuous function $f : X \rightarrow \mathbb{R}$, we
have
     $f(a) = f(b)$}
   \begin{proof}
     \step{<2>1}{\pflet{$f : X \rightarrow \mathbb{R}$ be continuous}}
     \step{<2>2}{For any $c \in \mathbb{R}$, we have $\inv{f}(c)$ is
$G_\delta$}
%TODO Extract lemma
     \begin{proof}
       \pf\ $\inv{f}(c) = \bigcap_{q \in \mathbb{Q}^+} \inv{f}(c - q, c + q)$
     \end{proof}
     \step{<2>3}{\pflet{$S_{nk} = \{ p \in C_{nk} : f(p) \neq f(p_{nk}) \}$}}
     \step{<2>4}{For all $n$, $k$, we have $S_{nk}$ is countable.}
     \begin{proof}
       \step{<3>1}{\pflet{$\inv{f}(p_{nk}) = \bigcap_{m=1}^\infty U_m$ where
$U_m$
           is open in $X$}}
       \step{<3>2}{For each $m$, \pick\ $B_m \in \mathcal{B}$ such that $p_{nk}
         \in B_m \subseteq U_m$}
       \step{<3>3}{$S_{nk} \subseteq \bigcup_{m=1}^\infty (C_{nk} \setminus
B_m)$}
       \step{<3>4}{Each $C_{nk} \setminus B_m$ is countable}
       \begin{proof}
         \step{<4>1}{\pflet{$m \in \mathbb{Z}$}}
         \step{<4>2}{$B_m$ cannot be of type 1}
         \step{<4>3}{If $B_m$ is of type 2 then $C_{nk} \setminus B_m$ is
finite.}
         \step{<4>4}{If $B_m$ is of type 3 or 4 then $C_{nk} \setminus B_m$ is
           empty.}
       \end{proof}
     \end{proof}
     \step{<2>5}{\pick\ $d \in [-1, 0]$ such that $\mathbb{R} \times \{ d \}$
       intersects none of the sets $S_{nk}$}
     \step{<2>6}{For $n$ odd, we have
       \[ f(n-1, d) = \lim_{k \rightarrow \infty} f(p_{nk})
       \enspace . \]}
     \begin{proof}
       \step{<3>1}{\pflet{$\epsilon > 0$}}
       \step{<3>2}{\pick\ $B \in \mathcal{B}$ such that $(n-1, d) \in B
\subseteq
         \inv{f}(f(n-1,d) - \epsilon, f(n-1,d) + \epsilon)$}
       \step{<3>3}{There exists $\delta > 0$ such that, for $x \in (n-1-\delta,
         n-1+\delta)$, we have $(x, d) \in B$}
       \step{<3>4}{\pick\ $K$ such that $1/K < \delta$}
       \step{<3>5}{\pflet{$k \geq K$}}
       \step{<3>6}{$f(n-1+1/k,d) = f(p_{nk})$}
       \step{<3>7}{$|f(n-1,d) - f(n-1+1/k,d)| < \epsilon$}
       \step{<3>8}{$|f(n-1,d) - f(p_{nk})| < \epsilon$}
     \end{proof}
     \step{<2>7}{For $n$ odd, we have
       \[ f(n+1, d) = \lim_{k \rightarrow \infty} f(p_{nk})
       \enspace . \]}
     \begin{proof}
      \pf\ Similar.
     \end{proof}
     \qedstep
     \begin{proof}
       \step{<3>1}{\assume{$f(a) \neq f(b)$}}
       \step{<3>2}{\assume{w.l.o.g.~$f(a) < f(b)$}}
       \step{<3>3}{\pick\ $B \in \mathcal{B}$ such that $a \in B \subseteq
         \inv{f}(-          \infty, (f(a) + f(b)) / 2)$}
       \step{<3>4}{\pflet{$m$ be even such that $B = \{a\} \cup \{(x,y) \in X :
x
<
m \}$}}
       \step{<3>5}{\pick\ $B \in \mathcal{B}$ such that $b \in B \subseteq
         \inv{f}((f(a) + f(b)) / 2, + \infty)$}
\step{<3>6}{\pflet{$m'$ be even such that $B = \{b\} \cup \{(x,y) \in X : x
>
m' \}$}}
\step{<3>7}{$f(m,d) = f(m', d)$}
\qedstep
     \end{proof}
   \end{proof}
   \step{<1>9}{$X$ is regular.}
   \step{<1>10}{$X$ is not completely regular.}
   \begin{proof}
     \pf\ $a$ and $b$ cannot be separated by a continuous function.
   \end{proof}
   \qed
 \end{proof}

  \begin{thm}[AC]
  A space is completely regular iff it is homeomorphic to a subspace of
$[0,1]^J$ for some $J$.
 \end{thm}

 \begin{proof}
  \pf
  \step{<1>1}{Every completely regular space is homeomorphic to a subspace of
    $[0,1]^J$ for some $J$.}
  \begin{proof}
    \step{<2>1}{\pflet{$X$ be completely regular}}
    \step{<2>2}{For every point $a$ and open set $U$ that contains $a$, \pick\
a
      continuous function $f_{aU}$ that is positive on $a$ and vanishes outside
      $U$}
    \step{<2>3}{The family $\{f_{aU}\}$ separates points from closed sets}
    \qedstep
    \begin{proof}
      \pf\ By the Imbedding Theorem.
    \end{proof}
  \end{proof}
  \step{<1>2}{Every subspace of $[0,1]^J$ is completely regular.}
  \begin{proof}
    \pf\ By Theorem \ref{thm:topology:completely_regular:product} and
Proposition \ref{prop:topology:regular:subspace}.
  \end{proof}
  \qed
 \end{proof}

 \begin{prop}
   The continuous image of a completely regular space is not necessarily completely regular.
 \end{prop}

 \begin{proof}
   \pf\ The identity map from the discrete two-point space to the indiscrete two-point space is continuous. \qed
 \end{proof}

  \section{Normal Spaces}

    \begin{df}[Normal Space]
    A \emph{normal} space is a $T_1$ space such that, for any disjoint closed
sets $A$, $B$, there exist disjoint open sets $U$, $V$ such that $A \subseteq
U$ and $B \subseteq V$.
  \end{df}

    \begin{thm}
   Every linearly ordered set is normal under the order topology.
  \end{thm}

  \begin{proof}
    \pf\ See Steen and Steerbach \emph{Counterexamples in Topology} Example 39.
\qed
  \end{proof}

    \begin{prop}
      \label{prop:topology:normal:S_Omega_times_S_Omega}
    The product space $S_\Omega \times \overline{S_\Omega}$ is not normal.
  \end{prop}

  \begin{proof}
   \pf
   \step{<1>1}{\pflet{$\Delta = \{ (x, x) : x \in \overline{S_\Omega} \}
\subseteq
       \overline{S_\Omega} \times \overline{S_\Omega} \}$}}
   \step{<1>2}{$\Delta$ is closed in $\overline{S_\Omega} \times
     \overline{S_\Omega}$} % TODO Extract lemma
   \step{<1>3}{\pflet{$A = \Delta \cap (S_\Omega \times \overline{S_\Omega})$}}
   \step{<1>4}{$A$ is closed in $S_\Omega \times \overline{S_\Omega}$}
   \step{<1>5}{\pflet{$B = S_\Omega \times \{ \Omega \}$}}
   \step{<1>6}{$B$ is closed}
   \step{<1>7}{$A \cap B = \emptyset$}
   \step{<1>8}{\assume{for a contradiction $U$ and $V$ are disjoint open sets
       including $A$ and $B$ respectively}}
   \step{<1>9}{For all $x \in S_\Omega$ there exists $\beta \in (x, \Omega)$
such
     that $(x, \beta) \notin U$}
   \begin{proof}
     \step{<2>1}{\pflet{$x \in S_\Omega$}}
     \step{<2>2}{$(x, \Omega) \in V$}
     \begin{proof}
       \pf\ $(x, \Omega) \in B \subseteq V$
     \end{proof}
     \step{<2>3}{\pick\ $y < \Omega$ such that $\{ x \} \times (y, \Omega]
       \subseteq V$}
     \begin{proof}
       \pf\ By Lemma \ref{lm:topology:order:open}.
     \end{proof}
     \step{<2>4}{\pick\ $\beta$ such that $x,y < \beta < \Omega$}
     \begin{proof}
       \pf\ Such a $\beta$ exists because $\Omega$ is a limit ordinal.
     \end{proof}
   \end{proof}
   \step{<1>10}{For $x \in S_\Omega$, \pflet{$\beta(x)$ be the least element of
$
       (x, \Omega)$ such that $(x, \beta(x)) \notin U$}}
   \step{<1>11}{\pflet{$b = \sup_{n=1}^\infty \beta^n(0)$}}
   \step{<1>12}{$\beta^n(0) \rightarrow b$ as $n \rightarrow \infty$}
   \step{<1>13}{$(\beta^n(0), \beta^{n+1}(0)) \rightarrow (b, b)$ as $n
     \rightarrow \infty$}
   \step{<1>14}{$(b, b) \in A$}
   \step{<1>15}{$(b, b) \in U$}
   \step{<1>16}{For all $n$ we have $(\beta^n(0), \beta^{n+1}(0)) \notin U$}
   \begin{proof}
     \pf\ By \stepref{<1>10}.
   \end{proof}
   \qedstep
   \begin{proof}
     \pf\ Steps \stepref{<1>12}, \stepref{<1>15} and \stepref{<1>16} form a
     contradiction.
   \end{proof}
   \qed
  \end{proof}

  \begin{cor}
   Not every completely regular space is normal.
  \end{cor}

  \begin{cor}
   An open subspace of a normal space is not necessarily normal.
  \end{cor}

  \begin{cor}
   The product of two normal spaces is not necessarily normal.
  \end{cor}

  \begin{prop}
   A closed subspace of a normal space is normal.
  \end{prop}

  \begin{proof}
   \pf
   \step{<1>1}{\pflet{$X$ be normal and $C \subseteq X$ be closed.}}
   \step{<1>2}{\pflet{$A$ and $B$ be closed in $C$}}
   \step{<1>3}{$A$ and $B$ are closed in $X$}
   \begin{proof}
     \pf\ By Corollary \ref{cor:topology:subspace:closed2}.
   \end{proof}
   \step{<1>4}{\pick\ disjoint open neighbourhoods $U$ and $V$ of $A$ and $B$
in
     $X$}
   \step{<1>5}{$U \cap C$ and $V \cap C$ are disjoint open neighourhoods of $A$
     and $B$ in $C$}
   \qed
  \end{proof}

  \begin{cor}
    Let $\{X_\alpha\}_{\alpha \in J}$ be a family of nonempty spaces. If
    $\prod_{\alpha \in J} X_\alpha$ is normal then each $X_\alpha$ is normal.
  \end{cor}

    \begin{prop}
    If the Continuum Hypothesis then $\mathbb{R}^\omega$ under the box topology
    is normal.
  \end{prop}

  \begin{proof}
    \pf\ See Rudin. The box product of countably many compact metric spaces.
    \emph{General Topology and Its Applications}, 2:293--298, 1972. \qed
  \end{proof}

    \begin{prop}[Stone (DC)]
      \label{prop:topology:normal:uncountable}
    If $J$ is uncountable then $\mathbb{R}^J$ is not normal.
  \end{prop}

  \begin{proof}
   \pf
   \step{<1>1}{\pflet{$X = (\mathbb{Z}^+)^J$} \prove{$X$ is not normal.}}
   \step{<1>2}{For $x \in X$ and $B \subseteq^{\mathrm{fin}} J$, \pflet{
       \[ U(x, B) = \{ y \in X : \forall \alpha \in B. y_\alpha = x_\alpha \}
       \enspace . \]
     }}
   \step{<1>3}{$\{ U(x, B) : x \in X, B \subseteq^{\mathrm{fin}} J \}$ is a
basis
     for $X$}
   \begin{proof}
     \step{<2>1}{\pflet{$x \in X$ and $\prod_{\alpha \in J} U_\alpha$ be a
basic
         open set including $x$, where $U_\alpha = \mathbb{Z}^+$ for all
         $\alpha$ except $\alpha_1, \ldots, \alpha_n$}}
     \step{<2>2}{$x \in U(x, \{ \alpha_1, \ldots, \alpha_n \}) \subseteq
       \prod_{\alpha \in J} U_\alpha$}
   \end{proof}
   \step{<1>4}{For $n \in \mathbb{Z}^+$, \pflet{$P_n = \{ x \in X : x \text{ is
         injective on } J \setminus \inv{x}(n) \}$}}
   \step{<1>5}{$P_1$ and $P_2$ are closed and disjoint.}
   \begin{proof}
     \step{<2>1}{$P_1$ is closed}
     \begin{proof}
       \step{<3>1}{\pflet{$x \in X \setminus P_1$}}
       \step{<3>2}{\pick\ $\alpha, \beta \in J$ such that $x_\alpha = x_\beta
\neq
         1$}
       \step{<3>3}{\pflet{$U_\gamma = \{ x_\alpha \}$ if $\gamma = \alpha$ or
           $\gamma = \beta$, $\mathbb{Z}^+$ for all other $\gamma \in J$}}
       \step{<3>4}{$x \in \prod_{\gamma \in J} U_\gamma \subseteq X \setminus
P_1$}
     \end{proof}
     \step{<2>2}{$P_2$ is closed}
     \begin{proof}
       \pf\ Similar.
     \end{proof}
     \step{<2>3}{$P_1 \cap P_2 = \emptyset$}
     \begin{proof}
       \pf\ If $x \in P_1 \cap P_2$ then $x$ is injective on $J$, contradicting
the fact that $J$ is uncountable.
     \end{proof}
   \end{proof}
   \step{<1>6}{\assume{for a contradiction $U$ and $V$ are disjoint open sets
       including $P_1$ and $P_2$}}
   \step{<1>7}{Given a sequence $(\alpha_i)$ of distinct elements of $J$ and a
     strictly increasing sequence $(n_i)$ of positive integers, \pflet{
       \begin{align*}
         B^{\alpha,n}_i & = \{ \alpha_1, \ldots, \alpha_{n_i} \} \\
         x^{\alpha,n}_i & \in X \\
         (x^{\alpha,n}_i)_\beta & = \begin{cases}
           j & \text{if } \beta = \alpha_j, 1 \leq j \leq n_{i-1} \\
             1 & \text{for all other values of } \beta
           \end{cases}
       \end{align*}
       for $i \geq 1$}}
   \step{<1>8}{\pick\ sequences $(\alpha_i)$, $(n_i)$ such that, for all $i
     \geq 1$, we have $U(x^{\alpha,n}_i, B^{\alpha, n}_i) \subseteq U$}
   \begin{proof}
     \step{<2>1}{\pflet{$x_1 \in X$ be given by $(x_1)_\alpha = 1$ for all
$\alpha
         \in J$}}
     \step{<2>2}{$x_1 \in U$}
     \begin{proof}
       \pf\ $x_1 \in P_1 \subseteq U$
     \end{proof}
     \step{<2>3}{\pick\ $B_1 \subseteq^{\mathrm{fin}} J$ such that $U(x_1, B_1)
       \subseteq U$}
     \begin{proof}
       \pf\ By \stepref{<1>3}.
     \end{proof}
     \step{<2>4}{\pflet{$n_1 = |B_1|$ and $B_1 = \{ \alpha_1, \ldots,
\alpha_{n_1}
         \}$}}
     \step{<2>5}{\assume{We have chosen $n_1$, \ldots, $n_k$ strictly
increasing
         and $\alpha_1$, \ldots, $\alpha_{n_k}$ such that, for $1 \leq i \leq
         k$, we have $U(x^{\alpha, n}_i, B^{\alpha, n}_i) \subseteq U$}}
     \step{<2>6}{$x^{\alpha, n}_{i+1} \in U$}
     \begin{proof}
       \pf\ $x^{\alpha, n}_{i+1} \in P_1 \subseteq U$
     \end{proof}
     \step{<2>7}{\pick\ $C \subseteq^{\mathrm{fin}} J$ such that $U(x^{\alpha,
n}_{i+1},        C) \subseteq U$}
     \step{<2>8}{\pflet{$n_{i+1}$ and $\alpha_{n_i + 1}, \ldots,
\alpha_{n_{i+1}}$
         be such that $B^{\alpha, n}_i \cup C = B^{\alpha, n}_{i+1}$}}
     \step{<2>9}{$U(x^{\alpha, n}_{i+1}, B^{\alpha, n}_{i+1}) \subseteq U$}
   \end{proof}
   \step{<1>9}{\pflet{$A = \{ \alpha_i : i \geq 1 \}$}}
   \step{<1>10}{\pflet{$y \in X$, $y_\beta = j$ if $\beta = \alpha_j$, $y_\beta
=
       2$ for $\beta \notin A$}}
   \step{<1>11}{\pick\ $B$ such that $U(y, B) \subseteq V$}
   \step{<1>12}{\pick\ $i$ such that $A \cap B \subseteq B^{\alpha, n}_i$}
   \step{<1>13}{$U(x^{\alpha, n}_{i+1}, B^{\alpha, n}_{i+1}) \cap U(y, B) \neq
     \emptyset$}
   \begin{proof}
     \pf\ $x^{\alpha, n}_{i+1} \in U(x^{\alpha, n}_{i+1}, B^{\alpha, n}_{i+1})
     \cap U(y, B)$
   \end{proof}
   \qedstep
   \begin{proof}
     \pf\ This contradicts the fact that $U$ and $V$ are disjoint
     (\stepref{<1>6}).
   \end{proof}
   \qed
  \end{proof}

    \begin{thm}[Urysohn Lemma]
   Let $X$ be a normal space. Let $A$ and $B$ be disjoint closed subsets of
   $X$. Then there exists a continuous map $f : X \rightarrow [0, 1]$ such that
   $f(x) = 0$ for all $x \in A$ and $f(x) = 1$ for all $x \in B$.
  \end{thm}

  \begin{proof}
   \pf
   \step{<1>1}{\pflet{$P$ be the set of all rational numbers in $[0, 1]$}}
   \step{<1>2}{For all $q \in P$, \pick\ an open set $U_q$ in $X$ such that $A
     \subseteq U_0$, $U_1 \subseteq X \setminus B$, and whenever $p < q$ then
     $\overline{U_p} \subseteq U_q$}
   \begin{proof}
     \step{<2>1}{\pick\ an enumeration $(q_n)$ of $P$ such that $q_1 = 1$ and
$q_2
       = 0$}
     \step{<2>2}{\pflet{$U_1 = X \setminus B$}}
     \step{<2>3}{\pick\ an open set $U_0$ such that $A \subseteq U_0$ and
       $\overline{U_0} \subseteq U_1$}
     \step{<2>4}{\assume{we have open sets $U_1$, $U_0$, \ldots, $U_{q_n}$ such
         that whenever $p < q$ then $\overline{U_p} \subseteq U_q$}}
     \step{<2>5}{$q_2 < q_{n+1} < q_1$}
     \step{<2>6}{\pflet{$q_k$ be greatest among $q_1$, \ldots, $q_n$ such that
         $q_k < q_{n+1}$, and $q_l$ be least such that $q_{n+1} < q_l$}}
     \step{<2>7}{\pick\ an open set $U_{q_{n+1}}$ such that $\overline{U_{q_k}}
       \subseteq U_{q_{n+1}}$ and $\overline{U_{q_{n+1}}} \subseteq U_{q_l}$}
     \step{<2>8}{For all $p, q \in \{ q_1, \ldots, q_{n+1} \}$, if $p < q$ then
       $\overline{U_p} \subseteq U_q$}
   \end{proof}
   \step{<1>3}{Extend the family $(U_q)$ to $\mathbb{Q}$ by defining: $U_q =
     \emptyset$ if $q < 0$ and $U_q = X$ if $q > 1$}
   \step{<1>4}{For all rationals $p$, $q$ with $p < q$ we have $\overline{U_p}
     \subseteq U_q$}
   \step{<1>5}{Define $f : X \rightarrow [0, 1]$ by $f(x) = \inf \{ q \in
     \mathbb{Q} : x \in U_q \}$}
   \begin{proof}
     \pf\ This set is nonempty since $x \in U_1$ and bounded below since if $x
     \in U_q$ then $q \geq 0$.
   \end{proof}
   \step{<1>6}{For all $x \in A$ we have $f(x) = 0$}
   \step{<1>7}{For all $x \in B$ we have $f(x) = 1$}
   \step{<1>8}{If $x \in \overline{U_r}$ then $f(x) \leq r$}
   \step{<1>9}{If $x \notin U_r$ then $f(x) \geq r$}
   \step{<1>10}{$f$ is continuous}
   \begin{proof}
     \step{<2>1}{\pflet{$x_0 \in X$}}
     \step{<2>2}{\pflet{$(c, d)$ be an open interval containing $f(x_0)$}
       \prove{There exists a neighbourhood $U$ of $x_0$ such that $f(U)
         \subseteq (c,   d)$}}
     \step{<2>3}{\pick\ rationals $p$, $q$ such that $c < p < f(x_0) < q < d$}
     \step{<2>4}{$x \notin \overline{U_p}$}
     \begin{proof}
       \pf\ By \stepref{<1>8}
     \end{proof}
     \step{<2>5}{$x \in U_q$}
     \begin{proof}
       \pf\ By \stepref{<1>9}
     \end{proof}
     \step{<2>6}{\pflet{$U = U_q \setminus \overline{U_p}$}}
   \end{proof}
   \qed
  \end{proof}

   \begin{df}[Vanish Precisely]
  Let $X$ be a set and $A \subseteq X$. Let $f : X \rightarrow [0,1]$. Then $f$
  \emph{vanishes precisely} on $A$ iff $\inv{f}(0) = A$.
 \end{df}

  \begin{thm}[CC]
    \label{thm:topology:normal:vanishes_precisely}
  Let $X$ be a normal space and $A \subseteq X$. Then there exists a continuous
function $f : X \rightarrow [0,1]$ such that $f$ vanishes precisely on $A$ if
and only if $A$ is a closed $G_\delta$ set.
 \end{thm}

 \begin{proof}
  \pf
  \step{<1>1}{If there exists $f$ such that $f$ vanishes precisely on $A$ then
$A$
    is closed.}
  \begin{proof}
    \pf\ This holds because $A = \inv{f}(0)$.
  \end{proof}
  \step{<1>2}{If there exists $f$ such that $f$ vanishes precisely on $A$ then
$A$
    is $G_\delta$.}
  \begin{proof}
    \pf\ This holds because $A = \bigcap_{q \in \mathbb{Q}^+} \inv{f}([0, q))$.
  \end{proof}
  \step{<1>3}{If $A$ is closed and $G_\delta$ then there exists $f$ that
vanishes
    precisely on $A$.}
  \begin{proof}
    \step{<2>1}{\pflet{$A = \bigcap_{n=1}^\infty U_n$}}
    \step{<2>2}{For $n \geq 1$, \pick\ $f_n : X \rightarrow [0,1 / 2^n]$ such
that
      $f(x) = 0$ for $x \in A$ and $f(x) = 1 / 2^n$ for $x \in X \setminus
      U_n$}
    \begin{proof}
      \pf\ By the Urysohn Lemma.
    \end{proof}
    \step{<2>3}{\pflet{$f : X \rightarrow [0,1]$ be given by $f(x) =
        \sum_{n=1}^\infty f_n(x)$}}
    \begin{proof}
      \pf\ The series converges for every $x$ by the Comparison Test.
    \end{proof}
    \step{<2>4}{$f$ is continuous}
    \begin{proof}
      \step{<3>1}{$f_n$ converges uniformly to $f$}
      \begin{proof}
        \pf\ By the Weierstrass M-test.
      \end{proof}
      \qedstep
      \begin{proof}
        \pf\ By the Uniform Limit Theorem.
      \end{proof}
    \end{proof}
    \step{<2>5}{$f(x) = 0$ for $x \in A$}
    \begin{proof}
      \pf\ From \stepref{<2>2}.
    \end{proof}
    \step{<2>6}{$f(x) > 0$ for $x \notin A$}
    \begin{proof}
      \step{<3>1}{\pflet{$x \notin A$}}
      \step{<3>2}{\pick\ $N$ such that $x \notin U_N$}
      \qedstep
      \begin{proof}
        \pf
        \begin{align*}
          f(x) & = \sum_{n=1}^\infty f_n(x) & (\text{\stepref{<2>3}}) \\
          & \geq f_N(x) \\
          & > 0 & (\text{\stepref{<2>2}})
        \end{align*}
      \end{proof}
    \end{proof}
  \end{proof}
  \qed
 \end{proof}

  \begin{thm}[Strong Form of Urysohn Lemma]
  Let $X$ be a normal space. Then there exists a continuous function $f : X
  \rightarrow [0,1]$ such that $\inv{f}(0) = A$ and $\inv{f}(1) = B$ if and
only if $A$ and $B$ are disjoint, closed and $G_\delta$.
 \end{thm}

 \begin{proof}
  \pf
  \step{<1>1}{If there exists a continuous function $f : X
    \rightarrow [0,1]$ such that $\inv{f}(0) = A$ and $\inv{f}(1) = B$ then $A$
    and $B$ are disjoint, closed and $G_\delta$}
  \begin{proof}
    \step{<2>1}{\assume{there exists a continuous function $f : X
        \rightarrow [0,1]$ such that $\inv{f}(0) = A$ and $\inv{f}(1) = B$}}
    \step{<2>2}{$A$ and $B$ are disjoint}
    \step{<2>3}{$A$ is closed and $G_\delta$}
    \begin{proof}
      \pf\ By Theorem \ref{thm:topology:normal:vanishes_precisely}.
    \end{proof}
    \step{<2>4}{$B$ is closed and $G_\delta$}
    \begin{proof}
      \pf\ Apply Theorem \ref{thm:topology:normal:vanishes_precisely} to $1-f$.
    \end{proof}
  \end{proof}
  \step{<1>2}{If $A$ and $B$ are disjoint, closed and $G_\delta$ then there
exists
    a continuous function $f : X
    \rightarrow [0,1]$ such that $\inv{f}(0) = A$ and $\inv{f}(1) = B$}
  \begin{proof}
    \step{<2>1}{\assume{$A$ and $B$ are disjoint, closed and $G_\delta$}}
    \step{<2>2}{\pick\ $g : X \rightarrow [0,1]$ that vanishes precisely on $A$
      and $h : X \rightarrow [0,1]$ that vanishes precisely on $B$}
    \step{<2>3}{\pflet{$f = g / (g + h)$}}
  \end{proof}
  \qed
 \end{proof}

  \begin{df}[Universal Extension Property]
   A topological space $Y$ has the \emph{universal extension property} iff, for
every normal space $X$ and closed subspace $A$ of $X$, every continuous
function $A
\rightarrow Y$ can be extended to a continuous function $X \rightarrow Y$.
 \end{df}

  \begin{thm}[Tietze Extension Theorem (DC)]
  Let $X$ be a normal space. Let $A$ be closed subspace of $X$.
  \begin{enumerate}
   \item Any continuous function $A \rightarrow [a,b]$ can be extended to a
continuous function $X \rightarrow [a,b]$.
\item Any continuous function $A \rightarrow \mathbb{R}$ can be extend to a
continuous function $X \rightarrow \mathbb{R}$.
  \end{enumerate}
 \end{thm}

 \begin{proof}
  \pf
  \step{<1>1}{Any continuous function $A \rightarrow [-1,1]$ can be extended to
a
    continuous function $X \rightarrow [-1,1]$}
  \begin{proof}
    \step{<2>1}{For every continuous function $f : A \rightarrow [-r, r]$,
there
      exists a continuous $g : X \rightarrow \mathbb{R}$ such that
      \begin{align*}
        |g(x)| & \leq \frac{1}{3} r & (x \in X) \\
        |g(x) - f(x)| & \leq \frac{2}{3} r & (x \in A)
      \end{align*}
    }
    \begin{proof}
      \step{<3>1}{\pflet{$f : A \rightarrow [-r, r]$ be continuous}}
      \step{<3>2}{\pflet{$I_1 = [-r, -\frac{1}{3}r]$}}
      \step{<3>3}{\pflet{$I_2 = [-\frac{1}{3}r, \frac{1}{3}r ]$}}
      \step{<3>4}{\pflet{$I_3 = [\frac{1}{3}r, r]$}}
      \step{<3>5}{\pflet{$B = \inv{f}(I_1)$}}
      \step{<3>6}{\pflet{$C = \inv{f}(I_3)$}}
      \step{<3>7}{\pick\ a continuous $g : X \rightarrow [-\frac{1}{3}r,
        \frac{1}{3} r]$ such that $g(x) = - \frac{1}{3} r$ for $x \in B$ and
        $g(x) = \frac{1}{3}r$ for $x \in C$}
      \begin{proof}
        \pf\ By the Urysohn Lemma, since $B$ and $C$ are closed disjoint
        subsets of $X$.
      \end{proof}
      \step{<3>8}{For all $x \in A$ we have $|g(x) - f(x)| \leq \frac{2}{3} r$}
      \begin{proof}
        \step{<4>1}{\pflet{$x \in A$}}
        \step{<4>2}{\case{$f(x) \in I_1$}}
        \begin{proof}
          \pf
          \begin{align*}
            |g(x) - f(x)| & = \left| - \frac{1}{3} r - f(x) \right| & (x \in
B) \\
            & \leq \frac{2}{3} r & (f(x) \in I_1)
          \end{align*}
        \end{proof}
        \step{<4>3}{\case{$f(x) \in I_2$}}
        \begin{proof}
          \pf\ In this case, $|g(x) - f(x)| \leq \frac{2}{3} r$ since $f(x),
          g(x) \in I_2$.
        \end{proof}
        \step{<4>4}{\case{$f(x) \in I_3$}}
        \begin{proof}
          \pf
          \begin{align*}
            |g(x) - f(x)| & = \left| \frac{1}{3} r - f(x) \right| & (x \in
C) \\
            & \leq \frac{2}{3} r & (f(x) \in I_3)
          \end{align*}
        \end{proof}
      \end{proof}
    \end{proof}
    \step{<2>2}{\pflet{$f : A \rightarrow [-1, 1]$ be continuous.}}
    \step{<2>3}{\pick\ a sequence of functions $(g_n)$ such that
      \begin{align*}
        |g_n(x)| & \leq \frac{1}{3} \left( \frac{2}{3} \right)^{n-1} & (x \in
X) \\
        |f(x) - g_1(x) - \cdots - g_n(x)| & \leq ( 2/3 )^n & (x \in A)
      \end{align*}
    }
    \begin{proof}
      \pf Given $g_1$, \ldots, $g_n$, we apply \stepref{<2>1} with $f = f - g_1
- \cdots - g_n$ and $r = (2/3)^n$.
    \end{proof}
    \step{<2>4}{\pflet{$g(x) = \sum_{n=1}^\infty g_n(x)$ for $x \in X$}}
    \begin{proof}
      \pf\ This series converges by the Comparison Test since
$\sum_{n=1}^\infty (2/3)^n$ converges.
    \end{proof}
    \step{<2>5}{$g$ is continuous.}
    \begin{proof}
      \step{<3>1}{$\sum_{n=1}^N g_n$ converges to $g$ uniformly}
      \begin{proof}
        \pf\ By the Weierstrass $M$-test.
      \end{proof}
      \qedstep
      \begin{proof}
        \pf\ By the Uniform Limit Theory.
      \end{proof}
    \end{proof}
    \step{<2>6}{For all $x \in A$ we have $g(x) = f(x)$}
    \begin{proof}
      \pf\ $|\sum_{n=1}^N g_n(x) - f(x)| \leq (2/3)^N \rightarrow 0$ as $N
\rightarrow \infty$.
    \end{proof}
    \step{<2>7}{For all $x \in X$ we have $-1 \leq g(x) \leq 1$}
    \begin{proof}
      \pf
      \begin{align*}
        |\sum_{n=1}^N g_n(x)| & \leq \sum_{n=1}^N |g_n(x) | \\
        & \leq 1/3 \sum_{n=1}^N (2/3)^{n-1} \\
        & \rightarrow 2/3 & \text{as } n \rightarrow \infty
      \end{align*}
    \end{proof}
  \end{proof}
  \step{<1>2}{Any continuous function $A \rightarrow (-1,1)$ can be extend to
    a continuous function $X \rightarrow (-1,1)$}
  \begin{proof}
    \step{<2>1}{\pflet{$f : A \rightarrow (-1, 1)$ be continuous}}
    \step{<2>2}{\pick\ a continuous $g : X \rightarrow [-1, 1]$ that extends
      $f$}
    \begin{proof}
      \pf\ By \stepref{<1>1}.
    \end{proof}
    \step{<2>3}{\pflet{$D = \inv{g}(-1) \cup \inv{g}(1)$}}
    \step{<2>4}{$D$ is closed in $X$}
    \begin{proof}
      \pf\ Since $g$ is continuous and $\{-1\}$, $\{1\}$ are closed in $[-1,1]$.
    \end{proof}
    \step{<2>5}{$D \cap A = \emptyset$}
    \begin{proof}
      \pf\ Since $g(A) = f(A) \subseteq (-1, 1)$.
    \end{proof}
    \step{<2>6}{\pick\ a continuous $\phi : X \rightarrow [0,1]$ such that
      $\phi(D) = \{ 0 \}$ and $\phi(A) = \{ 1 \}$}
    \begin{proof}
      \pf\ By the Urysohn Lemma.
    \end{proof}
    \step{<2>7}{\pflet{$h = g \phi$}}
    \step{<2>8}{$h$ is continuous}
    \step{<2>9}{$h$ extends $f$}
    \step{<2>10}{$\im h \subseteq (-1, 1)$}
  \end{proof}
  \qedstep
  \begin{proof}
    \pf\ The result follows because any closed interval in $\mathbb{R}$ is
    homeomorphic to $[-1,1]$ and $\mathbb{R} \cong (-1,1)$.
  \end{proof}
  \qed
 \end{proof}

  \begin{lm}[Shrinking Lemma (AC)]
   Let $X$ be a normal space. Let $\{ U_\alpha \}_{\alpha \in J}$ be a
point-finite indexed open covering of $X$. Then there exists an indexed open
covering $\{ V_\alpha \}_{\alpha \in J}$ such that $\overline{V_\alpha}
\subseteq U_\alpha$ for all $\alpha \in J$.
 \end{lm}

 \begin{proof}
  \pf
  \step{<1>1}{\pick\ a well-ordering $\prec$ on $J$}
  \step{<1>2}{\pick\ open sets $V_\alpha$ for $\alpha \in J$ such that
$A_\alpha
    \subseteq V_\alpha$ and $\overline{V_\alpha} \subseteq U_\alpha$, where
    \[ A_\alpha = X \setminus \bigcup_{\beta \prec \alpha} V_\beta \cup
    \bigcup_{\alpha \prec \beta} U_\beta \]}
  \begin{proof}
    \pf\ Apply transfinite induction to Proposition
    \ref{prop:topology:normal:shrinking}.
  \end{proof}
  \step{<1>3}{$\{ V_\alpha \}_{\alpha \in J}$ covers $X$}
  \begin{proof}
    \step{<2>1}{\pflet{$x \in X$}}
    \step{<2>2}{\pflet{$\alpha_1, \ldots, \alpha_n$ be the elements of $J$ such
        that $x \in U_{\alpha_i}$, where $\alpha_1 \prec \cdots \prec
\alpha_n$} \prove{$x \in V_{\alpha_i}$ for some $i$}}
    \step{<2>3}{\assume{$x \notin V_{\alpha_1}, \ldots, V_{\alpha_{n-1}}$}}
    \step{<2>4}{$x \in A_{\alpha_n}$}
    \step{<2>5}{$x \in V_{\alpha_n}$}
  \end{proof}
  \qed
 \end{proof}

  \begin{prop}[DC]
   $S_\Omega \times \overline{S_\Omega}$ is not normal.
 \end{prop}

 \begin{proof}
  \pf
  \step{<1>1}{\pflet{$\Delta = \{ (x, x) : x \in \overline{S_\Omega} \}$}}
  \step{<1>2}{$\Delta$ is closed in $\overline{S_\Omega}^2$}
  \begin{proof}
    \step{<2>1}{\pflet{$(x, y) \in \overline{S_\Omega}^2 \setminus \Delta$}}
    \step{<2>2}{\pick\ disjoint open sets $U$, $V$ such that $x \in U$ and $y
\in
      V$}
    \step{<2>3}{$(x, y) \in U \times V \subseteq \overline{S_\Omega}^2
\setminus
      \Delta$}
  \end{proof}
  \step{<1>3}{\pflet{$A = \Delta \cap (S_\Omega \times \overline{S_\Omega})$}}
  \step{<1>4}{$A$ is closed in $S_\Omega \times \overline{S_\Omega}$}
  \step{<1>5}{\pflet{$B = S_\Omega \times \{ \Omega \}$}}
  \step{<1>6}{$B$ is closed in $S_\Omega \times \overline{S_\Omega}$}
  \step{<1>7}{$A \cap B = \emptyset$}
  \step{<1>8}{\assume{for a contradiction $U$ and $V$ are disjoint open sets
      including $A$ and $B$ respectively}}
  \step{<1>9}{\pick\ a sequence $x_n$ in $S_\Omega$ such that $x_n < x_{n+1} <
    \Omega$ and $(x_n, x_{n+1}) \notin U$ for all $n$}
  \begin{proof}
    \step{<2>1}{\pflet{$x_n \in S_\Omega$}}
    \step{<2>2}{$(x_n, \Omega) \in V$}
    \step{<2>3}{\pick\ open sets $W \subseteq S_\Omega$, $X \subseteq
      \overline{S_\Omega}$ such that $x_n \in W$, $\Omega \in X$       and $W
      \times X \subseteq V$}
    \step{<2>4}{\pick\ $y < \Omega$ such that $(x_{n+1}, \Omega] \subseteq
      X$}
    \step{<2>5}{\pflet{$x_{n+1} = y + 1$}}
  \end{proof}
  \step{<1>10}{\pflet{$b$ be the supremum of $\{ x_n : n \geq 1 \}$}}
  \step{<1>11}{$(x_n, x_{n+1}) \rightarrow (b, b)$ as $n \rightarrow \infty$}
  \step{<1>12}{$(b, b) \in A$}
  \step{<1>13}{$(b, b) \in U$}
  \step{<1>14}{For all $n$ we have $(x_n, x_{n+1}) \notin U$}
  \qed
 \end{proof}

  \begin{prop}[AC]
   $\mathbb{R}_l$ is normal.
 \end{prop}

 \begin{proof}
  \pf
  \step{<1>1}{\pflet{$A$ and $B$ be disjoint closed sets in $\mathbb{R}_l$}}
  \step{<1>2}{For $a \in A$, \pick\ $x_a > a$ such that $[a, x_a)$ not
intersecting $B$}
  \step{<1>3}{For $b \in B$, \pick\ $x_b > b$ such that $[b, x_b)$ does not
    intersect $A$}
  \step{<1>4}{\pflet{$U = \bigcup_{a \in A} [a, x_a)$ and $V = \bigcup_{b \in B}
      [b, x_b)$}}
  \step{<1>5}{$U$ and $V$ are disjoint open sets including $A$ and $B$
    respectively.}
  \qed
 \end{proof}

 \begin{lm}
   \label{lm:closed_sorgenfrey}
  The set $L = \{ (x,-x) ; x \in \mathbb{R} \}$ as a subspace of
      $\mathbb{R}_l^2$ is closed
 \end{lm}

   \begin{proof}
    \step{<1>1}{\pflet{$(x, y) \notin L$, so $y \neq -x$} \prove{There exists a
        neighbourhood $U$ of $(x,y)$ that does not intersect $L$}}
    \step{<1>2}{\case{$y > -x$}}
    \begin{proof}
      \pf\ In this case, take $U = [x,+\infty) \times [y, + \infty)$
    \end{proof}
    \step{<1>3}{\case{$y < -x$}}
    \begin{proof}
      \pf\ In this case, take $U = [x,(x-y)/2) \times [y,(y-x)/2)$.
    \end{proof}
  \end{proof}

  \begin{prop}[AC]
  The Sorgenfrey plane is not normal.
 \end{prop}

 \begin{proof}
  \pf
  \step{<1>1}{\assume{for a contradiction the Sorgenfrey plane is normal.}}
  \step{<1>2}{\pflet{$L = \{ (x,-x) ; x \in \mathbb{R} \}$ as a subspace of
      $\mathbb{R}_l^2$}}
  \step{<1>3}{$L$ has the discrete topology.}
  \begin{proof}
    \step{<2>1}{\pflet{$(x, -x) \in L$} \prove{$\{ (x,-x) \}$ is open in $L$}}
    \step{<2>2}{$\{(x,-x)\} = ([x,+\infty) \times [-x,+\infty)) \cap L$}
  \end{proof}
  \step{<1>4}{Every subset of $L$ is closed in $\mathbb{R}_l^2$}
  \begin{proof}
    \pf\ By Corollary \ref{cor:topology:subspace:closed2}.
  \end{proof}
  \step{<1>5}{For every nonempty proper subset $A$ of $L$, \pick\ disjoint
    open sets $U_A$, $V_A$ containing $A$ and $L \setminus A$}
  \begin{proof}
    \pf\ By \stepref{<1>1} and \stepref{<1>4}.
  \end{proof}
  \step{<1>6}{\pflet{$D = \mathbb{Q}^2$}}
  \step{<1>7}{$D$ is dense in $\mathbb{R}_l^2$}
  \begin{proof}
    \pf\ Given any basic open set $[a,b) \times [c,d)$, pick rationals $q$, $r$
such that $a \leq q < b$ and $c \leq r < d$. Then $(q,r) \in ([a,b) \times
[c,d)) \cap D$
  \end{proof}
  \step{<1>8}{\pflet{$\theta : \mathcal{P} L \rightarrow \mathcal{P} D$ be the
function
\begin{align*}
 \theta(A) & = U_A \cap D & (\emptyset \neq A \neq L) \\
 \theta(\emptyset) & = \emptyset \\
 \theta(L) & = D
\end{align*}}}
\step{<1>9}{$\theta$ is injective}
\begin{proof}
  \step{<2>1}{\pflet{$A, B \subseteq L$ with $\theta(A) = \theta(B)$} \prove{$A
=
      B$}}
  \step{<2>2}{\case{$\emptyset \neq A \neq L$ and $\emptyset \neq B \neq L$}}
  \begin{proof}
    \step{<3>1}{$A \subseteq B$}
    \begin{proof}
      \step{<4>1}{\pflet{$x \in A$}}
      \step{<4>2}{$x \in U_A$}
      \begin{proof}
        \pf\ By \stepref{<1>5}
      \end{proof}
      \step{<4>3}{$x \in U_B$}
      \begin{proof}
        \pf\ By \stepref{<2>1}
      \end{proof}
      \step{<4>4}{$x \notin L \setminus B$}
      \begin{proof}
        \pf\ By \stepref{<1>5}
      \end{proof}
      \step{<4>5}{$x \in B$}
      \begin{proof}
        \pf\ Since $x \in L$ by \stepref{<4>1}
      \end{proof}
    \end{proof}
    \step{<3>2}{$B \subseteq A$}
    \begin{proof}
      \pf\ Similar.
    \end{proof}
  \end{proof}
  \step{<2>3}{\case{$\emptyset \neq A \neq L$ and $B = \emptyset$}}
  \begin{proof}
    \pf\ This implies $U_A \cap D = \emptyset$ which contradicts the fact that
    $D$ is dense.
  \end{proof}
  \step{<2>4}{\case{$\emptyset \neq A \neq L$ and $B = L$}}
  \begin{proof}
    \pf\ This implies $V_A \cap D = \emptyset$ which contradicts the fact that
    $D$ is dense.
  \end{proof}
  \step{<2>5}{\case{$A = B = \emptyset$}}
  \begin{proof}
    \pf\ Trivial
  \end{proof}
  \step{<2>6}{\case{$A = \emptyset$ and $B = L$}}
  \begin{proof}
    \pf\ This implies $D = \emptyset$ which is a contradiction.
  \end{proof}
  \step{<2>7}{\case{$A = B = L$}}
  \begin{proof}
    \pf\ Trivial
  \end{proof}
\end{proof}
\qedstep
\begin{proof}
  \pf\ This is a contradiction since $D$ is countable and $L$ is uncountable.
\end{proof}
  \qed
 \end{proof}

 \begin{prop}
   The continuous image of a normal space is not necessarily normal.
 \end{prop}

 \begin{proof}
   \pf\ The identity map from the discrete two-point space to the indiscrete two-point space is continuous. \qed
 \end{proof}

  \section{Completely Normal Spaces}

    \begin{df}[Completely Normal]
    A space $X$ is \emph{completely normal} iff every subspace is normal.
  \end{df}

      \begin{prop}
   A subspace of a completely normal space is completely normal.
  \end{prop}

  \begin{proof}
   \pf\ Immediate from definitions. \qed
  \end{proof}

    \begin{prop}
    \label{prop:topology:completely_normal:characterisation}
   Let $X$ be a topological space. Then $X$ is completely normal iff $X$ is
$T_1$ and, for any pair of separated sets $A$, $B$ in $X$, there exist disjoint
open sets including them.
  \end{prop}

  \begin{proof}
   \pf
   \step{<1>1}{If $X$ is completely normal then $X$ is
     $T_1$ and, for any pair of separated sets $A$, $B$ in $X$, there exist
     disjoint open sets including them.}
   \begin{proof}
     \step{<2>1}{\assume{$X$ is completely normal.}}
     \step{<2>2}{$X$ is $T_1$}
     \begin{proof}
       \pf\ Holds because $X$ is normal.
     \end{proof}
     \step{<2>3}{For any pair of separated sets $A$, $B$ in $X$, there exist
       disjoint open sets including them.}
     \begin{proof}
       \step{<3>1}{\pflet{$A$ and $B$ be separated in $X$}}
       \step{<3>2}{\pflet{$Y = X \setminus (\overline{A} \cap \overline{B})$}}
       \step{<3>3}{\pick\ disjoint open sets $U$, $V$ in $Y$ such that
         $\overline{A} \cap Y \subseteq U$ and $\overline{B} \cap Y \subseteq
         V$}
       \begin{proof}
         \pf\ $Y$ is normal by \stepref{<2>1}.
       \end{proof}
       \step{<3>4}{\pick\ open sets $U_0$, $V_0$ in $X$ such that $U = U_0 \cap
         Y$, $V = V_0 \cap Y$}
       \step{<3>5}{$A \subseteq U_0 \setminus \overline{B}$ and $B \subseteq
V_0
         \setminus \overline{A}$}
       \begin{proof}
         \pf\ Using \stepref{<3>1}.
       \end{proof}
     \end{proof}
   \end{proof}
   \step{<1>2}{If $X$ is $T_1$ and, for any pair of separated sets $A$, $B$ in
     $X$, there exist disjoint open sets including them, then $X$ is completely
     normal.}
   \begin{proof}
     \step{<2>1}{\assume{$X$ is $T_1$ and, for any pair of separated sets $A$,
$B$
         in $X$, there exist disjoint open sets including them}}
     \step{<2>2}{\pflet{$Y \subseteq X$}}
     \step{<2>3}{$Y$ is $T_1$}
     \begin{proof}
       \pf\ By Proposition \ref{prop:topology:T1:subspace}.
     \end{proof}
     \step{<2>4}{\pflet{$A$ and $B$ be disjoint closed sets in $Y$}}
     \step{<2>5}{$A$ and $B$ are separated in $X$}
     \begin{proof}
       \step{<3>1}{$\overline{A} \cap Y = A$ and $\overline{B} \cap Y = B$}
       \begin{proof}
         \pf\ By Proposition \ref{prop:topology:closure:closed2} and Theorem
         \ref{thm:topology:subspace:closure}.
       \end{proof}
       \step{<3>2}{$\overline{A} \cap B = \emptyset$}
       \begin{proof}
         \begin{align*}
           \overline{A} \cap B & = \overline{A} \cap \overline{B} \cap Y &
           (\text{\stepref{<3>1}}) \\
           & = A \cap B & (\text{\stepref{<3>1}}) \\
           & = \emptyset & (\text{\stepref{<2>4}})
         \end{align*}
       \end{proof}
       \step{<3>3}{$A \cap \overline{B} = \emptyset$}
       \begin{proof}
         \pf\ Similar.
       \end{proof}
     \end{proof}
     \step{<2>6}{\pick\ disjoint open sets $U$ and $V$ that include $A$ and $B$
       respectively.}
     \begin{proof}
       \pf\ By \stepref{<2>1}.
     \end{proof}
     \step{<2>7}{$U \cap Y$ and $V \cap Y$ are disjoint open sets in $Y$ that
       include $A$ and $B$ respectively.}
   \end{proof}
   \qed
  \end{proof}

  \begin{prop}
    \label{prop:topology:completely_normal:well_ordered}
   A well-ordered set in the order topology is completely normal.
  \end{prop}

  \begin{proof}
   \pf
   \step{<1>1}{\pflet{$X$ be a well-ordered set.}}
   \step{<1>2}{For all $a, b \in X$ with $a < b$, we have $(a, b]$ is open.}
   \begin{proof}
     \step{<2>1}{\case{$b$ is greatest in $X$}}
     \begin{proof}
       \pf\ This case holds by the definition of the order topology.
     \end{proof}
     \step{<2>2}{\case{$b$ is not greatest in $X$}}
     \begin{proof}
       \pf\ In this case, $(a, b] = (a, c)$ where $c$ is the successor of $b$.
     \end{proof}
   \end{proof}
   \step{<1>3}{\pflet{$A$ and $B$ be separated sets in $X$} \prove{There exist
       disjoint open sets $U$, $V$ including $A$ and $B$}}
   \step{<1>4}{\case{The least element of $X$ is not in $A$ or $B$}}
   \begin{proof}
     \step{<2>1}{\pflet{$U = \bigcup \{ (x, a] : a \in A, x < a, (x, a] \cap B
=
         \emptyset \}$}}
     \step{<2>2}{\pflet{$V = \bigcup \{ (y, b] : b \in B, y < b, (y, b] \cap A
=
         \emptyset \}$}}
     \step{<2>3}{$U$ is open}
     \begin{proof}
       \pf\ From \stepref{<1>2}.
     \end{proof}
     \step{<2>4}{$V$ is open}
     \begin{proof}
       \pf\ From \stepref{<1>2}.
     \end{proof}
     \step{<2>5}{$A \subseteq U$}
     \begin{proof}
       \step{<3>1}{\pflet{$a \in A$}}
       \step{<3>2}{\pick\ $W$ a neighbourhood of $a$ such that $W \cap B =
         \emptyset$}
       \begin{proof}
         \pf\ By \stepref{<1>3}.
       \end{proof}
       \step{<3>3}{\pick\ $x < a$ such that $(x, a] \subseteq W$}
       \begin{proof}
         \pf\ By Lemma \ref{lm:topology:order:open}
       \end{proof}
       \step{<3>4}{$a \in (x, a] \subseteq U$}
     \end{proof}
     \step{<2>6}{$B \subseteq V$}
     \begin{proof}
       \pf\ Similar.
     \end{proof}
     \step{<2>7}{$U \cap V = \emptyset$}
   \end{proof}
   \step{<1>5}{\case{$\bot \in A$}}
   \begin{proof}
     \step{<2>1}{\pick\ disjoint open sets $U$ and $V$ that include $A \setminus
       \{ \bot \}$ and $B$}
     \begin{proof}
       \pf\ From \stepref{<1>4}.
     \end{proof}
     \step{<2>2}{$U \cup \{ \bot \}$ and $V$ are disjoint open sets that include
       $A$ and $B$}
     \begin{proof}
       \pf\ $\{ \bot \}$ is open because it is $(- \infty, a)$ where $a$ is the
       successor of $\bot$.
     \end{proof}
   \end{proof}
   \qedstep
   \begin{proof}
     \pf\ By Proposition \ref{prop:topology:completely_normal:characterisation}.
   \end{proof}
   \qed
  \end{proof}

    \begin{prop}
   The product of two completely normal spaces is not necessarily completely
   normal.
  \end{prop}

  \begin{proof}
   \pf
   \step{<1>1}{$S_\Omega$ is completely normal.}
   \begin{proof}
     \pf\ By Proposition \ref{prop:topology:completely_normal:well_ordered}
   \end{proof}
   \step{<1>2}{$\overline{S_\Omega}$ is completely normal.}
   \begin{proof}
     \pf\ By Proposition \ref{prop:topology:completely_normal:well_ordered}
   \end{proof}
   \step{<1>3}{$S_\Omega \times \overline{S_\Omega}$ is not completely normal.}
   \begin{proof}
     \pf\ By Proposition \ref{prop:topology:normal:S_Omega_times_S_Omega}.
   \end{proof}
   \qed
  \end{proof}

    \begin{prop}
   A compact Hausdorff space is not necessarily completely normal.
  \end{prop}

  \begin{proof}
    \pf
    \step{<1>1}{\pick\ an uncountable set $J$}
    \step{<1>2}{$[0,1]^J$ is compact Hausdorff}
    \begin{proof}
      \pf\ By Tychonoff's Theorem and Theorem
      \ref{thm:topology:Hausdorff:product}.
    \end{proof}
    \step{<1>3}{$(0, 1)^J$ is not normal.}
    \begin{proof}
      \pf\ By Proposition \ref{prop:topology:normal:uncountable}, since $(0, 1)
\cong \mathbb{R}$.
    \end{proof}
    \qed
  \end{proof}

    \begin{prop}
    The space $\mathbb{R}_l$ is completely normal.
  \end{prop}

  \begin{proof}
   \pf
   \step{<1>1}{\pflet{$X \subseteq \mathbb{R}_l$}}
   \step{<1>2}{\pflet{$A$ and $B$ be disjoint closed sets in $X$.}}
   \step{<1>3}{\pick\ closed sets $C$ and $D$ such that $A = C \cap X$ and $B =
D
     \cap X$}
   \step{<1>4}{For $a \in A$, \pick\ $x_a > a$ such that $[a, x_a) \cap D =
     \emptyset$}
   \step{<1>5}{For $b \in B$, \pick\ $x_b > b$ such that $[b, x_b) \cap C =
     \emptyset$}
   \step{<1>6}{$\bigcup_{a \in A} [a, x_a) \cap X$ and $\bigcup_{b \in B} [b,
x_b)
     \cap X$ are disjoint open sets in $X$ that include $A$ and $B$}
   \qed
  \end{proof}

  \section{Perfectly Normal Spaces}

   \begin{df}[Perfectly Normal]
   A space is \emph{perfectly normal} iff it is normal and every closed set is
   $G_\delta$.
 \end{df}

  \begin{prop}
  Every perfectly normal space is completely normal.
 \end{prop}

 \begin{proof}
  \pf
  \step{<1>1}{\pflet{$X$ be perfectly normal.}}
  \step{<1>2}{\pflet{$A$ and $B$ be separated sets in $X$}}
  \step{<1>3}{\pick\ continuous functions $f, g : X \rightarrow [0, 1]$ that
    vanish precisely on $\overline{A}$ and $\overline{B}$, respectively.}
  \begin{proof}
    \pf\ By Theorem \ref{thm:topology:normal:vanishes_precisely}.
  \end{proof}
  \step{<1>4}{\pflet{$h = f - g$}}
  \step{<1>5}{$B \subseteq \inv{h}((0, + \infty))$ and $A
    \subseteq \inv{h}((-\infty, 0))$ }
  \qedstep
  \begin{proof}
    \pf\ By Proposition \ref{prop:topology:completely_normal:characterisation}.
  \end{proof}
  \qed
 \end{proof}

  \begin{prop}
   The space $\overline{S_\Omega}$ is not perfectly normal.
 \end{prop}

 \begin{proof}
   \pf\ The set $\{ \Omega \}$ is not $G_\delta$. \qed
 \end{proof}

  \chapter{Countability Axioms}

  \section{The First Countability Axiom}

  \begin{df}[First Countability Axiom]
    A topological space $X$ satisfies the \emph{first countability axiom}, or
    is \emph{first countable}, iff every point has a countable local basis.
  \end{df}

   \begin{prop}
  $S_\Omega$ is first countable.
 \end{prop}

 \begin{proof}
   \pf\ For every countable ordinal $\alpha > 0$, the set $\{ (\beta, \alpha +
   1) :    \beta < \alpha \}$ is a local basis at $\alpha$. The set $\{ \{ 0
   \} \}$ is a local basis at 0. \qed
 \end{proof}

  \begin{thm}[The Sequence Lemma (CC)]
    Let $X$ be a first countable space and $A \subseteq X$. If $x \in
    \overline{A}$, then there exists a sequence of points of $A$ that converges
    to $x$.
  \end{thm}

  \begin{proof}
    \pf
    \step{<1>1}{\pflet{$x \in \overline{A}$}}
    \step{<1>2}{\pick\ a countable basis $\{ B_n \}_{n \in \mathbb{Z}^+}$ at
      $x$.}
    \step{<1>3}{For $n \geq 1$, \pick\ a point $a_n \in B_1 \cap \cdots \cap
      B_n
      \cap A$ \prove{$a_n \rightarrow x$ as $n \rightarrow \infty$}}
    % TODO Extract lemma
    \begin{proof}
      \pf\ Using Countable Choice. Such an $a_n$ exists because $B_1 \cap
      \cdots
      \cap B_n$ is a neighbourhood of $x$. Apply Theorem
      \ref{thm:topology:closure:neighbourhoods}.
    \end{proof}
    \step{<1>4}{\pflet{$U$ be a neighbourhood of $x$}}
    \step{<1>5}{\pick\ $N$ such that $B_N \subseteq U$}
    \begin{proof}
      \pf\ From \stepref{<1>2}.
    \end{proof}
    \step{<1>6}{For $n \geq N$, we have $a_n \in U$}
    \begin{proof}
      \pf
      \begin{align*}
        a_n & \in B_1 \cap \cdots \cap B_n & (\text{\stepref{<1>3}}) \\
        & \subseteq B_N & (n \geq N) \\
        & \subseteq U & (\text{\stepref{<1>5}})
      \end{align*}
    \end{proof}
    \qed
  \end{proof}

  \begin{thm}[CC]
    Let $X$ and $Y$ be topological spaces where $X$ is first countable. Let $x
    \in X$. Suppose that, for every sequence $\{ x_n \}_{n \geq 1}$ such that
    $x_n \rightarrow x$ as $n \rightarrow \infty$, we have $f(x_n) \rightarrow
    f(x)$ as $n \rightarrow \infty$. Then $f$ is continuous at $x$.
  \end{thm}

  \begin{proof}
    \pf
    \step{<1>1}{\pflet{$V$ be a neighbourhood of $f(x)$}}
    \step{<1>2}{\assume{for a contradiction that, for every neighbourhood $U$
        of
        $x$, $f(U) \nsubseteq V$}}
    \step{<1>3}{\pick\ a countable local basis $\{ B_n \}_{n \geq 1}$}
    \step{<1>4}{For $n \geq 1$, \pick\ $a_n \in B_1 \cap \cdots \cap B_n$ such
      that
      $f(a_n) \notin V$}
    \step{<1>5}{$a_n \rightarrow x$ as $n \rightarrow \infty$}
    \begin{proof}
      \pf
      \step{<2>1}{\pflet{$U$ be a neighbourhood of $x$}}
      \step{<2>2}{\pick\ $N$ such that $B_N \subseteq U$}
      \step{<2>3}{For all $n \geq N$, $a_n \in U$}
      \begin{proof}
        \pf
        \begin{align*}
          a_n & \in B_1 \cap \cdots \cap B_n & (\text{\stepref{<1>4}}) \\
          & \subseteq B_N & (n \geq N) \\
          & \subseteq U & (\text{\stepref{<2>2}})
        \end{align*}
      \end{proof}
    \end{proof}
    \step{<1>6}{$f(a_n) \rightarrow f(x)$ as $n \rightarrow \infty$}
    \step{<1>7}{There exists $N$ such that, for all $n \geq N$, we have $f(a_n)
      \in
      V$}
    \qedstep
  \end{proof}

  \begin{lm}[CC]
    $\mathbb{R}^\omega$ under the box topology is not first countable.
  \end{lm}

  \begin{proof}
    \pf
    \step{<1>1}{\pflet{$\{ B_n \}_{n \geq 1}$ be any countable set of
        neighbourhoods of $\vec{0}$}}
    \step{<1>2}{For $n \geq 1$, \pick\ $U_{nm}$ for $m \geq 1$ such that
      $\vec{0}
      \in \prod_{m = 1}^\infty U_{nm} \subseteq B_n$}
    \step{<1>3}{For $n \geq 1$, \pick\ $a_n$, $b_n$ such that $0 \in (a_n, b_n)
      \subseteq U_{nn}$}
    \step{<1>4}{\pflet{$U = \prod_{n = 1}^\infty (a_n / 2, b_n / 2)$}}
    \step{<1>5}{$\vec{0} \in U$}
    \step{<1>6}{For all $n$, $B_n \nsubseteq U$}
    \qed
  \end{proof}

  \begin{lm}[CC]
    If $J$ is uncountable then $\mathbb{R}^J$ is not first countable.
  \end{lm}

  \begin{proof}
    \pf
    \step{<1>1}{\pflet{$\{ B_n \}_{n \geq 1}$ be a countable family of
        neighbourhoods of $\vec{0}$}}
    \step{<1>2}{For $n \geq 1$, \pick\ $U_{n \alpha}$ such that $\vec{0} \in
      \prod_{\alpha \in J} U_{n \alpha} \subseteq B_n$ where $U_{n \alpha}$ is
      open in $\mathbb{R}$ and $U_{n \alpha} = \mathbb{R}$ except for $\alpha =
      \alpha_{n1}, \ldots, \alpha_{nr_n}$}
    \step{<1>3}{\pick\ $\beta$ such that $\beta$ is different from
      $\alpha_{ni}$
      for all $n$, $i$}
    \step{<1>4}{\pflet{$V = \pi_\beta^{-1}((-1, 1))$}}
    \step{<1>5}{$\vec{0} \in V$}
    \step{<1>6}{$V \nsubseteq B_n$ for all $n$}
    \qed
  \end{proof}

  \begin{lm}
    $\mathbb{R}_l$ is first countable.
  \end{lm}

  \begin{proof}
    \pf\ For all $x \in \mathbb{R}$, $\{ [x, q) : q \in \mathbb{Q}, q > x \}$
    is a basis at $x$. \qed
  \end{proof}

  \begin{lm}
    The ordered square is first countable.
  \end{lm}

  \begin{proof}
    \pf
    \step{<1>1}{\pflet{$(x, y) \in I_o^2$} \prove{There exists a countable
        local
        basis $\mathcal{B}$ at $(x, y)$}}
    \step{<1>2}{\case{$(x, y) = (0, 0)$}}
    \begin{proof}
      \pf\ Take $\mathcal{B} = \{ [(0, 0), (0, q)) : q \in \mathbb{Q}, 0 < q <
      1
      \}$.
    \end{proof}
    \step{<1>3}{\case{$0 < y < 1$}}
    \begin{proof}
      \pf\ Take $\mathcal{B} = \{ ((x, q), (x, q')) : q, q' \in \mathbb{Q}, q <
      y < q' \}$.
    \end{proof}
    \step{<1>4}{\case{$x < 1, y = 1$}}
    \begin{proof}
      \pf\ Take $\mathcal{B} = \{ ((x, q), (q', 0)) : q, q' \in \mathbb{Q}, 0 <
      q < 1, x < q' < 1 \}$.
    \end{proof}
    \step{<1>5}{\case{$x > 0, y = 0$}}
    \begin{proof}
      \pf\ Take $\mathcal{B} = \{ ((q, 1), (x, q')) : q, q' \in \mathbb{Q}, 0 <
      q < x, 0 < q' < 1 \}$
    \end{proof}
    \step{<1>6}{\case{$(x, y) = (1, 1)$}}
    \begin{proof}
      \pf\ Take $\mathcal{B} = \{ ((1, q), (1, 1)] : q \in \mathbb{Q}, 0 < q <
      1
      \}$.
    \end{proof}
    \qed
  \end{proof}

    \begin{prop}
   A subspace of a first countable space is first countable.
  \end{prop}

  \begin{proof}
   \pf
   \step{<1>1}{\pflet{$X$ be a first countable space and $A \subseteq X$}}
   \step{<1>2}{\pflet{$a \in A$}}
   \step{<1>3}{\pick\ a countable basis $\mathcal{B}$ at $a$ in $X$}
   \step{<1>4}{$\{ B \cap A : B \in \mathcal{B}$ is a countable basis at $a$ in
     $A$.} % TODO Extract lemma
   \qed
  \end{proof}

  \begin{prop}[CC]
   A countable product of first countable spaces is first countable.
  \end{prop}

  \begin{proof}
   \pf
   \step{<1>1}{\pflet{$\{ X_n \}_{n \in \mathbb{Z}^+}$ be a countable family of
       first countable spaces.}}
   \step{<1>2}{\pflet{$\vec{x} \in \prod_{n=1}^\infty X_n$}}
   \step{<1>3}{\pick\ a countable basis $\mathcal{B}_n$ at $x_n$ in $X_n$ for
all
     $n$}
   \step{<1>4}{\pflet{$\mathcal{B}$ be the set of all sets $\prod_{i=1}^n U_n$
       where $U_n \in \mathcal{B}_n$ for finitely many $n$ and $U_n = X_n$ for
       all other $n$.}}
   \step{<1>5}{$\mathcal{B}$ is a countable basis at $\vec{x}$ in
     $\prod_{n=1}^\infty X_n$}
   \qed
  \end{proof}

  \begin{cor}
    The space $\mathbb{R}^\omega$ is first countable.
  \end{cor}

    \begin{prop}
   The space $S_\Omega$ is first countable.
  \end{prop}

  \begin{proof}
   \pf
   \step{<1>1}{\pflet{$\alpha \in S_\Omega$} \prove{$\alpha$ has a countable
local
       basis.}}
   \step{<1>2}{\case{$\alpha$ is zero or a successor ordinal.}}
   \begin{proof}
     \pf\ In this case, $\{ \{ \alpha \} \}$ is a local basis.
   \end{proof}
   \step{<1>3}{\case{$\alpha$ is a limit ordinal.}}
   \begin{proof}
     \step{<2>1}{\pick\ a countable sequence $(\beta_n)$ with supremum $\alpha$}
     \step{<2>2}{$\{ (\beta_n, \alpha + 1) : n \in \mathbb{Z}^+ \}$ is a local
       basis.}
   \end{proof}
   \qed
  \end{proof}

    \begin{prop}
      \label{prop:topology:first_countable:S_omega}
    The space $\overline{S_\Omega}$ is not first countable.
  \end{prop}

  \begin{proof}
   \pf
   \step{<1>1}{\assume{for a contradiction $\mathcal{B}$ is a countable local
       basis at $\Omega$}}
   \step{<1>2}{\pflet{$\alpha = \sup \{ \inf B : B \in \mathcal{B} \}$}}
   \step{<1>3}{$\alpha < \Omega$}
   \step{<1>4}{There is no $B \in \mathcal{B}$ such that $B \subseteq (\alpha,
+
     \infty)$}
   \qed
  \end{proof}

  \begin{prop}
   The continuous image of a first countable space is first countable.
  \end{prop}

  \begin{proof}
    \pf
    \step{<1>1}{\pflet{$X$ be a first countable space, $Y$ a space and $f : X
        \rightarrow Y$ continuous.}}
    \step{<1>2}{\pflet{$y \in f(X)$}}
    \step{<1>3}{\pick\ $x \in X$ such that $y = f(x)$}
    \step{<1>4}{\pick\ a countable local basis $\mathcal{B}$ at $x$}
    \step{<1>5}{$\{ f(B) : B \in \mathcal{B} \}$ is a countable local basis at
      $y$.}
    \qed
  \end{proof}

 \begin{prop}
   $S_\Omega \times \overline{S_\Omega}$ is not first countable.
 \end{prop}

 \begin{proof}
  \pf\ $(0, \Omega)$ has no countable basis. \qed
 \end{proof}

  \begin{prop}
  The Sorgenfrey plane is first countable.
 \end{prop}

 \begin{proof}
   \pf\ For any point $(a,b)$, the set $\{ [a, a + q) \times [b, b + r) : q, r
   \in \mathbb{Q} \}$ is a countable local basis at $(a,b)$. \qed
 \end{proof}


  \section{Separable Spaces}

    \begin{df}[Separable Space]
    A topological space $X$ is \emph{separable} iff it has a countable dense
    subset.
  \end{df}

    \begin{prop}
      \label{prop:topology:separable:S_omega}
   The space $S_\Omega$ is not separable.
  \end{prop}

  \begin{proof}
   \pf
   \step{<1>1}{\pflet{$D \subseteq S_\Omega$ be countable.}}
   \step{<1>2}{\pflet{$\alpha = \sup D$}}
   \step{<1>3}{$\overline{D} \subseteq (-\infty, \alpha]$}
   \qed
  \end{proof}

    \begin{prop}
    The space $\overline{S_\Omega}$ is not separable.
  \end{prop}

  \begin{proof}
   \pf
   \step{<1>1}{\pflet{$D \subseteq S_\Omega$ be countable.}}
   \step{<1>2}{\pflet{$\alpha = \sup \{ \beta \in D : \beta < \Omega \}$}}
   \step{<1>3}{$\alpha < \Omega$}
   \begin{proof}
     \pf\ $\alpha$ is the supremum of countably many countable ordinals.
   \end{proof}
   \step{<1>4}{$\overline{D} \subseteq (-\infty, \alpha] \cup \{ \Omega \}$}
   \qed
  \end{proof}

  \begin{cor}
    Not every compact Hausdorff space is separable.
  \end{cor}

  \begin{prop}
    Every open subspace of a separable space is separable.
  \end{prop}

  \begin{proof}
    \pf
    \step{<1>1}{\pflet{$X$ be a separable space with countable dense subset $D$.}}
    \step{<1>2}{\pflet{$U$ be an open subspace of $X$} \prove{$D \cap U$ is a countable dense subset of $U$.}}
    \step{<1>3}{$D \cap U$ is countable.}
    \step{<1>4}{\pflet{$V$ be an open set in $U$.}}
    \step{<1>5}{$V$ is open in $X$}
    \begin{proof}
      \pf\ Lemma \ref{lm:topology:subspace:open}
    \end{proof}
    \step{<1>6}{$V$ intersects $D$}
    \step{<1>7}{$V$ intensects $D \cap U$}
    \qed
  \end{proof}

  \begin{prop}[CC]
   The product of a countable family of separable spaces is separable.
  \end{prop}

  \begin{proof}
   \pf
   \step{<1>1}{\pflet{$(X_n)$ be a countable family of separable spaces.}}
   \step{<1>2}{For $n \geq 1$, \pick\ a dense set $D_n$ in $X_n$}
   \step{<1>3}{$\prod_{n=1}^\infty D_n$ is dense in $\prod_{n=1}^\infty X_n$.}
   \qed
  \end{proof}

  \begin{prop}
   The continuous image of a separable space is separable.
  \end{prop}

  \begin{proof}
   \pf
   \step{<1>1}{\pflet{$X$ be a separable space, $Y$ a space and $f : X
\rightarrow
       Y$ be continuous.}}
   \step{<1>2}{\pick\ a countable dense set $D$ in $X$}
   \step{<1>3}{$f(D)$ is dense in $f(X)$.}
   \qed
  \end{proof}

  \begin{cor}
    Let $\{ X_\alpha \}_{\alpha \in J}$ be a family of nonempty topological
    spaces. If $\prod_{\alpha \in J} X_\alpha$ is separable then each
$X_\alpha$ is separable.
  \end{cor}

  \begin{cor}
    $S_\Omega \times \overline{S_\Omega}$ is not separable.
  \end{cor}

 \begin{prop}
  The ordered square is not separable.
 \end{prop}

 \begin{proof}
   \pf\ $\{ \{x\} \times (0,1) : x \in [0,1] \}$ is an uncountable set of
disjoint open sets. \qed
 \end{proof}

  \begin{prop}
   $\mathbb{R}_l$ is separable.
 \end{prop}

 \begin{proof}
   \pf\ $\mathbb{Q}$ is dense. \qed
 \end{proof}

  \begin{prop}
  The Sorgenfrey plane is separable.
 \end{prop}

 \begin{proof}
   \pf\ $\mathbb{Q}^2$ is dense. \qed
 \end{proof}

 \begin{prop}
   Not every closed subspace of a separable space is separable.
 \end{prop}

 \begin{proof}
   \pf\ $\mathbb{R}_l^2$ is separable but the subspace $\{ (x, -x) : x \in \mathbb{R} \}$ is not. \qed
 \end{proof}

  \section{The Second Countability Axiom}

    \begin{df}[Second Countability Axiom]
    A topological space satisfies the \emph{second countability axiom}, or is
    \emph{second countable}, iff it has a countable basis.
  \end{df}

   \begin{prop}
  $S_\Omega$ is not second countable.
 \end{prop}

 \begin{proof}
   \pf\ $\{ \{ \alpha \} : \alpha \text{ is a countable successor ordinal} \}$
is an uncountable set of disjoint open sets. \qed
 \end{proof}

  \begin{prop}
    \label{prop:topology:second_countable:subspace}
   A subspace of a second countable space is second countable.
  \end{prop}

  \begin{proof}
   \pf
   \step{<1>1}{\pflet{$X$ be a second countable space and $A \subseteq X$}}
   \step{<1>2}{\pick\ a countable basis $\mathcal{B}$ for $X$}
   \step{<1>3}{$\{ B \cap A : B \in \mathcal{B} \}$ is a countable basis for
$A$}
   \qed
  \end{proof}

    \begin{prop}[CC]
   The product of countably many second countable spaces is second countable.
  \end{prop}

  \begin{proof}
   \pf
   \step{<1>1}{\pflet{$\{X_n\}_{n \in \mathbb{Z}^+}$ be a countable family of
       second countable spaces.}}
   \step{<1>2}{For $n \in \mathbb{Z}^+$, \pick\ a countable basis
$\mathcal{B}_n$
     for $X_n$.}
   \step{<1>3}{\pflet{$\mathcal{B}$ be the set of all sets of the form
       $\prod_{n=1}^\infty U_n$, where $U_n \in \mathcal{B}_n$ for finitely
many        $n$, and $U_n = X_n$ for all other $n$.}}
\step{<1>4}{$\mathcal{B}$ is a countable basis for $\prod_{n=1}^\infty X_n$}
\qed
  \end{proof}

    \begin{thm}[CC]
   Every second countable space is separable.
  \end{thm}

  \begin{proof}
   \pf
   \step{<1>1}{\pflet{$X$ be a second countable space.}}
   \step{<1>2}{\pick\ a countable basis $\mathcal{B}$ for $X$}
   \step{<1>3}{For $B \in \mathcal{B}$ nonempty, \pick\ a point $x_B \in B$}
   \step{<1>4}{$D = \{ x_B : B \in \mathcal{B} \setminus \{ \emptyset \} \}$ is
dense.}
   \begin{proof}
     \step{<2>1}{\pflet{$l \in X$} \prove{$l \in \overline{D}$}}
     \step{<2>2}{\pflet{$B \in \mathcal{B}$ such that $l \in B$}}
     \step{<2>3}{$x_B \in B \cap D$}
     \qedstep
     \begin{proof}
       \pf By Theorem \ref{thm:topology:closure:basis}
     \end{proof}
   \end{proof}
  \end{proof}

  \begin{cor}
    $S_\Omega \times \overline{S_\Omega}$ is not second countable.
  \end{cor}

  \begin{cor}
    The space $\mathbb{R}^\omega$ is separable.
  \end{cor}

\begin{cor}
	If $J$ is uncountable then $\mathbb{R}^J$ is not second countable.
\end{cor}

    \begin{prop}
   The ordered square is not second countable.
  \end{prop}

  \begin{proof}
   \pf
   \step{<1>1}{\pflet{$\mathcal{B}$ be any basis}}
   \step{<1>2}{For $x \in [0,1]$, \pick\ $B_x$ such that $x \in B_x \subseteq
((x,
     0), (x, 1))$}
   \step{<1>3}{The function $B_{(-)}$ is an injective function $[0, 1]
\rightarrow
     \mathcal{B}$}
   \step{<1>4}{$\mathcal{B}$ is uncountable.}
   \qed
  \end{proof}

  \begin{prop}
    The space $\overline{S_\Omega}$ is not second countable.
  \end{prop}

  \begin{proof}
    \pf\ It is not first countable (Proposition
    \ref{prop:topology:first_countable:S_omega}). \qed
  \end{proof}

  \begin{prop}
   The continuous image of a second countable space is second countable.
  \end{prop}

  \begin{proof}
   \pf
   \step{<1>1}{\pflet{$X$ be a second countable space, $Y$ a space and $f : X
       \rightarrow Y$ be continuous.}}
   \step{<1>2}{\pick\ a countable basis $\mathcal{B}$ for $X$.}
   \step{<1>3}{$\{ f(B) : B \in \mathcal{B}$ is a countable basis for $f(X)$}
   \qed
  \end{proof}

  \begin{thm}
    \label{thm:topology:normal:regular_lindelof}
    Every regular Lindel\"{o}f space is normal.
  \end{thm}

  \begin{proof}
   \pf
   \step{<1>1}{\pflet{$X$ be a regular Lindel\"{o}f space.}}
   \step{<1>2}{\pflet{$A$ and $B$ be disjoint closed sets in $X$.}}
   \step{<1>3}{$\{ U \text{ open in } X : \overline{U} \cap B = \emptyset \}$
     covers $A$}
       \begin{proof}
         \pf\ Proposition \ref{prop:topology:regular:closure}.
       \end{proof}
   \step{<1>4}{\pick\ a countable open covering $\{ U_n : n \in \mathbb{Z}^+
\}$
     of $A$ such that $\overline{U_n} \cap B = \emptyset$ for all $n$}
   \step{<1>5}{\pick\ a countable open covering $\{ V_n : n \in \mathbb{Z}^+
\}$
     of $B$ such that $\overline{V_n} \cap A = \emptyset$ for all $n$}
   \begin{proof}
     \pf\ Similar.
   \end{proof}
   \step{<1>6}{For $n \in \mathbb{Z}^+$, \pflet{$U_n' = U_n \setminus
       \bigcup_{i=1}^n \overline{V_i}$ and $V_n' = V_n \setminus
       \bigcup_{i=1}^n \overline{U_i}$}}
   \step{<1>7}{\pflet{$U' = \bigcup_{n=1}^\infty U_n'$ and $V =
       \bigcup_{n=1}^\infty V_n'$}}
   \step{<1>8}{$A \subseteq U'$ and $B \subseteq V'$}
   \step{<1>9}{$U' \cap V' = \emptyset$}
   \qed
  \end{proof}

\begin{cor}
	If $J$ is uncountable then $\mathbb{R}^J$ is not Lindel\"{o}f.
\end{cor}
    \begin{prop}
   Every second countable regular space is completely normal.
  \end{prop}

  \begin{proof}
   \pf
   \step{<1>1}{\pflet{$X$ be second countable and regular and $Y \subseteq X$}}
   \step{<1>2}{$Y$ is second countable}
   \begin{proof}
     \pf\ Proposition \ref{prop:topology:second_countable:subspace}.
   \end{proof}
   \step{<1>3}{$Y$ is regular}
   \begin{proof}
     \pf\ Proposition \ref{prop:topology:regular:subspace}
   \end{proof}
   \step{<1>4}{$Y$ is normal}
   \begin{proof}
     \pf\ Theorem \ref{thm:topology:normal:regular_lindelof}
   \end{proof}
   \qed
  \end{proof}

   \begin{prop}
   The space $\mathbb{R}^\omega$ is second countable.
 \end{prop}

 \begin{proof}
   \pf\ The sets $\prod_{n=0}^\infty U_n$ form a basis, where $U_n$ is an
   interval of the form $(q, r)$ for $q, r \in \mathbb{Q}$ for finitely many
   $n$, and $U_n = \mathbb{R}$ for all other $n$. \qed
 \end{proof}

  \begin{prop}[CC]
  In a second countable space, every discrete subspace is countable.
 \end{prop}

 \begin{proof}
  \pf
  \step{<1>1}{\pflet{$X$ be a second countable space}}
  \step{<1>2}{\pick\ a countable basis $\mathcal{B}$}
  \step{<1>3}{\pflet{$D \subseteq X$ be discrete}}
  \step{<1>4}{For $a \in D$, \pick\ $B_a \in \mathcal{B}$ such that $B_a \cap D
=
    \{ a \}$}
  \step{<1>5}{$a \mapsto B_a$ is injective}
  \qed
 \end{proof}

 \begin{prop}
  The space $\mathbb{R}_K$ is second countable.
\end{prop}

\begin{proof}
  \pf\ $\{ (a,b) : a, b \in \mathbb{R} \} \cup \{ (a, b) - K : a, b \in
  \mathbb{Q} \}$ is a basis. \qed
\end{proof}

\begin{cor}
  The space $\mathbb{R}_K$ is first countable.
\end{cor}

\begin{cor}
  The space $\mathbb{R}_K$ is separable.
\end{cor}

\begin{prop}
  Let $J$ be a set with $|J| > |\mathbb{R}|$. Then $\mathbb{R}^J$ is not separable.
\end{prop}

\begin{proof}
  \pf
  \step{<1>1}{\assume{$D$ is countable and dense in $\mathbb{R}^J$} \prove{$|J| \leq |\mathbb{R}|$}}
  \step{<1>2}{Define $f : J \rightarrow \mathcal{P} D$ by $f(\alpha) = D \cap \inv{\pi_\alpha}((0,1))$}
  \step{<1>3}{$f$ is injective}
  \begin{proof}
    \step{<2>1}{\pflet{$\alpha, \beta \in J$ with $\alpha \neq \beta$}}
    \step{<2>2}{\pick\ $x \in D \cap \inv{\pi_\alpha}((0,1)) \cap \inv{\pi_\beta}((2, 3))$}
    \step{<2>3}{$x \in f(\alpha)$ but $x \notin f(\beta)$}
  \end{proof}
  \qed
\end{proof}

\begin{cor}
  The product of a family of separable spaces is not necessarily separable.
\end{cor}

  \chapter{Connectedness}

  \section{Connected Spaces}

  \begin{df}[Separation]
    Let $X$ be a topological space. A \emph{separation} of $X$ is a pair of
    disjoint nonempty subsets whose union in $X$.
  \end{df}

  \begin{df}[Connected]
    A topological space is \emph{connected} iff it has no separation.
  \end{df}

  \begin{prop}
   $S_\Omega$ is not connected.
  \end{prop}

  \begin{proof}
    \pf\ $\{ 0 \}$ and $S_\Omega \setminus \{ 0 \}$ form a separation. \qed
  \end{proof}

  \begin{prop}
    A space $X$ is connected if and only if the only sets that are both closed
    and open are $\emptyset$ and $X$.
  \end{prop}

  \begin{proof}
    \pf\ Immediate from definitions. \qed
  \end{proof}

  \begin{prop}
    Let $Y$ be a subspace of $X$. Then a separation of $Y$ is a pair of
    disjoint
    nonempty sets $A$, $B$ such that $A \cup B = Y$ and neither of $A$, $B$
    contains a limit point of the other.
  \end{prop}

  \begin{proof}
    \pf
    \step{<1>1}{If $A$ and $B$ form a separation of $Y$ then $A$ and $B$ are
      disjoint and nonempty, $A \cup B = Y$, and neither of $A$, $B$ contains a
      limit point of the other.}
    \begin{proof}
      \step{<2>1}{\pflet{$A$ and $B$ be a separation of $Y$}}
      \step{<2>2}{$A$ and $B$ are disjoint and nonempty and $A \cup B = Y$}
      \begin{proof}
        \pf\ Immediate from the definition of separation.
      \end{proof}
      \step{<2>3}{$A$ does not contain a limit point of $B$}
      \begin{proof}
        \pf\ $B$ is closed in $Y$, hence contains all its limit points
        (Corollary \ref{cor:topology:limit_point:closed}), and so the result
        follows because $A$ and $B$ are disjoint.
      \end{proof}
      \step{<2>4}{$B$ does not contain a limit point of $A$}
      \begin{proof}
        \pf\ Similar.
      \end{proof}
    \end{proof}
    \step{<1>2}{If $A$ and $B$ are disjoint and nonempty, $A \cup B = Y$, and
      neither of $A$, $B$ contains a limit point of the other, then $A$ and
      $B$ are a separation of $Y$.}
    \begin{proof}
      \step{<2>1}{\assume{$A$ and $B$ are disjoint and nonempty, $A \cup B =
          Y$,
          and neither of $A$, $B$ contains a limit point of the other}}
      \step{<2>2}{$A$ is closed in $Y$}
      \begin{proof}
        \pf\ Every limit point of $A$ is not in $B$, so is in $A$. Apply
        Corollary \ref{cor:topology:limit_point:closed}.
      \end{proof}
      \step{<2>3}{$B$ is open in $Y$}
      \begin{proof}
        \pf $B = Y \setminus A$
      \end{proof}
      \step{<2>4}{$A$ is open in $Y$}
      \begin{proof}
        \pf\ Similar.
      \end{proof}
    \end{proof}
    \qed
  \end{proof}

  \begin{prop}
    \label{prop:topology:connected:separation_subspace}
    If the sets $C$ and $D$ form a separation of $X$, and $Y$ is a connected
    subspace of $X$, then $Y \subseteq C$ or $Y \subseteq D$.
  \end{prop}

  \begin{proof}
    \pf\ Otherwise, $Y \cap C$ and $Y \cap D$ would be a separation of $Y$. \qed
  \end{proof}

  \begin{prop}
    \label{prop:topology:connected:union}
    The union of a set of connected subspaces of $X$ that have a point in
    common
    is connected.
  \end{prop}

  \begin{proof}
    \pf
    \step{<1>1}{\pflet{$\mathcal{S}$ be a set of connected subspaces that have
        the
        point $a$ in common.}}
    \step{<1>2}{\assume{for a contradiction $U$ and $V$ form a separation of
        $\bigcup \mathcal{S}$}}
    \step{<1>3}{\assume{w.l.o.g. $a \in U$}}
    \step{<1>4}{For all $Y \in \mathcal{S}$ we have $Y \subseteq U$}
    \begin{proof}
      \pf\ By Proposition \ref{prop:topology:connected:separation_subspace}.
    \end{proof}
    \step{<1>5}{$V = \emptyset$}
    \qedstep
    \begin{proof}
      \pf\ This contradicts \stepref{<1>2}.
    \end{proof}
    \qed
  \end{proof}

  \begin{thm}
    \label{thm:topology:connected:closure}
    Let $A$ be a connected subspace of $X$. If $A \subseteq B \subseteq
    \overline{A}$ then $B$ is connected.
  \end{thm}

  \begin{proof}
    \pf
    \step{<1>1}{\assume{for a contradiction $U$ and $V$ are a separation of
        $B$}}
    \step{<1>2}{$A \subseteq U$ or $A \subseteq V$}
    \begin{proof}
      \pf\ By Proposition \ref{prop:topology:connected:separation_subspace}.
    \end{proof}
    \step{<1>3}{\assume{w.l.o.g. $A \subseteq U$}}
    \step{<1>4}{$\overline{A} \subseteq \overline{U}$}
    \begin{proof}
      \pf\ By Proposition \ref{prop:topology:closure:monotone}.
    \end{proof}
    \step{<1>5}{$B \subseteq \overline{U}$}
    \begin{proof}
      \pf\ Since $B \subseteq \overline{A}$.
    \end{proof}
    \step{<1>6}{The closure of $U$ in $B$ is $B$}
    \begin{proof}
      \pf\ By Theorem \ref{thm:topology:subspace:closure}.
    \end{proof}
    \step{<1>7}{$U = B$}
    \begin{proof}
      \pf\ Since $U$ is closed in $B$.
    \end{proof}
    \qedstep
    \begin{proof}
      \pf\ This contradicts \stepref{<1>1}.
    \end{proof}
    \qed
  \end{proof}

  \begin{thm}
    \label{thm:topology:connected:image}
    The image of a connected space under a continuous map is connected.
  \end{thm}

  \begin{proof}
    \pf\ Let $X$ be a connected space, $Y$ a topological space, and $f :
    X \twoheadrightarrow Y$ be surjective. If $U$ and $V$ form a separation
    of $Y$, then $f^{-1}(U)$ and $f^{-1}(V)$ form a separation of $X$. \qed
  \end{proof}

  \begin{cor}
    \label{cor:topology:connected:finer}
    Let $\mathcal{T}$ and $\mathcal{T}'$ be topologies on the same set $X$. If
    $\mathcal{T} \subseteq \mathcal{T}'$ and $X$ is connected under
    $\mathcal{T}'$ then $X$ is connected under $\mathcal{T}$.
  \end{cor}

  \begin{cor}
    Let $\{ X_\alpha \}_{\alpha \in J}$ be a family of topological spaces. If
    $\prod_{\alpha \in J} X_\alpha$ is connected then each $X_\alpha$ is
connected.
  \end{cor}

  \begin{cor}
   The Sorgenfrey plane is disconnected.
  \end{cor}

  \begin{prop}
    \label{prop:topology:connected:product}
    The product of a family of connected spaces is connected.
  \end{prop}

  \begin{proof}
    \pf
    \step{<1>1}{The product of two connected spaces is connected.}
    \begin{proof}
      \pf
      \step{<2>1}{\pflet{$X$ and $Y$ be connected spaces.}}
      \step{<2>2}{\assume{w.l.o.g.~$X$ and $Y$ are nonempty.}}
      \begin{proof}
        \pf\ If either is empty then $X \times Y = \emptyset$ is connected.
      \end{proof}
      \step{<2>3}{\assume{for a contradiction $U$ and $V$ are a separation of
          $X
          \times Y$.}}
      \step{<2>4}{\pick\ $b \in Y$}
      \begin{proof}
        \pf\ By \stepref{<2>2}.
      \end{proof}
      \step{<2>5}{For all $x \in X$, \pflet{$T_x = (X \times \{ b \}) \cup
          (\{x\}
          \times Y)$}}
      \step{<2>6}{For all $x \in X$, $T_x$ is connected}
      \begin{proof}
        \step{<3>1}{$X \times \{b\}$ is connected}
        \begin{proof}
          \pf\ It is homeomorphic to $X$.
        \end{proof}
        \step{<3>2}{$\{x\} \times Y$ is connected}
        \begin{proof}
          \pf\ It is homeomorphic to $Y$.
        \end{proof}
        \qedstep
        \begin{proof}
          \pf\ By Proposition \ref{prop:topology:connected:union}.
        \end{proof}
      \end{proof}
      \step{<2>7}{$X \times Y = \bigcup_{x \in X} T_x$}
      \qedstep
      \begin{proof}
        \step{<3>1}{\pick\ $a \in X$}
        \begin{proof}
          \pf\ By \stepref{<2>2}.
        \end{proof}
        \step{<3>2}{$(a, b) \in T_x$ for all $x \in X$}
        \qedstep
        \begin{proof}
          \pf\ By Proposition \ref{prop:topology:connected:union}.
        \end{proof}
      \end{proof}
    \end{proof}
    \step{<1>2}{\pflet{$\{ X_\alpha \}_{\alpha \in J}$ be a family of connected
        spaces.}}
    \step{<1>3}{\assume{w.l.o.g.~$\prod_{\alpha \in J} X_\alpha$ is nonempty}}
    \step{<1>4}{\pick\ $\vec{a} \in \prod_{\alpha \in J} X_\alpha$}
    \step{<1>5}{For $K$ a finite subset of $J$, \pflet{$X_K = \{ \vec{x} \in
        \prod_{\alpha \in J} X_\alpha : x_\alpha = a_\alpha \text{ for all }
        \alpha \in J \setminus K \}$}}
    \step{<1>6}{For all $K$, $X_K$ is connected.}
    \begin{proof}
      \pf\ It is homeomorphic to $\prod_{\alpha \in K} X_\alpha$, so it is
      connected by \stepref{<1>1}.
    \end{proof}
    \step{<1>7}{$\bigcup_{K \subseteq^{\mathrm{fin}} J} X_K$ is connected.}
    \begin{proof}
      \pf\ By Proposition \ref{prop:topology:connected:union} since $\vec{a}
      \in X_K$ for all $K$.
    \end{proof}
    \step{<1>8}{$\prod_{\alpha \in J} X_\alpha = \overline{\bigcup_{K
          \subseteq^{\mathrm{fin}} J} X_K}$}
    \begin{proof}
      \step{<2>1}{\pflet{$\vec{x} \in \prod_{\alpha \in J} X_\alpha$}}
      \step{<2>2}{\pflet{$U$ be an open neighbourhood of $\vec{x}$}}
      \step{<2>3}{\pick\ a basic open set $\prod_{\alpha \in J} V_\alpha$ such
        that $\vec{x} \in \prod_{\alpha \in J} V_\alpha \subseteq U$, where
        each $V_\alpha$ is open in $X_\alpha$, and $V_\alpha = X_\alpha$ except
        for  $\alpha \in K$ for some finite $K \subseteq J$
        \prove{$U$ intersects $X_K$}}
      \step{<2>4}{\pflet{$\vec{y} \in \prod_{\alpha \in J} X_\alpha$ with
          $y_\alpha = x_\alpha$ for $\alpha \in K$, $y_\alpha = a_\alpha$ for
          $\alpha \notin K$}}
      \step{<2>5}{$\vec{y} \in U \cap X_K$}
    \end{proof}
    \qedstep
  \end{proof}

  \begin{cor}
    For any set $I$, the space $\mathbb{R}^I$ under the product topology is connected.
  \end{cor}

  \begin{prop}
    $\mathbb{R}^\omega$ under the box topology is disconnected.
  \end{prop}

  \begin{proof}
    \pf\ The set of all bounded sequences and the set of all unbounded
    sequences
    form a separation. \qed
  \end{proof}

  \begin{df}[Totally Disconnected]
    A space is \emph{totally disconnected} iff the only connected subspaces are
    the singletons.
  \end{df}

  \begin{thm}
    Let $L$ be a linearly ordered set under the order topology. Then $L$ is
connected if and only if $L$ is a linear continuum.
  \end{thm}

  \begin{proof}
    \pf
    \step{<1>1}{If $L$ is a linear continuum then $L$ is connected.}
    \begin{proof}
    \step{<2>1}{\pflet{$L$ be a linear continuum.}}
    \step{<2>2}{\assume{for a contradiction $U$ and $V$ are a separation of
        $L$.}}
    \step{<2>3}{\pick\ $a \in U$ and $b \in V$}
    \step{<2>4}{\assume{w.l.o.g.~$a < b$}}
    \step{<2>5}{\pflet{$l = \sup \{ x \in A : x < b \}$}}
    \step{<2>6}{\case{$l \in A$}}
    \begin{proof}
      \step{<3>1}{\pick\ $a' > l$ such that $[l, a') \subseteq A$}
      \begin{proof}
        \pf\ By Lemma \ref{lm:topology:order:open}. We know $l$ is not greatest
        in $X$ because $l < b$.
      \end{proof}
      \step{<3>2}{\pick\ $a^*$ such that $l < a^* < a'$}
      \begin{proof}
        \pf\ $L$ is dense.
      \end{proof}
      \step{<3>3}{$l < a^*$, $a^* \in A$, $a^* < b$}
      \begin{proof}
        \pf\ If $b < a^*$ then $b \in A$ by \stepref{<3>1}.
      \end{proof}
      \qedstep
      \begin{proof}
        \pf\ This contradicts \stepref{<2>5}.
      \end{proof}
    \end{proof}
    \step{<2>7}{\case{$l \in B$}}
    \begin{proof}
      \step{<3>1}{\pick\ $b' < l$ such that $(b', l] \subseteq B$}
      \begin{proof}
        \pf\ By Lemma \ref{lm:topology:order:open}. We know $l$ is not least in
        $X$ because $a < l$.
      \end{proof}
      \step{<3>2}{\pick\ $b^*$ such that $b' < b^* < l$ \prove{$b^*$ is an
          upper
          bound for $\{ x \in A : x < b \}$}}
      \step{<3>3}{\pflet{$x \in A$ and $x < b$}}
      \step{<3>4}{$x \leq b^*$}
      \begin{proof}
        \pf\ If $b^* < x$ then $b^* < x \leq l$ and so $x \in B$ by
        \stepref{<3>1}.
      \end{proof}
      \qedstep
      \begin{proof}
        \pf\ This contradicts \stepref{<2>5}.
      \end{proof}
    \end{proof}
  \end{proof}
  \step{<1>2}{If $L$ is connected then $L$ is a linear continuum.}
  \begin{proof}
    \step{<2>1}{\assume{$L$ is connected}}
    \step{<2>2}{$L$ has the least upper bound property}
    \begin{proof}
      \step{<3>1}{\assume{for a contradiction $A \subseteq L$ is bounded above
with no least upper bound}}
      \step{<3>2}{\pflet{$U$ be the set of upper bounds of $A$}}
      \step{<3>3}{$U$ is open}
      \begin{proof}
        \step{<4>1}{\pflet{$u \in U$}}
        \step{<4>2}{\pick\ an upper bound $v$ for $A$ with $v < u$}
        \begin{proof}
          \pf\ $u$ is not the least upper bound for $A$ (\stepref{<3>1})
        \end{proof}
        \step{<4>3}{$u \in (v, +\infty) \subseteq U$}
      \end{proof}
      \step{<3>4}{\pflet{$V$ be the set of lower bounds of $U$}}
      \step{<3>5}{$U$ and $V$ form a separation of $L$}
      \begin{proof}
        \step{<4>1}{$V$ is open}
        \begin{proof}
          \pf\ Similar to \stepref{<3>3}.
        \end{proof}
        \step{<4>2}{$U$ and $V$ are disjoint}
        \begin{proof}
          \step{<5>1}{\assume{for a contradiction $x \in U \cap V$}}
          \step{<5>2}{\pick\ $u \in U$ such that $u < x$}
          \begin{proof}
            \pf\ $x$ is not the lowest upper bound of $A$
          \end{proof}
          \step{<5>3}{$x \leq u < x$}
        \end{proof}
        \step{<4>3}{$U \cup V = L$}
        \begin{proof}
          \step{<5>1}{\pflet{$x \in L \setminus U$}}
          \step{<5>2}{\pick\ $a \in A$ such that $x < a$}
          \step{<5>3}{$a \in V$}
          \step{<5>4}{$x \in V$}
        \end{proof}
      \end{proof}
    \end{proof}
    \step{<2>3}{For all $x,y \in L$, there exists $z \in L$ such that $x < z <
      y$}
    \begin{proof}
      \pf\ Otherwise $(-\infty, y)$ and $(x, +\infty)$ would form a separation
      of $L$.
    \end{proof}
  \end{proof}
    \qed
  \end{proof}

  \begin{cor}
    \label{cor:connected:real}
    The real line $\mathbb{R}$ is connected, and so is every ray and interval
    in $\mathbb{R}$.
  \end{cor}

  \begin{cor}
    The ordered square is connected.
  \end{cor}

\begin{cor}
  Not every closed subspace of a connected space is connected.
\end{cor}

\begin{proof}
  \pf\ The set $\{0,1\}$ is disconnected as a subspace of $\mathbb{R}$.
\end{proof}

\begin{cor}
  Not every open subspace of a connected space is connected.
\end{cor}

\begin{proof}
  \pf\ The space $\mathbb{R} \setminus \{ 0 \}$ is a disconnected open subspace of $\mathbb{R}$.
  \qed
\end{proof}

  \begin{thm}[Intermediate Value Theorem]
    Let $X$ be a connected space and $Y$ a linearly ordered set under the order
    topology. Let $f : X \rightarrow Y$ be continuous. Let $a, b \in X$ and $r
    \in Y$. If $f(a) < r < f(b)$, then there exists $c \in X$ such that $f(c) =
    r$.
  \end{thm}

  \begin{proof}
    \pf\ If not, then $f^{-1}((- \infty, r))$ and $f^{-1}((r, + \infty))$ would
    be a separation of $X$. \qed
  \end{proof}

   \begin{prop}
   Every connected regular space with more than one point is uncountable.
 \end{prop}

 \begin{proof}
  \pf
  \step{<1>1}{Every connected completely regular space with more than one point
is
    uncountable.}
  \begin{proof}
    \step{<2>1}{\pflet{$X$ be connected and completely regular and $a, b \in X$
        with $a \neq b$}}
    \step{<2>2}{\pick\ a continuous $f : X \rightarrow [0,1]$ such that $f(a) =
0$
      and $f(b) = 1$}
    \step{<2>3}{$f$ is surjective.}
    \begin{proof}
      \pf\ By the Intermediate Value Theorem.
    \end{proof}
  \end{proof}
  \step{<1>2}{Every connected regular space with more than one point is
    uncountable.}
  \begin{proof}
    \step{<2>1}{\assume{for a contradiction $X$ is connected, regular and
        countable with more than one point.}}
    \step{<2>2}{$X$ is Lindel\"{o}f}
    \step{<2>3}{$X$ is normal}
    \begin{proof}
      \pf\ By Theorem \ref{thm:topology:normal:regular_lindelof}
    \end{proof}
    \qedstep
    \begin{proof}
      \pf\ Contraditcting \stepref{<1>1}.
    \end{proof}
  \end{proof}
  \qed
 \end{proof}

  \begin{prop}
   $\overline{S_\Omega}$ is not conneced.
 \end{prop}

 \begin{proof}
   \pf\ $\{0\}$ is clopen. \qed
 \end{proof}

  \begin{prop}
   $\mathbb{R}_l$ is not connected.
 \end{prop}

 \begin{proof}
  \pf\ The set $[0, + \infty)$ is clopen. \qed
 \end{proof}

  \begin{prop}
   The space $\mathbb{R}^\omega$ under the uniform topology is not connected.
 \end{prop}

 \begin{proof}
  \pf\ The set of all bounded sequences and the set of all unbounded sequences
form a separation. \qed
 \end{proof}

 \begin{prop}
  The space $\mathbb{R}_K$ is connected.
\end{prop}

\begin{proof}
 \pf\ Easy. \qed
\end{proof}

  \section{Components and Local Connectedness}

  \begin{df}[(Connected) Component]
    Let $X$ be a topological space. Define an equivalence relation $\sim$ on
    $X$
    by: $x \sim y$ iff there exists a connected subspace $U \subseteq X$ such
    that $x \in U$ and $y \in U$. The \emph{(connected) components} of $X$ are
    the equivalence classes under $\sim$.

    We prove this is an equivalence relation.
  \end{df}

  \begin{proof}
    \pf
    \step{<1>1}{For all $x \in X$ we have $x \sim x$.}
    \begin{proof}
      \pf\ The subspace $\{x\} \subseteq X$ is connected.
    \end{proof}
    \step{<1>2}{For all $x, y \in X$, if $x \sim y$ then $y \sim x$.}
    \begin{proof}
      \pf\ Immediate from definitions.
    \end{proof}
    \step{<1>3}{For all $x, y, z \in X$, if $x \sim y$ and $y \sim z$ then $x
      \sim
      z$.}
    \begin{proof}
      \pf\ By Proposition \ref{prop:topology:connected:union}.
    \end{proof}
    \qed
  \end{proof}

  \begin{prop}
    \label{prop:topology:connected:subset}
    Let $X$ be a topological space. If $C \subseteq X$ is connected and
    nonempty, then there    exists a unique component $D$ of $X$ such that $C
    \subseteq D$.
  \end{prop}

  \begin{proof}
    \pf
    \step{<1>1}{\pick\ $a \in C$}
    \step{<1>2}{\pflet{$D$ be the $\sim$-equivalence class of $A$}}
    \step{<1>3}{$C \subseteq D$}
    \begin{proof}
      \pf\ For all $x \in C$ we have $a \sim x$ by definition.
    \end{proof}
    \step{<1>4}{$D$ is unique}
    \begin{proof}
      \pf\ This holds because the components are disjoint.
    \end{proof}
    \qed
  \end{proof}


  \begin{prop}[AC]
    \label{prop:topology:component:connected}
    Every component is connected.
  \end{prop}

  \begin{proof}
    \pf
    \step{<1>1}{\pflet{$C$ be a component of the topological space $X$}}
    \step{<1>2}{\pick\ $a \in C$}
    \step{<1>3}{For all $x \in C$, \pick\ a connected subspace $C_x$ of $X$
      containing both $a$ and $x$.}
    \begin{proof}
      \pf\ Such a $C_x$ exists since $a \sim x$.
    \end{proof}
    \step{<1>4}{$C = \bigcup_{x \in C} C_x$}
    \begin{proof}
      \pf\ This holds because $C_x \subseteq C$ by Proposition
      \ref{prop:topology:connected:subset}.
    \end{proof}
    \qedstep
    \begin{proof}
      \pf\ It follows that $C$ is connected by Proposition
      \ref{prop:topology:connected:union}.
    \end{proof}
    \qed
  \end{proof}

  \begin{prop}
    Every component is closed.
  \end{prop}

  \begin{proof}
    \pf\ From Theorem \ref{thm:topology:connected:closure}. \qed
  \end{proof}

   \begin{prop}
   The component of $\vec{a}$ in $\mathbb{R}^\omega$ under the uniform topology
   is $\{ \vec{b} : \vec{b} - \vec{a} \text{ is bounded} \}$.
 \end{prop}

 \begin{proof}
  \pf
  \step{<1>1}{$C = \{ \vec{b} : \vec{b} - \vec{a} \text{ is bounded} \}$ is
connected.}
\begin{proof}
  \step{<2>1}{\assume{$C = U \cup V$ is a separation of $C$ with $\vec{a} \in
U$}}
  \step{<2>2}{\pick\ $\vec{b} \in V$}
  \step{<2>3}{$\{ \epsilon : \epsilon \vec{b} + (1 - \epsilon) \vec{a} \in U
    \}$ and $\{ \epsilon : \epsilon \vec{b} + (1 - \epsilon) \vec{a} \in V \}$
    form a separation of $[0, 1]$}
\end{proof}
  \step{<1>2}{If $\vec{a}, \vec{b} \in C$ and $\vec{b} - \vec{a}$ is unbounded
    then $C$ is disconnected.}
  \begin{proof}
    \pf\ $\{ \vec{c} : \vec{c} - \vec{a} \text{ is bounded} \}$ and $\{ \vec{c}
    : \vec{c} - \vec{a} \text{ is unbounded} \}$
  \end{proof}
  \qed
 \end{proof}

  \begin{prop}
   Let $x, y \in \mathbb{R}^\omega$ under the box topology. Then $x$ and $y$
are in the same component iff $x-y$ is eventually zero.
 \end{prop}

 \begin{proof}
  \pf
  \step{<1>1}{For all $x \in \mathbb{R}^\omega$ the set $\{ y : x - y \text{ is
      eventulally zero} \}$ is connected}
  \begin{proof}
    \pf\ It is the union of the sets $C_N = \{ y : \forall n \geq N. y_n = 0
    \}$, each of which is connected because it is homeomorphic to
    $\mathbb{R}^{N-1}$.
  \end{proof}
  \step{<1>2}{If $x-y$ is not eventually zero then $x$ and $y$ are in different
components}
  \begin{proof}
    \step{<2>1}{\assume{$x-y$ is not eventually zero}}
    \step{<2>2}{Define $h : \mathbb{R}^\omega \rightarrow \mathbb{R}^\omega$
by:
      $h(z)_n = \begin{cases}
        z_n - x_n & \text{if } x_n = y_n \\
        n(z_n - x_n) / (y_n - x_n) & \text{if } x_n \neq y_n
      \end{cases}$}
    \step{<2>3}{$h$ is an automorphism of $\mathbb{R}^\omega$ under the box
      topology}
    \step{<2>4}{$h(x) = 0$}
    \step{<2>5}{$h(y)$ is unbounded}
    \qedstep
    \begin{proof}
      \pf\ The inverse image under $h$ of the set of bounded sequences and the
      set of unbounded sequences form a separation of $\mathbb{R}^\omega$ with
$x$ and $y$ in different sets.
    \end{proof}
    \qed
  \end{proof}
  \qed
 \end{proof}

  \section{Path Connectedness}

  \begin{df}[Path]
    Let $X$ be a topological space and $a, b \in X$. A \emph{path} from $a$ to
    $b$ is a continuous function $p : [0, 1] \rightarrow X$ such that $p(0) =
    a$ and $p(1) = b$.
  \end{df}

  \begin{df}[Path Connected]
    A topological space is \emph{path connected} iff there exists a path
    between any two points.
  \end{df}

  \begin{prop}
    \label{prop:topology:path_connected:connected}
    Every path connected space is connected.
  \end{prop}

  \begin{proof}
    \pf
    \step{<1>1}{\pflet{$X$ be a path connected space}}
    \step{<1>2}{\assume{for a contradiction $U$ and $V$ are a separation of
        $X$.}}
    \step{<1>3}{\pick\ $a \in U$ and $b \in V$}
    \step{<1>4}{\pick\ a path $p : [0,1] \rightarrow X$ from $a$ to $b$}
    \step{<1>5}{$p^{-1}(U)$ and $p^{-1}(V)$ form a separation of $[0,1]$.}
    \qedstep
    \begin{proof}
      \pf\ This contradicts Corollary \ref{cor:connected:real}.
    \end{proof}
  \end{proof}

  \begin{cor}
   $S_\Omega$ is not path connected.
  \end{cor}

  \begin{cor}
    $\overline{S_\Omega}$ is not path connected.
  \end{cor}

  \begin{cor}
    $\mathbb{R}_l$ is not path connected.
  \end{cor}

  \begin{cor}
   The Sorgenfrey plane is not path connected.
  \end{cor}

 \begin{cor}
   The space $\mathbb{R}^\omega$ under the uniform topology is not path connected.
connected.
 \end{cor}

 \begin{cor}
   The space $\mathbb{R}^\omega$ under the box topology is not path connected.
 \end{cor}

  \begin{prop}
    The long line is path connected.
  \end{prop}

  \begin{proof}
    \pf
    \step{<1>1}{\pflet{$a, b \in L$}}
    \step{<1>2}{\pick\ an ordinal $\alpha$ such that $a, b < (\alpha, 0)$}
    \step{<1>3}{There exists a path from $a$ to $b$}
    \begin{proof}
      \pf\ This holds because $[(0, 0), (\alpha, 0))$ is homeomorphic to $[0,
      1)$ by Proposition \ref{prop:order:long_line_zero_one}.
    \end{proof}
    \qed
  \end{proof}

\begin{cor}
  Not every closed subspace of a path connected space is path connected.
\end{cor}

\begin{proof}
  \pf\ Take any two-element subspace of the long line.
\end{proof}

\begin{cor}
  Not every open subspace of a path connected space is path connected.
\end{cor}

\begin{proof}
  \pf\ The space $\mathbb{R} \setminus \{ 0 \}$ is not path connected as a subspace of $\mathbb{R}$. \qed
\end{proof}

  \begin{df}[Path Component]
    Let $X$ be a topological space. Define an equivalence relation $\sim$ on
    $X$
    by: $x \sim y$ iff there exists a path from $x$ to $y$. The equivalence
    classes  are called the \emph{path components} of $X$.

    We prove this is an equivalence relation.
  \end{df}

  \begin{proof}
    \pf
    \step{<1>1}{For all $x \in X$ we have $x \sim x$}
    \begin{proof}
      \pf\ The constant path $p : [0,1] \rightarrow X$ where $p(t) = x$ is a
      path from $x$ to $x$.
    \end{proof}
    \step{<1>2}{If $x \sim y$ then $y \sim x$}
    \begin{proof}
      \pf\ If $p : [0,1] \rightarrow X$ is a path from $x$ to $y$ then $\lambda
      t. p(1-t)$ is a path from $y$ to $x$.
    \end{proof}
    \step{<1>3}{If $x \sim y$ and $y \sim z$ then $x \sim z$}
    \begin{proof}
      \step{<2>1}{\pflet{$p$ be a path from $x$ to $y$ and $q$ be a path from
          $y$
          to $z$.}}
      \step{<2>2}{\pflet{$r : [0,1] \rightarrow X$ where
          \[ r(t) = \begin{cases}
            p(2t) & \text{if } 0 \leq t \leq 1/2 \\
            q(2t-1) & \text{if } 1/2 \leq t \leq 1
          \end{cases} \]}}
      \step{<2>3}{$r$ is a path from $x$ to $z$.}
      \begin{proof}
        \pf\ $r$ is continuous by the Pasting Lemma.
      \end{proof}
    \end{proof}
    \qed
  \end{proof}

  \begin{prop}
    Every path component is path connected.
  \end{prop}

  \begin{proof}
    \pf\ By definition, if $x$ and $y$ are in the same path component then
    there
    is a path from $x$ to $y$. \qed
  \end{proof}

  \begin{prop}
    \label{prop:topology:path_connected:subset}
    If $A$ is a nonempty path connected subspace of the space $X$, then $A$ is
    included in a unique path component.
  \end{prop}

  \begin{proof}
    \pf
    \step{<1>1}{\pick\ $a \in A$}
    \step{<1>2}{\pflet{$C$ be the equivalence class of $a$ under $\sim$}}
    \step{<1>3}{$A \subseteq C$}
    \begin{proof}
      \pf\ For all $x \in A$, there exists a path from $a$ to $x$.
    \end{proof}
    \step{<1>4}{$C$ is unique}
    \begin{proof}
      \pf\ $C$ is the unique path component such that $a \in C$.
    \end{proof}
    \qed
  \end{proof}

  \begin{prop}
    Every path component is included in a component.
  \end{prop}

  \begin{proof}
    \pf\ From Propositions \ref{prop:topology:path_connected:connected} and
    \ref{prop:topology:connected:subset}. \qed
  \end{proof}

   \begin{prop}
  The ordered square is not path connected.
 \end{prop}

 \begin{proof}
  \pf
  \step{<1>1}{\assume{for a contradiction $p : [0,1] \rightarrow I_o^2$ is a
path
      from $(0, 0)$ to $(1, 1)$.}}
  \step{<1>2}{For all $x \in [0,1]$, $\inv{p}(\{ x \} \times (0, 1))$ is open
in
    $[0,1]$}
  \step{<1>3}{For all $x \in [0,1]$, \pick\ a rational $q_x \in \inv{p}(\{x\}
    \times (0,1))$}
  \step{<1>4}{$\{ q_x : x \in [0,1] \}$ is an uncountable set of rationals.}
  \qed
 \end{proof}

   \begin{prop}[AC]
     \label{prop:topology:path_conected:product}
  The product of a family of path connected spaces is path connected.
 \end{prop}

 \begin{proof}
  \pf
  \step{<1>1}{\pflet{$\{X_\alpha\}_{\alpha \in J}$ be a family of path
connected
      spaces and $a, b \in \prod_{\alpha \in J} X_\alpha$}}
  \step{<1>2}{For $\alpha \in J$, \pick\ a path $p_\alpha : [0,1] \rightarrow
    X_\alpha$ from $a_\alpha$ to $b_\alpha$}
  \step{<1>3}{Define $p : [0,1] \rightarrow \prod_{\alpha \in J} X_\alpha$ by
    $p(t)_\alpha = p_\alpha(t)$}
  \step{<1>4}{$p$ is a path from $a$ to $b$}
  \begin{proof}
    \pf\ By Theorem \ref{thm:topology:continuous:product}.
  \end{proof}
  \qed
 \end{proof}

 \begin{cor}
   For any set $I$, the space $\mathbb{R}^I$ in the product topology is path connected.
 \end{cor}

 \begin{prop}
  The space $\mathbb{R}_K$ is not path connected.
\end{prop}

\begin{proof}
 \pf
 \step{<1>1}{\assume{for a contradiction $p : [0,1] \rightarrow \mathbb{R}_K$
is
a
       path from 0 to 1}}
   \step{<1>2}{\pflet{$p : [0,1] \rightarrow \mathbb{R}_K$ be a path from 0 to
1}}
   \step{<1>3}{$p([0,1])$ is compact and connected in $\mathbb{R}_K$.}
   \begin{proof}
     \pf\ Theorem \ref{thm:topology:connected:image} and Proposition
\ref{prop:topology:compact:image}.
\end{proof}
   \step{<1>4}{$p([0,1])$ is connected in $\mathbb{R}$.}
   \begin{proof}
     \pf\ Corollary \ref{cor:topology:connected:finer}
   \end{proof}

   \step{<1>5}{$[0,1] \subseteq p([0,1])$}
   \begin{proof}
     \pf\ For any $x \in [0,1]$, if $x \notin p([0,1])$ then $p([0,1]) \cap
(-\infty, x)$ and $p([0,1]) \cap (x, + \infty)$ form a separation of $p([0,1])$.
\end{proof}
   \step{<1>6}{$[0,1]$ is compact in $\mathbb{R}_K$}
   \begin{proof}
     \pf\ Proposition \ref{prop:topology:compact:closed_is_compact}.
   \end{proof}
   \qedstep
   \begin{proof}
     \pf\ This contradicts Corollary \ref{cor:topology:compact_hausdorff:01K}.
   \end{proof}
   \qed
\end{proof}

\begin{prop}
 Let $f : X \rightarrow Y$ be continuous and surjective. If $X$ is path
connected then $Y$ is path connected.
\end{prop}

\begin{proof}
 \pf
 \step{<1>1}{\pflet{$a, b \in Y$}}
 \step{<1>2}{\pick\ $x, y \in X$ such that $f(x) = a$ and $f(y) = b$}
 \step{<1>3}{\pick\ a path $p : [0,1] \rightarrow X$ such that $p(0) = x$ and
   $p(1) = y$}
 \step{<1>4}{$f \circ p$ is a path from $a$ to $b$}
 \qed
\end{proof}

\begin{cor}
  Let $\{ X_\alpha \}_{\alpha \in J}$ be a family of non-empty topological
  spaces. If $\prod_{\alpha \in J} X_\alpha$ is path connected then each
$X_\alpha$ is path connected.
\end{cor}

  \section{Connected Subspaces of Euclidean Space}

  \begin{df}[Unit 2-Sphere]
    The \emph{unit 2-sphere} is $S^2 = \{ (x, y, z) \in \mathbb{R}^3 : x^2 +
    y^2 + z^2 = 1 \}$ as a subspace of $\mathbb{R}^3$.
  \end{df}

  \begin{df}[Unit Ball]
    For any $n \geq 1$, the \emph{closed unit ball} in $\mathbb{R}^n$ is
    \[ B^n = \{ \vec{x} \in \mathbb{R}^n : \| \vec{x} \| \leq 1 \} \enspace . \]
  \end{df}

  \begin{prop}
    Every open unit ball and closed unit ball in $\mathbb{R}^n$ is path
    connected.
  \end{prop}

  \begin{proof}
    \pf\ The straight line between any two points is a path in the ball. \qed
  \end{proof}

  \begin{df}[Punctured Euclidean Space]
    For $n \geq 1$, \emph{punctured Euclidean space} is $\mathbb{R}^n \setminus
    \{ \vec{0} \}$.
  \end{df}

  \begin{prop}
    Punctured Euclidean space in $\mathbb{R}^n$ is path connected iff $n > 1$.
  \end{prop}

  \begin{proof}
    \pf\ Easy. \qed
  \end{proof}

  \begin{df}[Unit Sphere]
    For $n \geq 1$, the \emph{unit sphere} $S^n$ is $\{ \vec{x} \in
    \mathbb{R}^{n+1} : \| \vec{x} \| = 1 \}$.
  \end{df}

  \begin{prop}
    In any number of dimensions, the unit sphere is path connected.
  \end{prop}

  \begin{proof}
    \pf\ Easy. \qed
  \end{proof}

  \begin{df}[Topologist's Sine Curve]
    The \emph{topologist's sine curve} is the closure of
    \[ S = \{ (x, \sin 1/x) : x \in \mathbb{R} \} \]
    in $\mathbb{R}^2$.
  \end{df}

  \begin{prop}
    The topologist's sine curve is connected.
  \end{prop}

  \begin{proof}
    \pf
    \step{<1>1}{$S = \{ (x, \sin 1/x) : x \in \mathbb{R} \}$ is connected.}
    \begin{proof}
      \step{<2>1}{The function $f : \mathbb{R} \rightarrow \mathbb{R}^2$ given
        by
        $f(x) = (x, \sin 1/x)$ is continuous.}
      \begin{proof}
        \pf\ By Theorem \ref{thm:topology:continuous:product}.
      \end{proof}
      \qedstep
      \begin{proof}
        \pf\ By Theorem \ref{thm:topology:connected:image}.
      \end{proof}
    \end{proof}
    \qedstep
    \begin{proof}
      \pf\ By Theorem \ref{thm:topology:connected:closure}.
    \end{proof}
    \qed
  \end{proof}

  \begin{prop}[CC]
    The topologist's sine curve is not path connected.
  \end{prop}

  \begin{proof}
    \pf
    \step{<1>1}{\pflet{$S = \{ (x, \sin 1/x) : x \in \mathbb{R} \}$}}
    \step{<1>2}{\assume{for a contradiction $p : [0, 1] \rightarrow
        \overline{S}$
        is a path from $(0, 0)$ to $(1, \sin 1)$.}}
    \step{<1>3}{$p^{-1}(\{0\} \times [-1, 1])$ is closed.}
    \step{<1>4}{$p^{-1}(\{0\} \times [-1 ,1])$ has a greatest element.}
    \begin{proof}
      \pf\ By Lemma \ref{lm:topology:continuum:closed}.
    \end{proof}
    \step{<1>5}{\pflet{$q : [0, 1] \rightarrow \overline{S}$ be a path such
        that:
        \begin{itemize}
          \item $q(0) \in \{ 0 \} \times [-1, 1]$
          \item $q(x) \in S$ for $x > 0$
        \end{itemize}}}
    \begin{proof}
      \pf\ Let $b$ be greatest in $p^{-1}(\{0\} \times [-1 ,1])$. Then $q$ is
      obtained by rescaling $p$ restricted to $[b, 1]$.
    \end{proof}
    \step{<1>6}{\pflet{$q(t) = (x(t), y(t))$ for $0 \leq t \leq 1$}}
    \step{<1>7}{$x(0) = 0$}
    \step{<1>8}{$x(t) > 0$ for $t > 0$}
    \step{<1>9}{$y(t) = \sin 1/x(t)$ for $t > 0$}
    \step{<1>10}{There exists a sequence $t_n \in [0, 1]$ such that $t_n
      \rightarrow 0$ as $n \rightarrow \infty$ and $y(t_n) = (-1)^n$ for all
      $n$.}
    \begin{proof}
      \step{<2>1}{For each $n$, \pick\ $u_n$ such that $0 < u_n < x(1/n)$ and
        $\sin 1/u_n = (-1)^n$.}
      \begin{proof}
        \pf\ Such a $u_n$ exists because $\sin 1/x$ takes values 1 and -1
        infinitely often in $(0, x(1/n))$.
      \end{proof}
      \step{<2>2}{For each $n$, \pick\ $t_n$ such that $0 < t_n < 1/n$ and
        $x(t_n)
        = u$}
      \begin{proof}
        \pf\ By the Intermediate Value Theorem.
      \end{proof}
    \end{proof}
    \qedstep
    \begin{proof}
      \pf\ This is a contradiction as $y(t_n) \rightarrow y(0)$ as $n
      \rightarrow \infty$ because $y$ is continuous.
    \end{proof}
    \qed
  \end{proof}

  \section{Local Connectedness}

  \begin{df}[Locally Connected]
    Let $X$ be a topological space and $x \in X$. Then $X$ is \emph{locally
      connected} at $x$ iff every neighbourhood of $x$ includes a connected
    neighbourhood of $x$.

    The space $X$ is \emph{locally connected} iff it is locally connected at
    every point.
  \end{df}

   \begin{prop}
  $S_\Omega$ is not locally connected.
 \end{prop}

 \begin{proof}
  \pf\ There is no connected neighbourhood of $\omega$. \qed
 \end{proof}

     \begin{prop}
     $\overline{S_\Omega}$ is not locally connected.
 \end{prop}

 \begin{proof}
  \pf\ There is no connected neighbourhood of $\omega$. \qed
 \end{proof}

  \begin{prop}
  	For any set $I$,
  	the space $\mathbb{R}^I$ is locally connected.
  \end{prop}

  \begin{proof}
  	\pf\ Every basic open set is the product of connected spaces, hence
  	connected. \qed
  \end{proof}

 \begin{prop}
    Let $X$ be a topological space. Then $X$ is locally connected if and only
    if, for every open set $U$ in $X$, every component of $U$ is open in $X$.
  \end{prop}

  \begin{proof}
    \pf
    \step{<1>1}{If $X$ is locally connected then, for every open set $U$ in
      $X$,
      every component of $U$ is open in $X$.}
    \begin{proof}
      \step{<2>1}{\assume{$X$ is locally connected.}}
      \step{<2>2}{\pflet{$U$ be open in $X$.}}
      \step{<2>3}{\pflet{$C$ be a component of $U$.}}
      \step{<2>4}{\pflet{$x \in C$} \prove{$C$ is a neighbourhood of $x$}}
      \step{<2>5}{$U$ is a neighbourhood of $x$ in $X$.}
      \begin{proof}
        \pf\ From \stepref{<2>2}, \stepref{<2>3} and \stepref{<2>4}.
      \end{proof}
      \step{<2>6}{\pick\ a connected neighbourhood $V$ of $x$ such that $V
        \subseteq U$.}
      \begin{proof}
        \pf\ Using \stepref{<2>1}.
      \end{proof}
      \step{<2>7}{$V \subseteq C$}
      \begin{proof}
        \pf\ By Proposition \ref{prop:topology:connected:subset}.
      \end{proof}
      \step{<2>8}{$C$ is a neighbourhood of $x$}
      \begin{proof}
        \pf\ By Proposition \ref{prop:topology:neighbourhood:monotone}.
      \end{proof}
      \qedstep
      \begin{proof}
        \pf\ By Proposition \ref{prop:topology:neighbourhood:open}.
      \end{proof}
    \end{proof}
    \step{<1>2}{If, for every open set $U$ in $X$, every component of $U$ is
      open
      in $X$, then $X$ is locally connected.}
    \begin{proof}
      \step{<2>1}{\assume{For every open set $U$ in $X$, every component of $U$
          is
          open in $X$.}}
      \step{<2>2}{\pflet{$x \in X$ and $N$ be a neighbourhood of $x$}}
      \step{<2>3}{\pick\ $U$ open such that $x \in U \subseteq N$}
      \step{<2>4}{\pflet{$C$ be the component of $U$ that contains $x$}}
      \step{<2>5}{$C$ is open in $X$}
      \begin{proof}
        \pf\ By \stepref{<2>1}.
      \end{proof}
      \step{<2>6}{$C$ is a connected neighbourhood of $x$ that is included in
        $N$}
    \end{proof}
    \qed
  \end{proof}

  \begin{cor}
   In a locally connected space, every component is open.
  \end{cor}

  \begin{cor}
    The space $\mathbb{R}^\omega$ under the box topology is not locally
connected.
  \end{cor}

\begin{cor}
  Not every closed subspace of a locally connected space is locally connected.
\end{cor}

\begin{proof}
  \pf\ The topologist's sine curve is not locally connected. \qed
\end{proof}

   \begin{prop}
   $S_\Omega \times \overline{S_\Omega}$ is not locally connected.
 \end{prop}

 \begin{proof}
  $(\omega, \omega)$ has no connected neighbourhood. \qed
 \end{proof}

 \begin{prop}
   $\mathbb{R}_l$ is not locally connected.
 \end{prop}

 \begin{proof}
  \pf\ 0 has no connected neighbourhood. \qed
 \end{proof}

  \begin{prop}
  The Sorgenfrey plane is not locally connected.
 \end{prop}

 \begin{proof}
  \pf\ Any basic open set $[a,b) \times [c,d)$ can be separated into $[a,b)
\times [c,e)$ and $[a,b) \times [e,d)$ for some $c < e < d$. \qed
 \end{proof}

  \begin{prop}
   The space $\mathbb{R}^\omega$ under the uniform topology is locally
connected.
 \end{prop}

 \begin{proof}
   \pf\ For any neighbourhood $U$ of a point $x$, the neighbourhood $U \cap \{
   y : y - x \text{ is bounded} \}$ is connected. \qed
 \end{proof}

 \begin{prop}
  The space $\mathbb{R}_K$ is not locally connected.
\end{prop}

\begin{proof}
 \pf\ The open set $(-1,1) - K$ does not include a connected neighbourhood of
0. \qed
\end{proof}

\begin{prop}
  Every open subspace of a locally connected space is locally connected.
\end{prop}

\begin{proof}
  \pf\ Follows easily from definition. \qed
\end{proof}

\begin{prop}[AC]
  The product of a family of locally connected spaces is locally connected.
\end{prop}

\begin{proof}
  \pf
  \step{<1>1}{\pflet{$\{X_\alpha\}_{\alpha \in J}$ be a family of locally connected spaces and $\vec{x} \in \prod_{\alpha \in J} X_\alpha$}}
  \step{<1>2}{\pflet{$\prod_{\alpha \in J} U_\alpha$ be any basic neighbourhood of $\vec{x}$, where each $U_\alpha$ is open in $X_\alpha$, and $U_\alpha = X_\alpha$ except for $\alpha = \alpha_1, \ldots, \alpha_n$}}
  \step{<1>3}{For $\alpha \in J$, \pick\ a connected neighbourhood $C_\alpha$ of $x_\alpha$ with $C_\alpha \subseteq U_\alpha$}
  \step{<1>4}{$\prod_{\alpha \in J} C_\alpha$ is connected}
  \begin{proof}
    \pf\ Proposition \ref{prop:topology:connected:product}
  \end{proof}
  \qed
\end{proof}

\begin{prop}
  Every discrete space is locally connected.
\end{prop}

\begin{proof}
  \pf\ For any point $x$, the set $\{x\}$ is a connected neighbourhood of $x$. \qed
\end{proof}

\begin{cor}
  The continuous image of a locally connected space is not necessarily locally connected.
\end{cor}

  \section{Local Path Connectedness}

  \begin{df}[Locally Path Connected]
    Let $X$ be a topological space and $x \in X$. Then $X$ is \emph{locally
      path connected at $x$} iff every neighbourhood of $x$ includes a path
    connected neighbourhood of $x$.

    The space $X$ is \emph{locally path connected} iff it is locally path
    connected
    at every point.
  \end{df}

   \begin{prop}
  $S_\Omega$ is not locally path connected.
 \end{prop}

 \begin{proof}
  \pf\ There is no path connected neighbourhood of $\omega$. \qed
 \end{proof}

  \begin{prop}
   $\overline{S_\Omega}$ is not locally path connected.
 \end{prop}

 \begin{proof}
  \pf\ There is no path connected neighbourhood of $\omega$. \qed
 \end{proof}

 \begin{prop}
   Not every closed subspace of a locally path connected space is locally path connected.
 \end{prop}

 \begin{proof}
   \pf\ The topologist's sine curve is not loally path connected. \qed
 \end{proof}

 \begin{prop}
   Every open subspace of a locally path connected space is locally path connected.
 \end{prop}

 \begin{proof}
   \pf\ Follows easily from definition. \qed
 \end{proof}

 \begin{prop}
  Every locally path connected space is locally connected.
 \end{prop}

 \begin{proof}
  \pf\ From Proposition \ref{prop:topology:path_connected:connected}. \qed
 \end{proof}

 \begin{cor}
   $\mathbb{R}_l$ is not locally path connected.
 \end{cor}

 \begin{cor}
  The Sorgenfrey plane is not locally path connected.
 \end{cor}

 \begin{cor}
   The space $\mathbb{R}^\omega$ under the box topology is not locally path
connected.
 \end{cor}

 \begin{cor}
   The space $\mathbb{R}_K$ is not locally path connected.
 \end{cor}

\begin{cor}
The topologist's sine curve is not locally path connected.
\end{cor}

\begin{prop}[AC]
  The product of a family of locally path connected spaces is locally path connected.
\end{prop}

\begin{proof}
  \pf
  \step{<1>1}{\pflet{$\{X_\alpha\}_{\alpha \in J}$ be a family of locally connected spaces and $\vec{x} \in \prod_{\alpha \in J} X_\alpha$}}
  % TODO Lemma about bases
  \step{<1>2}{\pflet{$\prod_{\alpha \in J} U_\alpha$ be any basic neighbourhood of $\vec{x}$, where each $U_\alpha$ is open in $X_\alpha$, and $U_\alpha = X_\alpha$ except for $\alpha = \alpha_1, \ldots, \alpha_n$}}
  \step{<1>3}{For $\alpha \in J$, \pick\ a path connected neighbourhood $C_\alpha$ of $x_\alpha$ with $C_\alpha \subseteq U_\alpha$}
  \step{<1>4}{$\prod_{\alpha \in J} C_\alpha$ is path connected}
  \begin{proof}
    \pf\ Proposition \ref{prop:topology:path_connected:product}
  \end{proof}
  \qed
\end{proof}

  \begin{prop}
    \label{prop:topology:locally_path_connected:open}
    Let $X$ be a topological space. Then $X$ is locally path connected if and
    only
    if, for every open set $U$ in $X$, every path component of $U$ is open in
    $X$.
  \end{prop}

  \begin{proof}
    \pf
    \step{<1>1}{If $X$ is locally path connected then, for every open set $U$
      in
      $X$,
      every path component of $U$ is open in $X$.}
    \begin{proof}
      \step{<2>1}{\assume{$X$ is locally path connected.}}
      \step{<2>2}{\pflet{$U$ be open in $X$.}}
      \step{<2>3}{\pflet{$C$ be a path component of $U$.}}
      \step{<2>4}{\pflet{$x \in C$} \prove{$C$ is a neighbourhood of $x$}}
      \step{<2>5}{$U$ is a neighbourhood of $x$ in $X$.}
      \begin{proof}
        \pf\ From \stepref{<2>2}, \stepref{<2>3} and \stepref{<2>4}.
      \end{proof}
      \step{<2>6}{\pick\ a path connected neighbourhood $V$ of $x$ such that $V
        \subseteq U$.}
      \begin{proof}
        \pf\ Using \stepref{<2>1}.
      \end{proof}
      \step{<2>7}{$V \subseteq C$}
      \begin{proof}
        \pf\ By Proposition \ref{prop:topology:path_connected:subset}.
      \end{proof}
      \step{<2>8}{$C$ is a neighbourhood of $x$}
      \begin{proof}
        \pf\ By Proposition \ref{prop:topology:neighbourhood:monotone}.
      \end{proof}
      \qedstep
      \begin{proof}
        \pf\ By Proposition \ref{prop:topology:neighbourhood:open}.
      \end{proof}
    \end{proof}
    \step{<1>2}{If, for every open set $U$ in $X$, every path component of $U$
      is
      open
      in $X$, then $X$ is locally path connected.}
    \begin{proof}
      \step{<2>1}{\assume{For every open set $U$ in $X$, every path component
          of
          $U$
          is
          open in $X$.}}
      \step{<2>2}{\pflet{$x \in X$ and $N$ be a neighbourhood of $x$}}
      \step{<2>3}{\pick\ $U$ open such that $x \in U \subseteq N$}
      \step{<2>4}{\pflet{$C$ be the path component of $U$ that contains $x$}}
      \step{<2>5}{$C$ is open in $X$}
      \begin{proof}
        \pf\ By \stepref{<2>1}.
      \end{proof}
      \step{<2>6}{$C$ is a path connected neighbourhood of $x$ that is included
        in
        $N$}
    \end{proof}
    \qed
  \end{proof}

  \begin{thm}[AC]
    Let $X$ be a topological space. If $X$ is locally path connected, then its
    components and its path components are the same.
  \end{thm}

  \begin{proof}
    \pf
    \step{<1>1}{\pflet{$P$ be a path component of $X$}}
    \step{<1>2}{\pflet{$C$ be the component such that $P \subseteq C$}
      \prove{$P
        =
        C$}}
    \step{<1>3}{\pflet{$Q = C \setminus P$}}
    \step{<1>4}{$P$ is open in $X$}
    \begin{proof}
      \pf\ By Proposition \ref{prop:topology:locally_path_connected:open}.
    \end{proof}
    \step{<1>5}{$Q$ is open in $X$}
    \begin{proof}
      \pf\ By Proposition \ref{prop:topology:locally_path_connected:open} since
      $Q$ is the union of the path components included in $C$ other than $P$.
    \end{proof}
    \step{<1>6}{$Q = \emptyset$}
    \begin{proof}
      \pf\ Otherwise $P$ and $Q$ would form a separation of $C$, contradicting
      \ref{prop:topology:component:connected}.
    \end{proof}
    \qed
  \end{proof}

   \begin{prop}
   $S_\Omega \times \overline{S_\Omega}$ is not locally path connected.
 \end{prop}

 \begin{proof}
  \pf\ $(\omega, \omega)$ has no path connected neighbourhood. \qed
 \end{proof}

 \begin{prop}
  The ordered square is not locally path connected.
 \end{prop}

 \begin{proof}
  \pf
  \step{<1>1}{\assume{for a contradiction $(1/2, 0)$ has a path connected
      neighbourhod $U$}}
  \step{<1>2}{\pick\ $a < 1/2$ such that $((a, 1), (1/2, 0)) \subseteq U$}
  \step{<1>3}{\pflet{$p : [0,1] \rightarrow I_o^2$ be a path from $(a, 1)$ to
      $(1/2, 0)$}}
  \step{<1>4}{For every $x$ such that $a < x < 1/2$, \pick\ a rational $q_x$
such
    that $p(q_x) \in ((x,0), (x,1))$}
  \step{<1>5}{$\{ q_x : a < x < 1/2 \}$ is an uncountable set of rationals.}
  \qed
 \end{proof}

 \begin{prop}
 	For any set $I$,
   the space $\mathbb{R}^I$ is locally path connected.
 \end{prop}

 \begin{proof}
  \pf\ Every basic open set is the product of path connected spaces, hence path
connected. \qed
 \end{proof}

 \begin{prop}
   The space $\mathbb{R}^\omega$ under the uniform topology is locally path
connected.
 \end{prop}

 \begin{proof}
  \pf\ Its components and path components are the same. \qed
 \end{proof}

 \begin{prop}
   Every discrete space is locally path connected.
 \end{prop}

 \begin{proof}
   \pf\ For any point $x$, the set $\{x\}$ is a path connected neighbourhood of $x$. \qed
 \end{proof}

 \begin{cor}
   The continuous image of a locally path connected space is not necessarily locally path connected.
 \end{cor}


 \section{Weak Local Connectedness}


  \begin{df}[Weakly Locally Connected]
    Let $X$ be a topological space and $x \in X$. Then $X$ is \emph{weakly
      locally connected at $x$} iff every neighbourhood of $x$ contains a
    connected subspace that contains a neighbourhood of $x$.
  \end{df}

  \chapter{Compact Spaces}

  \section{Countable Compactness}

  \begin{df}[Countably Compact]
    A topological space is \emph{countably compact} iff every countable open
    covering has a finite subcovering.
  \end{df}

  \section{Limit Point Compactness}

  \begin{df}[Limit Point Compact]
    A space is \emph{limit point compact} iff every infinite set has a
    limit point.
  \end{df}

   \begin{prop}[CC]
   $S_\Omega \times \overline{S_\Omega}$ is limit point compact.
 \end{prop}

 \begin{proof}
  \pf
  \step{<1>1}{\pflet{$A \subseteq S_\Omega \times \overline{S_\Omega}$ be
      infinite}}
  \step{<1>2}{\case{$\pi_1(A)$ is finite.}}
  \begin{proof}
    \step{<2>1}{\pick\ $x$ such that there are infinitely many $y$ such that
$(x,
      y) \in A$}
    \step{<2>2}{\pick\ a limit point $l$ of $\{ y : (x,y) \in A \}$}
    \step{<2>3}{$(x, l)$ is a limit point of $A$}
  \end{proof}
  \step{<1>3}{\case{$\pi_1(A)$ is infinite.}}
  \begin{proof}
    \step{<2>1}{\pick\ a limit point $l$ of $\pi_1(A)$.}
    \step{<2>2}{$l$ is a limit ordinal}
    \step{<2>3}{\pick\ a countable sequence $x_n$ with limit $l$}
    \step{<2>4}{For $n \geq 1$, \pick\ $a_n > x_n$ and $y_n$ such that $(a_n,
y_n)
      \in A$}
    \step{<2>5}{\case{$\{ y_n : n \geq 1 \}$ is finite}}
    \begin{proof}
      \step{<3>1}{\pick\ $y$ such that $y = y_n$ for infinitely many $n$}
      \step{<3>2}{$(l, y)$ is a limit point for $A$}
    \end{proof}
  <\step{<2>6}{\case{$\{ y_n : n \geq 1 \}$ is infinite}}
    \begin{proof}
      \step{<3>1}{\pick\ a limit point $m$ for $\{ y_n : n \geq 1 \}$}
      \step{<3>2}{$(l, m)$ is a limit point for $A$}
    \end{proof}
  \end{proof}
  \qed
 \end{proof}

  \begin{prop}
  The Sorgenfrey plane is not limit point compact.
 \end{prop}

 \begin{proof}
   \pf\ $\mathbb{Z}^2$ has no limit point. \qed
 \end{proof}

  \begin{prop}
   The space $\mathbb{R}^\omega$ under the box topology is not limit point
compact.
 \end{prop}

 \begin{proof}
  \pf The set of all constant sequences of integers is an infinite set with no
limit point. \qed
 \end{proof}

 \begin{prop}
   Not every open subspace of a limit point compact space is limit point compact.
 \end{prop}

 \begin{proof}
   \pf\ The space $[0,1]$ is limit point compact but $(0,1)$ is not. \qed
 \end{proof}

 \begin{prop}
   The product of two limit point compact spaces is not necessarily limit point compact.
 \end{prop}

 \begin{proof}
   \pf\ See Steen and Seebach \emph{Countexamples in Topology} Example 112. \qed
 \end{proof}

 \begin{prop}
   The continuous image of a limit point comapct space is not necessarily limit point comapct.
 \end{prop}

 \begin{proof}
   \pf\ Let $Y$ be a two-point set under the indiscrete topology. Then $\mathbb{N}
   \times Y$ is limit point compact, but $\mathbb{N}$ is not. \qed
 \end{proof}

  \section{Lindel\"{o}f Spaces}

    \begin{df}[Lindel\"{o}f Space]
    A topological space $X$ is \emph{Lindel\"{o}f} iff every open covering has
    a countable subcovering.
  \end{df}

    \begin{thm}[CC]
      \label{thm:topology:lindelof:second_countable}
    Every second countable space is Lindel\"{o}f.
  \end{thm}

  \begin{proof}
   \pf
   \step{<1>1}{\pflet{$X$ be a second countable space}}
   \step{<1>2}{\pick\ a countable basis $\mathcal{B}$ for $X$.}
   \step{<1>3}{\pflet{$\mathcal{A}$ be an open cover of $X$}}
   \step{<1>4}{For every $B \in \mathcal{B}$ such that there exists $U \in
     \mathcal{A}$ such that $B \subseteq U$, \pick\ $U_B \in \mathcal{A}$ such
     that $B \subseteq U_B$}
   \step{<1>5}{$\{ U_B : B \in \mathcal{B}, \exists U \in \mathcal{A}. B
\subseteq
     U \}$ covers $X$.}
   \begin{proof}
     \step{<2>1}{\pflet{$x \in X$}}
     \step{<2>2}{\pick\ $U \in \mathcal{A}$ such that $x \in U$}
     \step{<2>3}{\pick\ $B \in \mathcal{B}$ such that $x \in B \subseteq U$}
     \step{<2>4}{$x \in U_B$}
   \end{proof}
   \qed
  \end{proof}

  \begin{cor}
    The space $\mathbb{R}^\omega$ is Lindel\"{o}f.
  \end{cor}

  \begin{cor}
    The space $\mathbb{R}_K$ is Lindel\"{o}f.
  \end{cor}

    \begin{prop}
    The space $S_\Omega$ is not Lindel\"{o}f.
  \end{prop}

  \begin{proof}
    \pf $\{ (- \infty, \alpha) : \alpha \in S_\Omega \}$ is an open cover that
has no countable subcover. \qed
  \end{proof}

    \begin{prop}[CC]
    The space $\overline{S_\Omega}$ is Lindel\"{o}f.
  \end{prop}

  \begin{proof}
   \pf
   \step{<1>1}{\pflet{$\mathcal{A}$ be an open cover of $\overline{S_\Omega}$}}
   \step{<1>2}{\pick\ $U \in \mathcal{A}$ such that $\Omega \in U$}
   \step{<1>3}{\pick\ $\alpha < \Omega$ such that $(\alpha, \Omega] \subseteq U$}
   \step{<1>4}{For $\beta \leq \alpha$, \pick\ $U_\beta \in \mathcal{A}$ such that
     $\beta \in U_\beta$}
   \step{<1>5}{$\{ U \} \cup \{ U_\beta : \beta \leq \alpha \}$ is a countable
     subcover of $\mathcal{A}$.}
   \qed
  \end{proof}

  \begin{prop}[CC]
    The continuous image of a Lindel\"{o}f space is Lindel\"{o}f.
  \end{prop}

  \begin{proof}
   \pf
   \step{<1>1}{\pflet{$X$ be a Lindel\"{o}f space, $Y$ a space and $f : X
       \rightarrow Y$ continuous.}}
   \step{<1>2}{\pflet{$\mathcal{A}$ be an open covering of $Y$}}
   \step{<1>3}{$\{ \inv{f}(V) : V \in \mathcal{A} \}$ is an open covering of
$X$}
   \step{<1>4}{\pick\ a countable subcovering $\{ \inv{f}(V_1), \inv{f}(V_2),
     \ldots \}$ of $\{ \inv{f}(V) : V \in \mathcal{A} \}$}
   \step{<1>5}{$\{ V_1, V_2, \ldots \}$ is a countable subcovering of
     $\mathcal{A}$}
   \qed
  \end{proof}

 \begin{prop}
   The Sorgenfrey plane is not Lindel\"{o}f.
 \end{prop}

 \begin{proof}
  \pf
  \step{<1>1}{\pflet{$L = \{ (x, -x) : x \in \mathbb{R} \}$}}
  \step{<1>2}{$L$ is closed in $\mathbb{R}_l^2$}
  \begin{proof}
    \step{<2>1}{\pflet{$(x, y) \notin L$, so $y \neq -x$} \prove{There exists a
        neighbourhood $U$ of $(x,y)$ that does not intersect $L$}}
    \step{<2>2}{\case{$y > -x$}}
    \begin{proof}
      \pf\ In this case, take $U = [x,+\infty) \times [y, + \infty)$
    \end{proof}
    \step{<2>3}{\case{$y < -x$}}
    \begin{proof}
      \pf\ In this case, take $U = [x,(x-y)/2) \times [y,(y-x)/2)$.
    \end{proof}
  \end{proof}
  \step{<1>3}{\pflet{$\mathcal{U} = \{ \mathbb{R}_l^2 \setminus L \} \cup \{
[a,b)
      \times [-a,d) : a,b,d \in \mathbb{R} \}$}}
  \step{<1>4}{$\mathcal{U}$ is an open covering of $\mathbb{R}_l^2$}
  \step{<1>5}{No countable subset of $\mathcal{U}$ covers $\mathbb{R}_l^2$}
  \begin{proof}
    \pf\ Every set $[a,b) \times [-a,d)$ intersects $L$ in exactly one point,
namely $(a, -a)$.
  \end{proof}
  \qed
 \end{proof}

 \begin{cor}
  The Sorgenfrey plane is not second countable.
 \end{cor}

\begin{cor}
  The product of two Lindel\"{o}f spaces is not necessarily Lindel\"{o}f.
\end{cor}

\begin{prop}
	The space $\mathbb{R}^\omega$ under the box topology is not Lindel\"{o}f.
\end{prop}

\begin{proof}
	\pf\ The set $\{ \prod_{n=0}^\infty (a_n, a_n + 1) : \forall n. a_n \in \mathbb{Z} \}$ covers the space but has no countable subcover. \qed
\end{proof}

\begin{prop}
  Not every open subspace of a Lindel\"{o}f space is Lindel\"{o}f.
\end{prop}

\begin{proof}
  \pf\ The ordered square is Lindel\"{o}f but the subspace $[0,1]
  times (0,1)$ is not. \qed
\end{proof}

  \section{Compactness}

  \begin{df}[Compact]
    A topological space is \emph{compact} iff every open cover has a finite
    subcover.
  \end{df}

   \begin{prop}
     \label{prop:topology:compact:S_omega}
  $S_\Omega$ is not compact.
 \end{prop}

 \begin{proof}
   \pf\ The open covering $\{ (- \infty, \alpha) : \alpha \in S_\Omega \}$ has
   no finite subcovering. \qed
 \end{proof}

  \begin{prop}
   $\mathbb{R}_l$ is not compact.
 \end{prop}

 \begin{proof}
   \pf\ $\{ [n, n+1) : n \in \mathbb{Z} \}$ has no finite subcover. \qed
 \end{proof}

 \begin{prop}
   The space $\mathbb{R}^\omega$ under the box topology is not compact.
 \end{prop}

 \begin{proof}
   \pf\ The set $\{ \prod_{n=0}^\infty (a_n, a_n+1) : n \in \mathbb{Z}
\}$ is a cover that has no finite subcover. \qed
 \end{proof}

  \begin{prop}
    \label{prop:topology:compact:subspace}
    Let $Y$ be a subspace of $X$. Then $Y$ is compact if and only if every
    covering of $Y$ by sets open in $X$ contains a finite subcollection
    covering
    $Y$.
  \end{prop}

  \begin{proof}
    \pf
    \step{<1>1}{If $Y$ is compact then every covering of $Y$ by sets open in
      $X$
      contains a finite subcollection covering $Y$.}
    \begin{proof}
      \step{<2>1}{\assume{$Y$ is compact.}}
      \step{<2>2}{\pflet{$\mathcal{A}$ be a covering of $Y$ by sets open in
          $X$.}}
      \step{<2>3}{$\{ U \cap Y : U \in \mathcal{A} \}$ is an open covering of
        $Y$.}
      \step{<2>4}{\pick\ a finite subcovering $V_1$, \ldots, $V_n$ of $\{ U
        \cap
        Y
        : U \in \mathcal{A} \}$}
      \step{<2>5}{For $1 \leq i \leq n$, \pick\ $U_i \in \mathcal{A}$ such that
        $V_i = U_i \cap Y$.}
      \step{<2>6}{$\{ U_1, \ldots, U_n \}$ is a finite subset of $\mathcal{A}$
        that
        covers $Y$.}
    \end{proof}
    \step{<1>2}{If every covering of $Y$ by sets open in $X$ contains a finite
      subcollection covering $Y$ then $Y$ is compact.}
    \begin{proof}
      \step{<2>1}{\assume{Every covering of $Y$ by sets open in $X$ contains a
          finite subcollection covering $Y$.}}
      \step{<2>2}{\pflet{$\mathcal{A}$ be an open covering of $Y$}}
      \step{<2>3}{\pflet{$\mathcal{B} = \{ U \text{ open in } X : U \cap Y \in
          \mathcal{A} \}$}}
      \step{<2>4}{$\mathcal{B}$ covers $Y$}
      \step{<2>5}{\pick\ a finite subcollection $\{ U_1, \ldots, U_n \}
        \subseteq
        \mathcal{B}$ that covers $Y$}
      \step{<2>6}{$\{ U_1 \cap Y, \ldots, U_n \cap Y \}$ is a finite subcover
        of
        $\mathcal{A}$.}
    \end{proof}
    \qed
  \end{proof}

  \begin{prop}
    \label{prop:topology:compact:closed_is_compact}
    Every closed subspace of a compact space is compact.
  \end{prop}

  \begin{proof}
    \pf
    \step{<1>1}{\pflet{$X$ be a compact space and $Y \subseteq X$ be closed.}}
    \step{<1>2}{\pflet{$\mathcal{A}$ be a covering of $Y$ by spaces open in
        $X$}}
    \step{<1>3}{$\mathcal{A} \cup \{ X \setminus Y \}$ is an open covering of
      $X$.}
    \step{<1>4}{\pick\ a finite subcovering $\{ U_1, \ldots, U_n \}$ or $\{
      U_1,
      \ldots, U_n, X \setminus Y \}$}
    \step{<1>5}{$\{ U_1, \ldots, U_n \}$ is a finite subset of $\mathcal{A}$
      that
      covers $Y$.}
    \qedstep
    \begin{proof}
      \pf\ Proposition \ref{prop:topology:compact:subspace}.
    \end{proof}
    \qed
  \end{proof}

\begin{cor}
  Not every compact Hausdorff space is connected.
\end{cor}

\begin{proof}
  \pf\ The space $[0,1] \cup [2,3]$ is compact Hausdorff and disconnected. \qed
\end{proof}

\begin{cor}
  Not every compact Hausdorff space is path connected.
\end{cor}

\begin{cor}
  Not every compact Hausdorff space is locally connected.
\end{cor}

\begin{proof}
  The space $[0,1] \cap \mathbb{Q}$ is not locally connected.
\end{proof}

\begin{cor}
  Not every compact Hausdorff space is locally path connected.
\end{cor}

\begin{prop}
  Not every open subspace of a compact space is compact.
\end{prop}

\begin{proof}
  \pf\ The space $[0,1]$ is compact but $(0,1)$ is not. \qed
\end{proof}

  \begin{lm}
    \label{lm:topology:compact:regular}
    If $Y$ is a compact subspace of the Hausdorff space $X$ and $a \notin Y$,
    then there exist disjoint open sets $U$ and $V$ of $X$ containing $a$ and
    $Y$,
    respectively.
  \end{lm}

  \begin{proof}
    \pf
    \step{<1>1}{For $y \in Y$, there exist disjoint open sets $U$ and $V$ such
      that
      $a \in U$ and $y \in V$.}
    \step{<1>2}{$\{ V \text{ open in } X : \exists U \text{ open and disjoint
        from
      } V. a \in U \}$ is a covering of $Y$ by open sets in $X$.}
    \step{<1>3}{\pick\ a finite subset $\{ V_1, \ldots, V_n \}$ that covers
      $Y$.}
    \step{<1>4}{For $1 \leq i \leq n$, \pick $U_i$ disjoint from $V_i$ such
      that
      $a
      \in U_i$}
    \step{<1>5}{\pflet{$U = U_1 \cap \cdots \cap U_n$ and $V = V_1 \cup \cdots
        \cup
        V_n$}}
    \qed
  \end{proof}

  \begin{prop}
    \label{prop:topology:compact:compact_is_closed}
    Every compact subspace of a Hausdorff space is closed.
  \end{prop}

  \begin{proof}
    \pf
    \step{<1>1}{\pflet{$X$ be a Hausdorff space and $Y \subseteq X$ be
        compact.}}
    \step{<1>2}{Every point $a \notin Y$ has an open neighbourhood disjoint
      from
      $Y$.}
    \begin{proof}
      \pf\ By Lemma \ref{lm:topology:compact:regular}.
    \end{proof}
    \qedstep
    \begin{proof}
      \pf\ By Proposition \ref{prop:topology:neighbourhood:open}.
    \end{proof}
  \end{proof}


  \begin{prop}
    \label{prop:topology:compact:image}
    The image of a compact space under a continuous map is compact.
  \end{prop}

  \begin{proof}
    \pf
    \step{<1>1}{\pflet{$f : X \rightarrow Y$ be continuous where $X$ is
        compact.}}
    \step{<1>2}{\pflet{$\mathcal{A}$ be a covering of $f(X)$ by open sets in
        $Y$.}}
    \step{<1>3}{$\{ f^{-1}(U) : U \in \mathcal{A} \}$ is an open covering of
      $X$.}
    \step{<1>4}{\pick\ a finite subcovering $\{ f^{-1}(U_1), \ldots,
      f^{-1}(U_n)
      \}$}
    \step{<1>5}{$\{ U_1, \ldots, U_n \}$ is a finite subset of $\mathcal{A}$
      that
      covers $f(X)$.}
    \qedstep
    \begin{proof}
      \pf\ By Proposition \ref{prop:topology:compact:subspace}.
    \end{proof}
    \qed
  \end{proof}

  \begin{cor}
    Let $\{ X_\alpha \}_{\alpha \in J}$ be a family of topological spaces. If
    $\prod_{\alpha \in J} X_\alpha$ is compact then each $X_\alpha$ is compact.
  \end{cor}

  \begin{cor}
    $S_\Omega \times \overline{S_\Omega}$ is compact.
  \end{cor}

  \begin{cor}
   The Sorgenfrey plane is not compact.
  \end{cor}

  \begin{cor}
  	For any nonempty set $I$,
    the sapce $\mathbb{R}^I$ is not compact.
  \end{cor}

	\begin{cor}
		Let $\mathcal{T}$ and $\mathcal{T}'$ be topologies on the same
set $X$. If $\mathcal{T} \subseteq \mathcal{T}'$ and
$\mathcal{T}'$ is compact then $\mathcal{T}$ is compact.
	\end{cor}

        \begin{cor}
          The space $\mathbb{R}_K$ is not compact.
        \end{cor}

  \begin{thm}
    \label{thm:topology:compact:homeomorphism}
    Let $f : X \rightarrow Y$ be a bijective continuous function. If $X$ is
    compact and $Y$ is Hausdorff then $f$ is a homeomorphism.
  \end{thm}

  \begin{proof}
    \pf
    \step{<1>1}{\pflet{$C$ be closed in $X$}}
    \step{<1>2}{$C$ is compact}
    \begin{proof}
      \pf\ Proposition \ref{prop:topology:compact:closed_is_compact}.
    \end{proof}
    \step{<1>3}{$f(C)$ is compact}
    \begin{proof}
      \pf\ Proposition \ref{prop:topology:compact:image}
    \end{proof}
    \step{<1>4}{$f(C)$ is closed}
    \begin{proof}
      \pf\ Proposition \ref{prop:topology:compact:compact_is_closed}.
    \end{proof}
    \qedstep
    \begin{proof}
      \pf\ By Theorem
      \ref{thm:topology:continuous:characterisation}
      we have that      $f^{-1}$ is continuous.
    \end{proof}
    \qed
  \end{proof}

  \begin{cor}
  	\label{cor:topology:compact_hausdorff:finer_coarser}
	 Let $\mathcal{T}$ and $\mathcal{T}'$ be topologies on the same set $X$. If $\mathcal{T} \subseteq \mathcal{T}'$, $\mathcal{T}$ is Hausdorff and $\mathcal{T}'$ is compact then $\mathcal{T} = \mathcal{T}'$.
  \end{cor}

  \begin{cor}
    \label{cor:topology:compact_hausdorff:01K}
    The space $[0,1]$ is not compact as a subspace of $\mathbb{R}_K$.
  \end{cor}

  \begin{thm}[Tube Lemma]
    Let $A$ and $B$ be subspaces of $X$ and $Y$, respectively; let $N$ be an
    open set in $X \times Y$ including $A \times B$. If $A$ and $B$ are
    compact,
    then there exist open sets $U$ and $V$ in $X$ and $Y$, respectively, such
    that
    \[ A \times B \subseteq U \times V \subseteq N \enspace . \]
  \end{thm}

  \begin{proof}
    \pf
    \step{<1>1}{For all $a \in A$, there exist open sets $U$ and $V$ in $X$ and
      $Y$, respectively, such that
      \[ \{ a \} \times B \subseteq U \times V \subseteq N \enspace . \]}
    \begin{proof}
      \step{<2>1}{\pflet{$a \in A$}}
      \step{<2>2}{For all $b \in B$, there exist open sets $U$ and $V$ in $X$
        and
        $Y$, respectively, such that $(a, b) \in U \times V \subseteq N$.}
      \step{<2>3}{$\{ V \text{ open in } Y : \exists U \text{ open in } X. a
        \in
        U,
        U \times V \subseteq N \}$ covers $B$}
      \step{<2>4}{\pick\ a finite subset $\{ V_1, \ldots, V_n \}$ that covers
        $B$.}
      \step{<2>5}{For $1 \leq i \leq n$, \pick\ $U_i$ open in $X$ such that $a
        \in
        U_i$ and        $U_i \times V_i \subseteq N$}
      \step{<2>6}{\pflet{$U = U_1 \cap \cdots \cap U_n$ and $V = V_1 \cup
          \cdots
          \cup V_n$}}
    \end{proof}
    \step{<1>2}{$\{ U \text{ open in } X : \exists V \text{ open in } Y. B
      \subseteq V \text{ and } U \times V \subseteq N \}$ covers
      $A$.}
    \step{<1>3}{\pick\ a finite subset $\{ U_1, \ldots, U_n \}$ that covers
      $A$.}
    \step{<1>4}{For $1 \leq i \leq n$, \pick\ $V_i$ open in $B$ such that $B
      \subseteq V_i$ and $U_i \times V_i \subseteq N$.}
    \step{<1>5}{\pflet{$U = U_1 \cup \cdots \cup U_n$ and $V = V_1 \cap \cdots
        \cap
        V_n$}}
    \step{<1>6}{$A \times B \subseteq U \times V \subseteq N$}
    \qed
  \end{proof}

  \begin{prop}
    \label{prop:topology:compact:product}
    The product of two compact spaces is compact.
  \end{prop}

  \begin{proof}
    \pf
    \step{<1>1}{\pflet{$X$ and $Y$ be compact spaces.}}
    \step{<1>2}{\pflet{$\mathcal{A}$ be an open covering of $X \times Y$}}
    \step{<1>3}{For all $x \in X$, there exists a neighbourhood $W$ of $x$ such
      that $W \times Y$ is      covered by finitely many elements of
      $\mathcal{A}$.}
    \begin{proof}
      \step{<2>1}{\pflet{$x \in X$}}
      \step{<2>2}{$\{x\} \times Y$ is compact.}
      \begin{proof}
        \pf\ It is homeomorphic to $Y$.
      \end{proof}
      \step{<2>3}{\pick\ a finite subset $\{ U_1, \ldots, U_m \}$ of
        $\mathcal{A}$
        that covers $\{x\} \times Y$}
      \begin{proof}
        \pf\ By Proposition \ref{prop:topology:compact:subspace}.
      \end{proof}
      \step{<2>4}{There exists a neighbourhood $W$ of $x$ such that $W \times Y
        \subseteq U_1 \cup \cdots \cup U_m$}
      \begin{proof}
        \pf\ By the Tube Lemma.
      \end{proof}
    \end{proof}
    \step{<1>4}{$\{ W \text{ open in } X : W \times Y \text{ is covered by
        finitely
        many        elements of } \mathcal{A} \}$ is an open covering of $X$.}
    \step{<1>5}{\pick\ a finite subcovering $\{ W_1, \ldots, W_n \}$}
    \step{<1>6}{For $1 \leq i \leq n$, \pick\ a finite subset $\{ U_{i1},
      \ldots,
      U_{ir_i} \}$ of $\mathcal{A}$ that covers $W_i \times Y$}
    \step{<1>7}{$\{ U_{11}, \ldots, U_{nr_n} \}$ is a finite subcovering of
      $\mathcal{A}$.}
    \qed
  \end{proof}

  \begin{prop}
    \label{prop:topology:compact:finite_intersection}
    A topological space is compact if and only if every nonempty set of closed
    sets that has the finite intersection property has nonempty intersection.
  \end{prop}

  \begin{proof}
    \pf\ Immediate from definitions. \qed
  \end{proof}

  \begin{lm}
    \label{lm:topology:compact:projection_closed}
    If $Y$ is compact then $\pi_1 : X \times Y \rightarrow X$ is a closed map.
  \end{lm}

  \begin{proof}
    \pf
    \step{<1>1}{\pflet{$C \subseteq X \times Y$ be closed}}
    \step{<1>2}{\pflet{$x \in X \setminus \pi_1(C)$}}
    \step{<1>3}{For all $y \in Y$, we have $(x, y) \notin C$}
    \step{<1>4}{For all $y \in Y$, there exist open neighbourhoods $U$ of $x$
      and
      $V$ of $y$ such that $U \times V \subseteq (X \times Y) \setminus C$}
    \step{<1>5}{$\{ V \text{ open in } Y : \exists U \text{ an open
        neighbourhood
        of } x \text{ such that } U \times V \subseteq (X \times Y) \setminus C
      \}$ is an open covering of $Y$.}
    \step{<1>6}{\pick\ a finite subcovering $\{ V_1, \ldots, V_n \}$}
    \step{<1>7}{For $1 \leq i \leq n$, \pick\ an open neighbourhood $U_i$ of
      $x$
      such that $U_i \times V_i \subseteq (X \times Y) \setminus C$}
    \step{<1>8}{$x \in U_1 \cap \cdots \cap U_n \subseteq X \setminus \pi_1(C)$}
    \qed
  \end{proof}

  \begin{thm}
    Let $X$ be a compact space.
    Let $f_n : X \rightarrow \mathbb{R}$ be a sequence of continuous functions
    such that, for all $x \in X$, $f_n(x) \rightarrow f(x)$ as $n \rightarrow
    \infty$. If $f$ is continuous, and if the sequence $(f_n)_n$ is monotone
    increasing, and if $X$ is compact, then the convergence is uniform.
  \end{thm}

  \begin{proof}
    \pf
    \step{<1>1}{\pflet{$\epsilon > 0$} \prove{There exists $N$ such that, for
        all
        $n \geq N$, we have $|f_n(x) - f(x)| < \epsilon$}}
    \step{<1>2}{For $n \in \mathbb{Z}^+$, \pflet{$U_n = \{ x \in X : f(x) -
        f_n(x)
        <        \epsilon \}$}}
    \step{<1>3}{Each $U_n$ is open}
    \begin{proof}
      \pf\ Let $g(x) = f(x) - f_n(x)$. Then $g$ is continuous and $U_n =
      g^{-1}((- \infty, \epsilon))$.
    \end{proof}
    \step{<1>4}{$\{ U_n : n \geq 1 \}$ is an open covering of $X$}
    \begin{proof}
      \step{<2>1}{\pflet{$x \in X$}}
      \step{<2>2}{\pick\ $N$ such that, for all $n \geq N$, $|f(x) - f_n(x)| <
        \epsilon$}
      \begin{proof}
        \pf\ $f_n(x) \rightarrow f(x)$ as $n \rightarrow \infty$
      \end{proof}
      \step{<2>3}{$f(x) - f_N(x) < \epsilon$}
      \begin{proof}
        \pf\ This holds since the sequece $(f_n)_n$ is monotone.
      \end{proof}
    \end{proof}
    \step{<1>5}{\pick\ a finite subcovering $\{ U_{n_1}, \ldots, U_{n_k} \}$}
    \step{<1>6}{\pflet{$N = \max(n_1, \ldots, n_k)$}}
    \step{<1>7}{For all $n \geq N$ we have $|f_n(x) - f(x)| < \epsilon$}
    \qed
  \end{proof}

  \begin{lm}
    \label{lm:topology:compact:normal}
    Every compact Hausdorff space is normal.
  \end{lm}

  \begin{proof}
    \pf
    \step{<1>1}{\pflet{$A$ and $B$ be disjoint closed sets in the compact
        Hausdorff space $X$.}}
    \step{<1>2}{For all $a \in A$, there exist disjoint open sets $U$ and $V$
      such
      that $a \in U$ and $B \subseteq V$.}
    \begin{proof}
      \pf\ By Lemma \ref{lm:topology:compact:regular}.
    \end{proof}
    \step{<1>3}{$\{U \text{ open in } X : \exists V \text{ open in } Y. U \cap
      V
      = \emptyset, B \subseteq V \}$ is an open covering of $A$}
    \step{<1>4}{\pick\ a finite subcovering $\{ U_1, \ldots, U_n \}$}
    \step{<1>5}{For $1 \leq i \leq n$, \pick\ $V_i$ open in $Y$ such that $U_i
      \cap V_i = \emptyset$ and $B \subseteq V_i$}
    \step{<1>6}{\pflet{$U = U_1 \cup \cdots \cup U_n$ and $V = V_1 \cap \cdots
        \cap V_n$}}
    \qed
  \end{proof}

  \begin{thm}
    \label{thm:topology:compact:closed_interval}
    Let $X$ be a complete linearly ordered set under the order topology. Then
    every closed interval in $X$ is compact.
  \end{thm}

  \begin{proof}
    \pf
    \step{<1>1}{\pflet{$X$ be a complete linearly ordered set in the order
        topology}}
    \step{<1>2}{\pflet{$a, b \in X$, $a < b$} \prove{$[a, b]$ is compact}}
    \step{<1>3}{\pflet{$\mathcal{A}$ be a set of open sets that covers $[a,b]$}}
    \step{<1>4}{For all $x \in [a, b)$, there exists $y \in (x, b]$ such that
      $[x,
      y]$ is covered by at most two points of $\mathcal{A}$}
    \begin{proof}
      \step{<2>1}{\pflet{$x \in [a,b]$}}
      \step{<2>2}{\pick\ $U \in \mathcal{A}$ such that $x \in U$}
      \begin{proof}
        \pf\ By \stepref{<1>3} and \stepref{<2>1}
      \end{proof}
      \step{<2>3}{\pick\ $y \in (x, b]$ such that $[x, y) \subseteq U$}
      \begin{proof}
        \pf\ By Lemma \ref{lm:topology:order:open}.
      \end{proof}
      \step{<2>4}{\pick\ $V \in \mathcal{A}$ such that $y \in V$}
      \begin{proof}
        \pf\ By \stepref{<1>3} and \stepref{<2>3}.
      \end{proof}
      \step{<2>5}{$[x, y]$ is covered by $\{ U, V \}$}
      \begin{proof}
        \pf\ By \stepref{<2>3} and \stepref{<2>4}.
      \end{proof}
    \end{proof}
    \step{<1>5}{\pflet{$C = \{ y \in (a, b] : [a,y] \text{ is covered by a
          finite
          subset of } \mathcal{A} \}$}}
    \step{<1>6}{$C$ is nonempty}
    \begin{proof}
      \pf\ By \stepref{<1>4}.
    \end{proof}
    \step{<1>7}{\pflet{$c = \sup C$}}
    \begin{proof}
      \pf\ By \stepref{<1>1}.
    \end{proof}
    \step{<1>8}{$c \in C$}
    \begin{proof}
      \step{<2>1}{\pick\ $U \in \mathcal{A}$ such that $c \in U$}
      \step{<2>2}{\pick\ $y \in [a, c)$ such that $(y, c] \subseteq U$}
      \begin{proof}
        \pf\ By Lemma \ref{lm:topology:order:open}
      \end{proof}
      \step{<2>3}{\pick\ $z$ such that $y < z$ and $z \in C$}
      \begin{proof}
        \pf\ This exists because $y$ is not an upper bound for $C$.
      \end{proof}
      \step{<2>4}{\pick\ a finite $\mathcal{A}_0 \subseteq \mathcal{A}$ such
        that
        $[a, z]$ is covered by $\mathcal{A}_0$}
      \step{<2>5}{$[a, c]$ is covered by $\mathcal{A}_0 \cup \{ U \}$}
    \end{proof}
    \step{<1>9}{$c = b$}
    \begin{proof}
      \step{<2>1}{\assume{for a contradiction $c < b$}}
      \step{<2>2}{\pick\ $y \in (c, b]$ such that $[c, y]$ is covered by at
        most
        two elements of $\mathcal{A}$.}
      \begin{proof}
        \pf\ By \stepref{<1>4}
      \end{proof}
      \step{<2>3}{$y > c$ and $y \in C$}
      \qedstep
      \begin{proof}
        \pf\ This contradicts \stepref{<1>7}.
      \end{proof}
    \end{proof}
    \qedstep
  \end{proof}

  \begin{cor}
    \label{cor:topology:compact:real_closed_interval}
    Every closed interval in $\mathbb{R}$ is compact.
  \end{cor}

   \begin{cor}[CC]
     \label{cor:topology:limit_point_compact:S_omega}
  $S_\Omega$ is limit point compact.
 \end{cor}

 \begin{proof}
  \pf
  \step{<1>1}{\pflet{$A$ be an infinite subset of $S_\Omega$}}
  \step{<1>2}{\pick\ a countably infinite subset $B \subseteq A$}
  \step{<1>3}{\pflet{$b = \sup B$}}
  \step{<1>4}{$B \subseteq [0, b]$}
  \step{<1>5}{$[0, b]$ is compact}
  \begin{proof}
    \pf\ By the theorem.
  \end{proof}
  \step{<1>6}{$B$ has a limit point in $[0,b]$}
  \step{<1>7}{$A$ has a limit point in $[0,b]$}
  \qed
 \end{proof}

 \begin{cor}
  The ordered square is compact.
 \end{cor}

 \begin{cor}
  The ordered square is limit point compact.
 \end{cor}

\begin{cor}
  Not every subspace of a compact space is compact.
\end{cor}

\begin{proof}
  \pf\ $[0,1]$ is compact but $(0,1)$ is not. \qed
\end{proof}

  \begin{thm}[Extreme Value Theorem]
    Let $f : X \rightarrow Y$ be continuous where $Y$ is a linearly ordered set
    in the order topology. If $X$ is compact, then there exist $c, d \in X$
    such
    that, for all $x \in X$, we have $f(c) \leq f(x) \leq f(d)$.
  \end{thm}

  \begin{proof}
    \pf
    \step{<1>1}{$f(X)$ is compact.}
    \begin{proof}
      \pf\ By Proposition \ref{prop:topology:compact:image}.
    \end{proof}
    \step{<1>2}{$f(X)$ has a greatest element.}
    \begin{proof}
      \step{<2>1}{\assume{for a contradiction $f(X)$ has no greatest element.}}
      \step{<2>2}{$\{ (- \infty, f(x)) : x \in X \}$ is a set of open sets
        that covers $f(X)$.}
      \step{<2>3}{\pick\ a finite subset $\{ (- \infty, f(x_1)), \ldots, (-
        \infty,
        f(x_n)) \}$ that covers $f(X)$.}
      \begin{proof}
        \pf\ By Proposition \ref{prop:topology:compact:subspace}
      \end{proof}
      \step{<2>4}{\pflet{$f(x_N)$ be largest out of $f(x_1)$, \ldots, $f(x_n)$}}
      \step{<2>5}{$f(x_N) < f(x_N)$}
      \qedstep
      \begin{proof}
        \pf\ This is a contradiction.
      \end{proof}
    \end{proof}
    \step{<1>3}{$f(X)$ has a least element.}
    \begin{proof}
      \pf\ Similar.
    \end{proof}
    \qed
  \end{proof}

  \begin{thm}[DC]
    A nonempty compact Hausdorff space with no isolated points is uncountable.
  \end{thm}

  \begin{proof}
    \pf
    \step{<1>1}{\pflet{$X$ be a nonempty compact Hausdorff space with no
        isolated
        points.}}
    \step{<1>2}{For every nonempty open $U \subseteq X$ and point $x \in X$,
      there
      exists a nonempty open $V \subseteq U$ such that $x \notin \overline{V}$}
    \begin{proof}
      \step{<2>1}{\pflet{$U \subseteq X$ be nonempty and open and $x \in X$}}
      \step{<2>2}{\pick\ $y \in U$ such that $y \neq x$}
      \begin{proof}
        \pf\ This is possible because $U \neq \{ x \}$ since $x$ is not an
        isolated point.
      \end{proof}
      \step{<2>3}{\pick\ disjoint open neighbourhoods $W_1$ and $W_2$ of $x$
        and
        $y$}
      \begin{proof}
        \pf\ Since $X$ is Hausdorff
      \end{proof}
      \step{<2>4}{\pflet{$V = U \cap W_2$}}
      \step{<2>5}{$x \notin \overline{V}$}
      \begin{proof}
        \pf\ We have $\overline{V} \subseteq \overline{W_2} \subseteq X
        \setminus W_1$.
      \end{proof}
    \end{proof}
    \step{<1>3}{\pflet{$f : \mathbb{Z}^+ \rightarrow X$} \prove{$f$ is not
        surjective}}
    \step{<1>4}{\pick\ a sequence of open sets $V_1 \supseteq V_2 \supseteq
      \cdots$
      such that $f(n) \notin \overline{V_n}$}
    \begin{proof}
      \pf\ By \stepref{<1>2} and Dependent Choice.
    \end{proof}
    \step{<1>5}{\pick\ a point $b \in \bigcap_{i=1}^\infty \overline{V_i}$}
    \begin{proof}
      \pf\ By Proposition \ref{prop:topology:compact:finite_intersection}.
    \end{proof}
    \step{<1>6}{$b \neq f(n)$ for all $n$}
    \begin{proof}
      \pf\ For each $n$ we have $b \in \overline{V_n}$ (\stepref{<1>5}) and
      $f(n) \notin      \overline{V_n}$ (\stepref{<1>4}).
    \end{proof}
    \qed
  \end{proof}

  \begin{cor}
    Every closed interval in $\mathbb{R}$ is uncountable.
  \end{cor}

  \begin{thm}
    \label{thm:topology:compact:limit_point_compact}
    Every compact space is limit point compact.
  \end{thm}

  \begin{proof}
    \pf
    \step{<1>1}{\pflet{$X$ be a compact space.}}
    \step{<1>2}{\pflet{$A \subseteq X$ be a set with no limit points.}
      \prove{$A$
        is finite.}}
    \step{<1>3}{$A$ is closed.}
    \begin{proof}
      \pf\ By Corollary \ref{cor:topology:limit_point:closed}.
    \end{proof}
    \step{<1>4}{$A$ is compact.}
    \begin{proof}
      \pf\ By Proposition \ref{prop:topology:compact:closed_is_compact}.
    \end{proof}
    \step{<1>5}{$\{ U \text{ open in } X : U \cap A \text{ is a singleton} \}$
      covers $A$}
    \begin{proof}
      \step{<2>1}{\pflet{$a \in A$}}
      \step{<2>2}{\pick\ an open neighbourhood $U$ of $a$ such that $U$ does
        not
        intersect $A$ at a point other than $a$}
      \begin{proof}
        \pf\ One must exist because $a$ is not a limit point of $A$
        (\stepref{<1>2}).
      \end{proof}
      \step{<2>3}{$U \cap A = \{ a \}$}
    \end{proof}
    \step{<1>6}{\pick\ a finite subcover $\{ U_1, \ldots, U_n \}$}
    \begin{proof}
      \pf\ By \stepref{<1>4} using Proposition
      \ref{prop:topology:compact:subspace}.
    \end{proof}
    \step{<1>7}{For $1 \leq i \leq n$, \pflet{$U_i \cap A = \{ a_i \}$}}
    \step{<1>8}{$A = \{ a_1, \ldots, a_n \}$}
    \qed
  \end{proof}

  \begin{prop}
    \label{prop:topology:compact:union}
    Let $X$ be a space and $C, D \subseteq X$ be compact. Then $C \cup D$ is
    compact.
  \end{prop}

  \begin{proof}
    \pf
    \step{<1>1}{\pflet{$\mathcal{A}$ be a set of open sets that covers $C \cup
        D$}}
    \step{<1>2}{\pick\ a finite subset $\mathcal{A}_1$ that covers $C$ and a
      finite
      subset $\mathcal{A}_2$ that covers $D$.}
    \step{<1>3}{$\mathcal{A}_1 \cup \mathcal{A}_2$ is a finite subset of
      $\mathcal{A}$ that covers $C \cup D$.}
    \qedstep
  \end{proof}

    \begin{thm}
   Every compact Hausdorff space is normal.
  \end{thm}

  \begin{proof}
   \pf
   \step{<1>1}{\pflet{$X$ be a compact Hausdorff space.}}
   \step{<1>2}{\pflet{$A$ and $B$ be disjoint closed sets in $X$.}}
   \step{<1>3}{$\{ U \text{ open in } X : \exists V \text{ open in } X. B
     \subseteq V \wedge U \cap V = \emptyset\}$ covers $A$}
   \begin{proof}
     \step{<2>1}{$B$ is compact}
     \begin{proof}
       \pf\ By Proposition \ref{prop:topology:compact:closed_is_compact}.
     \end{proof}
     \qedstep
     \begin{proof}
       \pf\ By Lemma \ref{lm:topology:compact:regular}.
     \end{proof}
   \end{proof}
   \step{<1>4}{\pick\ a finite subcover $\{ U_1, \ldots, U_n \}$}
   \begin{proof}
     \pf\ $A$ is compact by Proposition
     \ref{prop:topology:compact:closed_is_compact}.
   \end{proof}
   \step{<1>5}{For $1 \leq i \leq n$, \pick\ $V_i$ open in $X$ such that $B
     \subseteq V_i$ and $U_i \cap V_i = \emptyset$}
   \step{<1>6}{\pflet{$U = U_1 \cup \cdots \cup U_n$ and $V = V_1 \cap \cdots
\cap
       V_n$}}
   \step{<1>7}{$U$ and $V$ are disjoint open sets, $A \subseteq U$ and $B
     \subseteq V$}
   \qed
  \end{proof}

  \begin{cor}
   The ordered square is normal.
  \end{cor}

  \begin{prop}
    Not every compact Hausdorff space is first countable.
  \end{prop}

  \begin{proof}
    \pf\ The space $\overline{S_\Omega}$ is compact Hausdorff but not first countable. \qed
  \end{proof}

\begin{cor}
  Not every compact Hausdorff space is second countable.
\end{cor}

\begin{thm}[Tychonoff (AC)]
  The product of a family of compact spaces is compact.
\end{thm}

\begin{proof}
  \pf
  \step{<1>1}{\pflet{$\{X_\alpha\}_{\alpha \in J}$ be a family of compact spaces.} \pflet{$X = \prod_{\alpha \in J} X_\alpha$}}
  \step{<1>2}{\pflet{$\mathcal{A} \subseteq \mathcal{P} X$ satisfy the finite intersection property.}
  \prove{$\bigcap_{A \in \mathcal{A}} \overline{A}$ is nonempty.}}
  \step{<1>3}{\pick\ a set $\mathcal{D} \subseteq \mathcal{P} X$ that includes $\mathcal{A}$ and is maximal with respect to the finite intersection property.}
  \begin{proof}
    \pf\ By Lemma \ref{lm:sets:finite_intersection_property:maximal}.
  \end{proof}
  \step{<1>4}{For $\alpha \in J$, \pick\ $x_\alpha \in \bigcap_{D \in \mathcal{D}} \overline{\pi_\alpha(D)}$}
  \begin{proof}
    \step{<2>1}{\pflet{$\alpha \in J$}}
    \step{<2>2}{$\{ \overline{\pi_\alpha(D)} : D \in \mathcal{D}\}$ satisfies the finite intersection property.}
    \qedstep
    \begin{proof}
      \pf\ By Proposition \ref{prop:topology:compact:finite_intersection}
    \end{proof}
  \end{proof}
  \step{<1>5}{\pflet{$x = (x_\alpha)_{\alpha \in J}$}}
  \step{<1>6}{For all $D \in \mathcal{D}$ we have $(x_\alpha)_{\alpha \in J} \in \overline{D}$}
  \begin{proof}
    \pf
    \step{<2>1}{Every subbasis element containing $x$ intersects every member of $\mathcal{D}$}
    \begin{proof}
      \step{<3>1}{\pflet{$\inv{\pi_\alpha(U)}$ be a subbasis element containing $x$ where $U$ is open in $X_
      \alpha$}}
      \step{<3>2}{\pflet{$D \in \mathcal{D}$}}
      \step{<3>3}{$U$ intersects $\pi_\alpha(D)$}
    \end{proof}
    \step{<2>2}{Every subbasis element containing $x$ is a member of $\mathcal{D}$}
    \begin{proof}
      \pf\ By Lemma \ref{lm:sets:finite_intersection_property:intersect_all}
    \end{proof}
    \step{<2>3}{Every basis element containing $x$ is a member of $\mathcal{D}$}
    \begin{proof}
      \pf\ By Lemma \ref{lm:sets:finite_intersection_property:finite_intersection}
    \end{proof}
    \step{<2>4}{Every basis element containing $x$ intersects every member of $\mathcal{D}$}
    \begin{proof}
      \pf\ This follows because $\mathcal{D}$ satisfies the finite intersection property.
    \end{proof}
  \end{proof}
  \qedstep
  \begin{proof}
    \pf\ By Proposition \ref{prop:topology:compact:finite_intersection}
  \end{proof}
  \qed
\end{proof}

\begin{thm}
  In a compact Hausdorff space, the components and the quasicomponents coincide.
\end{thm}

\begin{proof}
  \pf
  \step{<1>1}{\pflet{$X$ be a compact Hausdorff space and $x, y \in X$ lie in the same quasicomponent.} \prove{$x$ and $y$ are in the same component.}}
  \step{<1>2}{\pflet{$\mathcal{A}$ be the set of all closed subspaces $A$ of $X$ such that $x$ and $y$ lie in the same quasicomponent of $A$.}}
  \step{<1>3}{Every chain in $\mathcal{A}$ has a lower bound.}
  \begin{proof}
    \step{<2>1}{\pflet{$\mathcal{B} \subseteq \mathcal{A}$ be a chain} \prove{$Y = \bigcap \mathcal{B} \in \mathcal{A}$}}
    \step{<2>2}{\assume{for a contradiction $Y = C \cup D$ were $C$ and $D$ are disjoint and open in $Y$, $x \in C$ and $y \in D$}}
    \step{<2>3}{\pick\ disjoint open sets $U$ and $V$ in $X$ such that $C \subseteq U$ and $D \subseteq V$}
    \begin{proof}
      \pf\ By Lemma \ref{lm:topology:compact:normal}.
    \end{proof}
    \step{<2>4}{$\{ B \setminus (U \cup V) : B \in \mathcal{B} \}$ satisfies the finite intersection property.}
    \begin{proof}
      \step{<3>1}{\pflet{$B_1, \ldots, B_n \in \mathcal{B}$}}
      \step{<3>2}{$B_1 \cap \cdots \cap B_n \in \mathcal{B}$}
      \begin{proof}
        \pf\ By \stepref{<2>1}.
      \end{proof}
      \step{<3>3}{$B_1 \cap \cdots \cap B_n \setminus (U \cap V)$ is nonempty}
      \begin{proof}
        \pf\ $B_1 \cap \cdots \cap B_n \cap U$ and $B_1 \cap \cdots \cap B_n \cap V$ cannot be disjoint, because $x$ and $y$ are in the same quasicomponent of $B_1 \cap \cdots \cap B_n$.
      \end{proof}
    \end{proof}
    \step{<2>5}{$Y \setminus (U \cup V)$ is nonempty.}
    \begin{proof}
      \pf\ By Proposition \ref{prop:topology:compact:finite_intersection}.
    \end{proof}
    \qedstep
    \begin{proof}
      \pf\ This is a contradiction since $Y \setminus (U \cup V) = Y \setminus (C \cup D)$.
    \end{proof}
  \end{proof}
  \step{<1>4}{\pick\ a minimal element $D \in \mathcal{A}$}
  \begin{proof}
    \pf\ One exists by Zorn's Lemma.
  \end{proof}
  \step{<1>5}{$D$ is connected.}
  \begin{proof}
    \step{<2>1}{\assume[for a contradiction $D = U \uplus V$ is a separation of $D$]}
    \step{<2>2}{\case{$x, y \in U$}}
    \begin{proof}
      \pf\ In this case we have $U \in \mathcal{A}$ contradicting the minimality of $D$.
    \end{proof}
    \step{<2>3}{\case{$x \in U, y \in V$}}
    \begin{proof}
      \pf\ This is a contradiction because $x$ and $y$ are in the same quasicomponent of $D$.
    \end{proof}
    \step{<2>4}{\case{$x \in V, y \in U$}}
    \begin{proof}
      \pf\ Similar to \stepref{<2>3}.
    \end{proof}
    \step{<2>5}{\case{$x, y \in V$}}
    \begin{proof}
      \pf\ Similar to \stepref{<2>2}.
    \end{proof}
  \end{proof}
  \qed
\end{proof}

  \section{Perfect Maps}

    \begin{prop}
      \label{prop:topology:perfect:neighbourhood}
   Let $p : X \twoheadrightarrow Y$ be a closed continuous surjective map. For
   all $y \in Y$ and $U$ an open neighbourhood of $\inv{p}(y)$, there exists an
   open neighbourhood $W$ of $y$ such that $\inv{p}(W) \subseteq U$.
  \end{prop}

  \begin{proof}
   \pf\ Take $W = Y \setminus p(X \setminus U)$. \qed
  \end{proof}

      \begin{prop}[AC]
   Let $p : X \twoheadrightarrow Y$ be a closed continuous surjective map. If
$X$ is normal then $Y$ is normal.
  \end{prop}

  \begin{proof}
   \pf
   \step{<1>1}{\pflet{$A, B \subseteq Y$ be closed}}
   \step{<1>2}{$\inv{p}(A)$, $\inv{p}(B)$ are closed in $X$.}
   \step{<1>3}{\pick\ disjoint open sets $U$, $V$ of $\inv{p}(A)$, $\inv{p}(B)$
     respectively.}
   \step{<1>4}{For all $a \in A$, \pick\ an open neighbourhood $W_a$ of $a$
such
     that $\inv{p}(W_a) \subseteq U$}
   \begin{proof}
     \pf\ By Proposition \ref{prop:topology:perfect:neighbourhood}.
   \end{proof}
   \step{<1>5}{For all $b \in B$, \pick\ an open neighbourhood $W'_b$ of $b$
such
     that $\inv{p}(W'_b) \subseteq V$}
   \begin{proof}
     \pf\ By Proposition \ref{prop:topology:perfect:neighbourhood}.
   \end{proof}
   \step{<1>6}{\pflet{$W = \bigcup_{a \in A} W_a$ and $W' = \bigcup_{b \in B}
       W'_b$}}
   \step{<1>7}{$W \cap W' = \emptyset$}
   \begin{proof}
     \pf\ This holds because $\inv{p}(W) \subseteq U$, $\inv{p}(W') \subseteq
V$, and $p$ is surjective.
   \end{proof}
   \qed
  \end{proof}

  \begin{df}[Perfect Map]
    Let $X$ and $Y$ be topological spaces and $p : X \rightarrow Y$. Then $p$
    is
    \emph{perfect} iff $p$ is closed, continuous, surjective, and $p^{-1}(y)$
    is
    compact for all $y \in Y$.
  \end{df}

    \begin{prop}
      \label{prop:topology:perfect:Hausdorff}
   Let $p : X \rightarrow Y$ be a perfect map. If $X$ is Hausdorff then so is
$Y$.
  \end{prop}

  \begin{proof}
   \pf
   \step{<1>1}{\pflet{$a, b \in Y$ with $a \neq b$}}
   \step{<1>2}{\pick\ disjoint open neighbourhoods $U$ and $V$ of
$\inv{\pi}(a)$
     and $\inv{\pi}(b)$, respectively.}
   \begin{proof}
     \pf\ By Lemma \ref{lm:topology:compact:normal}.
   \end{proof}
   \step{<1>3}{\pick\ open neighbourhoods $W$ and $W'$ of $a$ and $b$ such that
     $\inv{\pi}(W) \subseteq U$ and $\inv{\pi}(W') \subseteq V$}
   \begin{proof}
     \pf\ By Proposition \ref{prop:topology:perfect:neighbourhood}.
   \end{proof}
   \step{<1>4}{$W$ and $W'$ are disjoint.}
   \qed
  \end{proof}

    \begin{prop}
   Let $p : X \twoheadrightarrow Y$ be perfect. If $X$ is regular then so is
$Y$.
  \end{prop}

  \begin{proof}
   \pf
   \step{<1>1}{$Y$ is $T_1$}
   \begin{proof}
     \pf\ By Proposition \ref{prop:topology:perfect:Hausdorff}.
   \end{proof}
   \step{<1>2}{\pflet{$C \subseteq Y$ be closed and $a \in Y \setminus C$}}
   \step{<1>3}{$\inv{p}(C)$ is closed and $\inv{p}(a)$ is disjoint from
     $\inv{p}(C)$.}
   \step{<1>4}{\pick\ disjoint open neighbourhoods $U$, $V$ of $\inv{p}(C)$,
     $\inv{p}(a)$ respectively.}
   \begin{proof}
     \pf\ By Lemma \ref{lm:topology:compact:regular}.
   \end{proof}
   \step{<1>5}{\pick\ an open neighbourhood $W'$ of $a$ such that $\inv{p}(W')
     \subseteq V$}
   \begin{proof}
     \pf\ By Proposition \ref{prop:topology:perfect:neighbourhood}.
   \end{proof}
   \step{<1>6}{For $c \in C$, \pick\ an open neighbourhood $W_c$ such that
     $\inv{p}(W_c) \subseteq U$}
   \begin{proof}
     \pf\ By Proposition \ref{prop:topology:perfect:neighbourhood}.
   \end{proof}
   \step{<1>7}{$W = \bigcup_{c \in C} W_c$ is an open neighbourhood of $C$
     disjoint from $W'$}
   \qed
  \end{proof}

    \begin{prop}[AC]
   Let $p : X \twoheadrightarrow Y$ be perfect. If $X$ is locally compact then
   so is $Y$.
  \end{prop}

  \begin{proof}
   \pf
   \step{<1>1}{\pflet{$b \in Y$}}
   \step{<1>2}{$\{ U \text{ open in } X : \exists C \subseteq X \text{ compact}.
U
     \subseteq C \}$ covers $\inv{p}(b)$}
   \step{<1>3}{\pick\ a finite subcover $\{ U_1, \ldots, U_n \}$}
   \step{<1>4}{For $1 \leq i \leq n$, \pick\ a compact $C_i \subseteq X$ such
that
     $U_i \subseteq C_i$}
   \step{<1>5}{For $1 \leq i \leq n$, \pick\ a neighbourhood $W_i$ of $b$ such
     that $\inv{p}(W_i) \subseteq U_i$}
   \begin{proof}
     \pf\ By Proposition \ref{prop:topology:perfect:neighbourhood}
   \end{proof}
   \step{<1>6}{$b \in W_1 \cup \cdots \cup W_n \subseteq p(C_1) \cup \cdots
\cup
     p(C_n)$}
   \step{<1>7}{$p(C_1) \cup \cdots \cup p(C_n)$ is compact.}
   \begin{proof}
     \step{<2>1}{Each $p(C_i)$ is compact.}
     \begin{proof}
       \pf\ By Proposition \ref{prop:topology:compact:image}.
     \end{proof}
     \qedstep
     \begin{proof}
        \pf\ By Proposition \ref{prop:topology:compact:union}.
      \end{proof}
   \end{proof}
   \qed
  \end{proof}


  \section{Sequential Compactness}

  \begin{df}[Sequentially Compact]
    A space is \emph{sequentially compact} iff every sequence has a convergent
    subsequence.
  \end{df}

   \begin{prop}
   $\overline{S_\Omega}$ is not sequentially compact.
 \end{prop}

 \begin{proof}
  \pf\ $\Omega$ is a limit point of $S_\Omega$ but is not the limit of any
sequence of points in $S_\Omega$. \qed
 \end{proof}

  \section{Local Compactness}

  \begin{df}[Local Compactness]
    Let $X$ be a topological space.

    For $x \in X$, the space $X$ is \emph{locally compact} at $x$ iff there
    exists a compact subspace $C \subseteq X$ that includes a neighbourhood of
    $x$.

    The space $X$ is \emph{locally compact} iff it is locally compact at every
    point.
  \end{df}

   \begin{prop}
  Every complete linearly ordered set is locally compact under the order
  topology.
 \end{prop}

 \begin{proof}
  \pf
  \step{<1>1}{\pflet{$L$ be a complete linearly ordered set and $x \in L$}
\prove{There exists a compact subspace $C \subseteq L$ that includes a
  neighbourhood $U$ of $x$}}
\step{<1>2}{\case{$x$ is least and greatest in $L$}}
\begin{proof}
  \pf\ In this case, $L = \{ x \}$ is compact.
\end{proof}
\step{<1>3}{\case{$x$ is least in $L$ but not greatest}}
\begin{proof}
  \step{<2>1}{\pick\ $a < x$}
  \step{<2>2}{Take $C = [a,x]$ and $U = (a, x]$}
\end{proof}
  \step{<1>4}{\case{$x$ is greatest in $L$ but not least}}
  \begin{proof}
    \pf\ Similar.
  \end{proof}
  \step{<1>5}{\case{$x$ is neither least nor greatest}}
  \begin{proof}
    \step{<2>1}{\pick\ $a < x$ and $b > x$}
    \step{<2>2}{Take $C = [a,b]$ and $U = (a, b)$}
  \end{proof}
  \qed
 \end{proof}

 \begin{cor}
  For every ordinal $\alpha$, the space $S_\alpha$ is locally compact.
 \end{cor}


  \begin{thm}
    Every closed subspace of a locally compact Hausdorff space is locally
    compact.
  \end{thm}

  \begin{proof}
    \pf
    \step{<1>1}{\pflet{$X$ be locally compact Hausdorff and $C \subseteq X$ be
        closed.}}
    \step{<1>2}{\pflet{$x \in C$}}
    \step{<1>3}{\pick\ $D \subseteq X$ compact and $U \subseteq D$ open such
      that
      $x \in U$}
    \step{<1>4}{$D$ is closed.}
    \begin{proof}
      \pf\ Proposition \ref{prop:topology:compact:compact_is_closed}.
    \end{proof}
    \step{<1>5}{$C \cap D$ is closed}
    \begin{proof}
      \pf\ Propositon \ref{prop:topology:closed:intersection}.
    \end{proof}
    \step{<1>6}{$C \cap D$ is compact}
    \begin{proof}
      \pf\ Proposition \ref{prop:topology:compact:closed_is_compact}.
    \end{proof}
    \qedstep
    \begin{proof}
      \pf\ $C \cap D \subseteq C$ is compact and includes the open
      neighbourhood
      $U \cap C$ of $x$.
    \end{proof}
    \qed
  \end{proof}

   \begin{prop}
   Let $\{X_\alpha\}_{\alpha \in J}$ be a family of nonempty topological
spaces. If    $\prod_{\alpha \in J} X_\alpha$ is locally compact, then each
$X_\alpha$ is locally compact.
 \end{prop}

 \begin{proof}
  \pf
  \step{<1>1}{\pflet{$\alpha \in J$ and $x_\alpha \in X_\alpha$}}
  \step{<1>2}{\pick\ $x_\beta \in X_\beta$ for all $\beta \in J \setminus \{
    \alpha \}$}
  \step{<1>3}{\pick\ a compact subspace $C \subseteq \prod_{\alpha \in J}
    X_\alpha$ that a neighbourhood $U$ of $x$ included in $C$}
  \step{<1>4}{\pick\ a basic open set $\prod_{\alpha \in J} U_\alpha$ such that
$x
    \in \prod_{\alpha \in J} U_\alpha \subseteq U$}
  \step{<1>5}{$x_\alpha \in U_\alpha \subseteq \pi_\alpha(C)$}
  \step{<1>6}{$\pi_\alpha(C)$ is compact.}
  \begin{proof}
    \pf\ By Proposition \ref{prop:topology:compact:image}.
  \end{proof}
  \qed
 \end{proof}

 \begin{cor}
  The Sorgenfrey plane is not locally compact.
 \end{cor}

  \begin{prop}
   Let $\{ X_\alpha \}_{\alpha \in J}$ be a family of locally compact spaces
such that $X_\alpha$ is compact for all but finitely many values of $\alpha$.
Then $\prod_{\alpha \in J} X_\alpha$ is locally compact.
 \end{prop}

 \begin{proof}
  \pf
  \step{<1>1}{\assume{$X_\alpha$ is compact if $\alpha \neq \alpha_1, \ldots,
      \alpha_n$}}
  \step{<1>2}{\pflet{$\vec{x} \in \prod_{\alpha \in J} X_\alpha$}}
  \step{<1>3}{For $1 \leq i \leq n$, \pick\ $C_{\alpha_i} \subseteq
X_{\alpha_i}$
    compact     and $U_{\alpha_i}$ open such that $x_{\alpha_i} \in
    U_{\alpha_i} \subseteq C_{\alpha_i}$}
  \step{<1>4}{For $\alpha \neq \alpha_1, \ldots, \alpha_n$, \pflet{$C_\alpha =
      U_\alpha = X_\alpha$}}
  \step{<1>5}{$\vec{x} \in \prod_{\alpha \in J} U_\alpha \subseteq
\prod_{\alpha
      \in J} C_\alpha$}
  \step{<1>6}{$\prod_{\alpha \in J} C_\alpha$ is compact}
  \begin{proof}
    \pf\ By Tychonoff's Theorem.
  \end{proof}
  \qed
 \end{proof}

  \begin{prop}
   $\mathbb{R}_l$ is not locally compact.
 \end{prop}

 \begin{proof}
  \pf\ $[0, +\infty)$ can be partitioned into infinitely many disjoint open
sets, which therefore do not have a finite subcover. \qed
 \end{proof}

    \begin{prop}
   Let $\{X_\alpha\}_{\alpha \in J}$ be a family of nonempty topological
spaces. If    $\prod_{\alpha \in J} X_\alpha$ is locally compact, then all but
finitely many of the $X_\alpha$ are compact.
 \end{prop}

 \begin{proof}
  \pf
  \step{<1>1}{\pick\ a point $a = (a_\alpha)_{\alpha \in J} \in \prod_{\alpha
\in
      J}       X_\alpha$}
  \step{<1>2}{\pick\ a compact $C \subseteq \prod_{\alpha \in J} X_\alpha$ that
    includes the       basic neighbourhood $\prod_{\alpha \in J} U_\alpha$ of
    $a$,  where $U_\alpha = X_\alpha$ for all $\alpha$ except $\alpha =
    \alpha_1,  \ldots, \alpha_n$}
  \step{<1>3}{For $\alpha \neq \alpha_1, \ldots, \alpha_n$, we have $X_\alpha$
is
    compact.}
  \begin{proof}
    \pf\ $X_\alpha$ is homeomorphic to a closed subspace of $C$.
  \end{proof}
  \qed
 \end{proof}

 \begin{cor}
 	For any infinite set $I$, the space $\mathbb{R}^I$ is not locally compact.
 \end{cor}

  \begin{prop}
   $[0,1]^\omega$ is not compact under the uniform topology.
 \end{prop}

 \begin{proof}
   \pf $\{ a_i : i \geq 0 \}$ is an infinite set with no limit point, where
$a_i$ is the point with $i$th component 1 and all other components 0. \qed
 \end{proof}

 \begin{cor}
   $\mathbb{R}^\omega$ under the uniform topology is not locally compact.
 \end{cor}

 \begin{proof}
  \pf
  \step{<1>1}{\assume{$\mathbb{R}^\omega$ is locally compact}}
  \step{<1>2}{\pflet{$C$ be a compact subspace such that $B(\vec{0}, \epsilon)
      \subseteq C$}}
  \step{<1>3}{$\overline{B(\vec{0}, \epsilon)}$ is compact.}
  \qedstep
  \begin{proof}
    \pf\ This contradicts the proposition.
  \end{proof}
  \qed
 \end{proof}

 \begin{prop}
   Not every subspace of a locally compact Hausdorff space is locally compact.
 \end{prop}

 \begin{proof}
   \pf\ $\mathbb{R}$ is locally compact Hausdoff, $\mathbb{Q}$ is not locally compact. \qed
 \end{proof}

 \begin{prop}
   The continuous image of a locally compact Hausdorff space is not necessarily locally compact.
 \end{prop}

 \begin{proof}
   \pf
   \step{<1>1}{\pflet{$\{ q_0, q_1, \ldots \}$ be an enumeration of $[0,1] \cap \mathbb{Q}$.}}
   \step{<1>2}{Define $f : (0, +\infty) \setminus \mathbb{Z} \rightarrow [0,1] \cap \mathbb{Q}$ by: $f(x) = q_n$ for $x \in (n, n+1)$}
   \step{<1>3}{$f$ is continuous.}
   \begin{proof}
     \pf\ The inverse image of any set is a union of open intervals.
   \end{proof}
   \qed
 \end{proof}

  \section{Compactifications}

  \begin{df}[Compactification]
    Let $X$ and $Y$ be spaces. Then $Y$ is a \emph{compactification} of $X$ iff
    $Y$ is a compact Hausdorff space and $X$ is a subspace of $Y$ with
    $\overline{X} = Y$.
  \end{df}

  \begin{df}[One-Point Compactification]
    A \emph{one-point compactification} of $X$ is a compactification $Y$ of $X$
    such that $Y \setminus X$ is a singleton.
  \end{df}

  \begin{thm}
    \label{thm:topology:locally_compact:one_point_compactification}
    Let $X$ be a topological space. Then $X$ is locally compact Hausdorff if
    and
    only if there exists a space $Y$ such that:
    \begin{enumerate}
      \item $X$ is a subspace of $Y$
      \item The set $Y \setminus X$ is a singleton.
      \item $Y$ is a compact Hausdorff space.
    \end{enumerate}
    If $Y$ and $Y'$ are two spaces satisfying these conditions, then there
    exists a
    unique homeomorphism between $Y$ and $Y'$ that is the identity on $X$.
  \end{thm}

  \begin{proof}
    \pf
    \step{<1>1}{If $X$ is locally compact Hausdorff then there exists a space
      $Y$
      satisfying 1--3.}
    \begin{proof}
      \step{<2>1}{\pflet{$Y = X \cup \{ \infty \}$ under the topology
          $\mathcal{T} = \{ U \subseteq X : U \text{ is open in } X \} \cup \{
          Y
          \setminus C : C \subseteq X \text{ is compact} \}$.}}
      \begin{proof}
        \step{<3>1}{$Y \in \mathcal{T}$}
        \begin{proof}
          \pf\ This holds because $Y = Y \setminus \emptyset$.
        \end{proof}
        \step{<3>2}{For all $U, V \in \mathcal{T}$ we have $U \cap V \in
          \mathcal{T}$.}
        \begin{proof}
          \step{<4>1}{\pflet{$U, V \in \mathcal{T}$}}
          \step{<4>2}{\case{$U$, $V$ are open in $X$}}
          \begin{proof}
            \pf\ In this case, $U \cap V$ is open in $X$.
          \end{proof}
          \step{<4>3}{\case{$U$ is open in $X$, $V = Y \setminus C$ where $C
              \subseteq X$ is compact.}}
          \begin{proof}
            \step{<5>1}{$U \cap V = U \setminus C$}
            \step{<5>2}{$C$ is closed in $X$}
            \begin{proof}
              \pf\ Proposition \ref{prop:topology:compact:compact_is_closed}.
            \end{proof}
            \step{<5>3}{$U \cap V$ is open in $X$}
          \end{proof}
          \step{<4>4}{\case{$U = Y \setminus C$ where $C \subseteq X$ is
              compact,
              $V$ is open in $X$.}}
          \begin{proof}
            \pf\ Similar.
          \end{proof}
          \step{<4>5}{\case{$U = Y \setminus C$, $V = Y \setminus D$ where $C,
              D
              \subseteq X$ are compact.}}
          \begin{proof}
            \step{<5>1}{$U \cap V = Y \setminus (C \cup D)$}
            \step{<5>2}{$C$ and $D$ are closed in $X$}
            \begin{proof}
              \pf\ Proposition \ref{prop:topology:compact:compact_is_closed}.
            \end{proof}
            \step{<5>3}{$C \cup D$ is closed in $X$}
            \begin{proof}
              \pf\ Proposition \ref{prop:topology:closed:union}.
            \end{proof}
            \step{<5>4}{$C \cup D$ is compact.}
            \begin{proof}
              \pf\ By Proposition
              \ref{prop:topology:compact:union}. \qed
            \end{proof}
          \end{proof}
        \end{proof}
        \step{<3>3}{For all $\mathcal{A} \subseteq \mathcal{T}$ we have
          $\bigcup
          \mathcal{A} \in \mathcal{T}$.}
        \begin{proof}
          \step{<4>1}{\pflet{$\mathcal{A} \subseteq \mathcal{T}$}}
          \step{<4>2}{\case{Every element of $\mathcal{A}$ is an open set in
              $X$.}}
          \begin{proof}
            \pf\ In this case, $\bigcup \mathcal{A}$ is open in $X$.
          \end{proof}
          \step{<4>3}{\case{There exists $C$ compact in $X$ such that $Y
              \setminus
              C \in \mathcal{A}$}}
          \begin{proof}
            \step{<5>1}{$\bigcup \mathcal{A} = Y \setminus (\bigcap \{ D
              \subseteq X : D \text{ compact}, Y \setminus D \in \mathcal{A} \}
              \setminus \bigcup \{ U \text{ open in } X : U \in \mathcal{A}
              \})$}
            \begin{proof}
              \pf\ Set theory.
            \end{proof}
            \step{<5>2}{$\bigcap \{ D
              \subseteq X : D \text{ compact}, Y \setminus D \in \mathcal{A} \}
              \setminus \bigcup \{ U \text{ open in } X : U \in \mathcal{A} \}$
              is
              compact.}
            \begin{proof}
              \pf\ It is a closed subset of the compact set $C$.
            \end{proof}
          \end{proof}
        \end{proof}
      \end{proof}
      \step{<2>2}{$X$ is a subspace of $Y$}
      \begin{proof}
        \step{<3>1}{For every open set $U$ of $X$, there exists $V$ open in $Y$
          such that $U = V \cap X$}
        \begin{proof}
          \pf\ Take $V = U$.
        \end{proof}
        \step{<3>2}{For every open set $V$ in $Y$, we have $V \cap X$ is open
          in
          $X$.}
        \begin{proof}
          \step{<4>1}{\pflet{$V$ be open in $Y$}}
          \step{<4>2}{\case{$V$ is open in $X$}}
          \begin{proof}
            \pf\ In this case, $V \cap X = V$.
          \end{proof}
          \step{<4>3}{\case{$V = Y \setminus C$ where $C \subseteq X$ is
              compact.}}
          \begin{proof}
            \step{<5>1}{$C$ is closed in $X$.}
            \begin{proof}
              \pf\ By Proposition \ref{prop:topology:compact:compact_is_closed}.
            \end{proof}
            \step{<5>2}{$V \cap X = X \setminus C$}
          \end{proof}
        \end{proof}
      \end{proof}
      \step{<2>3}{$Y \setminus X = \{ \infty \}$}
      \step{<2>4}{$Y$ is compact.}
      \begin{proof}
        \step{<3>1}{\pflet{$\mathcal{A}$ be an open covering of $Y$}}
        \step{<3>2}{\pick\ $U \in \mathcal{A}$ such that $\infty \in U$}
        \step{<3>3}{\pick\ $C \subseteq X$ compact such that $U = Y \setminus
          C$.}
        \step{<3>4}{$\{ V \cap X : V \in \mathcal{A} \}$ is set of open sets
          that
          covers $C$}
        \step{<3>5}{\pick\ a finite subset $\{ V_1, \ldots, V_n \}$ such that
          $\{
          V_1 \cap X, \ldots, V_n \cap X \}$ covers $C$.}
        \step{<3>6}{$\{ U, V_1, \ldots, V_n \}$ is a finite subcover of $Y$.}
      \end{proof}
      \step{<2>5}{$Y$ is Hausdorff.}
      \begin{proof}
        \step{<3>1}{\pflet{$x, y \in Y$ with $x \neq y$} \prove{There exist
            disjoint open neighbourhoods $U$, $V$ of $x$ and $y$.}}
        \step{<3>2}{\case{$x, y \in X$}}
        \begin{proof}
          \pf\ In this case, we just use the fact that $X$ is Hausdorff.
        \end{proof}
        \step{<3>3}{\case{$x = \infty$, $y \in X$}}
        \begin{proof}
          \step{<4>1}{\pick\ $C \subseteq X$ compact such that $C$ includes an
            open
            neighbourhood $V$ of $y$}
          \step{<4>2}{\pflet{$U = Y \setminus C$}}
        \end{proof}
        \step{<3>4}{\case{$x \in X$, $y = \infty$}}
        \begin{proof}
          \pf\ Simlar.
        \end{proof}
      \end{proof}
    \end{proof}
    \step{<1>2}{If there exists a space $Y$ satisfying 1--3 then $X$ is locally
      compact Hausdorff.}
    \begin{proof}
      \step{<2>1}{\pflet{$Y$ be a space satisfying 1--3}}
      \step{<2>2}{\pflet{$\infty$ be the point in $Y \setminus X$}}
      \step{<2>3}{$X$ is locally compact}
      \begin{proof}
        \step{<3>1}{\pflet{$x \in X$}}
        \step{<3>2}{\pick\ disjoint open neighbourhoods $U$ of $x$ and $V$ of
          $\infty$}
        \step{<3>3}{$X \setminus V$ is compact and includes $U$}
        \begin{proof}
          \pf\ $X \setminus V = Y \setminus V$ is compact because it is a
          closed
          subset of $Y$ (Proposition
          \ref{prop:topology:compact:closed_is_compact}).
        \end{proof}
      \end{proof}
      \step{<2>4}{$X$ is Hausdorff.}
      \begin{proof}
        \pf\ By Corollary \ref{cor:topology:Hausdorff:subspace}.
      \end{proof}
    \end{proof}
    \step{<1>3}{If $Y$ and $Y'$ are two spaces satisfying 1--3 then there
      exists a
      unique homemorphism between $Y$ and $Y'$ that is the identity on $X$.}
    \begin{proof}
      \step{<2>1}{\pflet{$Y$ and $Y'$ be two spaces that satisfy 1--3.}}
      \step{<2>2}{\pflet{$Y \setminus X = \{ p \}$ and $Y' \setminus X = \{ q
          \}$}}
      \step{<2>3}{\pflet{$h : Y \rightarrow Y'$ be given by
          \begin{align*}
            h(x) & = x & (x \in X) \\
            h(p) & = q
          \end{align*}}}
      \step{<2>4}{$h$ is a homeomorphism}
      \begin{proof}
        \step{<3>1}{$h$ is bijective.}
        \step{<3>2}{$h$ is continuous.}
        \begin{proof}
          \step{<4>1}{\pflet{$V \subseteq Y'$ be open.} \prove{$\inv{h}(V)$ is
              open.}}
          \step{<4>2}{\case{$V \subseteq X$}}
          \begin{proof}
            \step{<5>1}{$\inv{h}(V) = V$}
            \step{<5>2}{$V$ is open in $X$}
            \begin{proof}
              \pf\ Condition 1 for $Y'$.
            \end{proof}
            \step{<5>3}{$V$ is open in $Y$}
            \begin{proof}
              \pf\ Condition 1 for $Y$.
            \end{proof}
          \end{proof}
          \step{<4>3}{\case{$q \in V$}}
          \begin{proof}
            \step{<5>1}{$Y' \setminus V$ is compact.}
            \begin{proof}
              \pf\ Proposition \ref{prop:topology:compact:closed_is_compact}.
            \end{proof}
            \step{<5>2}{$Y' \setminus V$ is closed in $Y$.}
            \begin{proof}
              \pf\ Proposition \ref{prop:topology:compact:compact_is_closed}.
            \end{proof}
            \step{<5>3}{$\inv{h}(V) = Y \setminus (Y' \setminus V)$}
          \end{proof}
        \end{proof}
        \step{<3>3}{$\inv{h}$ is continuous.}
        \begin{proof}
          \pf\ Similar.
        \end{proof}
      \end{proof}
      \step{<2>5}{If $h' : Y \rightarrow Y'$ is a homeomorphism such that $h'
        \restriction_X = \id{X}$ then $h' = h$}
    \end{proof}
    \qed
  \end{proof}

  \begin{thm}
    \label{thm:topology:locally_compact:neighbourhood}
    Let $X$ be a Hausdorff space. Then $X$ is locally compact if and only if,
    for all $x \in X$ and any neighbourhood $U$ of $x$, there exists an open
    neighbourhood $V$ of $x$ such that $\overline{V}$ is compact and
    $\overline{V}
    \subseteq U$.
  \end{thm}

  \begin{proof}
    \pf
    \step{<1>1}{If $X$ is locally compact then, for all $x \in X$ and any
      neighbourhood $U$ of $x$, there exists an open
      neighbourhood $V$ of $x$ such that $\overline{V}$ is compact and
      $\overline{V}
      \subseteq U$.}
    \begin{proof}
      \step{<2>1}{\assume{$X$ is locally compact.}}
      \step{<2>2}{\pflet{$x \in X$ and $U$ be a neighbourhood of $x$.}}
      \step{<2>3}{\pflet{$Y$ be the one-point compactification of $X$.}}
      \begin{proof}
        \pf\ By Theorem
        \ref{thm:topology:locally_compact:one_point_compactification}.
      \end{proof}
      \step{<2>4}{\pflet{$C = Y \setminus U$}}
      \step{<2>5}{$C$ is compact}
      \begin{proof}
        \pf\ By Proposition \ref{prop:topology:compact:closed_is_compact}.
      \end{proof}
      \step{<2>6}{\pick\ disjoint open sets $V$, $W$ containing $x$ and $C$}
      \begin{proof}
        \pf\ Lemma \ref{lm:topology:compact:regular}
      \end{proof}
      \step{<2>7}{$V$ is open in $X$}
      \begin{proof}
        \pf\ $V \subseteq X$ since $\infty \in W$.
      \end{proof}
      \step{<2>8}{The closure of $V$ in $X$ is compact}
      \begin{proof}
        \step{<3>1}{The closure of $V$ is $X$ is the same as the closure of $V$
          in
          $Y$.}
        \begin{proof}
          \pf\ The point $\infty$ cannot be a limit point of $V$ since $W$ is a
          neighbourhood disjoint from $V$.
        \end{proof}
        \step{<3>2}{The closure of $V$ in $Y$ is compact.}
        \begin{proof}
          \pf\ By Proposition \ref{prop:topology:compact:closed_is_compact}.
        \end{proof}
      \end{proof}
      \step{<2>9}{$\overline{V} \subseteq U$}
      \begin{proof}
        \pf
        \begin{align*}
          \overline{V} & \subseteq Y \setminus W \\
          & \subseteq Y \setminus C \\
          & = U
        \end{align*}
      \end{proof}
    \end{proof}
    \step{<1>2}{If, for all $x \in X$ and any neighbourhood $U$ of $x$, there
      exists an
      open
      neighbourhood $V$ of $x$ such that $\overline{V}$ is compact and
      $\overline{V}
      \subseteq U$, then $X$ is locally compact.}
    \begin{proof}
      \step{<2>1}{\assume{for all $x \in X$ and any neighbourhood $U$ of $x$,
          there
          exists an
          open
          neighbourhood $V$ of $x$ such that $\overline{V}$ is compact and
          $\overline{V}
          \subseteq U$}}
      \step{<2>2}{\pflet{$x \in X$} \prove{There exists $C \subseteq X$ compact
          such
          that $C$ includes a neighbourhood $U$ of $x$}}
      \step{<2>3}{\pick\ an open neighbourhood $V$ of $x$ such that
        $\overline{V}$ is
        compact and $\overline{V} \subseteq X$}
      \step{<2>4}{Take $C = \overline{V}$ and $U = V$}
    \end{proof}
    \qed
  \end{proof}

  \begin{cor}
    Every open subspace of a locally compact Hausdorff space is locally compact.
  \end{cor}

  \begin{cor}
    A space is locally compact Hausdorff if and only if it is an open subspace
    of a compact Hausdorff space.
  \end{cor}

  \begin{cor}
   Every locally compact Hausdorff space is completely regular.
  \end{cor}

  \begin{cor}
    The space $\mathbb{R}_K$ is not locally compact.
  \end{cor}

    \begin{lm}[AC]
      \label{lm:topology:locally_compact:quotient}
   If $p : X \rightarrow Y$ is a quotient map and $Z$ is a locally compact
Hausdorff space, then the map
\[ \pi = p \times \id{Z} : X \times Z \rightarrow Y \times Z \]
is a quotient map.
  \end{lm}

  \begin{proof}
   \pf
   \step{<1>1}{$\pi$ is surjective.}
   \begin{proof}
     \pf\ This holds because $p$ is surjective.
   \end{proof}
   \step{<1>2}{$\pi$ is continuous.}
   \begin{proof}
     \pf\ By Theorem \ref{thm:topology:continuous:product}. % TODO Extract lemma
   \end{proof}
   \step{<1>3}{For $A \subseteq Y \times Z$, if $\inv{\pi}(A)$ is open in $X
     \times Z$ then $A$ is open in $Y \times Z$.}
   \begin{proof}
     \step{<2>1}{\pflet{$A \subseteq Y \times Z$}}
     \step{<2>2}{\assume{$\inv{\pi}(A)$ is open in $X \times Z$}}
     \step{<2>3}{\pflet{$(y, z) \in A$}}
     \step{<2>4}{\pick\ $x \in X$ such that $p(x) = y$}
     \begin{proof}
       \pf\ Since $p$ is surjective.
     \end{proof}
     \step{<2>5}{\pick\ open sets $U_1$, $V$ with $\overline{V}$ compact such
that
       $(x, y) \in U_1 \times V$ and $U_1 \times \overline{V} \subseteq
       \inv{\pi}(A)$}
     \begin{proof}
       \pf\ Using Theorem \ref{thm:topology:locally_compact:neighbourhood}
     \end{proof}
     \step{<2>6}{\pick\ a sequence of open sets $U_1$, $U_2$, \ldots in $X$
such
       that $\inv{p}(p(U_n)) \subseteq U_{n+1}$ and $U_n \times \overline{V}
       \subseteq \inv{\pi}(A)$ for all $n$}
     \begin{proof}
       \step{<3>1}{\pflet{$U$ be open with $U \times \overline{V} \subseteq
           \inv{\pi}(A)$} \prove{There exists $W$ open with $\inv{p}(p(U))
           \subseteq W$ and $W \times \overline{V} \subseteq \inv{\pi}(A)$}}
       \step{<3>2}{For all $x \in \inv{p}(p(U))$, \pick\ open sets $U_x$, $V_x$
         such that $x \in U_x$, $\overline{V} \subseteq V_x$ and $U_x \times
         V_x \subseteq \inv{\pi}(A)$}
       \begin{proof}
         \pf\ By the Tube Lemma.
       \end{proof}
       \step{<3>3}{\pflet{$W = \bigcup_{x \in \inv{p}(p(U))} U_x$}}
     \end{proof}
     \step{<2>7}{\pflet{$U = \bigcup_{n=1}^\infty U_n$}}
     \step{<2>8}{$U$ is saturated with respect to $p$}
     \begin{proof}
       \step{<3>1}{\pflet{$a \in U$, $b \in X$, $p(a) = p(b)$}}
       \step{<3>2}{\pick\ $n$ such that $a \in U_n$}
       \step{<3>3}{$b \in \inv{p}(p(U_n))$}
       \step{<3>4}{$b \in U_{n+1}$}
       \step{<3>5}{$b \in U$}
     \end{proof}
     \step{<2>9}{$p(U)$ is open in $Y$}
     \begin{proof}
       \pf\ By Lemma \ref{lm:topology:quotient:saturated}.
     \end{proof}
     \step{<2>10}{$(y, z) \in p(U) \times V \subseteq A$}
     \qedstep
     \begin{proof}
       \pf\ By Proposition \ref{prop:topology:neighbourhood:open}.
     \end{proof}
   \end{proof}
   \qed
  \end{proof}

    \begin{thm}
   Let $p : A \rightarrow B$ and $q : C \rightarrow D$ be quotient maps. If $B$
and $C$ are locally compact Hausdorff spaces, then $p \times q : A \times C
\rightarrow B \times D$ is a quotient map.
  \end{thm}

  \begin{proof}
    \pf\ This holds by Lemma
\ref{lm:topology:locally_compact:quotient} and Proposition
\ref{prop:topology:quotient:composite} because $p \times q = (\id{B} \times q)
\circ (p \times
\id{C})$. \qed
  \end{proof}

  \chapter{Metric Spaces}

  \section{The Metric Topology}

  \begin{df}[Metric]
    A \emph{metric} on a set $X$ is a function $d : X \times X \rightarrow
    \mathbb{R}$ such that, for all $x, y, z \in X$:
    \begin{enumerate}
      \item $d(x, y) \geq 0$;
      \item $d(x, y) = 0$ if and only if $x = y$;
      \item $d(x, y) = d(y, x)$;
      \item \textbf{Triangle Inequality}

      $d(x, z) \leq d(x, y) + d(y, z)$
    \end{enumerate}
    A \emph{metric space} $X$ consists of a set $X$ and a metric on $X$. We
    call $d(x, y)$ the \emph{distance} between $x$ and $y$.
  \end{df}

  \begin{df}[Open Ball]
    Let $X$ be a metric space with metric $d$, $x \in X$ and $\epsilon > 0$.
    The \emph{open       ball} with \emph{centre} $x$ and \emph{radius}
    $\epsilon$
    is
    \[ B_d(x, \epsilon) = \{ y \in X : d(x, y) < \epsilon \} \enspace . \]
  \end{df}

  \begin{lm}
    \label{lm:topology:metric:balls}
    Let $X$ be a metric space, $x, y \in X$ and $\epsilon > 0$. If $y \in B(x,
    \epsilon)$, then there exists $\delta$ such that $0 < \delta < \epsilon$ and
    \[ B(y, \delta) \subseteq B(x, \epsilon) \enspace . \]
  \end{lm}

  \begin{proof}
    \pf
    \step{<1>1}{\pflet{$\delta = \epsilon - d(x, y)$}}
    \step{<1>2}{\pflet{$z \in B(y, \delta)$}}
    \step{<1>3}{$d(x, z) < \epsilon$}
    \begin{proof}
      \pf
      \begin{align*}
        d(x, z) & \leq d(x, y) + d(y, z) & (\text{Triangle Inequality}) \\
        & < d(x, y) + \delta & (\text{\stepref{<1>2}}) \\
        & = \epsilon & (\text{\stepref{<1>1}})
      \end{align*}
    \end{proof}
    \qed
  \end{proof}

  \begin{df}[Metric Topology]
    Let $d$ be a metric on $X$. The \emph{metric topology} on $X$ induced by
    $d$ is the topology generated by the basis consisting of the open balls.

    We prove this is a topology.
  \end{df}

  \begin{proof}
    \pf
    \step{<1>1}{Every point is in an open ball.}
    \begin{proof}
      \pf\ $x \in B(x, 1)$
    \end{proof}
    \step{<1>2}{If $B_1$, $B_2$ are open balls and $x \in B_1 \cap B_2$, then
      there
      exists an open ball $B_3$ such that $x \in B_3 \subseteq B_1 \cap B_2$.}
    \begin{proof}
      \step{<2>1}{\pflet{$x \in B(y, \epsilon_1) \cap B(z, \epsilon_2)$}}
      \step{<2>2}{\pick\ $\delta_1$, $\delta_2$ such that $0 < \delta_1 <
        \epsilon_1$, $0 < \delta_2 < \epsilon_2$, $B(x, \delta_1) \subseteq
        B(y,
        \epsilon_1)$ and $B(x, \delta_2) \subseteq B(z, \epsilon_2)$.}
      \begin{proof}
        \pf\ Lemma \ref{lm:topology:metric:balls}.
      \end{proof}
      \step{<2>3}{\pflet{$\delta = \min(\delta_1, \delta_2)$}}
      \step{<2>4}{$x \in B(x, \delta) \subseteq B(y, \epsilon_1) \cap B(y,
        \epsilon_2)$}
    \end{proof}
    \qedstep
    \begin{proof}
      \pf\ Lemma \ref{lm:topology:basis:generate}.
    \end{proof}
  \end{proof}

  \begin{lm}
    \label{lm:topology:metric:open}
    A set $U$ is open in the metric topology induced by $d$ if and only if, for
    all $x \in U$, there exists $\epsilon > 0$ such that $B(x, \epsilon)
    \subseteq
    U$.
  \end{lm}

  \begin{proof}
    \pf
    \step{<1>1}{If $U$ is open then, for all $x \in U$, there exists $\epsilon
      > 0$
      such that $B(x, \epsilon) \subseteq U$.}
    \begin{proof}
      \step{<2>1}{\assume{$U$ is open.}}
      \step{<2>2}{\pflet{$x \in U$}}
      \step{<2>3}{\pick\ $B(y, \delta)$ such that $x \in B(y, \delta) \subseteq
        U$}
      \step{<2>4}{\pick\ $\epsilon$ such that $0 < \epsilon < \delta$ and $B(x,
        \epsilon) \subseteq B(y, \delta)$}
      \begin{proof}
        \pf\ Lemma \ref{lm:topology:metric:balls}.
      \end{proof}
      \step{<2>5}{$B(x, \epsilon) \subseteq U$}
      \begin{proof}
        \pf\ From \stepref{<2>3} and \stepref{<2>4}.
      \end{proof}
    \end{proof}
    \step{<1>2}{If, for all $x \in U$, there exists $\epsilon > 0$ such that
      $B(x,
      \epsilon) \subseteq U$, then $U$ is open.}
    \begin{proof}
      \pf\ Immediate from definition of metric topology.
    \end{proof}
    \qed
  \end{proof}

  \begin{lm}
    Let $d$ and $d'$ be two metrics on the set $X$. Let $\mathcal{T}$ and
    $\mathcal{T}'$ be the topologies the induce, respectively. Then
    $\mathcal{T} \subseteq \mathcal{T}'$ if and only if, for all $x \in X$ and
    $\epsilon > 0$, there exists $\delta > 0$ such that $B_{d'}(x, \delta)
    \subseteq B_d(x, \epsilon)$.
  \end{lm}

  \begin{proof}
    \pf
    \step{<1>1}{If $\mathcal{T} \subseteq \mathcal{T}'$ then, for all $x \in X$
      and
      $\epsilon > 0$, there exists $\delta > 0$ such that $B_{d'}(x, \delta)
      \subseteq B_d(x, \epsilon)$.}
    \begin{proof}
      \step{<2>1}{\assume{$\mathcal{T} \subseteq \mathcal{T}'$}}
      \step{<2>2}{\pflet{$x \in X$ and $\epsilon > 0$}}
      \step{<2>3}{$B_d(x, \epsilon) \in \mathcal{T}'$}
      \begin{proof}
        \pf\ From \stepref{<2>1}.
      \end{proof}
      \step{<2>4}{There exists $\delta > 0$ such that $B_{d'}(x, \delta)
        \subseteq
        B_d(x, \epsilon)$}
      \begin{proof}
        \pf\ By Lemma \ref{lm:topology:metric:open}.
      \end{proof}
    \end{proof}
    \step{<1>2}{If, for all $x \in X$ and $\epsilon > 0$, there exists $\delta
      >
      0$
      such that $B_{d'}(x, \delta) \subseteq B_d(x, \epsilon)$, then
      $\mathcal{T}
      \subseteq \mathcal{T}'$}
    \begin{proof}
      \step{<2>1}{\assume{For all $x \in X$ and $\epsilon > 0$, there exists
          $\delta
          > 0$ such that $B_{d'}(x, \delta) \subseteq B_d(x, \epsilon)$.}}
      \step{<2>2}{\pflet{$U \in \mathcal{T}$} \prove{$U \in \mathcal{T}'$}}
      \step{<2>3}{\pflet{$x \in U$}}
      \step{<2>4}{\pick\ $\epsilon > 0$ be such that $B_d(x, \epsilon)
        \subseteq
        U$}
      \begin{proof}
        \pf\ By Lemma \ref{lm:topology:metric:open}.
      \end{proof}
      \step{<2>5}{\pick\ $\delta > 0$ such that $B_{d'}(x, \delta) \subseteq
        B_d(x,
        \epsilon)$}
      \begin{proof}
        \pf\ By \stepref{<2>1}.
      \end{proof}
      \step{<2>6}{$B_{d'}(x, \delta) \subseteq U$}
      \begin{proof}
        \pf\ By \stepref{<2>4} and \stepref{<2>5}.
      \end{proof}
      \qedstep
      \begin{proof}
        \pf\ By Lemma \ref{lm:topology:metric:open}.
      \end{proof}
    \end{proof}
    \qed
  \end{proof}

  \begin{df}[Metrizable]
    A topological space is \emph{metrizable} if and only if there exists a
    metric that induces its topology.
  \end{df}

  \begin{lm}
    Every discrete space is metrizable.
  \end{lm}

  \begin{proof}
    \pf\ The discrete topology is induced by the metric $d(x, y) = 1$ if $x
    \neq
    y$, 0 if $x = y$. \qed
  \end{proof}

  \begin{prop}
    The continuous image of a metrizable space is not necessarily metrizable.
  \end{prop}

  \begin{proof}
    \pf\ The identity map from the discrete two-point space to the indiscrete two-point space is continuous. \qed
  \end{proof}

  \begin{lm}
    $\mathbb{R}$ is metrizable.
  \end{lm}

  \begin{proof}
    \pf\ The standard topology is induced by the metric $d(x, y) = |x-y|$. \qed
  \end{proof}

  \begin{df}[Bounded]
    Let $X$ be a metric space and $A \subseteq X$. Then $A$ is \emph{bounded}
    iff $\{ d(x, y) : x, y \in A \}$ is bounded above, in which case its
    \emph{diameter} is
    \[ \diam A = \sup_{x, y \in A} d(x, y) \enspace . \]
  \end{df}

  \begin{lm}
    \label{lm:topology:metric:subspace}
    Let $(X, d)$ be a metric space and $A \subseteq X$. Then $d \restriction_{A
      \times A}$ is a metric on $A$ that induces the subspace topology.
  \end{lm}

  \begin{proof}
    \pf
    \step{<1>1}{$d \restriction_{A \times A}$ is a metric on $A$.}
    \begin{proof}
      \pf\ Each of the axioms for a metric follows immediately from the same
      axiom for $d$.
    \end{proof}
    \step{<1>2}{The topology induced by $d \restriction_{A \times A}$ is the
      product topology.}
    \begin{proof}
      \pf\ Both are the topology generated by the basis consisting of all
      the open balls $B_{d \restriction_{A \times A}}(a, \epsilon) = B_d(a,
      \epsilon)  \cap      A$.
    \end{proof}
    \qed
  \end{proof}

  \begin{lm}
    Every metric space is Hausdorff.
  \end{lm}

  \begin{proof}
    \pf
    \step{<1>1}{\pflet{$X$ be a metric space and $x, y \in X$ with $x \neq y$.}}
    \step{<1>2}{\pflet{$\epsilon = d(x, y)$}}
    \step{<1>3}{$B(x, \epsilon / 2)$ and $B(y, \epsilon / 2)$ are disjoint
      neighbourhoods of $x$ and $y$.}
    \qed
  \end{proof}

  \begin{thm}
    Every metric space is first countable.
  \end{thm}

  \begin{proof}
    \pf\ $\{ B(x, q) : q \in \mathbb{Q}^+ \}$ is a local basis at $x$. \qed
  \end{proof}

  \begin{cor}
  	If $J$ is infinite then the space $\mathbb{R}^J$ is not metrizable.
  \end{cor}

  \begin{df}[Standard Bounded Metric]
    Let $d$ be a metric on $X$. The \emph{standard bounded metric}
    corresponding to $d$ is
    \[ \overline{d}(x, y) = \min(d(x, y), 1) \enspace . \]

    We prove this is a metric.
  \end{df}

  \begin{proof}
    \pf
    \step{<1>1}{$\overline{d}(x, y) \geq 0$}
    \begin{proof}
      \pf\ This holds because $d(x, y) \geq 0$ ($d$ is a metric) and $1 > 0$.
    \end{proof}
    \step{<1>2}{$\overline{d}(x, y) = 0$ iff $x = y$}
    \begin{proof}
      \pf\ Immediate from definition.
    \end{proof}
    \step{<1>3}{$\overline{d}(x, y) = \overline{d}(y, x)$}
    \begin{proof}
      \pf\ Immediate from definition.
    \end{proof}
    \step{<1>4}{$\overline{d}(x, z) \leq \overline{d}(x, y) + \overline{d}(y,
      z)$}
    \begin{proof}
      \step{<2>1}{\case{$d(x, y) \leq 1$, $d(y, z) \leq 1$}}
      \begin{proof}
        \pf
        \begin{align*}
          \overline{d}(x, z) & \leq d(x, z) \\
          & \leq d(x, y) + d(y, z) \\
          & = \overline{d}(x, y) + \overline{d}(y, z)
        \end{align*}
      \end{proof}
      \step{<2>2}{\case{$d(y, z) > 1$}}
      \begin{proof}
        \pf
        \begin{align*}
          \overline{d}(x, z) & \leq 1 \\
          & \leq \overline{d}(x, y) + 1 \\
          & = \overline{d}(x, y) + \overline{d}(y, z)
        \end{align*}
      \end{proof}
      \step{<2>3}{\case{$d(x, y) > 1$}}
      \begin{proof}
        \pf\ Similar.
      \end{proof}
    \end{proof}
    \qed
  \end{proof}

  \begin{thm}
    Let $d$ be a metric on $X$. Then the standard bounded metric $\overline{d}$
    corresponding to $d$ induces the same topology as $d$.
  \end{thm}

  \begin{proof}
    \pf
    \step{<1>1}{\pflet{$\mathcal{T}$ be the topology induced by $d$ and
        $\mathcal{T}'$ be the topology induced by $\overline{d}$.}}
    \step{<1>2}{$\mathcal{T} \subseteq \mathcal{T}'$}
    \begin{proof}
      \step{<2>1}{\pflet{$x \in X$ and $\epsilon > 0$}}
      \step{<2>2}{\pflet{$\delta = \min(\epsilon, 1/2)$}}
      \step{<2>3}{$B_{\overline{d}}(x, \delta) \subseteq B_d(x, \epsilon)$}
      \begin{proof}
        \step{<3>1}{\pflet{$y \in B_{\overline{d}}(x, \delta)$}}
        \step{<3>2}{$\overline{d}(x, y) < \delta$}
        \step{<3>3}{$\overline{d}(x, y) < 1$}
        \begin{proof}
          \pf\ From \stepref{<2>2} and \stepref{<3>2}.
        \end{proof}
        \step{<3>4}{$\overline{d}(x, y) = d(x, y)$}
        \begin{proof}
          \pf\ From \stepref{<3>3} and the definition of $\overline{d}$.
        \end{proof}
        \step{<3>5}{$d(x, y) < \epsilon$}
        \begin{proof}
          \pf\ By \stepref{<2>2} and \stepref{<3>2} and \stepref{<3>4}.
        \end{proof}
      \end{proof}
    \end{proof}
    \step{<1>3}{$\mathcal{T}' \subseteq \mathcal{T}$}
    \begin{proof}
      \step{<2>1}{\pflet{$x \in X$ and $\epsilon > 0$}}
      \step{<2>2}{$B_d(x, \epsilon) \subseteq B_{\overline{d}}(x, \epsilon)$}
      \begin{proof}
        \pf\ This holds because $\overline{d}(x, y) \leq d(x, y)$.
      \end{proof}
    \end{proof}
    \qed
  \end{proof}

  \begin{df}[Square Metric]
    The \emph{square metric} on $\mathbb{R}^n$ is defined by
    \[ \rho(\vec{x}, \vec{y}) = \max(|x_1 - y_1|, \ldots, |x_n - y_n|) \enspace
    . \]

    We prove this is a metric.
  \end{df}

  \begin{proof}
    \pf
    \step{<1>1}{$\rho(\vec{x}, \vec{y}) \geq 0$}
    \begin{proof}
      \pf\ Immediate from definitions.
    \end{proof}
    \step{<1>2}{$\rho(\vec{x}, \vec{y}) = 0$ iff $\vec{x} = \vec{y}$}
    \begin{proof}
      \pf\ Immediate from definitions.
    \end{proof}
    \step{<1>3}{$\rho(\vec{x}, \vec{y}) = \rho(\vec{y}, \vec{x})$}
    \begin{proof}
      \pf\ Immediate from definitions.
    \end{proof}
    \step{<1>4}{$\rho(\vec{x}, \vec{z}) \leq \rho(\vec{x}, \vec{y}) +
      \rho(\vec{y},
      \vec{z})$}
    \begin{proof}
      \step{<2>1}{For all $i$, we have $|x_i - z_i| \leq |x_i - y_i| + |y_i -
        z_i|$}
      \step{<2>2}{For all $i$, $|x_i - z_i| \leq \rho(\vec{x}, \vec{y}) +
        \rho(\vec{y}, \vec{z})$}
      \step{<2>3}{$\rho(\vec{x}, \vec{z}) \leq \rho(\vec{x}, \vec{y}) +
        \rho(\vec{y},
        \vec{z})$}
    \end{proof}
    \qed
  \end{proof}

  \begin{thm}
    The square metric induces the standard topology on $\mathbb{R}^n$.
  \end{thm}

  \begin{proof}
    \pf
    \step{<1>1}{\pflet{$\mathcal{T}_\rho$ be the topology induced by the square
        metric and $\mathcal{T}_s$ the standard topology.}}
    \step{<1>2}{$\mathcal{T}_\rho \subseteq \mathcal{T}_s$}
    \begin{proof}
      \pf\ This holds because $B_\rho(\vec{x}, \epsilon) = (x_1 - \epsilon, x_1
      + \epsilon) \times \cdots \times (x_n - \epsilon, x_n + \epsilon)$.
    \end{proof}
    \step{<1>3}{$\mathcal{T}_s \subseteq \mathcal{T}_\rho$}
    \begin{proof}
      \step{<2>1}{\pflet{$B = U_1 \times \cdots \times U_n$ be a
          basic          open set in $\mathcal{T}_s$, where each $U_i$ is open
          in $\mathbb{R}$.}}
      \step{<2>2}{\pflet{$\vec{x} \in B$}}
      \step{<2>3}{For $1 \leq i \leq n$, \pick\ $\epsilon_i > 0$ such that
        $(x_i       -
        \epsilon_i, x_i + \epsilon_i) \subseteq U_i$}
      \step{<2>4}{\pflet{$\epsilon = \min(\epsilon_1, \ldots, \epsilon_n$)}}
      \step{<2>5}{$B_\rho(\vec{x}, \epsilon) \subseteq B$}
    \end{proof}
    \qed
  \end{proof}

  \begin{lm}
    \label{lm:topology:metric:product}
    The product of a countable family of metrizable spaces is metrizable.
  \end{lm}

  \begin{proof}
    \pf
    \step{<1>1}{\pflet{$\{ X_n \}_{n \in \mathbb{Z}^+}$ be a family of metric
        spaces with metrics bounded by 1, $X = \prod_{n=1}^\infty X_n$.}}
    \step{<1>2}{\pflet{$D : X \times X \rightarrow \mathbb{R}$ be given by
        \[ D(\vec{x}, \vec{y}) = \sup_{n \geq 1} \frac{d(x_n, y_n)}{n} \enspace
        . \]}}
    \step{<1>3}{$D$ is a metric on $X$.}
    \begin{proof}
      \step{<2>1}{$D(\vec{x}, \vec{y}) \geq 0$}
      \begin{proof}
        \pf\ Immediate from definitions.
      \end{proof}
      \step{<2>2}{$D(\vec{x}, \vec{y}) = 0$ iff $\vec{x} = \vec{y}$}
      \begin{proof}
        \pf\ Immediate from definitions.
      \end{proof}
      \step{<2>3}{$D(\vec{x}, \vec{y}) = D(\vec{y}, \vec{x})$}
      \begin{proof}
        \pf\ Immediate from definitions.
      \end{proof}
      \step{<2>4}{$D(\vec{x}, \vec{z}) \leq D(\vec{x}, \vec{y}) + D(\vec{y},
        \vec{z})$}
      \begin{proof}
        \step{<3>1}{For all $n$, we have $\frac{d(x_n, z_n)}{n} \leq
          \frac{d(x_n,
            y_n)}{n} + \frac{d(y_n, z_n)}{n}$}
        \step{<3>2}{For all $n$, we have $\frac{d(x_n, z_n)}{n} \leq D(\vec{x},
          \vec{y}) + D(\vec{y}, \vec{z})$}
        \step{<3>3}{$D(\vec{x}, \vec{z}) \leq D(\vec{x}, \vec{y}) + D(\vec{y},
          \vec{z})$}
      \end{proof}
    \end{proof}
    \step{<1>4}{\pflet{$\mathcal{T}_D$ be the topology induced by $D$ and
        $\mathcal{T}_p$ the product topology.}}
    \step{<1>5}{$\mathcal{T}_D \subseteq \mathcal{T}_p$}
    \begin{proof}
      \step{<2>1}{\pflet{$U \in \mathcal{T}_D$} \prove{$U \in \mathcal{T}_p$}}
      \step{<2>2}{\pflet{$\vec{x} \in U$}}
      \step{<2>3}{\pick\ $\epsilon > 0$ such that $B_D(\vec{x}, \epsilon)
        \subseteq
        U$}
      \step{<2>4}{\pick\ $N$ such that $1 / N < \epsilon$}
      \step{<2>5}{\pflet{$V = B(x_1, \epsilon) \times \cdots \times B(x_N,
          \epsilon) \times X_{N+1} \times X_{N+2} \times \cdots$}}
      \step{<2>6}{$\vec{x} \in V \subseteq B_D(\vec{x}, \epsilon)$}
    \end{proof}
    \step{<1>6}{$\mathcal{T}_p \subseteq \mathcal{T}_D$}
    \begin{proof}
      \step{<2>1}{\pflet{$U = \prod_{n=1}^\infty U_n$ be a basic open set in
          $\mathcal{T}_p$, where each $U_n$ is open in $X_n$, and $U_n = X_n$
          for $n > N$.}}
      \step{<2>2}{\pflet{$\vec{x} \in U$} \prove{There exists $\epsilon > 0$
          such
          that $B_D(\vec{x}, \epsilon) \subseteq U$.}}
      \step{<2>3}{For $n \leq N$, \pick\ $\epsilon_n > 0$ such that $B(x_n,
        \epsilon_n) \subseteq U_n$}
      \step{<2>4}{\pflet{$\epsilon = \min(\epsilon_1, \epsilon_2 / 2, \ldots,
          \epsilon_n / n)$}}
      \step{<2>5}{\pflet{$\vec{y} \in B_D(\vec{x}, \epsilon)$}}
      \step{<2>6}{For $n \leq N$, $y_n \in U_n$}
      \begin{proof}
        \step{<3>1}{$D(\vec{x}, \vec{y}) < \epsilon$}
        \step{<3>2}{$d(x_n, y_n) / n < \epsilon$}
        \step{<3>3}{$d(x_n, y_n) / n < \epsilon_n / n$}
        \qedstep
        \begin{proof}
          \pf\ By \stepref{<2>3}.
        \end{proof}
      \end{proof}
    \end{proof}
    \qed
  \end{proof}

  \begin{cor}
    The space $\mathbb{R}^\omega$ is metrizable.
  \end{cor}

  \begin{df}[Uniform Metric]
    Let $J$ be a set. The \emph{uniform metric} $\overline{\rho}$ on
    $\mathbb{R}^J$ is defined by
    \[ \overline{\rho}(\vec{x}, \vec{y}) = \sup_{\alpha \in J}
    \overline{d}(x_\alpha, y_\alpha) \enspace . \]
    where $\overline{d}$ is the standard bounded metric on $\mathbb{R}$. The
    \emph{uniform topology} is the topology induced by the uniform metric.

    We prove this is a metric.
  \end{df}

  \begin{proof}
    \pf
    \step{<1>1}{$\overline{\rho}(\vec{x}, \vec{y}) \geq 0$}
    \begin{proof}
      \pf\ Immediate from definitions.
    \end{proof}
    \step{<1>2}{$\overline{\rho}(\vec{x}, \vec{y}) = 0$ iff $\vec{x} = \vec{y}$}
    \begin{proof}
      \pf\ Immediate from definitions.
    \end{proof}
    \step{<1>3}{$\overline{\rho}(\vec{x}, \vec{y}) = \overline{\rho}(\vec{y},
      \vec{x})$}
    \begin{proof}
      \pf\ Immediate from definitions.
    \end{proof}
    \step{<1>4}{$\overline{\rho}(\vec{x}, \vec{z}) \leq
      \overline{\rho}(\vec{x},
      \vec{y}) + \overline{\rho}(\vec{y}, \vec{z})$}
    \begin{proof}
      \pf
      \step{<2>1}{For all $\alpha \in J$, $\overline{d}(x_\alpha, z_\alpha)
        \leq
        \overline{d}(x_\alpha, y_\alpha) + \overline{d}(y_\alpha, z_\alpha)$}
      \step{<2>2}{For all $\alpha \in J$, $\overline{d}(x_\alpha, z_\alpha)
        \leq
        \overline{\rho}(\vec{x}, \vec{y}) + \overline{\rho}(\vec{y}, \vec{z})$}
      \step{<2>3}{$\overline{\rho}(\vec{x}, \vec{z}) \leq
        \overline{\rho}(\vec{x},
        \vec{y}) + \overline{\rho}(\vec{y}, \vec{z})$}
    \end{proof}
    \qed
  \end{proof}

  \begin{thm}[DC]
  	\label{thm:topology:product:compare}
    The uniform topology on $\mathbb{R}^J$ is finer than the product topology
    and coarser than the box topology. These three topologies are different iff
    $J$
    is infinite.
  \end{thm}

  \begin{proof}
    \pf
    \step{<1>1}{The uniform topology is finer than the product topology.}
    \begin{proof}
      \step{<2>1}{\pflet{$B = \prod_{\alpha \in J} U_\alpha$ be a basic open
          set
          in           the product topology, where each $U_\alpha$ is open in
          $\mathbb{R}$,           and $U_\alpha = \mathbb{R}$ except for
          $\alpha         =
          \alpha_1, \ldots,           \alpha_n$.}}
      \step{<2>2}{\pflet{$\vec{x} \in U$}}
      \step{<2>3}{For $1 \leq i \leq n$, \pick\ $0 < \epsilon_i < 1$ such that
        $(x_{\alpha_i} - \epsilon_i, x_{\alpha_i} + \epsilon_i) \subseteq
        U_{\alpha_i}$.}
      \step{<2>4}{\pflet{$\epsilon = \min(\epsilon_1, \ldots, \epsilon_n)$}}
      \step{<2>5}{$B_{\overline{\rho}}(\vec{x}, \epsilon) \subseteq B$}
      \begin{proof}
        \step{<3>1}{\pflet{$\vec{y} \in B_{\overline{\rho}}(\vec{x},
            \epsilon)$}}
        \step{<3>2}{For $1 \leq i \leq n$, we have $y_i \in U_{\alpha_i}$}
        \begin{proof}
          \step{<4>1}{\pflet{$1 \leq i \leq n$}}
          \step{<4>2}{$\overline{d}(x_{\alpha_i}, y_{\alpha_i}) < \epsilon_i$}
          \begin{proof}
            \pf\ From \stepref{<2>4} and \stepref{<3>1}.
          \end{proof}
          \step{<4>3}{$d(x_{\alpha_i}, y_{\alpha_i}) < \epsilon_i$}
          \begin{proof}
            \pf\ From \stepref{<4>2} since $\epsilon_i < 1$ (\stepref{<2>3}).
          \end{proof}
          \qedstep
          \begin{proof}
            \pf\ By \stepref{<2>3}.
          \end{proof}
        \end{proof}
      \end{proof}
    \end{proof}
    \step{<1>2}{The uniform topology is coarser than the box topology.}
    \begin{proof}
      \step{<2>1}{\pflet{$\vec{x} \in \mathbb{R}^J$ and $\epsilon > 0$}
        \prove{$B_{\overline{\rho}}(\vec{x}, \epsilon)$ is open in the box
          topology.}}
      \step{<2>2}{\case{$\epsilon < 1$}}
      \begin{proof}
        \pf\ In this case, $B(\vec{x}, \epsilon) = \prod_{\alpha \in J}
        (x_\alpha - \epsilon, x_\alpha + \epsilon)$.
      \end{proof}
      \step{<2>3}{\case{$\epsilon \geq 1$}}
      \begin{proof}
        \pf\ In this case, $B(\vec{x}, \epsilon) = \mathbb{R}^J$.
      \end{proof}
    \end{proof}
    \step{<1>3}{If $J$ is finite then the product topology is the same as the
      box
      topology.}
    \begin{proof}
      \pf\ Immediate from definitions.
    \end{proof}
    \step{<1>4}{If $J$ is infinite then the uniform topology is distinct from
      the product topology.}
    \begin{proof}
      \step{<2>1}{$B(\vec{0}, 1/2)$ is not open in the product topology.}
      \begin{proof}
        \step{<3>1}{$\vec{0} \in B(\vec{0}, 1/2)$}
        \step{<3>2}{\pflet{$\prod_{\alpha \in J} U_\alpha$ be any basic open
            set
            containing $\vec{0}$, where $U_\alpha$ is open in $\mathbb{R}$ for
            all $\alpha$, and $U_\alpha = \mathbb{R}$ except for $\alpha =
            \alpha_1, \ldots, \alpha_n$}}
        \step{<3>3}{\pick\ $\alpha_0 \in J$ such that $\alpha_0 \neq \alpha_1,
          \ldots,           \alpha_n$}
        \step{<3>4}{\pflet{$\vec{x}$ be such that $x_{\alpha_0} = 1$, and
            $x_\alpha = 0$ for $\alpha \neq \alpha_0$.}}
        \step{<3>5}{$\vec{x} \in \prod_{\alpha \in J} U_\alpha$}
        \step{<3>6}{$\vec{x} \notin B(\vec{0}, 1/2)$}
      \end{proof}
    \end{proof}
    \step{<1>5}{If $J$ is infinite then the uniform topology is distinct from
      the
      box topology.}
    \begin{proof}
      \step{<2>1}{\pick\ a countable sequence $\alpha_1$, $\alpha_2$, \ldots in
        $J$}
      \step{<2>2}{\pflet{$U = \prod_{\alpha \in J} U_\alpha$, where
          $U_{\alpha_n}
          = (-1/n, 1/n)$ for all $n$, and $U_\alpha = \mathbb{R}$ for all other
          $\alpha$.} \prove{$U$ is not open in the uniform topology.}}
      \step{<2>3}{$\vec{0} \in U$}
      \step{<2>4}{\pflet{$\epsilon > 0$} \prove{$B(\vec{0}, \epsilon)
          \nsubseteq
          U$}}
      \step{<2>5}{\pick\ $N$ such that $1/N < \epsilon$}
      \step{<2>6}{\pflet{$\vec{x}$ be such that $x_{\alpha_N} = 1/N$ and
          $x_\alpha
          = 0$ for all other $\alpha$}}
      \step{<2>7}{$\vec{x} \in B(\vec{0}, \epsilon)$}
      \step{<2>8}{$\vec{x} \notin U$}
    \end{proof}
    \qed
  \end{proof}

   \begin{prop}
   The space $\mathbb{R}^\omega$ under the uniform topology is not second
countable.
 \end{prop}

 \begin{proof}
   \pf\ The set of all sequences of 0s and 1s is discrete but uncountable. \qed
 \end{proof}

\begin{cor}
  Not every metric space is second countable.
\end{cor}

  \begin{thm}
    \label{thm:topology:metric:continuous}
    Let $X$ and $Y$ be metric spaces. Let $f : X \rightarrow Y$ and $x \in X$.
    Then $f$ is continuous at $x$ if and only if, for every $\epsilon > 0$,
    there
    exists $\delta > 0$ such that, for all $x' \in X$, if $d(x, x') < \delta$
    then
    $d(f(x), f(x')) < \epsilon$.
  \end{thm}

  \begin{proof}
    \pf
    \step{<1>1}{If $f$ is continuous at $x$ then, for every $\epsilon > 0$,
      there
      exists $\delta > 0$ such that, for all $x' \in X$, if $d(x, x') < \delta$
      then $d(f(x), f(x')) < \epsilon$.}
    \begin{proof}
      \step{<2>1}{\assume{$f$ is continuous at $x$.}}
      \step{<2>2}{\pflet{$\epsilon > 0$}}
      \step{<2>3}{\pick\ a neighbourhood $U$ of $x$ such that $f(U) \subseteq
        B(f(x), \epsilon)$}
      \begin{proof}
        \pf\ One exists by \stepref{<2>1}, since $B(f(x), \epsilon)$ is a
        neighbourhood of $f(x)$.
      \end{proof}
      \step{<2>4}{\pick\ $\delta > 0$ such that $B(x, \delta) \subseteq U$}
      \begin{proof}
        \pf\ By \stepref{<2>3} and Lemma \ref{lm:topology:metric:open}.
      \end{proof}
      \step{<2>5}{\pflet{$x' \in X$ with $d(x, x') < \delta$}}
      \step{<2>6}{$x' \in U$}
      \begin{proof}
        \pf\ From \stepref{<2>4} and \stepref{<2>5}.
      \end{proof}
      \step{<2>7}{$f(x') \in B(f(x), \epsilon)$}
      \begin{proof}
        \pf\ From \stepref{<2>3} and \stepref{<2>6}.
      \end{proof}
    \end{proof}
    \step{<1>2}{If, for all $\epsilon > 0$, there exists $\delta > 0$ such
      that,
      for all $x' \in X$, if $d(x, x') < \delta$ then $d(f(x), f(x')) <
      \epsilon$, then $f$ is continuous at $x$.}
    \begin{proof}
      \step{<2>1}{\assume{For all $\epsilon > 0$ there exists $\delta > 0$ such
          that, for all $x' \in X$, if $d(x, x') < \delta$ then $d(f(x), f(x'))
          <
          \epsilon$.}}
      \step{<2>2}{\pflet{$V$ be a neighbourhood of $f(x)$}}
      \step{<2>3}{\pick\ $\epsilon > 0$ such that $B(f(x), \epsilon) \subseteq
        V$} % TODO Extract lemma
      \begin{proof}
        \pf\ By Lemma \ref{lm:topology:metric:open}.
      \end{proof}
      \step{<2>4}{\pick\ $\delta > 0$ such that, for all $x' \in X$, if $d(x,
        x')
        <
        \delta$ then $d(f(x), f(x')) < \epsilon$.}
      \begin{proof}
        \pf\ By \stepref{<2>1} and \stepref{<2>3}.
      \end{proof}
      \step{<2>5}{$B(x, \delta)$ is a neighbourhood of $x$}
      \begin{proof}
        \pf\ By the definition of the metric topology.
      \end{proof}
      \step{<2>6}{$f(B(x, \delta)) \subseteq V$}
      \begin{proof}
        \step{<3>1}{\pflet{$x' \in B(x, \delta)$}}
        \step{<3>2}{$d(f(x), f(x')) < \epsilon$}
        \begin{proof}
          \pf\ From \stepref{<2>4}.
        \end{proof}
        \step{<3>3}{$x' \in V$}
        \begin{proof}
          \pf\ From \stepref{<2>3}.
        \end{proof}
      \end{proof}
    \end{proof}
    \qed
  \end{proof}

  \begin{lm}
    Addition is a continuous function $\mathbb{R}^2 \rightarrow \mathbb{R}$.
  \end{lm}

  \begin{proof}
    \pf
    \step{<1>1}{\pflet{$(x, y) \in \mathbb{R}^2$ and $\epsilon > 0$}}
    \step{<1>2}{\pflet{$\delta = \epsilon / 2$}}
    \step{<1>3}{\pflet{$(x', y') \in \mathbb{R}^2$ be such that $\rho((x, y),
        (x',
        y')) < \delta$, where $\rho$ is the square metric}}
    \step{<1>4}{$|x - x'| < \delta$ and $|y - y'| < \delta$}
    \step{<1>5}{$|(x + y) - (x' + y')| < \epsilon$}
    \begin{proof}
      \pf
      \begin{align*}
        |(x + y) - (x' + y')| & \leq |x - x'| + |y - y'| \\
        & < 2 \delta & (\text{\stepref{<1>4}}) \\
        & = \epsilon & (\text{\stepref{<1>2}})
      \end{align*}
    \end{proof}
    \qedstep
    \begin{proof}
      \pf\ By Theorem \ref{thm:topology:metric:continuous}.
    \end{proof}
    \qed
  \end{proof}

  \begin{lm}
    Additive inverse is a continuous function $- : \mathbb{R} \rightarrow
    \mathbb{R}$.
  \end{lm}

  \begin{proof}
    \pf\ If $|x - y| < \epsilon$ then $|(-x)-(-y)| < \epsilon$. \qed
  \end{proof}


  \begin{lm}
    Multiplication is a continuous function $\cdot : \mathbb{R}^2 \rightarrow
    \mathbb{R}$.
  \end{lm}

  \begin{proof}
    \pf
    \step{<1>1}{\pflet{$(x,y) \in \mathbb{R}^2$ and $\epsilon > 0$}}
    \step{<1>2}{\pflet{$\delta = \min(1, \epsilon / (|x| + |y| + 1)$}}
    \step{<1>3}{\pflet{$(x', y') \in \mathbb{R}^2$ and $\rho((x, y), (x', y'))
        <
        \delta$}}
    \step{<1>4}{$|xy - x'y'| < \epsilon$}
    \begin{proof}
      \pf
      \begin{align*}
        |xy - x'y'| & = |x (y' - y) + y (x' - x) + (x - x') (y - y')| \\
        & \leq |x| |y'- y| + |y| |x' - x| + |x - x'| |y - y'| \\
        & < |x| \delta + |y| \delta + \delta^2 & (\text{\stepref{<1>3}}) \\
        & = \delta (|x| + |y| + \delta) \\
        & \leq \delta (|x| + |y| + 1) & (\text{\stepref{<1>2}}) \\
        & \leq \epsilon & (\text{\stepref{<1>2}})
      \end{align*}
    \end{proof}
    \qed
  \end{proof}

  \begin{lm}
    Multiplicative inverse is a continuous function $(\ )^{-1} : \mathbb{R}
    \setminus \{ 0 \} \rightarrow \mathbb{R}$.
  \end{lm}

  \begin{proof}
    \pf
    \step{<1>1}{\pflet{$f : \mathbb{R} \setminus \{ 0 \} \rightarrow
        \mathbb{R}$
        be defined by $f(x) = x^{-1}$.}}
    \step{<1>2}{\pflet{$a, b \in \mathbb{R}$ with $a < b$} \prove{$f^{-1}((a,
        b))$
        is open}}
    \step{<1>3}{\case{$0 < a < b$}}
    \begin{proof}
      \pf\ $f^{-1}((a, b)) = (b^{-1}, a^{-1})$
    \end{proof}
    \step{<1>4}{\case{$a < 0 < b$}}
    \begin{proof}
      \pf\ $f^{-1}((a, b)) = (-\infty, a^{-1}) \cup (b^{-1}, + \infty)$
    \end{proof}
    \step{<1>5}{\case{$a < b < 0$}}
    \begin{proof}
      \pf\ $f^{-1}((a, b)) = (b^{-1}, a^{-1})$
    \end{proof}
    \qed
  \end{proof}

  \begin{df}[Uniform Convergence]
    Let $X$ be a set and $Y$ a metric space. Let $f_n : X \rightarrow Y$ for $n
    \geq 1$, and $f : X \rightarrow Y$. Then $f_n$ \emph{converges uniformly}
    to
    $f$ as $n \rightarrow \infty$ iff, for all $\epsilon > 0$, there exists $N$
    such that, for all $x \in X$ and $n \geq N$, $d(f_n(x), f(x)) < \epsilon$.
  \end{df}

  \begin{thm}[Uniform Limit Theorem]
    Let $X$ be a topological space and $Y$ a metric space. Let $f_n : X
    \rightarrow Y$ for $n \geq 1$ and $f : X \rightarrow Y$. If $f_n$ converges
    uniformly to $f$ and each $f_n$ is continuous, then $f$ is continuous.
  \end{thm}

  \begin{proof}
    \pf
    \step{<1>1}{\pflet{$x \in X$ and $\epsilon > 0$}}
    \step{<1>2}{\pick\ $N$ such that, for all $x' \in X$ and $\delta > 0$,
      $d(f_n(x'), f(x')) < \epsilon / 3$}
    \step{<1>3}{\pick\ $\delta > 0$ such that, for all $x' \in X$, if $d(x, x')
      <
      \delta$ then $d(f_N(x), f_N(x')) < \epsilon / 3$}
    \step{<1>4}{For all $x' \in X$, if $d(x, x') < \delta$ then $d(f(x), f(x'))
      <
      \epsilon$}
    \begin{proof}
      \pf
      \begin{align*}
        d(f(x), f(x')) & \leq d(f(x), f_N(x)) + d(f_N(x), f_N(x')) + d(f_N(x'),
        f(x')) \\
        & < \epsilon / 3 + \epsilon / 3 + \epsilon / 3 \\
        & = \epsilon
      \end{align*}
    \end{proof}
    \qed
  \end{proof}

  \begin{lm}
    Let $X$ be a set. Let $f_n : X \rightarrow \mathbb{R}$ for $n \geq 1$ and
    $f : X \rightarrow \mathbb{R}$. Then $f_n$ converges uniformly to $f$ if
    and only if $f_n$ converges to $f$ in $\mathbb{R}^X$ under the uniform
    topology.
  \end{lm}

  \begin{proof}
    \pf
    \step{<1>1}{If $f_n$ converges uniformly to $f$ then $f_n$ converges to $f$
      under the uniform topology.}
    \begin{proof}
      \step{<2>1}{\assume{$f_n$ converges uniformly to $f$}}
      \step{<2>2}{\pflet{$\epsilon > 0$}}
      \step{<2>3}{\pick\ $N$ such that, for all $x \in X$ and $n \geq N$,
        $d(f_n(x), f(x)) < \epsilon / 2$}
      \step{<2>4}{$\overline{\rho}(f_n, f) \leq \epsilon / 2$}
      \step{<2>5}{$\overline{\rho}(f_n, f) < \epsilon$}
    \end{proof}
    \step{<1>2}{If $f_n$ converges to $f$ under the uniform topology then $f_n$
      converges uniformly to $f$.}
    \begin{proof}
      \step{<2>1}{\assume{$f_n$ converges to $f$ under the uniform topology.}}
      \step{<2>2}{\pflet{$\epsilon > 0$}}
      \step{<2>3}{\pick\ $N$ such that, for all $n \geq N$,
        $\overline{\rho}(f_n,
        f) < \epsilon$}
      \step{<2>4}{For all $n \geq N$ and $x \in X$, $d(f_n(x), f(x)) <
        \epsilon$}
    \end{proof}
    \qed
  \end{proof}

  \begin{thm}
    Every monotone increasing sequence of real numbers that is bounded above
    converges to its supremum.
  \end{thm}

  \begin{proof}
    \pf
    \step{<1>1}{\pflet{$\{ s_n \}_{n \geq 1}$ be a monotone increasing sequence
        of
        real numbers bounded above with supremum $l$.}}
    \step{<1>2}{\pflet{$\epsilon > 0$}}
    \step{<1>3}{$l - \epsilon$ is not an upper bound for $\{ s_n : n \geq 1
      \}$.}
    \step{<1>4}{\pick\ $N$ such that $x_N > l - \epsilon$}
    \step{<1>5}{For all $n \geq N$, we hawe $l - \epsilon < x_n \leq l$}
    \step{<1>6}{For all $n \geq N$, we have $|x_n - l| < \epsilon$}
    \qed
  \end{proof}

  \begin{df}[Infinite Series]
    Let $\{ a_n \}_{n \geq 1}$ be a sequence of real numbers. The
    \emph{infinite series} $\sum_{n=1}^\infty a_n$ \emph{converges} to $s$ iff
    $\sum_{n=1}^N a_n \rightarrow s$ as $N \rightarrow \infty$.
  \end{df}

  \begin{prop}
    If $\sum_{n=1}^\infty a_n = s$ and $\sum_{n=1}^\infty b_n = t$ then
    $\sum_{n=1}^\infty (c a-n + b_n) = cs + t$.
  \end{prop}

  \begin{proof}
    \pf This holds because $\sum_{n=1}^N (c a_n + b_n) = c \sum_{n=1}^N a_n +
    \sum_{n=1}^N b_n \rightarrow cs + t$ as $N \rightarrow \infty$. \qed
  \end{proof}

  \begin{thm}[Comparison Test]
    If $|a_i| \leq b_i$ for all $i$ and $\sum_{i=1}^\infty b_i$ converges, then
    $\sum_{i=1}^\infty a_i$ converges.
  \end{thm}

  \begin{proof}
    \pf
    \step{<1>1}{$\sum_{i=1}^\infty |a_i|$ converges}
    \begin{proof}
      \pf\ $\sum_{i=1}^N |a_i|$ is a monotone increasing sequence bounded above
      by $\sum_{i=1}^\infty b_i$.
    \end{proof}
    \step{<1>2}{\pflet{$c_i = |a_i| + a_i$}}
    \step{<1>3}{$\sum_{i=1}^\infty c_i$ converges}
    \begin{proof}
      \pf\ $\sum_{i=1}^N c_i$ is a monotone increasing sequence bounded above
      by
      $2 \sum_{i=1}^\infty |a_i|$.
    \end{proof}
    \qedstep
    \begin{proof}
      \pf\ Since $a_i = c_i - |a_i|$.
    \end{proof}
  \end{proof}

  \begin{lm}
    \label{lm:topology:metric:tail}
    If $\sum_{n=1}^\infty a_n$ converges then $\sum_{n=N}^\infty a_n
    \rightarrow 0$ as $N \rightarrow \infty$.
  \end{lm}

  \begin{proof}
    \pf
    \begin{align*}
      \sum_{n=N}^\infty a_n & = \sum_{n=1}^\infty a_n - \sum_{n=1}^{N-1} a_n \\
      & \rightarrow \sum_{n=1}^\infty a_n - \sum_{n=1}^\infty a_n \\
      & = 0
    \end{align*}
    as $N \rightarrow \infty$. \qed
  \end{proof}

  \begin{thm}[Weierstrass M-Test]
    Let $X$ be a set and $f_n : X \rightarrow \mathbb{R}$ for $n \geq 1$. If
    $|f_n(x)| \leq M_n$ for all $n \geq 1$ and all $x \in X$, and if
    $\sum_{n=1}^\infty M_n$ converges, then
    \[ \sum_{n=1}^N f_n(x) \rightarrow \sum_{n=1}^\infty f_n(x) \]
    uniformly in $x$ as $N \rightarrow \infty$.
  \end{thm}

  \begin{proof}
    \pf
    \step{<1>1}{For $N \geq 1$, \pflet{$s_N : X \rightarrow \mathbb{R}$,
        $s_N(x)
        =
        \sum_{n=1}^N f_n(x)$}}
    \step{<1>2}{For all $x \in X$, $\sum_{n=1}^\infty f_n(x)$ converges.}
    \begin{proof}
      \pf\ By the Comparison Test.
    \end{proof}
    \step{<1>3}{\pflet{$s : X \rightarrow \mathbb{R}$, $s(x) =
        \sum_{n=1}^\infty
        f_n(x)$.}}
    \step{<1>4}{For $N \geq 1$, \pflet{$r_N = \sum_{n = N+1}^\infty M_n$}}
    \step{<1>5}{For $1 \leq N < K$, we have $|s_K(x) - s_N(x)| \leq r_N$ for
      all
      $x
      \in X$}
    \begin{proof}
      \pf
      \begin{align*}
        |s_K(x) - s_N(x)| & = \left| \sum_{n = N+1}^K f_n(x) \right| \\
        & \leq \sum_{n=N+1}^K |f_n(x)| \\
        & \leq \sum_{n=N+1}^K M_n \\
        & \leq \sum_{n=N+1}^\infty M_n \\
        & = r_N
      \end{align*}
    \end{proof}
    \step{<1>6}{For $N \geq 1$ and $x \in X$ we have $|s(x) - s_N(x)| \leq r_N$}
    \begin{proof}
      \pf\ Let $K \rightarrow \infty$ in \stepref{<1>5}.
    \end{proof}
    \step{<1>7}{\pflet{$\epsilon > 0$}}
    \step{<1>8}{\pick\ $N$ such that, for all $N' \geq N$, we have $r_{N'} <
      \epsilon$}
    \begin{proof}
      \pf\ Such an $N$ exists by Lemma \ref{lm:topology:metric:tail}.
    \end{proof}
    \step{<1>9}{For all $N' \geq N$ and $x \in X$ we have $|s_{N'}(x) - s(x)| <
      \epsilon$}
    \qed
  \end{proof}

  \begin{df}
    Let $X$ be a metric space. Let $x \in X$ and $A \subseteq X$ be nonempty.
    The     \emph{distance} from $x$ to $A$ is
    \[ d(x, A) = \inf_{a \in A} d(x, a) \enspace . \]
  \end{df}

  \begin{lm}
    \label{lm:topology:metric:dist_continuous}
    Let $X$ be a metric space and $A \subseteq X$ be nonempty. Then the
    function
    $d(-, A) : X \rightarrow \mathbb{R}$ is continuous.
  \end{lm}

  \begin{proof}
    \pf
    \step{<1>1}{\pflet{$x \in X$ and $\epsilon > 0$}}
    \step{<1>2}{\pflet{$y \in X$ with $d(x, y) < \epsilon$}}
    \step{<1>3}{$|d(x, A) - d(y, A)| < \epsilon$}
    \begin{proof}
      \pf
      \step{<2>1}{$d(x, A) - d(y, A) < \epsilon$}
      \begin{proof}
        \pf
        \begin{align*}
          d(x, A) & = \inf_{a \in A} d(x, a) \\
          & \leq \inf_{a \in A} (d(x, y) + d(y, a)) \\
          & = d(x, y) + \inf_{a \in A} d(y, a) \\
          & = d(x, y) + d(y, A) \\
          & < \epsilon + d(y, A)
        \end{align*}
      \end{proof}
      \step{<2>2}{$d(y, A) - d(x, A) < \epsilon$}
      \begin{proof}
        \pf\ Similar.
      \end{proof}
    \end{proof}
    \qedstep
    \begin{proof}
      \pf\ By Theorem \ref{thm:topology:metric:continuous}.
    \end{proof}
    \qed
  \end{proof}

  \begin{df}[Shrinking Map]
    Let $X$ be a metric space and $f : X \rightarrow X$. Then $f$ is a
    \emph{shrinking map} iff, for all $x, y \in X$ with $x \neq y$, we have
    $d(f(x), f(y)) < d(x, y)$.
  \end{df}

  \begin{df}[Contraction]
    Let $X$ be a metric space and $f : X \rightarrow X$. Then $f$ is a
    \emph{contraction} iff there exists $\alpha < 1$ such that, for all $x, y
    \in X$,
    \[ d(f(x), f(y)) \leq \alpha d(x, y) \enspace . \]
  \end{df}

    \begin{prop}
   Every separable metric space is second countable.
  \end{prop}

  \begin{proof}
   \pf
   \step{<1>1}{\pflet{$X$ be a separable metric space.}}
   \step{<1>2}{\pick\ a countable dense set $D$}
   \step{<1>3}{\pflet{$\mathcal{B} = \{ B(d, q) : d \in D, q \in \mathbb{Q}^+
\}$}}
   \step{<1>4}{$\mathcal{B}$ is a countable basis for $X$}
   \qed
  \end{proof}

  \begin{cor}
    The space $\mathbb{R}^\omega$ under the uniform topology is not separable.
  \end{cor}

  \begin{cor}
    Not every metric space is separable.
  \end{cor}

\begin{cor}
	The space $\mathbb{R}^\omega$ under the box topology is not separable.
\end{cor}
    \begin{prop}[CC]
      \label{prop:topology:metric:Lindelof_second_countable}
    Every Lindel\"{o}f metric space is second countable.
  \end{prop}

  \begin{proof}
   \pf
   \step{<1>1}{\pflet{$X$ be a Lindel\"{o}f metric space.}}
   \step{<1>2}{For all $n \in \mathbb{Z}^+$, \pick\ a countable covering
     $\mathcal{A}_n$ of      $X$ by $1/n$-balls}
   \begin{proof}
     \pf\ One exists by the Lindel\"{o}f condition, since the set of all
$1/n$-balls covers $X$.
   \end{proof}
   \step{<1>3}{$\bigcup_{n=1}^\infty \mathcal{A}_n$ is a countable basis.}
    \qed
  \end{proof}

  \begin{cor}
    The space $\mathbb{R}^\omega$ under the uniform topology is not
Lindel\"{o}f.
  \end{cor}

\begin{cor}
  Not every metric space is Lindel\"{o}f.
\end{cor}

  \begin{prop}
    The space $\mathbb{R}_l$ is not metrizable.
  \end{prop}

  \begin{proof}
    \pf\ It is Lindel\"{o}f but not second countable. \qed
  \end{proof}

    \begin{prop}
   The ordered square is not metrizable.
  \end{prop}

  \begin{proof}
   \pf\ It is compact but not second countable. \qed
  \end{proof}

  \begin{prop}
    The space $\mathbb{R}^\omega$ under the uniform topology is not second
countable.
  \end{prop}

  \begin{proof}
    \pf\ It contains a subspace homeomorphic to $\mathbb{R}$. \qed
  \end{proof}

    \begin{thm}[AC]
      \label{thm:topology:metric:normal}
   Every metrizable space is normal.
  \end{thm}

  \begin{proof}
   \pf
   \step{<1>1}{\pflet{$X$ be a metric space.}}
   \step{<1>2}{\pflet{$A$ and $B$ be disjoint closed subspaces of $X$.}}
   \step{<1>3}{For $a \in A$, \pick\ $\epsilon_a > 0$ such that $B(a,
\epsilon_a)$
     does not intersect $B$.}
   \step{<1>4}{For $b \in B$, \pick\ $\epsilon_b > 0$ such that $B(b,
\epsilon_b)$
     does not intersect $A$.}
   \step{<1>5}{\pflet{$U = \bigcup_{a \in A} B(a, \epsilon_a / 2)$}}
   \step{<1>6}{\pflet{$V = \bigcup_{b \in B} B(b, \epsilon_b / 2)$}}
   \step{<1>7}{$U \cap V = \emptyset$}
   \begin{proof}
     \step{<2>1}{\pflet{$z \in U \cap V$}}
     \step{<2>2}{\pick\ $a \in A$ and $b \in B$ such that $z \in B(a, \epsilon_a
/
       2)$ and $z \in B(b, \epsilon_b / 2)$}
     \step{<2>3}{\assume{w.l.o.g.~ $\epsilon_a \leq \epsilon_b$}}
     \step{<2>4}{$a \in B(b, \epsilon_b)$}
     \begin{proof}
       \pf
       \begin{align*}
         d(a, b) & \leq d(a, z) + d(b, z) & (\text{Triangle Inequality}) \\
         & < \epsilon_a / 2 + \epsilon_b / 2 & (\text{\stepref{<2>2}}) \\
         & \leq \epsilon_b & (\text{\stepref{<2>3}})
       \end{align*}
     \end{proof}
     \qedstep
     \begin{proof}
       \pf\ This contradicts \stepref{<1>4}.
     \end{proof}
   \end{proof}
   \qed
  \end{proof}

  \begin{cor}
    The space $\mathbb{R}^\omega$ is normal.
  \end{cor}

  \begin{cor}
    The space $\mathbb{R}_K$ is not metlizable.
  \end{cor}

    \begin{prop}
   Every metrizable space is completely normal.
  \end{prop}

  \begin{proof}
   \pf\ Every subspace is metrizable (Lemma
\ref{lm:topology:metric:subspace}) hence normal (Theorem
\ref{thm:topology:metric:normal}). \qed
  \end{proof}

   \begin{prop}
  Every metrizable space is perfectly normal.
 \end{prop}

 \begin{proof}
  \pf
  \step{<1>1}{\pflet{$X$ be a metric space.}}
  \step{<1>2}{$X$ is normal.}
  \begin{proof}
    \pf\ Theorem \ref{thm:topology:metric:normal}
  \end{proof}
  \step{<1>3}{Every closed set is $G_\delta$.}
  \begin{proof}
    \pf\ If $A$ is closed then $A = \bigcap_{q \in \mathbb{Q}^+} \{ x \in X :
    d(A, x) < q \}$.
  \end{proof}
  \qed
 \end{proof}

  \begin{thm}[Urysohn Metrization Theorem (CC)]
  Every second countable regular space is metrizable.
 \end{thm}

 \begin{proof}
  \pf
  \step{<1>1}{\pflet{$X$ be a second countable regular space.}}
  \step{<1>2}{$X$ is normal.}
  \step{<1>3}{\pick\ a countable basis $\mathcal{B} = \{ B_n : n \in
\mathbb{Z}^+
    \}$}
  \step{<1>4}{For every pair of integers $m$, $n$ with $\overline{B_m}
\subseteq
    B_n$, \pick\ a continuous function $g_{mn} : X \rightarrow [0, 1]$ such
    that $g_{mn}(\overline{B_m}) = \{ 1\}$ and $g_{mn}(X \setminus B_n) = \{ 0
    \}$}
  \begin{proof}
    \pf\ By the Urysohn Lemma.
  \end{proof}
  \step{<1>5}{The set $\{ g_{mn} : \overline{U_m} \subseteq U_n \}$ separates
    points from     closed sets in $X$}
  \begin{proof}
    \step{<2>1}{\pflet{$x \in X$ and $U$ be a neighbourhood of $x$}}
    \step{<2>2}{\pick\ $B_n \in \mathcal{B}$ such that $x \in B_n \subseteq U$}
    \step{<2>3}{\pick\ $V$ open such that $x \in V$ and $\overline{V} \subseteq
      B_n$}
    \step{<2>4}{\pick\ $B_m \in \mathcal{B}$ such that $x \in B_m \subseteq V$}
    \step{<2>5}{$g_{mn}(x) = 1$ and $g_{mn}$ vanishes outside $U$}
  \end{proof}
\step{<1>6}{$X$ is imbeddable in $[0,1]^\omega$}
\begin{proof}
  \pf\ By the Imbedding Theorem.
\end{proof}
\qedstep
 \end{proof}

 \begin{cor}
   The space $\mathbb{R}^\omega$ under the box topology is not second countable.
 \end{cor}

 \begin{prop}
  Not every second countable Hausdorff space is metrizable.
 \end{prop}

 \begin{proof}
  \pf\ $\mathbb{R}_K$ is second countable and Hausdorff but not metrizable
(because it is not regular). \qed
 \end{proof}

  \begin{prop}
   There exists a space that is completely normal, first countable,
Lindel\"{o}f and separable but not metrizable.
 \end{prop}

 \begin{proof}
   \pf\ The space $\mathbb{R}_l$ is all of these. \qed
 \end{proof}

  \begin{prop}
   $\overline{S_\Omega}$ is not metrizable.
 \end{prop}

 \begin{proof}
  \pf\ It is compact but not sequentially compact. \qed
 \end{proof}

  \begin{prop}
  Every compact metric space is second countable.
 \end{prop}

 \begin{proof}
  \pf
  \step{<1>1}{\pflet{$X$ be a compact etric space}}
  \step{<1>2}{For every $n \geq 1$, \pick\ a finite covering $\mathcal{A}_n$ of
$X$ by     open balls of radius $1/n$}
\begin{proof}
  \pf\ Such a covering exists because $\{ B_{1/n}(x) : x \in X\}$ covers $X$.
\end{proof}
\step{<1>3}{$\bigcup_{n=1}^\infty \mathcal{A}_n$ is a countable basis for $X$}
\qed
 \end{proof}

 \begin{cor}
   The space $\mathbb{R}^\omega$ under the uniform topology is not compact.
 \end{cor}

 \begin{cor}
   The space $\mathbb{R}^\omega$ under the uniform topology is not limit
point compact.
 \end{cor}

 \begin{prop}
 	The space $\mathbb{R}^\omega$ under the box topology is not locally compact.
 \end{prop}

 \begin{proof}
 	\pf
 	\step{<1>1}{\assume{$\mathbb{R}^\omega$ under the box topology is locally
 			compact.}}
 	\step{<1>2}{For every point $x$, there exists a basic open set $B = \prod_{i=0}^\infty U_i$ such that $x \in B$ and $\overline{B}$ is compact.}
 	\step{<1>3}{The box topology on $\overline{B}$ is the same as the product topology on $\overline{B}$}
 	\begin{proof}
 		\pf\ By Corollary \ref{cor:topology:compact_hausdorff:finer_coarser}.
 	\end{proof}
 	\step{<1>4}{The box topology on $\overline{B}$ is strictly finer than the product topology.}
 	\begin{proof}
 		\pf By Theorem \ref{thm:topology:product:compare}.
 	\end{proof}
 	\qed
 \end{proof}

 \begin{prop}
 Not every metrizable space is connected.
\end{prop}

\begin{proof}
 \pf\ The discrete space with two points is metrizable but not connected. \qed
\end{proof}

\begin{cor}
 Not every metrizable space is path connected.
\end{cor}

\begin{prop}
  Not every metric space is limit point compact.
\end{prop}

\begin{proof}
  \pf\ The space $\mathbb{R}$ is not limit point compact. \qed
\end{proof}

\begin{prop}
  Not every metric space is locally compact.
\end{prop}

\begin{proof}
  The space $\mathbb{R}^\omega$ in the uniform topology is not locally compact.
\end{proof}

  \section{Isometries}

  \begin{df}[Isometry]
    Let $X$ be a metric space. An \emph{isometry} of $X$ is a function $f : X
    \rightarrow X$ such that, for all $x, y \in X$,
    \[ d(f(x), f(y)) = d(x, y) \enspace . \]
  \end{df}

  \section{Lebesgue Numbers}

  \begin{df}[Lebesgue Number]
    Let $X$ be a metric space and $\mathcal{A}$ an open covering of $X$. A
    \emph{Lebesgue number} for $\mathcal{A}$ is a real $\delta > 0$ such that,
    for every nonempty set $A \subseteq X$ of diameter $< \delta$, there exists
    $U \in \mathcal{A}$ such that $A \subseteq U$.
  \end{df}

  \begin{lm}[Lebesgue Number Lemma]
    In a compact metric space, every open covering has a Lebesgue number.
  \end{lm}

  \begin{proof}
    \pf
    \step{<1>1}{\pflet{$X$ be a compact metric space and $\mathcal{A}$ an open
        covering of $X$} \prove{There exists a Lebesgue number $\delta$ for
        $\mathcal{A}$.}}
    \step{<1>2}{\assume{w.l.o.g.~$X \notin \mathcal{A}$}}
    \begin{proof}
      \pf\ If $X \in \mathcal{A}$ then we can take $\delta = 1$.
    \end{proof}
    \step{<1>3}{\pick\ a finite subcovering $\{ U_1, \ldots, U_n \} \subseteq
      \mathcal{A}$ that covers $X$}
    \step{<1>4}{For $1 \leq i \leq n$, \pflet{$C_i = X \setminus U_i$}}
    \step{<1>5}{\pflet{$f : X \rightarrow \mathbb{R}$ be defined by
        \[ f(x) = 1/n \sum_{i=1}^n d(x, C_i) \enspace . \]}}
    \begin{proof}
      \pf\ Each $C_i$ is nonempty by \stepref{<1>2}.
    \end{proof}
    \step{<1>6}{For all $x \in X$ we have $f(x) > 0$}
    \begin{proof}
      \step{<2>1}{\pflet{$x \in X$}}
      \step{<2>2}{\pick\ $i$ such that $x \in U_i$}
      \begin{proof}
        \pf\ By \stepref{<1>3}.
      \end{proof}
      \step{<2>3}{\pick\ $\epsilon > 0$ such that $B(x, \epsilon) \subseteq
        U_i$}
      \begin{proof}
        \pf\ By Lemma \ref{lm:topology:metric:open}.
      \end{proof}
      \step{<2>4}{$d(x, C_i) \geq \epsilon$}
    \end{proof}
    \step{<1>7}{$f$ is continuous}
    \begin{proof}
      \pf\ From Lemma \ref{lm:topology:metric:dist_continuous}.
    \end{proof}
    \step{<1>8}{\pflet{$\delta = \min f(X)$} \prove{For every nonempty set $A
        \subseteq X$        with diameter $< \delta$, there exists $U \in
        \mathcal{A}$ such that $A        \subseteq U$}}
    \begin{proof}
      \pf\ $f(X)$ has a minimum by the Extreme Value Theorem.
    \end{proof}
    \step{<1>9}{\pflet{$A \subseteq X$ be nonempty with $\diam A < \delta$}}
    \step{<1>10}{\pick\ $x_0 \in A$}
    \step{<1>11}{\pflet{$i$ be such that $d(x_0, C_i)$ is greatest among
        $d(x_0,
        C_1)$, \ldots, $d(x_0, C_n)$}}
    \step{<1>12}{$\delta \leq d(x_0, C_i)$}
    \begin{proof}
      \pf
      \begin{align*}
        \delta & \leq f(x_0) & (\text{\stepref{<1>8}}) \\
        & = 1/n \sum_{j=1}^n d(x_0, C_j) & (\text{\stepref{<1>5}}) \\
        & \leq 1/n \sum_{j=1}^n d(x_0, C_i) & (\text{\stepref{<1>11}}) \\
        & = d(x_0, C_i)
      \end{align*}
    \end{proof}
    \step{<1>13}{$x_0 \in U_i$}
    \begin{proof}
      \pf\ $x_0 \notin C_i$ because $d(x_0, C_i) > 0$.
    \end{proof}
    \qed
  \end{proof}


  \begin{thm}[DC]
    Let $X$ be a metrizable space. Then the following are equivalent:
    \begin{enumerate}
      \item $X$ is compact.
      \item $X$ is limit point compact.
      \item $X$ is sequentially compact.
    \end{enumerate}
  \end{thm}

  \begin{proof}
    \pf
    \step{<1>1}{$1 \Rightarrow 2$}
    \begin{proof}
      \pf\ Theorem \ref{thm:topology:compact:limit_point_compact}.
    \end{proof}
    \step{<1>2}{$2 \Rightarrow 3$}
    \begin{proof}
      \step{<2>1}{\assume{$X$ is limit point compact.}}
      \step{<2>2}{\pflet{$(x_n)$ be a sequence in $X$} \prove{$(x_n)$ has a
          convergent subsequence.}}
      \step{<2>3}{\case{$\{x_n : n \in \mathbb{Z}^+ \}$ is finite.}}
      \begin{proof}
        \pf\ In this case, $(x_n)$ has a constant subsequence.
      \end{proof}
      \step{<2>4}{\case{$\{x_n : n \in \mathbb{Z}^+ \}$ is infinite.}}
      \begin{proof}
        \step{<3>1}{\pick\ a limit point $l$ of $\{ x_n : n \in \mathbb{Z}^+
          \}$}
        \step{<3>2}{For every poisitive integer $r$, \pick\ $n_r$ such that
          $n_r
          >
          n_{r-1}$ and $d(x_{n_r}, l) < 1/r$}
        \begin{proof}
          \pf\ There always exists such an $n_r$ since $B(l, 1/r)$ intersects
          $\{ x_n : n \in \mathbb{Z}^+ \}$ in infinitely many points by
          Theorem \ref{thm:topology:T1:limit_point}.
        \end{proof}
        \step{<3>3}{$x_{n_r} \rightarrow l$ as $r \rightarrow \infty$}
      \end{proof}
    \end{proof}
    \step{<1>3}{$3 \Rightarrow 1$}
    \begin{proof}
      \step{<2>1}{\assume{$X$ is sequentially compact.}}
      \step{<2>2}{Every open covering of $X$ has a Lebesgue number.}
      \begin{proof}
        \step{<3>1}{\pflet{$\mathcal{A}$ be an open covering of $X$.}}
        \step{<3>2}{\assume{for a contradiction that, for all $\delta > 0$,
            there
            exists a set $C \subseteq X$ with $\diam C < \delta$ such that
            there
            is no $U \in \mathcal{A}$ such that $C \subseteq U$}}
        \step{<3>3}{For $n \geq 1$, \pick\ $C_n \subseteq X$ with $\diam C_n <
          1/n$
          such that there is no $U \in \mathcal{A}$ such that $C_n \subseteq U$}
        \step{<3>4}{For $n \geq 1$, \pick\ $x_n \in C_n$}
        \step{<3>5}{\pick\ a convergent subsequence $(x_{n_r})$ of $(x_n)$}
        \begin{proof}
          \pf\ By \stepref{<2>1}.
        \end{proof}
        \step{<3>6}{\pflet{$x_{n_r} \rightarrow l$ as $r \rightarrow \infty$}}
        \step{<3>7}{\pick\ $U \in \mathcal{A}$ with $l \in U$}
        \begin{proof}
          \pf\ By \stepref{<3>1}
        \end{proof}
        \step{<3>8}{\pick\ $\epsilon > 0$ such that $B(l, \epsilon) \subseteq
          U$}
        \begin{proof}
          \pf\ By Lemma \ref{lm:topology:metric:open}.
        \end{proof}
        \step{<3>9}{\pick\ $R$ such that $1/n_R < \epsilon / 2$ and $d(x_{n_R},
          l) <
          \epsilon / 2$}
        \begin{proof}
          \pf\ By \stepref{<3>6}
        \end{proof}
        \step{<3>10}{$C_{n_R} \subseteq U$}
        \begin{proof}
          \pf
          \begin{align*}
            C_{n_R} & \subseteq B(x_{n_R}, 1/n_R) & (\text{\stepref{<3>3},
              \stepref{<3>4}}) \\
            & \subseteq B(x_{n_R}, \epsilon / 2) & (\text{\stepref{<3>9}}) \\
            & \subseteq B(l, \epsilon) & (\text{\stepref{<3>9}}) \\
            & \subseteq U & (\text{\stepref{<3>8}})
          \end{align*}
        \end{proof}
        \qedstep
        \begin{proof}
          \pf\ This contradicts \stepref{<3>3}.
        \end{proof}
      \end{proof}
      \step{<2>3}{For all $\epsilon > 0$, there exists a finite covering of $X$
        by
        $\epsilon$-balls.}
      \begin{proof}
        \step{<3>1}{\pflet{$\epsilon > 0$}}
        \step{<3>2}{\assume{for a contradiction there is no finite covering of
            $X$
            by $\epsilon$-balls.}}
        \step{<3>3}{\pick\ a sequence $(x_n)$ in $X$ such that, for all $n$,
          \[ x_n \notin B(x_1, \epsilon) \cup \cdots \cup B(x_{n-1}, \epsilon)
          \enspace . \]}
        \step{<3>4}{For all $m$, $n$ with $m > n$ we have $d(x_m, x_n) \geq
          \epsilon$}
        \step{<3>5}{Any $\epsilon / 2$-ball contains at most one element of
          $(x_n)$.}
        \step{<3>6}{$(x_n)$ has no convergent subsequence.}
        \qedstep
        \begin{proof}
          \pf\ This contradicts \stepref{<2>1}.
        \end{proof}
      \end{proof}
      \step{<2>4}{\pflet{$\mathcal{A}$ be an open covering of $X$}}
      \step{<2>5}{\pick\ a Lebesgue number $\delta$ for $\mathcal{A}$}
      \begin{proof}
        \pf\ By \stepref{<2>2}.
      \end{proof}
      \step{<2>6}{\pick\ a finite covering $\{ B_1, \ldots, B_n \}$ of $X$ by
        $\delta / 3$-balls.}
      \begin{proof}
        \pf\ By \stepref{<2>3}.
      \end{proof}
      \step{<2>7}{For $1 \leq i \leq n$, \pick\ $U_i \in \mathcal{A}$ such that
        $B_i \subseteq U_i$}
      \step{<2>8}{$\{ U_1, \ldots, U_n \}$ covers $X$.}
    \end{proof}
    \qed
  \end{proof}

  \begin{cor}
   $S_\Omega$ is not metrizable.
  \end{cor}

  \begin{proof}
   \pf\ It is limit point compact (Corollary
\ref{cor:topology:limit_point_compact:S_omega}) but not compact (Proposition
\ref{prop:topology:compact:S_omega}). \qed
  \end{proof}

  \begin{cor}
    The space $\mathbb{R}^\omega$ is not limit point compact.
  \end{cor}

  \section{Uniform Continuity}

  \begin{df}[Uniform Continuity]
    Let $X$ and $Y$ be metric spaces and $f : X \rightarrow Y$. Then $f$ is
    \emph{uniformly continuous} iff, for all $\epsilon > 0$, there exists
    $\delta > 0$ such that, for all $x, y \in X$, if $d(x, y) < \delta$ then
    $d(f(x), f(y)) < \epsilon$.
  \end{df}

  \begin{thm}[Uniform Continuity Theorem]
    Let $X$ be a compact metric space, $Y$ a metric space, and $f : X
    \rightarrow Y$ be continuous. Then $f$ is uniformly continuous.
  \end{thm}

  \begin{proof}
    \pf
    \step{<1>1}{\pflet{$\epsilon > 0$} \prove{There exists $\delta > 0$ such
        that,
        for all $x, y \in X$, if $d(x, y) < \delta$ then $d(f(x), f(y)) <
        \epsilon$.}}
    \step{<1>2}{\pflet{$\mathcal{A} = \{ f^{-1}(B(y, \epsilon / 2)) : y \in Y
        \}$}}
    \step{<1>3}{$\mathcal{A}$ is an open covering of $X$}
    \step{<1>4}{\pick\ a Lebesgue number $\delta$ for $\mathcal{A}$. \prove{For
        all
        $x, y \in X$, if $d(x, y) < \delta$ then $d(f(x), f(y)) < \epsilon$}}
    \begin{proof}
      \pf\ By the Lebesgue Number Lemma
    \end{proof}
    \step{<1>5}{\pflet{$x, y \in X$ with $d(x, y) < \delta$}}
    \step{<1>6}{$\diam \{ x, y \} < \delta$}
    \step{<1>7}{\pick\ $z \in Y$ such that $\{ x, y \} \subseteq f^{-1}(B(z,
      \epsilon / 2))$}
    \step{<1>8}{$d(f(x), f(y)) < \epsilon$}
    \qed
  \end{proof}

  \section{Locally Metrizable Spaces}

   \begin{df}[Locally Metrizable]
   A space is \emph{locally metrizable} iff every point has a metrizable
neighbourhood.
 \end{df}

  \begin{prop}
  Every metrizable space is locally metrizable.
 \end{prop}

 \begin{proof}
  \pf\ Trivial. \qed
 \end{proof}

 \begin{cor}
   The space $\mathbb{R}^\omega$ is locally metrizable.
 \end{cor}

   \begin{prop}
  A compact Hausdorff space is metrizable if and only if it is locally
metrizable.
 \end{prop}

 \begin{proof}
  \pf
  \step{<1>1}{\pflet{$X$ be a locally metrizable compact Hausdorff space}}
  \step{<1>2}{$X$ is regular}
  \begin{proof}
    \pf\ Lemma \ref{lm:topology:compact:normal}
  \end{proof}
  \step{<1>3}{$X$ is second countable}
  \begin{proof}
    \step{<2>1}{$\{ U : U \text{ open in } X \text{ and metrizable} \}$ covers
$X$}
    \step{<2>2}{\pick\ a finite subcover $U_1$, \ldots, $U_n$}
    \step{<2>3}{For $1 \leq i \leq n$, \pick\ a countable basis $\mathcal{B}_i$
of
      $U_i$}
    \step{<2>4}{$\mathcal{B}_1 \cup \cdots \cup \mathcal{B}_n$ is a basis for
$X$}
  \end{proof}
  \qedstep
  \begin{proof}
    \pf\ By the Urysohn Metrization Theorem.
  \end{proof}
  \qed
 \end{proof}

 \begin{cor}
   $\overline{S_\Omega}$ is not locally metrizable.
 \end{cor}

 \begin{cor}
  The ordered square is not locally metrizable.
 \end{cor}

  \begin{prop}
 Every subspace of a locally metrizable space is locally metrizable.
\end{prop}

\begin{proof}
 \pf
 \step{<1>1}{\pflet{$X$ be locally metrizable and $Y \subseteq X$}}
 \step{<1>2}{\pflet{$y \in Y$}}
 \step{<1>3}{\pick\ a metrizable neighbourhood $U$ of $y$ in $X$}
 \step{<1>4}{$U \cap Y$ is a metrizable neighbourhood of $y$ in $Y$}
 \qed
\end{proof}

\begin{cor}
  $S_\Omega \times \overline{S_\Omega}$ is not locally metrizable.
\end{cor}

\begin{proof}
  \pf\ It has a subspace homeomorphic to $\overline{S_\Omega}$. \qed
\end{proof}

 \begin{prop}[CC]
   Every locally metrizable regular Lindel\"{o}f space is metrizable.
 \end{prop}

 \begin{proof}
  \pf
  \step{<1>1}{\pflet{$X$ be a locally metrizable regular Lindel\"{o}f space.}}
  \step{<1>2}{Every point in $X$ has an open second countable neighbourhood.}
  \begin{proof}
    \step{<2>1}{\pflet{$x \in X$}}
    \step{<2>2}{\pick\ an open metrizable $U$ containing $x$}
    \begin{proof}
      \pf\ $X$ is locally metrizable (\stepref{<1>1})
    \end{proof}
    \step{<2>3}{\pick\ an open $V$ such that $x \in V \subseteq \overline{V}
\subseteq
      U$}
    \begin{proof}
      \pf\ Proposition \ref{prop:topology:regular:closure}
    \end{proof}
    \step{<2>4}{$\overline{V}$ is Lindel\"{o}f}
    \begin{proof}
      \pf\ Proposition \ref{prop:topology:Lindelof:subspace}
    \end{proof}
    \step{<2>5}{$\overline{V}$ is second countable}
    \begin{proof}
      \pf\ Proposition \ref{prop:topology:metric:Lindelof_second_countable}
    \end{proof}
  \end{proof}
  \step{<1>3}{\pick\ a countable covering of secound countable open sets
    $\mathcal{U}$}
  \begin{proof}
    \pf\ $X$ is Lindel\"{o}f (\stepref{<1>1})
  \end{proof}
  \step{<1>4}{For $U \in \mathcal{U}$, \pick\ a countable basis $\mathcal{B}_U$}
  \step{<1>5}{$\bigcup_{U \in \mathcal{U}} \mathcal{B}_U$ is a countable basis
for
    $X$}
  \begin{proof}
    \step{<2>1}{\pflet{$x \in U$ where $U$ is open in $X$}}
    \step{<2>2}{\pick\ $V \in \mathcal{U}$ such that $x \in V$}
    \step{<2>3}{There exists $B \in \mathcal{B}_V$ such that $x \in B \subseteq
U
\cap       V$}
  \end{proof}
  \qedstep
  \begin{proof}
    \pf\ By the Urysohn Metrization Theorem.
  \end{proof}
  \qed
 \end{proof}

 \begin{cor}
   $\mathbb{R}_l$ is not locally metrizable.
 \end{cor}

  \begin{prop}
   The Sorgenfrey plane is not locally metrizable.
 \end{prop}

 \begin{proof}
  \pf
  \step{<1>1}{\pflet{$U$ be any neighbourhood of $(0,0)$} \prove{$U$ is not
      Lindel\"{o}f}}
  \step{<1>2}{\pick\ $a > 0$ such that $[0,a)^2 \subseteq U$}
  \step{<1>3}{\pflet{$L = \{ (x, a-x) : 0 < x < a \}$}}
  \step{<1>4}{$L$ is closed in $U$}
  \begin{proof}
    \pf\ By Lemma \ref{lm:closed_sorgenfrey} since $(x,y) \mapsto (x,a+y)$ is
    a homeomorphism of $\mathbb{R}_l^2$ with itself.
  \end{proof}
  \step{<1>5}{\pflet{$\mathcal{U} = \{ U \setminus L \} \cup \{ ([x,b) \times
      [a-x,c)) \cap U : b > a, c > a - x \}$}}
  \step{<1>6}{$\mathcal{U}$ covers $U$}
  \step{<1>7}{No countable subset of $\mathcal{U}$ covers $U$}
  \begin{proof}
    \pf\ Every set of the for $[x,b) \times [a-x,c)$ intersects $L$ in exactly
one point.
  \end{proof}
  \qed
 \end{proof}

 \begin{cor}
  The Sorgenfrey plane is not metrizable.
 \end{cor}

\begin{prop}
  The space $\mathbb{R}_K$ is locally metrizable.
\end{prop}

\begin{proof}
 \pf\ The set $(-1, 1) - K$ is a metrizable neighbourhood of 0. For any other
point $p$, pick an open interval around $p$ that does not contain 0. \qed
\end{proof}
\begin{prop}
  The product of two locally metrizable spaces is locally metrizable.
\end{prop}

\begin{proof}
  \pf
  \step{<1>1}{\pflet{$X$ and $Y$ be locally metrizable}}
  \step{<1>2}{\pflet{$(a, b) \in X \times Y$}}
  \step{<1>3}{\pick\ metrizable neighbourhoods $U$ of $a$ and $V$ of $b$}
  \step{<1>4}{$U \times V$ is a metrizable neighbourhood of $(a, b)$.}
  \begin{proof}
    \pf\ By Lemma \ref{lm:topology:metric:product}.
  \end{proof}
  \qed
\end{proof}


\begin{prop}
  The product of two locally metrizable spaces is locally metrizable.
\end{prop}

\begin{proof}
  \pf
  \step{<1>1}{\pflet{$X$ and $Y$ be locally metrizable}}
  \step{<1>2}{\pflet{$(a, b) \in X \times Y$}}
  \step{<1>3}{\pick\ metrizable neighbourhoods $U$ of $a$ and $V$ of $b$}
  \step{<1>4}{$U \times V$ is a metrizable neighbourhood of $(a, b)$.}
  \begin{proof}
    \pf\ By Lemma \ref{lm:topology:metric:product}.
  \end{proof}
  \qed
\end{proof}

\begin{prop}
  The space $\mathbb{R}_K^\omega$ is not locally metrizable.
\end{prop}

\begin{proof}
  \pf\ If it were, then there would be a basic open set $\prod_n U_n$ that is metrizable, but then $\mathbb{R}_K$ would be metrizable as it is homeomorphic to a subspace of $\prod_n U_n$. \qed
\end{proof}

\begin{cor}
  The product of a countable family of locally metrizable spaces is not necessarily locally metrizable.
\end{cor}

\begin{prop}
  The continuous image of a locally metrizable space is not necessarily locally metrizable.
\end{prop}

\begin{proof}
  \pf\ The identity map from the discrete two-point space to the indiscrete two-point space is continuous. \qed
\end{proof}

 \chapter{Manifolds}

 \section{Manifolds}

  \begin{df}[Manifold]
   Let $m \geq 1$. An \emph{$m$-manifold} is a second countable Hausdorff space
such that each point has a neighbourhood that is homeomorphic to an open subset
of $\mathbb{R}^m$.

A \emph{curve} is a 1-manifold and a \emph{surface} is a 2-manifold.
 \end{df}

  \begin{df}[Support]
   Let $X$ be a topological space and $\phi : X \rightarrow \mathbb{R}$ be a
   function. Then the \emph{support} of $\phi$ is the closure of
   $\inv{\phi}(\mathbb{R} \setminus \{ 0 \})$.
 \end{df}

  \begin{df}[Partition of Unity]
   Let $X$ be a topological space. Let $\{ U_1, \ldots, U_n \}$ be a finite
indexed open covering of $X$. An indexed family of continuous functions
$\phi_1, \ldots, \phi_n : X \rightarrow [0,1]$ is a \emph{partition of unity}
dominated by $\{ U_1, \ldots, U_n \}$ iff:
\begin{enumerate}
 \item $\supp \phi_i \subseteq U_i$ for all $i$;
 \item $\sum_{i=1}^n \phi_i(x) = 1$ for all $x \in X$.
\end{enumerate}
 \end{df}

  \begin{thm}[Existence of Finite Partitions of Unity]
    \label{thm:topology:manifolds:partition}
   Let $X$ be a normal space.
   Let $\{ U_1, \ldots, U_n \}$ be a finite indexed open covering of $X$. Then
   there exists a partition of unity dominated by $\{ U_1, \ldots, U_n \}$.
 \end{thm}

 \begin{proof}
  \pf
  \step{<1>1}{For every finite indexed open covering $\{ U_1, \ldots, U_n \}$
of
    $X$, there exists a finite indexed open covering $\{ V_1, \ldots, V_n \}$
    such that $\overline{V_i} \subseteq U_i$}
  \begin{proof}
    \step{<2>1}{For $1 \leq k \leq n$, there exist open sets $V_1$, \ldots,
$V_k$
      such that $\overline{V_i} \subseteq U_i$ for all $i$ and $\{ V_1, \ldots,
      V_k, U_{k+1}, \ldots, U_n \}$ covers $X$}
    \begin{proof}
      \step{<3>1}{\assume{as an induction hypothesis that $0 leq k < k$ and
there
          exist open sets $V_1$, \ldots, $V_k$ such that $\overline{V_i}
          \subseteq U_i$ for all $i$  and $\{ V_1, \ldots, V_k, U_{k+1},
          \ldots, U_n \}$ covers $X$}}
      \step{<3>2}{\pflet{$A = X \setminus (V_1 \cup \cdots \cup V_k) \setminus
          (U_{k+2} \cup \cdots \cup U_n)$}}
      \step{<3>3}{$A$ is closed}
      \step{<3>4}{$A \subseteq U_{k+1}$}
      \begin{proof}
        \pf\ Since $\{ V_1, \ldots, V_k, U_{k+1}, \ldots, U_n \}$ covers $X$
      \end{proof}
      \step{<3>5}{\pick\ an open set $V_{k+1}$ such that $A \subseteq V_{k+1}$
and
        $\overline{V_{k+1}} \subseteq U_{k+1}$}
      \begin{proof}
        \pf\ By Proposition \ref{prop:topology:regular:closure}
      \end{proof}
      \step{<3>6}{$\{ V_1, \ldots, V_k, V_{k+1}, U_{k+2}, \ldots, U_n \}$
covers
        $X$}
    \end{proof}
  \end{proof}
  \step{<1>2}{\pick\ an open covering $\{ V_1, \ldots, V_n \}$ with
    $\overline{V_i} \subseteq U_i$ for all $i$}
  \begin{proof}
    \pf\ By \stepref{<1>1}.
  \end{proof}
  \step{<1>3}{\pick\ an open covering $\{ W_1, \ldots, W_n \}$ with
    $\overline{W_i} \subseteq V_i$ for all $i$}
  \begin{proof}
    \pf\ By \stepref{<1>1}.
  \end{proof}
  \step{<1>4}{For $1 \leq i \leq n$, \pick\ a continuous function $\psi_i : X
    \rightarrow [0,1]$ such that $\psi_i(\overline{W_i}) = \{ 1 \}$ and
    $\psi_i(X \setminus V_i) = \{ 0 \}$}
  \begin{proof}
    \pf\ By the Urysohn Lemma.
  \end{proof}
  \step{<1>5}{\pflet{$\Psi : X \rightarrow \mathbb{R}$ where $\Psi(x) =
      \sum_{i=1}^n \psi_i(x)$}}
  \step{<1>6}{$\Psi(x) > 0$ for all $x \in X$}
  \begin{proof}
    \step{<2>1}{\pflet{$x \in X$}}
    \step{<2>2}{\pick\ $i$ such that $x \in W_i$}
    \step{<2>3}{$\psi_i(x) = 1$}
  \end{proof}
  \step{<1>7}{For $1 \leq j \leq n$, \pflet{$\phi_j(x) =
      \frac{\psi_j(x)}{\Psi(x)}$}}
  \step{<1>8}{$\psi_1$, \ldots, $\psi_n$ are a partition of unity dominated by
    $\{ U_1, \ldots, U_n \}$}
  \begin{proof}
    \step{<2>1}{$\supp \psi_i \subseteq U_i$}
    \begin{proof}
      \step{<3>1}{$\inv{\psi_i}(\mathbb{R} \setminus \{ 0 \}) \subseteq V_i$}
      \begin{proof}
        \pf\ By \stepref{<1>4}
      \end{proof}
      \step{<3>2}{$\supp \psi_i \subseteq \overline{V_i}$}
      \begin{proof}
        \pf\ Proposition \ref{prop:topology:closure:monotone}
      \end{proof}
    \end{proof}
    \step{<2>2}{$\sum_{i=1}^n \psi_i(x) = 1$ for all $x \in X$}
  \end{proof}
  \qed
 \end{proof}

 \begin{thm}
   \label{thm:topology:manifolds:compact_Hausdorff}
  Let $X$ be a compact Hausdorff space. Suppose that, for every $x \in X$,
there exists a neighbourhood $U$ of $x$ and a positive integer $k$ such that
$U$ can be imbedded in $\mathbb{R}^k$. Then there exists a positive integer $N$
such that $X$ can be imbedded in $\mathbb{R}^N$.
 \end{thm}

  \begin{proof}
  \pf
  \step{<1>1}{\pick\ a finite open covering $\{ U_1, \ldots, U_n \}$ of $X$
such
    that each $U_i$ can be imbedded in $\mathbb{R}^k$ for some $k$}
  \begin{proof}
    \pf\ Since $\{ U \text{ open in } X : U \text{ can be imbedded in }
    \mathbb{R}^k \text{ for some } k \}$ covers $X$.
  \end{proof}
  \step{<1>2}{For $1 \leq i \leq n$, \pick\ a positive integer $k_i$ and an
imbedding $g_i : U_i \rightarrow
\mathbb{R}^{k_i}$}
  \step{<1>3}{\pick\ a partition of unity $\phi_1$, \ldots, $\phi_n$ dominated
by
    $\{ U_1, \ldots, U_n \}$}
  \begin{proof}
    \step{<2>1}{$X$ is normal}
    \begin{proof}
      \pf\ By Lemma \ref{lm:topology:compact:normal}.
    \end{proof}
    \qedstep
    \begin{proof}
      \pf\ Theorem \ref{thm:topology:manifolds:partition}
    \end{proof}
  \end{proof}
  \step{<1>4}{For $1 \leq i \leq n$, \pflet{$A_i = \supp \phi_i$}}
  \step{<1>5}{For $1 \leq i \leq n$, \pflet{$h_i : X \rightarrow
      \mathbb{R}^{k_i}$
be
defined by
\[ h_i(x) = \begin{cases}
  \phi_i(x) g_i(x) & \text{for } x \in U_i \\
  \vec{0} & \text{for } x \in X \setminus A_i
\end{cases} \]}}
\begin{proof}
  \pf\ If $x \in U_i$ and $x \in X \setminus A_i$ then $x \notin \supp \phi_i$
so $\phi_i(x) = 0$
\end{proof}
\step{<1>6}{\pflet{$N = n + k_1 + \cdots + k_n$}}
\step{<1>7}{\pflet{$F : X \rightarrow \mathbb{R}^N$ be the function
    \[ F(x) = (\phi_1(x), \ldots, \phi_n(x), h_1(x), \ldots, h_n(x)) \]}}
\step{<1>8}{$F$ is an imbedding}
\begin{proof}
  \step{<2>1}{$F$ is continuous}
  \begin{proof}
    \pf\ Each $h_i$ is continuous by Theorem
    \ref{thm:topology:continuous:local}.
  \end{proof}
  \step{<2>2}{$F$ is injective}
  \begin{proof}
    \step{<3>1}{\assume{$F(x) = F(y)$}}
    \step{<3>2}{\pick\ $i$ such that $\phi_i(x) > 0$}
    \begin{proof}
      \pf\ Since $\sum_i \phi_i(x) = 1$ (\stepref{<1>3})
    \end{proof}
    \step{<3>3}{$\phi_i(y) = 0$}
    \begin{proof}
      \pf\ By \stepref{<3>1}
    \end{proof}
    \step{<3>4}{$x, y \in U_i$}
    \begin{proof}
      \pf\ Since $\supp \phi_i \subseteq U_i$
    \end{proof}
    \step{<3>5}{$h_i(x) = h_i(y)$}
    \begin{proof}
      \pf\ By \stepref{<3>1}
    \end{proof}
    \step{<3>6}{$g_i(x) = g_i(y)$}
    \begin{proof}
      \pf\ By \stepref{<1>5}
    \end{proof}
    \step{<3>7}{$x = y$}
    \begin{proof}
      \pf\ By \stepref{<1>2}
    \end{proof}
  \end{proof}
  \qedstep
  \begin{proof}
    \pf\ By Theorem \ref{thm:topology:compact:homeomorphism}
  \end{proof}
\end{proof}
\qed
 \end{proof}

   \begin{cor}
   Every compact manifold can be imbedded in $\mathbb{R}^N$ for some $N$.
 \end{cor}

 \begin{prop}
  The line with two origins is a second countable $T_1$ space where every point
  has a neighbourhood that is homeomorphic to an open subset of $\mathbb{R}$,
but it is not a 1-manifold.
 \end{prop}

  \chapter{Normed Spaces}

  \section{The Norm on $\mathbb{R}^n$}

  \begin{df}[Norm]
    Given $\vec{x} = (x_1, \ldots, x_n) \in \mathbb{R}^n$, the \emph{norm}
    $\|\vec{x}\|$ is defined by
    \[ \|\vec{x}\| = \sqrt{x_1^2 + \cdots + x_n^2} \enspace . \]
  \end{df}

  \begin{df}[Vector Sum]
    Define the \emph{sum} of $\vec{x}, \vec{y} \in \mathbb{R}^n$ by
    \[ \vec{x} + \vec{y} = (x_1 + y_1, \ldots, x_n + y_n) \enspace . \]
  \end{df}

  \begin{df}[Scalar Product]
    Given $c \in \mathbb{R}$ and $\vec{x} \in \mathbb{R}^n$, define the
    \emph{scalar product} $c \vec{x}$ to be
    \[ c \vec{x} = (c x_1, \ldots, c x_n) \enspace . \]
  \end{df}

  \begin{df}[Inner Product]
    The \emph{inner product} of $\vec{x}, \vec{y} \in \mathbb{R}^n$ is
    \[ \vec{x} \cdot \vec{y} = x_1 y_1 + \cdots + x_n y_n \enspace . \]
  \end{df}

  \begin{lm}
    \label{lm:norm:distribute}
    \[ \vec{x} \cdot (\vec{y} + \vec{z}) = \vec{x} \cdot \vec{y} + \vec{x}
    \cdot \vec{z} \]
  \end{lm}

  \begin{proof}
    \pf\ Both are equal to $\sum_{i=1}^n (x_i y_i + x_i z_i)$. \qed
  \end{proof}

  \begin{lm}
    \label{lm:norm:cauchy_schwarz}
    \[ |\vec{x} \cdot \vec{y}| \leq \|\vec{x}\| \|\vec{y}\| \]
  \end{lm}

  \begin{proof}
    \pf
    \step{<1>1}{\case{$\vec{x} = \vec{0}$ or $\vec{y} = \vec{0}$}}
    \begin{proof}
      \pf\ In this case, both sides are 0.
    \end{proof}
    \step{<1>2}{\case{$\vec{x} \neq \vec{0} \neq \vec{y}$}}
    \begin{proof}
      \step{<2>1}{\pflet{$a = 1 / \| \vec{x} \|$, $b = 1 / \| \vec{y} \|$}}
      \step{<2>2}{$2 + 2 a b \vec{x} \cdot \vec{y} \geq 0$}
      \begin{proof}
        \step{<3>1}{$\| a \vec{x} + b \vec{y} \|^2 \geq 0$}
        \step{<3>2}{$\sum_{i=1}^n (a x_i + b y_i)^2 \geq 0$}
        \step{<3>3}{$a^2 \sum_{i=1}^n x_i^2 + b^2 \sum_{i=1}^n y_i^2 + 2 a b
          \sum_{i=1}^n x_i y_i \geq 0$}
        \step{<3>4}{$a^2 \| \vec{x} \|^2 + b^2 \| \vec{y} \|^2 + 2 a b \vec{x}
          \cdot \vec{y} \geq 0$}
      \end{proof}
      \step{<2>3}{$2 - 2 a b \vec{x} \cdot \vec{y} \geq 0$}
      \begin{proof}
        \pf\ Similar.
      \end{proof}
      \step{<2>4}{$2 - 2 a b |\vec{x} \cdot \vec{y}| \geq 0$}
      \begin{proof}
        \pf\ From \stepref{<2>2} and \stepref{<2>3}.
      \end{proof}
      \step{<2>5}{$|\vec{x} \cdot \vec{y}| \leq 1 / a b$}
    \end{proof}
    \qed
  \end{proof}

  \begin{lm}
    \label{lm:norm:triangle}
    \[ \| \vec{x} + \vec{y} \| \leq \| \vec{x} \| + \| \vec{y} \| \]
  \end{lm}

  \begin{proof}
    \pf
    \begin{align*}
      \| \vec{x} + \vec{y} \|^2 & = (\vec{x} + \vec{y}) \cdot (\vec{x} +
      \vec{y}) \\
      & = \| \vec{x} \| ^2 + 2 \vec{x} \cdot \vec{y} + \| \vec{y} \|^2 &
      (\text{Lemma \ref{lm:norm:distribute}}) \\
      & \leq \| \vec{x} \|^2 + 2 \| \vec{x} \| \| \vec{y} \| + \| \vec{y} \|^2
      &
      (\text{Lemma \ref{lm:norm:cauchy_schwarz}}) \\
      & = (\| \vec{x} \| + \| \vec{y} \|)^2 & \qed
    \end{align*}
  \end{proof}

  \begin{df}[Euclidean Metric]
    The \emph{euclidean metric} on $\mathbb{R}^n$ is given by
    \[ d(\vec{x}, \vec{y}) = \| \vec{x} - \vec{y} \| \]

    We prove this is a metric.
  \end{df}

  \begin{proof}
    \pf
    \step{<1>1}{$d(\vec{x}, \vec{y}) \geq 0$}
    \begin{proof}
      \pf\ Immediate from definitions.
    \end{proof}
    \step{<1>2}{$d(\vec{x}, \vec{y}) = 0$ iff $\vec{x} = \vec{y}$}
    \begin{proof}
      \pf\ Immediate from definitions.
    \end{proof}
    \step{<1>3}{$d(\vec{x}, \vec{y}) = d(\vec{y}, \vec{x})$}
    \begin{proof}
      \pf\ Immediate from definitions.
    \end{proof}
    \step{<1>4}{$d(\vec{x}, \vec{z}) \leq d(\vec{x}, \vec{y}) + d(\vec{y},
      \vec{z})$}
    \begin{proof}
      \pf\ From Lemma \ref{lm:norm:triangle}.
    \end{proof}
    \qed
  \end{proof}

  \begin{lm}
    \label{lm:topology:metric:euclidean_square}
    Let $d$ be the euclidean topology on $\mathbb{R}^n$ and $\rho$ the square
    topology. Then, for all $x, y \in \mathbb{R}^n$, we have
    \[ \rho(x, y) \leq d(x, y) \leq \sqrt{n} \rho(x, y) \]
  \end{lm}

  \begin{proof}
    \pf
    \step{<1>1}{$\rho(x, y) \leq d(x, y)$}
    \begin{proof}
      \step{<2>1}{For $1 \leq i \leq n$ we have $|x_i - y_i| \leq d(x, y)$}
      \begin{proof}
        \pf\ By the definition of the euclidean metric.
      \end{proof}
      \qedstep
      \begin{proof}
        \pf\ By the definition of the square metric.
      \end{proof}
    \end{proof}
    \step{<1>2}{$d(x, y) \leq \sqrt{n} \rho(x, y)$}
    \begin{proof}
      \pf
      \begin{align*}
        d(x, y) & = \sqrt{(x_1 - y_1)^2 + \cdots + (x_n - y_n)^2} \\
        & \leq \sqrt{\rho(x,y)^2 + \cdots + \rho(x, y)^2} \\
        & = \sqrt{n \rho(x,y)^2} \\
        & = \sqrt{n} \rho(x, y)
      \end{align*}
    \end{proof}
    \qed
  \end{proof}

  \begin{cor}
    The euclidean metric induces the standard topology on $\mathbb{R}^n$.
  \end{cor}

  \begin{df}
    Let $l_2$ be the set of sequences $\vec{a} \in \mathbb{R}^\omega$ such that
    $\sum_{n=1}^\infty a_n^2 < \infty$.
  \end{df}

  \begin{lm}
    \label{lm:norm:l2}
    If $\vec{a}, \vec{b} \in l_2$ then $\sum_{n=1}^\infty |a_n b_n| < \infty$.
  \end{lm}

  \begin{proof}
    \pf
    \begin{align*}
      \sum_{n=1}^N |a_n b_n| & \leq \sqrt{(\sum_{n=1}^N a_n^2) (\sum_{n=1}^N
        b_n^2)} & (\text{Lemma \ref{lm:norm:cauchy_schwarz}}) \\
      & \rightarrow \sqrt{\sum_{n=1}^\infty a_n^2) (\sum_{n=1}^\infty b_n^2)}
      \text{ as } n \rightarrow \infty
    \end{align*}
    \qed
  \end{proof}

  \begin{lm}
    If $\vec{a}, \vec{b} \in l_2$ then $\vec{a} + \vec{b} \in l_2$.
  \end{lm}

  \begin{proof}
    \pf
    \begin{align*}
      \sum_{n=1}^\infty (a_n + b_n)^2 & = \sum_{n=1}^\infty a_n^2 + 2
      \sum_{n=1}^\infty a_n b_n + \sum_{n=1}^\infty b_n^2 \\
      & \leq \sum_{n=1}^\infty a_n^2 + 2 \sum_{n=1}^\infty |a_n b_n| +
      \sum_{n=1}^\infty b_n^2 \\
      & < \infty & (\text{Lemma \ref{lm:norm:l2}})
    \end{align*}
    \qed
  \end{proof}

  \begin{lm}
    If $c \in \mathbb{R}$ and $\vec{a} \in l_2$ then $c \vec{a} \in l_2$.
  \end{lm}

  \begin{proof}
    \pf $\sum_{n=1}^\infty (c a_n)^2 = c^2 \sum_{n=1}^\infty a_n^2$. \qed
  \end{proof}

  \begin{df}[The $l^2$-metric]
    The \emph{$l^2$-metric} is defined on $l_2$ by
    \[ d(\vec{a}, \vec{b}) = \left[ \sum_{n=1}^\infty (a_n -
    b_n)^2 \right]^{\frac{1}{2}} \enspace . \] The topology induced by this
    metric
    is the \emph{$l^2$-topology}. We write $l_2$ for this set under the
    $l^2$-topology.

    We prove this is a metric.
  \end{df}

  \begin{proof}
    \pf
    \step{<1>1}{$d(\vec{a}, \vec{b}) \geq 0$}
    \begin{proof}
      \pf\ Immediate from definitions.
    \end{proof}
    \step{<1>2}{$d(\vec{a}, \vec{b}) = 0$ iff $\vec{a} = \vec{b}$}
    \begin{proof}
      \pf\ Immediate from definitions.
    \end{proof}
    \step{<1>3}{$d(\vec{a}, \vec{b}) = d(\vec{b}, \vec{a})$}
    \begin{proof}
      \pf\ Immediate from definitions.
    \end{proof}
    \step{<1>4}{$d(\vec{a}, \vec{c}) \leq d(\vec{a}, \vec{b}) + d(\vec{b},
      \vec{c})$}
    \begin{proof}
      \pf\ $\sqrt{\sum_{i=1}^N (a_n - c_n)^2}  \leq \sqrt{\sum_{i=1}^N (a_n
        - b_n)^2} + \sqrt{\sum_{i=1}^N (b_n - c_n)^2}$ since the euclidean
      metric on $\mathbb{R}^N$ is a metric.
    \end{proof}
    \qed
  \end{proof}

  \begin{df}[Hilbert Cube]
    The \emph{Hilbert cube} is $\prod_{n=1}^\infty [0, 1/n]$ as a subspace of
    the $l_2$.
  \end{df}

  \begin{df}[Isometric Imbedding]
    Let $X$, $Y$ be metric spaces and $f : X \rightarrow Y$. Then $f$ is an
    \emph{isometric imbedding} iff, for all $x, y \in X$, $d(f(x), f(y)) = d(x,
    y)$.
  \end{df}

  \begin{lm}
    Every isometric imbedding is an imbedding.
  \end{lm}

  \begin{proof}
    \pf
    \step{<1>1}{\pflet{$f : X \rightarrow Y$ be an isometric imbedding.}}
    \step{<1>2}{$f$ is continuous.}
    \begin{proof}
      \pf\ If $d(x, y) < \epsilon$ then $d(f(x), f(y)) < \epsilon$.
    \end{proof}
    \step{<1>3}{$f$ is injective.}
    \begin{proof}
      \pf\ If $f(x) = f(y)$ then $d(f(x), f(y)) = 0$ so $d(x, y) = 0$ hence $x
      =y $.
    \end{proof}
    \step{<1>4}{$f^{-1} : f(X) \rightarrow X$ is continuous.}
    \begin{proof}
      \pf\ If $d(f^{-1}(x), f^{-1}(y)) < \epsilon$ then $d(x, y) < \epsilon$.
    \end{proof}
    \qed
  \end{proof}

  \chapter{Topological Groups}

  \section{Topological Groups}

  \begin{df}[Topological Group]
    A \emph{topological group} $G$ consists of a group $G$ that is also a $T_1$
    space such that $\cdot : G^2 \rightarrow G$ and $(\ )^{-1} : G \rightarrow
    G$ are continuous.
  \end{df}

  \begin{prop}
    Every topological group is homogeneous.
  \end{prop}

  \begin{proof}
    \pf
    \step{<1>1}{\pflet{$G$ be a topological group.}}
    \step{<1>2}{\pflet{$x, y \in G$}}
    \step{<1>3}{\pflet{$f : G \rightarrow G$ be given by $f(g) = yx^{-1}z$}}
    \step{<1>4}{$f$ is a homeomorphism}
    \step{<1>5}{$f(x) = y$}
    \qed
  \end{proof}

  \begin{df}[Symmetric]
    Let $G$ be a topological group. A neighbourhood $U$ of $e$ is
    \emph{symmetric} iff $U = U^{-1}$.
  \end{df}

  \begin{prop}
    \label{prop:topological_group:neighbourhood}
    For every neighbourhood $U$ of $e$, there exists a symmetric neighbourhood
    $V$ of $e$ such that $VV \subseteq U$.
  \end{prop}

  \begin{proof}
    \pf
    \step{<1>1}{\pflet{$m : G^2 \rightarrow G$ be the multiplication function}}
    \step{<1>2}{$ee \in U$}
    \step{<1>3}{$(e, e) \in m^{-1}(U)$}
    \step{<1>4}{\pick\ neighbourhoods $U_1$, $U_2$ of $e$ such that $(e, e) \in
      U_1
      \times U_2 \subseteq m^{-1}(U)$}
    \step{<1>5}{\pflet{$V' = U_1 \cap U_2$}}
    \step{<1>6}{$V'V' \subseteq U$}
    \step{<1>7}{\pflet{$f : G^2 \rightarrow G$ be the function $f(x, y) =
        xy^{-1}$}}
    \step{<1>8}{$(e, e) \in f^{-1}(V')$}
    \step{<1>9}{\pick\ a neighbourhood $W$ of $e$ such that $W W^{-1} \subseteq
      V'$} % TODO Extract lemma
    \step{<1>10}{\pflet{$V = W W^{-1}$}}
    \step{<1>11}{$V$ is a neighbourhood of $e$}
    \begin{proof}
      \pf\ $V$ is open because $V = \bigcup_{a \in W^{-1}} Wa$.  % TODO Extract
      % lemma
    \end{proof}
    \step{<1>12}{$V$ is symmetric}
    \step{<1>13}{$VV \subseteq U$}
    \qed
  \end{proof}

  \begin{prop}
    Every topological group is regular.
  \end{prop}

  \begin{proof}
    \pf
    \step{<1>1}{\pflet{$G$ be a topological group}}
    \step{<1>2}{\pflet{$A \subseteq G$ be closed and $a \notin A$}}
    \step{<1>3}{$G \setminus A a^{-1}$ is a neighbourhood of $e$}
    \step{<1>4}{\pick\ a symmetric neighbourhood $V$ of $e$ such that $VV
      \subseteq
      G \setminus A a^{-1}$}
    \begin{proof}
      \pf\ Proposition \ref{prop:topological_group:neighbourhood}.
    \end{proof}
    \step{<1>5}{$VA$ and $Va$ are disjoint neighbourhoods of $A$ and $a$}
    \qed
  \end{proof}

  \begin{prop}
    The long line is not second countable.
  \end{prop}

  \begin{proof}
    \pf Let $\mathcal{B}$ be a basis for $L$. Then, for every countable ordinal
    $\alpha$, $\mathcal{B}$ mst contain a basic open set that contains
    $(\alpha, 1/2)$ but not $(\beta, 1/2)$ for any other $\beta$. Therefore,
    $\mathcal{B}$ is uncountable. \qed
  \end{proof}

  \begin{cor}
    The long line cannot be imbedded in $\mathbb{R}$.
  \end{cor}

  \begin{thm}
    Let $f : X \rightarrow Y$. Let $Y$ be compact Hausdorff. Then $f$ is
    continuous if and only if the graph of $f$ is closed in $X \times Y$.
  \end{thm}

  \begin{proof}
    \pf
    \step{<1>1}{\pflet{$G_f$ be the graph of $f$.}}
    \step{<1>2}{If $f$ is continuous then the graph of $f$ is closed.}
    \begin{proof}
      \step{<2>1}{\assume{$f$ is continuous.}}
      \step{<2>2}{\pflet{$(x, y) \in (X \times Y) \setminus G_f$}}
      \step{<2>3}{$y \neq f(x)$}
      \step{<2>4}{\pick\ disjoint open neighbourhoods $U$ of $f(x)$ and $V$ of
        $y$}
      \begin{proof}
        \pf\ $Y$ is Hausdorff.
      \end{proof}
      \step{<2>5}{$(x, y) \in f^{-1}(U) \times V \subseteq (X \times Y)
        \setminus
        G_f$}
      \qedstep
    \end{proof}
    \step{<1>3}{If the graph of $f$ is closed then $f$ is continuous.}
    \begin{proof}
      \step{<2>1}{\assume{$G_f$ is closed.}}
      \step{<2>2}{\pflet{$x_0 \in X$ and $V$ be an open neighbourhood of
          $f(x_0)$}}
      \step{<2>3}{$G_f \cap (X \times (Y \setminus V))$ is closed}
      \step{<2>4}{$\pi_1(G_f \cap (X \times (Y \setminus V)))$ is closed}
      \begin{proof}
        \pf\ Lemma \ref{lm:topology:compact:projection_closed}
      \end{proof}
      \step{<2>5}{$x_0 \in X \setminus \pi_1(G_f \cap (X \times (Y \setminus
        V)))
        \subseteq f^{-1}(V)$}
      \qedstep
    \end{proof}
    \qed
  \end{proof}

  \begin{thm}
    Let $X$ be a compact Hausdorff space. Let $\mathcal{A}$ be a set of closed
    connected subspaces of $X$ that is linearly ordered by proper inclusion.
    Then
    \[ Y = \bigcap \mathcal{A} \]
    is connected.
  \end{thm}

  \begin{proof}
    \pf
    \step{<1>1}{\assume{for a contradiction $C$ and $D$ form a separation of
        $Y$}}
    \step{<1>2}{\pick\ disjoint $U$ and $V$ open in $X$ such that $C = U \cap
      Y$ and $D = V       \cap Y$}
    \begin{proof}
      \step{<2>1}{$C$ and $D$ are compact}
      \begin{proof}
        \step{<3>1}{$Y$ is compact}
        \begin{proof}
          \pf\ $Y$ is a closed subset of $X$, hence compact by Proposition
          \ref{prop:topology:compact:closed_is_compact}.
        \end{proof}
        \qedstep
        \begin{proof}
          \pf\ $C$ and $D$ are closed subsets of $Y$ hence compact by
          Proposition \ref{prop:topology:compact:closed_is_compact}.
        \end{proof}
      \end{proof}
      \qedstep
      \begin{proof}
        \pf\ By Lemma \ref{lm:topology:compact:normal}.
      \end{proof}
    \end{proof}
    \step{<1>3}{For all $A \in \mathcal{A}$, we have $A \setminus (U \cup V)$
      is
      nonempty}
    \begin{proof}
      \pf\ Since $A$ is connected.
    \end{proof}
    \step{<1>4}{$\{ A \setminus (U \cup V) : A \in \mathcal{A} \}$ has the
      finite intersection property}
    \begin{proof}
      \pf\ This holds because $\mathcal{A}$ is linearly ordered under proper
      inclusion.
    \end{proof}
    \step{<1>5}{$\bigcap_{A \in \mathcal{A}} (A \setminus (U \cup V))$ is
      nonempty}
    \begin{proof}
      \pf\ By Proposition \ref{prop:topology:compact:finite_intersection}.
    \end{proof}
    \qed
  \end{proof}

  \begin{thm}
    Let $A \subseteq \mathbb{R}^n$. Then the following are equivalent:
    \begin{enumerate}
      \item $A$ is compact.
      \item $A$ is closed and bounded under the euclidean metric.
      \item $A$ is closed and bounded under the square metric.
    \end{enumerate}
  \end{thm}

  \begin{proof}
    \pf
    \step{<1>1}{$1 \Rightarrow 2$}
    \begin{proof}
      \step{<2>1}{\assume{$A$ is compact.}}
      \step{<2>2}{$A$ is closed.}
      \begin{proof}
        \pf\ By Proposition \ref{prop:topology:compact:compact_is_closed}.
      \end{proof}
      \step{<2>3}{$\{ B(\vec{0}, n) : n \in \mathbb{Z}^+ \}$ covers $A$}
      \step{<2>4}{\pick\ a finite subcover $\{ B(\vec{0}, n_1), \ldots,
        B(\vec{0},
        n_k) \}$}
      \step{<2>5}{\pflet{$N = \max(n_1, \ldots, n_k)$}}
      \step{<2>6}{For all $x, y \in A$ we have $d(x, y) < 2 N$}
      \begin{proof}
        \pf\ We have $d(x, y) \leq d(\vec{0}, x) + d(\vec{0}, y) < N + N$.
      \end{proof}
    \end{proof}
    \step{<1>2}{$2 \Rightarrow 3$}
    \begin{proof}
      \pf\ If $d(x, y) < \epsilon$ for all $x, y \in A$ then $\rho(x, y) <
      \epsilon \sqrt{n}$ by Lemma \ref{lm:topology:metric:euclidean_square}.
    \end{proof}
    \step{<1>3}{$3 \Rightarrow 1$}
    \begin{proof}
      \step{<2>1}{\assume{$A$ is closed and $\rho(x, y) < \epsilon$ for all $x,
          y
          \in A$}}
      \step{<2>2}{\pick\ $x_0 \in A$}
      \step{<2>3}{\pflet{$b = \rho(\vec{0}, x_0)$}}
      \step{<2>4}{\pflet{$P = \epsilon + b$}}
      \step{<2>5}{$A \subseteq [-P, P]^n$}
      \begin{proof}
        \pf For any $y \in A$ we have
        \begin{align*}
          \rho(\vec{0}, y) & \leq \rho(\vec{0}, x_0)
          + \rho(x_0, y) & (\text{Triangle Inequality}) \\
          & < b + \epsilon & (\text{\stepref{<2>3}, \stepref{<2>1}}) \\
          & = P & (\text{\stepref{<2>4}})
        \end{align*}
      \end{proof}
      \step{<2>6}{$[-P, P]^n$ is compact.}
      \begin{proof}
        \pf\ By Corollary \ref{cor:topology:compact:real_closed_interval} and
        Proposition \ref{prop:topology:compact:product}.
      \end{proof}
      \qedstep
      \begin{proof}
        \pf\ By Proposition \ref{prop:topology:compact:closed_is_compact}.
      \end{proof}
    \end{proof}
    \qed
  \end{proof}

  \begin{thm}[AC]
    \label{thm:topology:convergence:compact}
   Let $X$ be a topological space. Then $X$ is compact if and only if every
   nonempty net in $X$ has a convergent subnet.
  \end{thm}

  \begin{proof}
   \pf
   \step{<1>1}{If $X$ is compact then every nonempty net in $X$ has a convergent
     subnet.}
   \begin{proof}
     \step{<2>1}{\assume{$X$ is compact.}}
     \step{<2>2}{\pflet{$(x_\alpha)_{\alpha \in J}$ be a nonempty net in $X$}}
     \step{<2>3}{For $\alpha \in J$, \pflet{$B_\alpha = \{ \beta \in J : \alpha
         \leq \beta \}$.}}
     \step{<2>4}{$\{ B_\alpha : \alpha \in J \}$ has the finite intersection
       property.}
     \begin{proof}
       \step{<3>1}{\pflet{$\alpha_1, \ldots, \alpha_n \in J$}}
       \step{<3>2}{\pick\ $\beta \in J$ such that $\alpha_1 \leq \beta$, \ldots,
         $\alpha_n \leq \beta$}
       \step{<3>3}{$x_\beta \in B_{\alpha_1} \cap \cdots \cap B_{\alpha_n}$}
     \end{proof}
     \step{<2>5}{\pick\ $l \in \bigcap_{\alpha \in J} B_\alpha$}
     \begin{proof}
       \pf\ Proposition \ref{prop:topology:compact:finite_intersection}.
     \end{proof}
     \step{<2>6}{\pflet{$K = \{ \alpha \in J : x_\alpha = l \}$}}
     \step{<2>7}{$K$ is cofinal in $J$}
     \begin{proof}
       \step{<3>1}{\pflet{$\alpha \in J$}}
       \step{<3>2}{$l \in B_\alpha$}
       \begin{proof}
         \pf\ By \stepref{<2>5}.
       \end{proof}
       \step{<3>3}{There exists $\beta \geq \alpha$ such that $x_\beta = l$.}
     \end{proof}
     \step{<2>8}{$(x_\alpha)_{\alpha \in K}$ is a subnet of $(x_\alpha)_{\alpha
         \in J}$ that converges to $l$.}
   \end{proof}
   \step{<1>2}{If every nonempty net in $X$ has a convergent subnet then $X$ is
     compact.}
   \begin{proof}
     \step{<2>1}{\assume{Every nonempty net in $X$ has a convergent subnet}}
     \step{<2>2}{\pflet{$\mathcal{A}$ be a nonempty set of closed sets with the
         finite intersection property.}}
     \step{<2>3}{\pflet{$J$ be the poset of all finite intersections of elements
         of          $\mathcal{A}$ under $\supseteq$}}
     \step{<2>4}{\pick\ $x_C \in C$ for all $C \in J$}
     \begin{proof}
       \pf\ These are all nonempty by \stepref{<2>2}.
     \end{proof}
     \step{<2>5}{\pick\ an accumulation point $l$ of $(x_C)$ \prove{$l \in
         \bigcap \mathcal{A}$}}
     \begin{proof}
      \pf\ One exists by Lemma \ref{lm:topology:accumulation_point:subnet}.
     \end{proof}
     \step{<2>6}{\pflet{$C \in \mathcal{A}$} \prove{$l \in C$}}
     \step{<2>7}{\pflet{$U$ be a neighbourhood of $l$} \prove{$U$ intersects
         $C$}}
     \step{<2>8}{\pick\ $D \subseteq C$ such that $x_D \in U$}
     \begin{proof}
       \pf\ By \stepref{<2>5}.
     \end{proof}
     \step{<2>9}{$U$ intersects $C$}
     \step{<2>10}{$l \in C$}
     \begin{proof}
      \pf\ By Theorem \ref{thm:topology:closure:neighbourhoods} since $C$ is
      closed (\stepref{<2>2}).
     \end{proof}
     \qedstep
     \begin{proof}
       \pf\ Proposition \ref{prop:topology:compact:finite_intersection}.
     \end{proof}
   \end{proof}
   \qed
  \end{proof}

  \begin{cor}[AC]
   Let $G$ be a topological group. Let $A$ and $B$ be subsets of $G$. If $A$ is
closed in $G$ and $B$ is compact then $AB$ is closed in $G$.
  \end{cor}

  \begin{proof}
   \pf
   \step{<1>1}{\pflet{$c \in \overline{AB}$} \prove{$c \in AB$}}
   \step{<1>2}{\pick\ a net $(x_\alpha)_{\alpha \in J}$ that converges to $c$}
   \begin{proof}
     \pf\ By Theorem \ref{thm:topology:convergence:closure}.
   \end{proof}
   \step{<1>3}{For $\alpha \in J$, \pick\ $a_\alpha \in A$ and $b_\alpha \in B$
     such that $x_\alpha = a_\alpha b_\alpha$}
   \step{<1>4}{\pick\ a convergent subnet $(b_{g(\beta)})_{\beta \in K}$ of
     $(b_\alpha)_{\alpha \in J}$}
   \begin{proof}
     \pf\ By Theorem \ref{thm:topology:convergence:compact}.
   \end{proof}
   \step{<1>5}{\pflet{$b_{g(\beta)} \rightarrow b$}}
   \step{<1>6}{$b \in B$}
   \begin{proof}
     \step{<2>1}{$B$ is closed}
     \begin{proof}
       \pf\ By Proposition \ref{prop:topology:compact:compact_is_closed}.
     \end{proof}
     \qedstep
     \begin{proof}
     \pf\ By Theorem \ref{thm:topology:convergence:closure}
   \end{proof}
   \end{proof}
   \step{<1>7}{$a_{g(\beta)} \rightarrow cb^{-1}$}
   \begin{proof}
     \pf\ By Theorem \ref{thm:topology:convergence:continuous}
   \end{proof}
   \step{<1>8}{$cb^{-1} \in A$}
   \begin{proof}
     \pf\ By Theorem \ref{thm:topology:convergence:closure}
   \end{proof}
   \step{<1>9}{$c \in AB$}
   \qedstep
   \begin{proof}
     \pf\ By Proposition \ref{prop:topology:closure:closed2}.
   \end{proof}
  \end{proof}

  \begin{prop}
    Let $A_0 + A_1$ be the sum of $A_0$ and $A_1$ with injections $i_0 : A_0
\rightarrow A_0 + A_1$ and $i_1 : A_1 \rightarrow A_0 + A_1$.

Let $g : B \rightarrow A_0 + A_1$ be a function.

Let $B_0$ be the pullback of $i_0$ and $g$ with projections $j_0 : B_0
\rightarrow B$ and $k_0 : B_0 \rightarrow A_0$.

Let $B_1$ be the pullback of $i_1$ and $g$ with projection s$j_1 : B_1
\rightarrow B$ and $k_1 : B_1 \rightarrow A_1$.

Then $B$ is the sum of $B_0$ and $B_1$ with injections $j_0$ and $j_1$.

    \[ \xymatrix{
      B_0 \ar[r]^{j_0} \ar[d] & B \ar[d]^g & B_1 \ar[l]^{j_1} \ar[d] \\
      A_0 \ar[r]_-{i_0} & A_0 + A_1 & A_1 \ar[l]_-{i_1}
    } \]
  \end{prop}

  \begin{proof}
   \pf
   \step{<1>1}{\pflet{$X$ be any set and $x : B_0 \rightarrow X$, $y : B_1
       \rightarrow X$}}
  \end{proof}

  \begin{prop}[CC]
    \label{prop:topology:Lindelof:basis}
    Let $X$ be a space and $\mathcal{B}$ be a basis for $X$. Suppose that every
    subset of $\mathcal{B}$ that covers $X$ has a countable subcover. Then $X$
    is Lindel\"{o}f.
  \end{prop}

  \begin{proof}
   \pf
   \step{<1>1}{\pflet{$\mathcal{A}$ be an open cover of $X$.}}
   \step{<1>2}{$\{ B \in \mathcal{B} : \exists U \in \mathcal{A}. B \subseteq U
     \}$ covers $X$.}
   \step{<1>3}{\pick\ a countable subcover $\mathcal{B}_0$}
   \step{<1>4}{For $B \in \mathcal{B}_0$, \pick\ $U_B \in \mathcal{A}$ such that
     $B \subseteq U_B$}
   \step{<1>5}{$\{ U_B : B \in \mathcal{B}_0 \}$ is a countable subcover of
     $\mathcal{A}$.}
   \qed
  \end{proof}

  \begin{prop}[CC]
    The space $\mathbb{R}_l$ is Lindel\"{o}f.
  \end{prop}

  \begin{proof}
    \pf
    \step{<1>1}{\pflet{$\mathcal{A}$ be a set of basis elements $[a, b)$ that
        covers $X$} \prove{$\mathcal{A}$ has a countable subcover.}}
    \step{<1>2}{\pflet{$C = \bigcup \{ (a, b) : [a, b) \in \mathcal{A} \}$}}
    \step{<1>3}{$\mathbb{R} \setminus C$ is countable.}
    \begin{proof}
      \step{<2>1}{For all $x \in \mathbb{R} \setminus C$, \pick\ a rational $q_x$
        such that there exists $b$ such that $q_x \in [x, b) \in \mathcal{A}$}
      \begin{proof}
        \step{<3>1}{\pick\ $[a, b) \in \mathcal{A}$ such that $x \in [a, b)$}
        \step{<3>2}{$x = a$}
        \begin{proof}
          \pf\ If not we would have $x \in C$
        \end{proof}
        \step{<3>3}{There exists a rational in $(a, b)$}
      \end{proof}
      \step{<2>2}{For $x, y \in \mathbb{R} \setminus C$, if $x < y$ then $q_x <
q_y$}
      \begin{proof}
        \step{<3>1}{\pick\ $b$, $c$ such that $q_x \in [x, b) \in \mathcal{A}$ and
          $q_y \in [y, c) \in \mathcal{A}$}
        \begin{proof}
          \pf\ By \stepref{<2>1}.
        \end{proof}
        \step{<3>2}{$b \leq y$}
        \begin{proof}
          \pf\ Otherwise we would have $y \in (x, b) \subseteq C$.
        \end{proof}
        \step{<3>3}{$q_x < q_y$}
        \begin{proof}
          \pf\ $q_x < b \leq y \leq q_y$
        \end{proof}
      \end{proof}
      \step{<2>3}{The map $q_{-} : \mathbb{R} \setminus C \rightarrow \mathbb{Q}$
        is injective.}
    \end{proof}
    \step{<1>4}{For $x \in \mathbb{R} \setminus C$, \pick\ $[a_x, b_x) \in
      \mathcal{A}$ such that $a_x \leq x < b_x$}
    \step{<1>5}{\pick\ a countable subset $( (a_n, b_n) )_{n \in \mathbb{Z}^+}$ of
      $\{ (a, b) : [a, b) \in \mathcal{A} \}$ that covers $C$}
    \begin{proof}
      \step{<2>1}{The set $C$ as a subspace of $\mathbb{R}$ with the standard
        topology is second countable.}
      \step{<2>2}{The set $C$ as a subspace of $\mathbb{R}$ with the standard
        topology is Lindel\"{o}f.}
      \begin{proof}
        \pf\ By Theorem \ref{thm:topology:lindelof:second_countable}.
      \end{proof}
    \end{proof}
    \step{<1>6}{$\{ [a_x, b_x) : x \in \mathbb{R} \setminus C \} \cup \{ [a_n,
      b_n) : n \in \mathbb{Z}^+ \}$ is a countable subcover of $\mathcal{A}$.}
    \qedstep
    \begin{proof}
      \pf\ By Proposition \ref{prop:topology:Lindelof:basis}.
    \end{proof}
    \qed
  \end{proof}

  \begin{prop}[AC]
    The space $\mathbb{R}_l$ is not second countable.
  \end{prop}

  \begin{proof}
   \pf
   \step{<1>1}{\pflet{$\mathcal{B}$ be any basis for $\mathbb{R}_l$}}
   \step{<1>2}{For $x \in \mathbb{R}$, \pick\ $B_x \in \mathcal{B}$ such that $x
     \in B_x \subseteq [x, x+1)$}
   \step{<1>3}{The mapping $B_{(-)}$ is an injective function $\mathbb{R}
     \rightarrow \mathcal{B}$}
   \begin{proof}
     \pf\ For any $x$ we have $x = \min B_x$.
   \end{proof}
   \step{<1>4}{$\mathcal{B}$ is uncountable.}
   \qed
  \end{proof}

    \begin{prop}
    The product of a Lindel\"{o}f space and a compact space is Lindel\"{o}f.
  \end{prop}

  \begin{proof}
    \pf
    \step{<1>1}{\pflet{$X$ be a Lindel\"{o}f space and $Y$ a compact space.}}
    \step{<1>2}{\pflet{$\mathcal{A}$ be an open covering of $X \times Y$}}
    \step{<1>3}{For all $x \in X$, there exists a neighbourhood $W$ of $x$ such
      that $W \times Y$ is      covered by finitely many elements of
      $\mathcal{A}$.}
    \begin{proof}
      \step{<2>1}{\pflet{$x \in X$}}
      \step{<2>2}{$\{x\} \times Y$ is compact.}
      \begin{proof}
        \pf\ It is homeomorphic to $Y$.
      \end{proof}
      \step{<2>3}{\pick\ a finite subset $\{ U_1, \ldots, U_m \}$ of
        $\mathcal{A}$
        that covers $\{x\} \times Y$}
      \begin{proof}
        \pf\ By Proposition \ref{prop:topology:compact:subspace}.
      \end{proof}
      \step{<2>4}{There exists a neighbourhood $W$ of $x$ such that $W \times Y
        \subseteq U_1 \cup \cdots \cup U_m$}
      \begin{proof}
        \pf\ By the Tube Lemma.
      \end{proof}
    \end{proof}
    \step{<1>4}{$\{ W \text{ open in } X : W \times Y \text{ is covered by
        finitely
        many        elements of } \mathcal{A} \}$ is an open covering of $X$.}
    \step{<1>5}{\pick\ a countable subcovering $\{ W_1, W_2, \ldots \}$}
    \step{<1>6}{For $i \geq 1$, \pick\ a finite subset $\{ U_{i1},
      \ldots,
      U_{ir_i} \}$ of $\mathcal{A}$ that covers $W_i \times Y$}
    \step{<1>7}{$\{ U_{1j} : i \geq 1, 1 \leq j \leq r_i \}$ is a countable
subcovering of
      $\mathcal{A}$.}
    \qed
  \end{proof}

  \begin{prop}
    \label{prop:topology:normal:shrinking}
   Let $X$ be a $T_1$ space. Then $X$ is normal if and only if, for every
closed set $A$ and open set $U \supseteq A$, there exists an open set $V
\supseteq A$ such that $\overline{V} \subseteq U$.
  \end{prop}

  \begin{proof}
   \pf
   \step{<1>1}{If $X$ is normal then,  for every
closed set $A$ and open set $U \supseteq A$, there exists an open set $V
\supseteq A$ such that $\overline{V} \subseteq U$.}
\begin{proof}
  \step{<2>1}{\assume{$X$ is normal.}}
  \step{<2>2}{\pflet{$A$ be a closed set and $U$ an open set with $A \subseteq U$}}
  \step{<2>3}{\pick\ disjoint open sets $V$, $W$ such that $A \subseteq V$ andn $X
    \setminus U \subseteq W$}
  \step{<2>4}{$\overline{V} \subseteq U$}
  \begin{proof}
    \pf
    \begin{align*}
      \overline{V} & \subseteq X \setminus W \\
      & \subseteq U
    \end{align*}
  \end{proof}
\end{proof}
\step{<1>2}{If, for every
closed set $A$ and open set $U \supseteq A$, there exists an open set $V
\supseteq A$ such that $\overline{V} \subseteq U$, then $X$ is normal.}
\begin{proof}
  \step{<2>1}{\assume{ for every
closed set $A$ and open set $U \supseteq A$, there exists an open set $V
\supseteq A$ such that $\overline{V} \subseteq U$.}}
\step{<2>2}{\pflet{$A$, $B$ be disjoint closed sets}}
\step{<2>3}{\pick\ an open set $V$ such that $A \subseteq V$ and $\overline{V}
  \subseteq X \setminus B$}
\step{<2>4}{$A \subseteq V$ and $B \subseteq X \setminus \overline{V}$}
\end{proof}
\qed
  \end{proof}

  \begin{df}[Action]
    Let $G$ be a topological group and $X$ a topological space. An
\emph{action} of $G$ on $X$ is a continuous function $\cdot : G \times X
\rightarrow X$ such that, for all $g, h \in G$ and $x \in X$:
\begin{enumerate}
 \item $e \cdot x = x$
 \item $g \cdot (h \cdot x) = gh \cdot x$
\end{enumerate}
  \end{df}

  \begin{df}[Orbit Space]
   Let $G$ be a topological group, $X$ a topological space, and $\cdot : G
   \times X \rightarrow X$ an action of $G$ on $X$. Then the \emph{orbit space}
$X / G$ is the quotient space of $X$ by the equivalence relation $\sim$
generated by $x \sim g \cdot x$ for all $x \in X$, $g \in G$.
  \end{df}

  \begin{thm}
   Let $G$ be a topological group. Let $X$ be a topological space. Let
$\cdot : G \times X \rightarrow X$ be an action of $G$ on $X$. Then the
canonical map $\pi : X \twoheadrightarrow X / G$ is perfect.
  \end{thm}

  \begin{proof}
     \step{<1>1}{$\pi$ is closed.}
     \begin{proof}
       \step{<2>1}{\pflet{$A \subseteq X$ be closed.}}
       \step{<2>2}{$GA = \{ g \cdot a : g \in G, a \in A \}$ is closed}
       \begin{proof}
         \step{<3>1}{\pflet{$z \notin GA$}}
         \step{<3>2}{For all $g \in G$ we have $g \cdot z \notin A$}
         \step{<3>3}{For $g \in G$, there exist $U$ an open neighbourhood of $g$
           and            $V$ an open neighbourhood of $z$ such that $UV$ does
           not  intersect $A$}
         \step{<3>4}{$\{ U \text{ open in } G : \exists V \text{ an open
             neighbourhood of } z . UV \cap A = \emptyset \}$ covers $G$}
         \step{<3>5}{\pick\ a finite subcover $\{ U_1, \ldots, U_n \}$}
         \step{<3>6}{For $1 \leq i \leq n$, \pick\ $V_i$ an open neighbourhood of
           $z$ such that $U_i V_i \cap A = \emptyset$}
         \step{<3>7}{$z \in V_1 \cap \cdots \cap V_n \subseteq X \setminus GA$}
       \end{proof}
       \step{<2>3}{$\pi(A)$ is closed}
       \begin{proof}
         $\inv{\pi}(\pi(A)) = GA$
       \end{proof}
     \end{proof}
     \step{<1>2}{$\pi$ is continuous.}
     \begin{proof}
      \pf\ By definition of the quotient topology.
     \end{proof}
     \step{<1>3}{$\pi$ is surjective.}
     \begin{proof}
      \pf\ By definition.
     \end{proof}
     \step{<1>4}{For all $a \in X / G$ we have $\inv{\pi}(a)$ is compact.}
     \begin{proof}
       \step{<2>1}{\pflet{$a \in X / G$}}
       \step{<2>2}{\pick\ $x \in X$ such that $a = \pi(x)$}
       \step{<2>3}{$\inv{\pi}(a) = \{ gx : g \in G \}$}
       \step{<2>4}{$\inv{\pi}(a)$ is homeomorphic to $G$}
     \end{proof}
   \qed
  \end{proof}

  \begin{cor}
   If $X$ is Hausdorff then so is $X / G$.
  \end{cor}

  \begin{cor}
   If $X$ is regular then so is $X / G$.
  \end{cor}

  \begin{cor}
   If $X$ is normal then so is $X / G$.
  \end{cor}

  \begin{cor}
    If $X$ is locally compact then so is $X / G$.
  \end{cor}

  \begin{cor}
   If $X$ is second countable then so is $X / G$.
  \end{cor}

  \begin{prop}
   Let $p : X \twoheadrightarrow Y$ be perfect. If $X$ is second countable then
so is $Y$.
  \end{prop}

  \begin{proof}
   \pf
   \step{<1>1}{\pick\ a countable basis $\mathcal{B}$ for $X$}
   \step{<1>2}{\pflet{$\mathcal{J} = \{ J \subseteq^{\mathrm{fin}} \mathcal{B} :
       \exists W \text{ open in } Y. \inv{p}(W) \subseteq \bigcup J \}$}}
   \step{<1>3}{For every $J \in \mathcal{J}$, \pflet{$W_J = \bigcup \{ W \text{
         open in } Y : \inv{p}(W) \subseteq \bigcup J \}$.} \prove{$\{ W_J : J
\in \mathcal{J} \}$        is a basis for $Y$.}}
   \step{<1>4}{$y \in V$ where $V$ is open in $Y$}
   \step{<1>5}{$\{ B \in \mathcal{B} : x \in B \subseteq \inv{p}(V) \}$ covers
     $\inv{p}(y)$}
   \step{<1>6}{\pick\ a countable subcover $J \subseteq^{\mathrm{fin}}
     \mathcal{B}$}
   \step{<1>7}{$y \in W_J \subseteq V$}
   \begin{proof}
     \step{<2>1}{$\inv{p}(y) \subseteq \bigcup J$}
     \step{<2>2}{\pick\ an open neighbourhood $W$ of $y$ such that $\inv{p}(W)
       \subseteq \bigcup J$}
     \begin{proof}
       \pf\ By Proposition \ref{prop:topology:perfect:neighbourhood}.
     \end{proof}
     \step{<2>3}{$W \subseteq W_J$}
   \end{proof}
   \qed
  \end{proof}

  \begin{prop}
   A subspace of a $T_1$ space is $T_1$.
  \end{prop}

  \begin{proof}
   \pf
   \step{<1>1}{\pflet{$X$ be $T_1$ and $Y \subseteq X$}}
   \step{<1>2}{\pflet{$a \in Y$}}
   \step{<1>3}{$\{a\}$ is closed in $X$}
   \step{<1>4}{$\{a\}$ is closed in $Y$}
   \begin{proof}
     \pf\ By Corollary \ref{cor:topology:subspace:closed}.
   \end{proof}
   \qed
  \end{proof}

  \begin{prop}[DC]
   Not every topological group is normal.
  \end{prop}

  \begin{proof}
   \pf\ From Proposition \ref{prop:topology:normal:uncountable}. \qed
  \end{proof}

  \begin{thm}
   A subspace of a completely regular space is completely regular.
  \end{thm}

  \begin{proof}
   \pf
   \step{<1>1}{\pflet{$X$ be completely regular and $Y \subseteq X$}}
   \step{<1>2}{\pflet{$a \in Y$ and $A$ be closed in $Y$ such that $a \notin A$}}
   \step{<1>3}{\pick\ $C$ closed in $X$ such that $A = X \cap C$}
   \step{<1>4}{\pick\ a continuous function $f : X \rightarrow [0,1]$ such that
     $f(a) = 0$ and $f(C) = \{ 1 \}$}
   \step{<1>5}{$f \restriction Y : Y \rightarrow [0,1]$ is a continuous function
     such that $(f \restriction Y)(a) = 0$ and $(f \restriction Y)(A) = \{ 1
     \}$}
   \qed
 \end{proof}

 \begin{prop}[DC]
  Every topological group is completely regular.
 \end{prop}

 \begin{proof}
  \pf
  \step{<1>1}{\pflet{$G$ be a topological group}}
  \step{<1>2}{\pflet{$x \in G$ and $A \subseteq G$ be closed such that $x \notin
      A$} \prove{There exists a continuous $f : G \rightarrow [0,1]$ such that
      $f(x) = 0$ and $f(A) = \{ 1 \}$}}
  \step{<1>3}{\assume{w.l.o.g.~$x = e$}}
  \begin{proof}
    \pf\ $\lambda y. x^{-1}y$ is an automorphism of $G$ that maps $x$ to $e$.
  \end{proof}
  \step{<1>4}{\pick\ a sequence $V_n$ ($n \geq 0$) of symmetric neighbourhoods
of $e$
disjoint from $A$     such that $V_n V_n \subseteq V_{n-1}$ for all $n$}
\begin{proof}
  \step{<2>1}{\pflet{$V_0 = X \setminus A$}}
  \step{<2>2}{Given $V_n$, \pick\ a symmetric neighbourhood $V_{n+1}$ of $e$
such that
    $V_{n+1} V_{n+1} \subseteq V_n$}
  \begin{proof}
    \pf\ By Proposition \ref{prop:topological_group:neighbourhood}.
  \end{proof}
\end{proof}
  \step{<1>5}{For every dyadic rational $p$, define an open set $U(p)$
as follows:
\begin{align*}
 U(1/2^n) & = V_n & (n \geq 0) \\
 U((2k+1)/2^{n+1}) & = V_{n+1} U(k/2^n) & (0 < k < 2^n) \\
 U(p) & = \emptyset & (p \leq 0) \\
 U(p) & = G & (p > 1)
\end{align*}
}
\step{<1>6}{For all $k$ and $n$, we have
  \[ V_n U(k / 2^n) \subseteq U((k+1)/2^n) \]}
\begin{proof}
  \step{<2>1}{$k \leq 0$}
  \begin{proof}
    \pf\ In this case, $V_n U(k / 2^n) = \emptyset$
  \end{proof}
  \step{<2>2}{$k = 1$ and $n > 0$}
  \begin{proof}
    \pf
    \begin{align*}
     V_n U(1 / 2^n) & = V_n V_n \\
     & \subseteq V_{n-1} \\
     & = U(1 / 2^{n-1})
    \end{align*}
  \end{proof}
  \step{<2>3}{$k = 2a$ for some $0 < a < 2^{n-1}$}
  \begin{proof}
    \pf
    \begin{align*}
      V_n U(2a / 2^n) & = V_n U(a / 2^{n-1}) \\
      & = U(2a+1 / 2^n)
    \end{align*}
  \end{proof}
  \step{<2>4}{$k = 2a+1$ for some $0 < a < 2^{n-1}$}
  \begin{proof}
    \pf
    \begin{align*}
      V_n U((2a+1) / 2^n) & = V_n V_n U(a / 2^{n-1}) \\
      & \subseteq V_{n-1} U(a / 2^{n-1}) \\
    & \subseteq U((a+1) / 2^{n-1})
    \end{align*}
  \end{proof}
  \step{<2>5}{$k \geq 2^n$}
  \begin{proof}
    \pf\ In this case, $U((k+1)/2^n) = G$.
  \end{proof}
\end{proof}
\step{<1>7}{Define $f : G \rightarrow [0,1]$ by
  \[ f(x) = \inf \{ p : x \in U(p) \} \]}
\begin{proof}
  \pf\ This set is nonempty because $x \in U(1)$ and bounded below because if
$x \in U(p)$ then $p > 0$.
\end{proof}
\step{<1>8}{For $n > 0$ we have $\overline{U(k / 2^n)} \subseteq V_n U(k/2^n)$}
\begin{proof}
  \step{<2>1}{\pflet{$x \in \overline{U(k / 2^n)}$}}
  \step{<2>2}{$V_n x$ is a neighbourhood of $x$}
  \step{<2>3}{\pick\ $y \in V_n x \cap U(k / 2^n)$}
  \step{<2>4}{\pick\ $z \in V_n$ such that $y = zx$}
  \step{<2>5}{$x = z^{-1} y$}
\end{proof}
\step{<1>9}{For $p$ and $q$ dyadic rationals, if $p < q$ then $\overline{U(p)}
  \subseteq U(q)$}
\step{<1>10}{If $x \in \overline{U(p)}$ then $f(x) \leq p$}
\begin{proof}
  \step{<2>1}{For all $q > p$ we have $x \in U(q)$}
  \step{<2>2}{For all $q > p$ we have $f(x) \leq q$}
\end{proof}
\step{<1>11}{If $x \notin U(p)$ then $f(x) \geq p$}
\begin{proof}
  \pf\ If $x \notin U(p)$ and $x \in U(q)$ then $q > p$.
\end{proof}
\step{<1>12}{$f$ is continuous}
\begin{proof}
  \step{<2>1}{\pflet{$x_0 \in X$}}
  \step{<2>2}{\pflet{$c < f(x_0) < d$} \prove{There exist a neighbourhood $U$ of
      $x_0$ such that $f(U) \subseteq (c, d)$}}
  \step{<2>3}{\pick\ rational numbers $p$, $q$ such that $c < p < f(x_0) < q < d$}
  \step{<2>4}{$x \notin \overline{U(p)}$}
  \step{<2>5}{$x \in U(q)$}
  \step{<2>6}{Take $U = U(q) \setminus \overline{U(p)}$}
\end{proof}
\step{<1>13}{$f(e) = 0$}
\begin{proof}
  \pf\ We have $e \in U(1/2^n)$ for all $n$.
\end{proof}
\step{<1>14}{$f(A) = \{ 1 \}$}
\begin{proof}
  \pf\ If $x \in A$ and $x \in U(p)$ then $p > 1$.
\end{proof}
\qed
 \end{proof}

 \begin{df}[Bijection]
   A function $f : A \rightarrow B$ is a \emph{bijection}, $f : A \cong B$, iff
   there exists a function $\inv{f} : B \rightarrow A$, the \emph{inverse} of
   $f$, such that $\inv{f} \circ f = \id{A}$ and $f \circ \inv{f} = \id{B}$.
 \end{df}

 \begin{thm}
  Let $Y$ be a normal space. Then $Y$ is an absolute retract if and only if $Y$
  has the universal extension property.
 \end{thm}

 \begin{proof}
  \pf
  \step{<1>1}{If $Y$ is an absolute retract then $Y$ has the universal extension
    property.}
  \begin{proof}
    \step{<2>1}{\assume{$Y$ is an absolute retract.}}
    \step{<2>2}{\pflet{$X$ be a normal space, $A$ a closed subspace of $X$ and $f
        : A \rightarrow Y$ a continuous function.}}
    \step{<2>3}{\pflet{$Z_f$ be the quotient space of $X \cup Y$ under: $a \sim
        f(a)$ for all $a \in A$}}
    \step{<2>4}{\pflet{$p : X \cup Y \twoheadrightarrow Z_f$ be the quotient
map}}
    \step{<2>5}{For all $x_1, x_2 \in X$ we have $p(x_1) = p(x_2)$ iff
$x_1 = x_2$ or $x_1, x_2 in A$ and $f(x_1) =
f(x_2)$; and for $x \in X$ and $y \in Y$ we have $p(x) = p(y)$ iff $f(x) = y$;
and for $y_1, y_2 \in Y$ we have $p(y_1) = p(y_2)$ iff $y_1 = y_2$}
    \step{<2>6}{$p$ imbeds $Y$ into a closed subspace of $Z_f$}
    \begin{proof}
      \step{<3>1}{$p$ is injective on $Y$}
      \step{<3>2}{$\inv{p} : p(Y) \rightarrow Y$ is continuous}
      \begin{proof}
        \step{<4>1}{\pflet{$U \subseteq Y$ be open} \prove{$p(U)$ is open}}
        \step{<4>2}{$\inv{p}(p(U)) = \inv{f}(U) \cup U$}
      \end{proof}
      \step{<3>3}{$p(Y)$ is closed}
      \begin{proof}
        \pf\ $\inv{p}(p(Y)) = A \cup Y$
      \end{proof}
    \end{proof}
    \step{<2>7}{$Z_f$ is normal}
    \begin{proof}
      \step{<3>1}{$Z_f$ is $T_1$}
      \begin{proof}
        \pf\ For $y \in Y$ we have $\inv{p}(y) = \inv{f}(y) \cup \{ y \}$ which
        is closed.
      \end{proof}
      \step{<3>2}{Any two disjoint closed sets in $Z_f$ can be
separated by a continuous function.}
      \begin{proof}
        \step{<4>1}{\pflet{$C$ and $D$ be disjoint closed sets in $Z_f$}}
        \step{<4>2}{\pick\ $g : Y \rightarrow [0,1]$ such that $g(Y \cap
          \inv{p}(C)) = \{ 0           \}$ and $g(Y \cap \inv{p}(D)) = \{ 1 \}$}
        \begin{proof}
          \pf\ By the Urysohn Lemma.
        \end{proof}
        \step{<4>3}{\pick\ $h : X \rightarrow [0,1]$ such that $h(X \cap
          \inv{p}(C)) = \{ 0           \}$ and $h(X \cap \inv{p}(D)) = \{ 1 \}$
          and $h$ agrees with $g \circ f$ on $A$}
        \begin{proof}
          \pf\ By the Tietze Extension Theorem applied to $A \cup (X \cap
          \inv{p}(C)) \cup (X \cap \inv{p}(D))$.
        \end{proof}
        \step{<4>4}{\pflet{$k : Z_f \rightarrow [0,1]$ be the continuous function
such that             $k(p(x)) = h(x)$             for $x \in X$ and $k(p(y)) =
g(y)$ for             $y \in Y$}}
\begin{proof}
  \pf\ By the Pasting Lemma
\end{proof}
        \step{<4>5}{$k(C) = \{ 0 \}$}
        \step{<4>6}{$k(D) = \{ 1 \}$}
      \end{proof}
      \qedstep
      \begin{proof}
        \pf\ If $g$ is such a continuous function then $\inv{g}([0, 1/2))$ and
        $\inv{g}((1/2,1])$ are disjoint open sets that include $A$ and $B$
        respectively.
      \end{proof} % TODO Extract lemma
    \end{proof}
    \step{<2>8}{\pick\ a retraction $r : Z_f \rightarrow p(Y)$}
    \step{<2>9}{$\inv{p} \circ r \circ p : X \rightarrow Y$ extends $f$}
  \end{proof}
  \step{<1>2}{If $Y$ has the universal extension property then $Y$ is an absolute
    retract.}
  \begin{proof}
    \step{<2>1}{\assume{$Y$ has the universal extension property}}
    \step{<2>2}{\pflet{$Z$ be a normal space, $Y_0$ a closed subspace of $Z$, and
        $\phi : Y \cong Y_0$ a homeomorphism}}
    \step{<2>3}{\pick\ a continuous extension $f : Z \rightarrow Y$ of
      $\inv{\phi}$}
    \step{<2>4}{$\phi \circ f$ is a retraction}
  \end{proof}
  \qed
 \end{proof}

 \begin{thm}
  Every manifold is metrizable.
 \end{thm}

 \begin{proof}
  \pf
  \step{<1>1}{\pflet{$X$ be an $m$-manifold.}}
  \step{<1>2}{$X$ is regular.}
  \begin{proof}
    \step{<2>1}{$X$ is $T_1$}
    \step{<2>2}{\pflet{$x \in X$ and $U$ be a neighbourhood of $x$}}
    \step{<2>3}{\pick\ a neighbourhood $V$ of $x$ that is imbeddable in
      $\mathbb{R}^m$}
    \step{<2>4}{\pick\ a neighbourhood $W$ of $x$ such that $\overline{W}
      \subseteq U \cap V$}
    \begin{proof}
      \pf\ One exists since $V$ is regular (Proposition
      \ref{prop:topology:regular:subspace})
    \end{proof}
    \step{<2>5}{$x \in W$ and $\overline{W} \subseteq U$}
    \qedstep
    \begin{proof}
      \pf\ Proposition \ref{prop:topology:regular:closure}
    \end{proof}
  \end{proof}
  \qedstep
  \begin{proof}
    \pf\ By the Urysohn Metrization Theorem.
  \end{proof}
  \qed
 \end{proof}

 \begin{thm}
  Let $X$ be a compact Hausdorff space in which every point has a neighbourhood
  that is imbeddable in $\mathbb{R}^m$. Then $X$ is an $m$-manifold.
 \end{thm}

 \begin{proof}
  \pf
  \step{<1>1}{There exists $N$ such that $X$ is imbeddable in $\mathbb{R}^N$}
  \begin{proof}
    \pf\ Theorem \ref{thm:topology:manifolds:compact_Hausdorff}
  \end{proof}
  \step{<1>2}{$X$ is second countable.}
  \begin{proof}
    \pf\ Proposition \ref{prop:topology:second_countable:subspace}
  \end{proof}
  \qed
 \end{proof}

 \begin{prop}
  $S_\Omega$ is locally metrizable.
 \end{prop}

 \begin{proof}
  \pf\ For any $\alpha \in S_\Omega$, the neighbourhood $[0, \alpha] = (-
  \infty, \alpha + 1)$ is imbeddable in $\mathbb{R}$. \qed
 \end{proof}

 \begin{prop}[DC]
   $\overline{S_\Omega}$ is compact.
 \end{prop}

 \begin{proof}
  \pf
     \pf
   \step{<1>1}{\pflet{$\mathcal{A}$ be an open cover of $\overline{S_\Omega}$}}
   \step{<1>2}{\assume{for a contradiction there is no finite subcover of
       $\mathcal{A}$}}
   \step{<1>3}{There exists a sequence of sets $U_n \in \mathcal{A}$ and ordinals
     $\alpha_n$ such that $\alpha_{n+1} < \alpha_n$ for all $n$ and $\alpha_n
     \in U_n$ for all $n$}
   \begin{proof}
     \step{<2>1}{\pflet{$\alpha_1 = \Omega$}}
     \step{<2>2}{Given $\alpha_1$, \ldots, $\alpha_n$ and $U_1$, \ldots,
       $U_{n-1}$ with $0 \neq \alpha_n < \alpha_{n-1} < \cdots < \alpha_1$ and
       $\alpha_i \in U_i$ for $i < n$, \pick\ $U_n \in \mathcal{A}$ with
       $\alpha_n \in U_n$}
     \begin{proof}
       \pf\ By \stepref{<1>1}.
     \end{proof}
     \step{<2>3}{\pick\ $\alpha_{n+1} < \alpha_n$ such that $(\alpha_{n+1},
       \alpha_n] \subseteq U_n$}
     \begin{proof}
       \pf\ By Lemma \ref{lm:topology:order:open}.
     \end{proof}
     \step{<2>4}{$\alpha_{n+1} \neq 0$}
     \begin{proof}
       \pf\ If $\alpha_{n+1} = 0$ then $U_1$, \ldots, $U_n$ cover
       $\overline{S_\Omega}$, contradicting \stepref{<1>2}.
     \end{proof}
   \end{proof}
   \qedstep
   \begin{proof}
     \pf\ This is a contradiction because the ordinals are well-ordered.
   \end{proof}
   \qed
 \end{proof}

 \begin{prop}
   $\mathbb{R}_l$ is not limit point compact.
 \end{prop}

 \begin{proof}
   \pf\ $\mathbb{Z}$ has no limit point. \qed
 \end{proof}

 \begin{prop}
   \label{prop:topology:Lindelof:subspace}
   Every closed subspace of a Lindel\"{o}f space is Lindel\"{o}f.
 \end{prop}

 \begin{proof}
  \pf
  \step{<1>1}{\pflet{$X$ be Lindel\"{o}f and $A \subseteq X$ be closed}}
  \step{<1>2}{\pflet{$\mathcal{U}$ be an open covering of $A$}}
  \step{<1>3}{$\{ U \text{ open in } X : U \cap A \in \mathcal{U} \} \cup \{ X
    \setminus A \}$ covers $X$}
  \step{<1>4}{\pick\ a countable subcovering $\mathcal{V}$}
  \step{<1>5}{$\{ U \cap A : U \in \mathcal{V}, U \neq X \setminus A \}$ is a
    countable subcover of $\mathcal{U}$}
  \qed
 \end{proof}

 \begin{prop}
   $\mathbb{R}^\omega$ is locally connected.
 \end{prop}

 \begin{proof}
  \pf This holds because every basic open set is connected, being the product
of a family of connected spaces. \qed
 \end{proof}

\begin{prop}
	The space $\mathbb{R}^\omega$ under the box topology is not first countable.
\end{prop}

\begin{proof}
	\pf
	\step{<1>1}{\assume{for a contradiction $\{ U_n \}_{n \geq 0}$ is a countable basis at 0.}}
	\step{<1>2}{For $n \geq 1$, \pick\ a basic open set $B_n = \prod_{j=0}^\infty (a_{nj}, b_{nj})$ such that $0 \in B_n \subseteq U_n$}
	\step{<1>3}{$\prod_{n=0}^\infty (a_{nn}/2, b_{nn}/2)$ is a neighbourhood of 0 that does not include any $U_n$}
	\qed
\end{proof}

\begin{prop}
	The space $\mathbb{R}^\omega$ under the box topology is not locally metrizable.
\end{prop}

\begin{proof}
	\pf
	\step{<1>1}{\pflet{$U$ be any neighbourhood of $0$}}
	\step{<1>2}{\pflet{$A$ be the set of all sequences in $U$ with all coordinates positive}}
	\step{<1>3}{$0 \in \overline{A}$}
	\step{<1>4}{There is no sequence of points of $A$ converging to 0.}
	\step{<1>5}{$U$ is not metrizable.}
	\begin{proof}
		\pf\ By the Sequence Lemma.
	\end{proof}
	\qed
\end{proof}

\begin{prop}
	For any nonempty set $I$, the space $\mathbb{R}^I$ is not limit point compact.
\end{prop}

\begin{proof}
	\pf\ $\mathbb{Z}^I$ is an infinite set with no limit point. \qed
\end{proof}

\begin{prop}
	The space $\mathbb{R}^{[0,1]}$ is separable.
\end{prop}

\begin{proof}
  \pf\ The set $D$ is dense where $D$ is the set of all functions $f : [0,1]
\rightarrow \mathbb{Q}$ such that there exists a sequence of rationals $0 = q_0
< q_1 < \cdots < q_N = 1$ such that $f$ is constant on $[q_i, q_{i+1})$ for $0
\leq i < N$. \qed
\end{proof}

\begin{prop}
  If $J$ is uncountable then $\mathbb{R}^J$ is not locally metrizable.
\end{prop}

\begin{proof}
  \pf\ Every point has a neighbourhood homeomorphic to $\mathbb{R}^J$. \qed
\end{proof}

\begin{prop}
  The space $\mathbb{R}_K$ is not limit point compact.
\end{prop}

\begin{proof}
  \pf\ The set $\mathbb{Z}$ has no limit point. \qed
\end{proof}

\begin{prop}
 The topologist's sine curve is not locally connected.
\end{prop}

\begin{proof}
 \pf\ There is no connected neighbourhood of $(0,0)$. \qed
\end{proof}

\begin{cor}
Not every metric space is locally connected.
\end{cor}

\begin{cor}
Not every metric space is locally path connected.
\end{cor}

\begin{prop}
  Not every metric space is compact.
\end{prop}

\begin{proof}
  \pf\ The space $\mathbb{R}$ is not compact. \qed
\end{proof}

\begin{prop}
  Every closed subspace of a limit point compact space is limit point compact.
\end{prop}

\begin{proof}
  \pf
  \step{<1>1}{\pflet{$X$ be a limit point compact space and $C \subseteq X$ be closed.}}
  \step{<1>2}{\pflet{$A \subseteq C$ be infinite.}}
  \step{<1>3}{\pick\ a limit point $l$ of $A$ in $X$}
  \step{<1>4}{$l \in C$}
  \begin{proof}
    \step{<2>1}{$l$ is a limt point of $C$}
    \begin{proof}
      \pf\ By Lemma \ref{lm:topology:limit_point:subset}.
    \end{proof}
    \qedstep
    \begin{proof}
      \pf\ By Corollary \ref{cor:topology:limit_point:closed}.
    \end{proof}
  \end{proof}
  \step{<1>5}{$l$ is a limit point of $A$ in $C$.}
  \begin{proof}
    \pf\ By Proposition \ref{prop:topology:subspace:limit_point}.
  \end{proof}
  \qed
\end{proof}

\begin{prop}
  For any part $i : S \hookrightarrow X$ of a set $X$, we have $\emptyset \subseteq_X i$.
\end{prop}

\begin{proof}
  \pf\ We have $i \circ \magic_S = \magic_X$ by the uniqueness of $\magic_X$. \qed
\end{proof}

\end{document}
