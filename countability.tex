\chapter{Countability Axioms}

\section{The First Countability Axiom}

\begin{df}[First Countability Axiom]
  A topological space $X$ satisfies the \emph{first countability axiom}, or
  is \emph{first countable}, iff every point has a countable local basis.
\end{df}

 \begin{prop}
$S_\Omega$ is first countable.
\end{prop}

\begin{proof}
 \pf\ For every countable ordinal $\alpha > 0$, the set $\{ (\beta, \alpha +
 1) :    \beta < \alpha \}$ is a local basis at $\alpha$. The set $\{ \{ 0
 \} \}$ is a local basis at 0. \qed
\end{proof}

\begin{thm}[The Sequence Lemma (CC)]
  \label{sequence_lemma}
  Let $X$ be a first countable space and $A \subseteq X$. If $x \in
  \overline{A}$, then there exists a sequence of points of $A$ that converges
  to $x$.
\end{thm}

\begin{proof}
  \pf
  \step{<1>1}{\pflet{$x \in \overline{A}$}}
  \step{<1>2}{\pick\ a countable basis $\{ B_n \}_{n \in \mathbb{Z}^+}$ at
    $x$.}
  \step{<1>3}{For $n \geq 1$, \pick\ a point $a_n \in B_1 \cap \cdots \cap
    B_n
    \cap A$ \prove{$a_n \rightarrow x$ as $n \rightarrow \infty$}}
  % TODO Extract lemma
  \begin{proof}
    \pf\ Using Countable Choice. Such an $a_n$ exists because $B_1 \cap
    \cdots
    \cap B_n$ is a neighbourhood of $x$. Apply Theorem
    \ref{thm:topology:closure:neighbourhoods}.
  \end{proof}
  \step{<1>4}{\pflet{$U$ be a neighbourhood of $x$}}
  \step{<1>5}{\pick\ $N$ such that $B_N \subseteq U$}
  \begin{proof}
    \pf\ From \stepref{<1>2}.
  \end{proof}
  \step{<1>6}{For $n \geq N$, we have $a_n \in U$}
  \begin{proof}
    \pf
    \begin{align*}
      a_n & \in B_1 \cap \cdots \cap B_n & (\text{\stepref{<1>3}}) \\
      & \subseteq B_N & (n \geq N) \\
      & \subseteq U & (\text{\stepref{<1>5}})
    \end{align*}
  \end{proof}
  \qed
\end{proof}

\begin{thm}[CC]
  Let $X$ and $Y$ be topological spaces where $X$ is first countable. Let $x
  \in X$. Suppose that, for every sequence $\{ x_n \}_{n \geq 1}$ such that
  $x_n \rightarrow x$ as $n \rightarrow \infty$, we have $f(x_n) \rightarrow
  f(x)$ as $n \rightarrow \infty$. Then $f$ is continuous at $x$.
\end{thm}

\begin{proof}
  \pf
  \step{<1>1}{\pflet{$V$ be a neighbourhood of $f(x)$}}
  \step{<1>2}{\assume{for a contradiction that, for every neighbourhood $U$
      of
      $x$, $f(U) \nsubseteq V$}}
  \step{<1>3}{\pick\ a countable local basis $\{ B_n \}_{n \geq 1}$}
  \step{<1>4}{For $n \geq 1$, \pick\ $a_n \in B_1 \cap \cdots \cap B_n$ such
    that
    $f(a_n) \notin V$}
  \step{<1>5}{$a_n \rightarrow x$ as $n \rightarrow \infty$}
  \begin{proof}
    \pf
    \step{<2>1}{\pflet{$U$ be a neighbourhood of $x$}}
    \step{<2>2}{\pick\ $N$ such that $B_N \subseteq U$}
    \step{<2>3}{For all $n \geq N$, $a_n \in U$}
    \begin{proof}
      \pf
      \begin{align*}
        a_n & \in B_1 \cap \cdots \cap B_n & (\text{\stepref{<1>4}}) \\
        & \subseteq B_N & (n \geq N) \\
        & \subseteq U & (\text{\stepref{<2>2}})
      \end{align*}
    \end{proof}
  \end{proof}
  \step{<1>6}{$f(a_n) \rightarrow f(x)$ as $n \rightarrow \infty$}
  \step{<1>7}{There exists $N$ such that, for all $n \geq N$, we have $f(a_n)
    \in
    V$}
  \qedstep
\end{proof}

\begin{lm}[CC]
  $\mathbb{R}^\omega$ under the box topology is not first countable.
\end{lm}

\begin{proof}
  \pf
  \step{<1>1}{\pflet{$\{ B_n \}_{n \geq 1}$ be any countable set of
      neighbourhoods of $\vec{0}$}}
  \step{<1>2}{For $n \geq 1$, \pick\ $U_{nm}$ for $m \geq 1$ such that
    $\vec{0}
    \in \prod_{m = 1}^\infty U_{nm} \subseteq B_n$}
  \step{<1>3}{For $n \geq 1$, \pick\ $a_n$, $b_n$ such that $0 \in (a_n, b_n)
    \subseteq U_{nn}$}
  \step{<1>4}{\pflet{$U = \prod_{n = 1}^\infty (a_n / 2, b_n / 2)$}}
  \step{<1>5}{$\vec{0} \in U$}
  \step{<1>6}{For all $n$, $B_n \nsubseteq U$}
  \qed
\end{proof}

\begin{lm}[CC]
  If $J$ is uncountable then $\mathbb{R}^J$ is not first countable.
\end{lm}

\begin{proof}
  \pf
  \step{<1>1}{\pflet{$\{ B_n \}_{n \geq 1}$ be a countable family of
      neighbourhoods of $\vec{0}$}}
  \step{<1>2}{For $n \geq 1$, \pick\ $U_{n \alpha}$ such that $\vec{0} \in
    \prod_{\alpha \in J} U_{n \alpha} \subseteq B_n$ where $U_{n \alpha}$ is
    open in $\mathbb{R}$ and $U_{n \alpha} = \mathbb{R}$ except for $\alpha =
    \alpha_{n1}, \ldots, \alpha_{nr_n}$}
  \step{<1>3}{\pick\ $\beta$ such that $\beta$ is different from
    $\alpha_{ni}$
    for all $n$, $i$}
  \step{<1>4}{\pflet{$V = \pi_\beta^{-1}((-1, 1))$}}
  \step{<1>5}{$\vec{0} \in V$}
  \step{<1>6}{$V \nsubseteq B_n$ for all $n$}
  \qed
\end{proof}

\begin{lm}
  $\mathbb{R}_l$ is first countable.
\end{lm}

\begin{proof}
  \pf\ For all $x \in \mathbb{R}$, $\{ [x, q) : q \in \mathbb{Q}, q > x \}$
  is a basis at $x$. \qed
\end{proof}

\begin{lm}
  The ordered square is first countable.
\end{lm}

\begin{proof}
  \pf
  \step{<1>1}{\pflet{$(x, y) \in I_o^2$} \prove{There exists a countable
      local
      basis $\mathcal{B}$ at $(x, y)$}}
  \step{<1>2}{\case{$(x, y) = (0, 0)$}}
  \begin{proof}
    \pf\ Take $\mathcal{B} = \{ [(0, 0), (0, q)) : q \in \mathbb{Q}, 0 < q <
    1
    \}$.
  \end{proof}
  \step{<1>3}{\case{$0 < y < 1$}}
  \begin{proof}
    \pf\ Take $\mathcal{B} = \{ ((x, q), (x, q')) : q, q' \in \mathbb{Q}, q <
    y < q' \}$.
  \end{proof}
  \step{<1>4}{\case{$x < 1, y = 1$}}
  \begin{proof}
    \pf\ Take $\mathcal{B} = \{ ((x, q), (q', 0)) : q, q' \in \mathbb{Q}, 0 <
    q < 1, x < q' < 1 \}$.
  \end{proof}
  \step{<1>5}{\case{$x > 0, y = 0$}}
  \begin{proof}
    \pf\ Take $\mathcal{B} = \{ ((q, 1), (x, q')) : q, q' \in \mathbb{Q}, 0 <
    q < x, 0 < q' < 1 \}$
  \end{proof}
  \step{<1>6}{\case{$(x, y) = (1, 1)$}}
  \begin{proof}
    \pf\ Take $\mathcal{B} = \{ ((1, q), (1, 1)] : q \in \mathbb{Q}, 0 < q <
    1
    \}$.
  \end{proof}
  \qed
\end{proof}

  \begin{prop}
 A subspace of a first countable space is first countable.
\end{prop}

\begin{proof}
 \pf
 \step{<1>1}{\pflet{$X$ be a first countable space and $A \subseteq X$}}
 \step{<1>2}{\pflet{$a \in A$}}
 \step{<1>3}{\pick\ a countable basis $\mathcal{B}$ at $a$ in $X$}
 \step{<1>4}{$\{ B \cap A : B \in \mathcal{B}$ is a countable basis at $a$ in
   $A$.} % TODO Extract lemma
 \qed
\end{proof}

\begin{prop}[CC]
 A countable product of first countable spaces is first countable.
\end{prop}

\begin{proof}
 \pf
 \step{<1>1}{\pflet{$\{ X_n \}_{n \in \mathbb{Z}^+}$ be a countable family of
     first countable spaces.}}
 \step{<1>2}{\pflet{$\vec{x} \in \prod_{n=1}^\infty X_n$}}
 \step{<1>3}{\pick\ a countable basis $\mathcal{B}_n$ at $x_n$ in $X_n$ for
all
   $n$}
 \step{<1>4}{\pflet{$\mathcal{B}$ be the set of all sets $\prod_{i=1}^n U_n$
     where $U_n \in \mathcal{B}_n$ for finitely many $n$ and $U_n = X_n$ for
     all other $n$.}}
 \step{<1>5}{$\mathcal{B}$ is a countable basis at $\vec{x}$ in
   $\prod_{n=1}^\infty X_n$}
 \qed
\end{proof}

\begin{cor}
  The space $\mathbb{R}^\omega$ is first countable.
\end{cor}

  \begin{prop}
 The space $S_\Omega$ is first countable.
\end{prop}

\begin{proof}
 \pf
 \step{<1>1}{\pflet{$\alpha \in S_\Omega$} \prove{$\alpha$ has a countable
local
     basis.}}
 \step{<1>2}{\case{$\alpha$ is zero or a successor ordinal.}}
 \begin{proof}
   \pf\ In this case, $\{ \{ \alpha \} \}$ is a local basis.
 \end{proof}
 \step{<1>3}{\case{$\alpha$ is a limit ordinal.}}
 \begin{proof}
   \step{<2>1}{\pick\ a countable sequence $(\beta_n)$ with supremum $\alpha$}
   \step{<2>2}{$\{ (\beta_n, \alpha + 1) : n \in \mathbb{Z}^+ \}$ is a local
     basis.}
 \end{proof}
 \qed
\end{proof}

  \begin{prop}
    \label{prop:topology:first_countable:S_omega}
  The space $\overline{S_\Omega}$ is not first countable.
\end{prop}

\begin{proof}
 \pf
 \step{<1>1}{\assume{for a contradiction $\mathcal{B}$ is a countable local
     basis at $\Omega$}}
 \step{<1>2}{\pflet{$\alpha = \sup \{ \inf B : B \in \mathcal{B} \}$}}
 \step{<1>3}{$\alpha < \Omega$}
 \step{<1>4}{There is no $B \in \mathcal{B}$ such that $B \subseteq (\alpha,
+
   \infty)$}
 \qed
\end{proof}

\begin{prop}
 The continuous image of a first countable space is first countable.
\end{prop}

\begin{proof}
  \pf
  \step{<1>1}{\pflet{$X$ be a first countable space, $Y$ a space and $f : X
      \rightarrow Y$ continuous.}}
  \step{<1>2}{\pflet{$y \in f(X)$}}
  \step{<1>3}{\pick\ $x \in X$ such that $y = f(x)$}
  \step{<1>4}{\pick\ a countable local basis $\mathcal{B}$ at $x$}
  \step{<1>5}{$\{ f(B) : B \in \mathcal{B} \}$ is a countable local basis at
    $y$.}
  \qed
\end{proof}

\begin{prop}
 $S_\Omega \times \overline{S_\Omega}$ is not first countable.
\end{prop}

\begin{proof}
\pf\ $(0, \Omega)$ has no countable basis. \qed
\end{proof}

\begin{prop}
The Sorgenfrey plane is first countable.
\end{prop}

\begin{proof}
 \pf\ For any point $(a,b)$, the set $\{ [a, a + q) \times [b, b + r) : q, r
 \in \mathbb{Q} \}$ is a countable local basis at $(a,b)$. \qed
\end{proof}


\section{Separable Spaces}

  \begin{df}[Separable Space]
  A topological space $X$ is \emph{separable} iff it has a countable dense
  subset.
\end{df}

  \begin{prop}
    \label{prop:topology:separable:S_omega}
 The space $S_\Omega$ is not separable.
\end{prop}

\begin{proof}
 \pf
 \step{<1>1}{\pflet{$D \subseteq S_\Omega$ be countable.}}
 \step{<1>2}{\pflet{$\alpha = \sup D$}}
 \step{<1>3}{$\overline{D} \subseteq (-\infty, \alpha]$}
 \qed
\end{proof}

  \begin{prop}
  The space $\overline{S_\Omega}$ is not separable.
\end{prop}

\begin{proof}
 \pf
 \step{<1>1}{\pflet{$D \subseteq S_\Omega$ be countable.}}
 \step{<1>2}{\pflet{$\alpha = \sup \{ \beta \in D : \beta < \Omega \}$}}
 \step{<1>3}{$\alpha < \Omega$}
 \begin{proof}
   \pf\ $\alpha$ is the supremum of countably many countable ordinals.
 \end{proof}
 \step{<1>4}{$\overline{D} \subseteq (-\infty, \alpha] \cup \{ \Omega \}$}
 \qed
\end{proof}

\begin{cor}
  Not every compact Hausdorff space is separable.
\end{cor}

\begin{prop}
  Every open subspace of a separable space is separable.
\end{prop}

\begin{proof}
  \pf
  \step{<1>1}{\pflet{$X$ be a separable space with countable dense subset $D$.}}
  \step{<1>2}{\pflet{$U$ be an open subspace of $X$} \prove{$D \cap U$ is a countable dense subset of $U$.}}
  \step{<1>3}{$D \cap U$ is countable.}
  \step{<1>4}{\pflet{$V$ be an open set in $U$.}}
  \step{<1>5}{$V$ is open in $X$}
  \begin{proof}
    \pf\ Lemma \ref{lm:topology:subspace:open}
  \end{proof}
  \step{<1>6}{$V$ intersects $D$}
  \step{<1>7}{$V$ intensects $D \cap U$}
  \qed
\end{proof}

\begin{prop}[CC]
 The product of a countable family of separable spaces is separable.
\end{prop}

\begin{proof}
 \pf
 \step{<1>1}{\pflet{$(X_n)$ be a countable family of separable spaces.}}
 \step{<1>2}{For $n \geq 1$, \pick\ a dense set $D_n$ in $X_n$}
 \step{<1>3}{$\prod_{n=1}^\infty D_n$ is dense in $\prod_{n=1}^\infty X_n$.}
 \qed
\end{proof}

\begin{prop}
 The continuous image of a separable space is separable.
\end{prop}

\begin{proof}
 \pf
 \step{<1>1}{\pflet{$X$ be a separable space, $Y$ a space and $f : X
\rightarrow
     Y$ be continuous.}}
 \step{<1>2}{\pick\ a countable dense set $D$ in $X$}
 \step{<1>3}{$f(D)$ is dense in $f(X)$.}
 \qed
\end{proof}

\begin{cor}
  Let $\{ X_\alpha \}_{\alpha \in J}$ be a family of nonempty topological
  spaces. If $\prod_{\alpha \in J} X_\alpha$ is separable then each
$X_\alpha$ is separable.
\end{cor}

\begin{cor}
  $S_\Omega \times \overline{S_\Omega}$ is not separable.
\end{cor}

\begin{prop}
The ordered square is not separable.
\end{prop}

\begin{proof}
 \pf\ $\{ \{x\} \times (0,1) : x \in [0,1] \}$ is an uncountable set of
disjoint open sets. \qed
\end{proof}

\begin{prop}
 $\mathbb{R}_l$ is separable.
\end{prop}

\begin{proof}
 \pf\ $\mathbb{Q}$ is dense. \qed
\end{proof}

\begin{prop}
The Sorgenfrey plane is separable.
\end{prop}

\begin{proof}
 \pf\ $\mathbb{Q}^2$ is dense. \qed
\end{proof}

\begin{prop}
 Not every closed subspace of a separable space is separable.
\end{prop}

\begin{proof}
 \pf\ $\mathbb{R}_l^2$ is separable but the subspace $\{ (x, -x) : x \in \mathbb{R} \}$ is not. \qed
\end{proof}

\section{The Second Countability Axiom}

  \begin{df}[Second Countability Axiom]
  A topological space satisfies the \emph{second countability axiom}, or is
  \emph{second countable}, iff it has a countable basis.
\end{df}

 \begin{prop}
$S_\Omega$ is not second countable.
\end{prop}

\begin{proof}
 \pf\ $\{ \{ \alpha \} : \alpha \text{ is a countable successor ordinal} \}$
is an uncountable set of disjoint open sets. \qed
\end{proof}

\begin{prop}
  \label{prop:topology:second_countable:subspace}
 A subspace of a second countable space is second countable.
\end{prop}

\begin{proof}
 \pf
 \step{<1>1}{\pflet{$X$ be a second countable space and $A \subseteq X$}}
 \step{<1>2}{\pick\ a countable basis $\mathcal{B}$ for $X$}
 \step{<1>3}{$\{ B \cap A : B \in \mathcal{B} \}$ is a countable basis for
$A$}
 \qed
\end{proof}

  \begin{prop}[CC]
 The product of countably many second countable spaces is second countable.
\end{prop}

\begin{proof}
 \pf
 \step{<1>1}{\pflet{$\{X_n\}_{n \in \mathbb{Z}^+}$ be a countable family of
     second countable spaces.}}
 \step{<1>2}{For $n \in \mathbb{Z}^+$, \pick\ a countable basis
$\mathcal{B}_n$
   for $X_n$.}
 \step{<1>3}{\pflet{$\mathcal{B}$ be the set of all sets of the form
     $\prod_{n=1}^\infty U_n$, where $U_n \in \mathcal{B}_n$ for finitely
many        $n$, and $U_n = X_n$ for all other $n$.}}
\step{<1>4}{$\mathcal{B}$ is a countable basis for $\prod_{n=1}^\infty X_n$}
\qed
\end{proof}

  \begin{thm}[CC]
 Every second countable space is separable.
\end{thm}

\begin{proof}
 \pf
 \step{<1>1}{\pflet{$X$ be a second countable space.}}
 \step{<1>2}{\pick\ a countable basis $\mathcal{B}$ for $X$}
 \step{<1>3}{For $B \in \mathcal{B}$ nonempty, \pick\ a point $x_B \in B$}
 \step{<1>4}{$D = \{ x_B : B \in \mathcal{B} \setminus \{ \emptyset \} \}$ is
dense.}
 \begin{proof}
   \step{<2>1}{\pflet{$l \in X$} \prove{$l \in \overline{D}$}}
   \step{<2>2}{\pflet{$B \in \mathcal{B}$ such that $l \in B$}}
   \step{<2>3}{$x_B \in B \cap D$}
   \qedstep
   \begin{proof}
     \pf By Theorem \ref{thm:topology:closure:basis}
   \end{proof}
 \end{proof}
\end{proof}

\begin{cor}
  $S_\Omega \times \overline{S_\Omega}$ is not second countable.
\end{cor}

\begin{cor}
  The space $\mathbb{R}^\omega$ is separable.
\end{cor}

\begin{cor}
If $J$ is uncountable then $\mathbb{R}^J$ is not second countable.
\end{cor}

  \begin{prop}
 The ordered square is not second countable.
\end{prop}

\begin{proof}
 \pf
 \step{<1>1}{\pflet{$\mathcal{B}$ be any basis}}
 \step{<1>2}{For $x \in [0,1]$, \pick\ $B_x$ such that $x \in B_x \subseteq
((x,
   0), (x, 1))$}
 \step{<1>3}{The function $B_{(-)}$ is an injective function $[0, 1]
\rightarrow
   \mathcal{B}$}
 \step{<1>4}{$\mathcal{B}$ is uncountable.}
 \qed
\end{proof}

\begin{prop}
  The space $\overline{S_\Omega}$ is not second countable.
\end{prop}

\begin{proof}
  \pf\ It is not first countable (Proposition
  \ref{prop:topology:first_countable:S_omega}). \qed
\end{proof}

\begin{prop}
 The continuous image of a second countable space is second countable.
\end{prop}

\begin{proof}
 \pf
 \step{<1>1}{\pflet{$X$ be a second countable space, $Y$ a space and $f : X
     \rightarrow Y$ be continuous.}}
 \step{<1>2}{\pick\ a countable basis $\mathcal{B}$ for $X$.}
 \step{<1>3}{$\{ f(B) : B \in \mathcal{B}$ is a countable basis for $f(X)$}
 \qed
\end{proof}

\begin{thm}
  \label{thm:topology:normal:regular_lindelof}
  Every regular Lindel\"{o}f space is normal.
\end{thm}

\begin{proof}
 \pf
 \step{<1>1}{\pflet{$X$ be a regular Lindel\"{o}f space.}}
 \step{<1>2}{\pflet{$A$ and $B$ be disjoint closed sets in $X$.}}
 \step{<1>3}{$\{ U \text{ open in } X : \overline{U} \cap B = \emptyset \}$
   covers $A$}
     \begin{proof}
       \pf\ Proposition \ref{prop:topology:regular:closure}.
     \end{proof}
 \step{<1>4}{\pick\ a countable open covering $\{ U_n : n \in \mathbb{Z}^+
\}$
   of $A$ such that $\overline{U_n} \cap B = \emptyset$ for all $n$}
 \step{<1>5}{\pick\ a countable open covering $\{ V_n : n \in \mathbb{Z}^+
\}$
   of $B$ such that $\overline{V_n} \cap A = \emptyset$ for all $n$}
 \begin{proof}
   \pf\ Similar.
 \end{proof}
 \step{<1>6}{For $n \in \mathbb{Z}^+$, \pflet{$U_n' = U_n \setminus
     \bigcup_{i=1}^n \overline{V_i}$ and $V_n' = V_n \setminus
     \bigcup_{i=1}^n \overline{U_i}$}}
 \step{<1>7}{\pflet{$U' = \bigcup_{n=1}^\infty U_n'$ and $V =
     \bigcup_{n=1}^\infty V_n'$}}
 \step{<1>8}{$A \subseteq U'$ and $B \subseteq V'$}
 \step{<1>9}{$U' \cap V' = \emptyset$}
 \qed
\end{proof}

\begin{cor}
If $J$ is uncountable then $\mathbb{R}^J$ is not Lindel\"{o}f.
\end{cor}
  \begin{prop}
 Every second countable regular space is completely normal.
\end{prop}

\begin{proof}
 \pf
 \step{<1>1}{\pflet{$X$ be second countable and regular and $Y \subseteq X$}}
 \step{<1>2}{$Y$ is second countable}
 \begin{proof}
   \pf\ Proposition \ref{prop:topology:second_countable:subspace}.
 \end{proof}
 \step{<1>3}{$Y$ is regular}
 \begin{proof}
   \pf\ Proposition \ref{prop:topology:regular:subspace}
 \end{proof}
 \step{<1>4}{$Y$ is normal}
 \begin{proof}
   \pf\ Theorem \ref{thm:topology:normal:regular_lindelof}
 \end{proof}
 \qed
\end{proof}

 \begin{prop}
 The space $\mathbb{R}^\omega$ is second countable.
\end{prop}

\begin{proof}
 \pf\ The sets $\prod_{n=0}^\infty U_n$ form a basis, where $U_n$ is an
 interval of the form $(q, r)$ for $q, r \in \mathbb{Q}$ for finitely many
 $n$, and $U_n = \mathbb{R}$ for all other $n$. \qed
\end{proof}

\begin{prop}[CC]
In a second countable space, every discrete subspace is countable.
\end{prop}

\begin{proof}
\pf
\step{<1>1}{\pflet{$X$ be a second countable space}}
\step{<1>2}{\pick\ a countable basis $\mathcal{B}$}
\step{<1>3}{\pflet{$D \subseteq X$ be discrete}}
\step{<1>4}{For $a \in D$, \pick\ $B_a \in \mathcal{B}$ such that $B_a \cap D
=
  \{ a \}$}
\step{<1>5}{$a \mapsto B_a$ is injective}
\qed
\end{proof}

\begin{prop}
The space $\mathbb{R}_K$ is second countable.
\end{prop}

\begin{proof}
\pf\ $\{ (a,b) : a, b \in \mathbb{R} \} \cup \{ (a, b) - K : a, b \in
\mathbb{Q} \}$ is a basis. \qed
\end{proof}

\begin{cor}
The space $\mathbb{R}_K$ is first countable.
\end{cor}

\begin{cor}
The space $\mathbb{R}_K$ is separable.
\end{cor}

\begin{prop}
Let $J$ be a set with $|J| > |\mathbb{R}|$. Then $\mathbb{R}^J$ is not separable.
\end{prop}

\begin{proof}
\pf
\step{<1>1}{\assume{$D$ is countable and dense in $\mathbb{R}^J$} \prove{$|J| \leq |\mathbb{R}|$}}
\step{<1>2}{Define $f : J \rightarrow \mathcal{P} D$ by $f(\alpha) = D \cap \inv{\pi_\alpha}((0,1))$}
\step{<1>3}{$f$ is injective}
\begin{proof}
  \step{<2>1}{\pflet{$\alpha, \beta \in J$ with $\alpha \neq \beta$}}
  \step{<2>2}{\pick\ $x \in D \cap \inv{\pi_\alpha}((0,1)) \cap \inv{\pi_\beta}((2, 3))$}
  \step{<2>3}{$x \in f(\alpha)$ but $x \notin f(\beta)$}
\end{proof}
\qed
\end{proof}

\begin{cor}
The product of a family of separable spaces is not necessarily separable.
\end{cor}
