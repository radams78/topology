\chapter{Countability Axioms}

\section{The First Countability Axiom}

\begin{df}[First Countable]
  A topological space satisfies the \emph{first countability axiom}, or is \emph{first countable}, iff every point has a countable local basis.
\end{df}

\begin{prop}
  A space is first countable iff every point has a countable basis $\{ B_1, B_2, \ldots \}$ such that $B_1 \supseteq B_2 \supseteq \cdots$.
\end{prop}

\begin{proof}
  \pf\ If a point has a countable basis $\{ C_1, C_2, \ldots \}$, then take $B_n = C_1 \cap \cdots \cap C_n$. \qed
\end{proof}

\begin{prop}[Sequence Lemma (CC)]
  Let $X$ be a first countable space, $A \subseteq X$ and $a \in \overline{A}$. Then there exists a sequence in $A$ that converges to $a$.
\end{prop}

\begin{proof}
  \pf
  \step{<1>1}{\pick\ a basis $\{ B_1, B_2, \ldots \}$ at $a$ with $B_1 \supseteq B_2 \supseteq \cdots$}
  \step{<1>2}{For $n = 1, 2, \ldots$, \pick\ $a_n \in B_n \cap A$ \prove{For every neighbourhood $U$ of $a$, there exists $N$ such that, for all $n \geq N$, we have $a_n \in U$.}}
  \begin{proof}
    \pf\ Using Proposition \ref{prop:closure:membership}.
  \end{proof}
  \step{<1>3}{\pflet{$U$ be a neighbourhood of $a$.}}
  \step{<1>4}{\pick\ $N$ such that $B_N \subseteq U$}
  \step{<1>5}{For all $n \geq N$ we have $a_n \in U$}
  \begin{proof}
    \pf\ Since $a_n \in B_n \subseteq B_N \subseteq U$.
  \end{proof}
  \qed
\end{proof}

\begin{ex}
  \enumerate
  \item
  The space $\mathbb{R}_l$ is first countable. For any $a \in \mathbb{R}$ the set of all intervals $[a,q)$ for $q$ rational is a local basis at $a$.
  \item
  The ordered square is first countable. For:
  \begin{itemize}
    \item
    A basis at $(0,0)$ is the set of all intervals of the form $((0,0),(0,q))$ with $q > 0$ rational.
    \item
    For $0 < y < 1$, a basis at $(x,y)$ is the set of all intervals of the form $((x,q), (x,r))$ with $q, r$ rational, $q < y < r$.
    \item
    For $x < 1$, a basis at $(x, 1)$ is the set of all intervals of the form $((x,q), (r, 0))$ with $q, r$ rational, $q < 1$ and $r > x$.
    \item
    For $x > 0$, a basis at $(x, 0)$ is the set of all intervals of the form $((q, 0), (x, r))$ with $q, r$ rational, $q < x$ and $r > 0$.
    \item
    A basis at $(1,1)$ is the set of all intervals of the form $((1,q),(1,1))$ with $q < 1$ rational.
  \end{itemize}
  \item
  The space $\mathbb{R}^\omega$ under the box topology is not first countable. Let $A$ be the set of all sequences whose members are all positive. Then $\vec{0} \in \overline{A}$ but there is no sequence in $A$ that converges to $\vec{0}$. For, given any sequence $((a_{mn})_n)_m$ in $A$, the open set
  \[ \prod_{n=1}^\infty (-a_{nn},a_{nn}) \]
  contains $\vec{0}$ but does not contain any member of the sequence.
\end{ex}

\begin{ex}
  For $J$ uncountable, the space $\mathbb{R}^J$ under the product topology is not first countable.
\end{ex}

\begin{proof}
  \pf
  \step{<1>1}{\pflet{$A = \{ (x_\alpha) \in \mathbb{R}^J : x_\alpha = 1 \text{ for all but finitely many } \alpha \}$}}
  \step{<1>2}{$\vec{0} \in \overline{A}$}
  \begin{proof}
    \step{<2>1}{\pflet{$\vec{0} \in \prod_{\alpha \in J} U_\alpha$ where each $U_\alpha$ is open in $\mathbb{R}$ and $U_\alpha = \mathbb{R}$ for all but finitely many $\alpha$, say $\alpha_1$, \ldots, $\alpha_n$}}
    \step{<2>2}{\pflet{$x_\alpha = 0$ for $\alpha = \alpha_1, \ldots, \alpha_n$ and $x_\alpha = 1$ for all other $\alpha$}}
    \step{<2>3}{$(x_\alpha) \in \prod_{\alpha \in J} U_\alpha \cap A$}
  \end{proof}
  \step{<1>3}{There is no sequence of points in $A$ that converges to $\vec{0}$.}
  \begin{proof}
    \step{<2>1}{\assume{for a contradiction $((a_{n\alpha})_\alpha)_n$ is a sequence in $A$ that converges to $\vec{0}$}}
    \step{<2>2}{For $n = 1, 2, \ldots$, \pflet{$J_n = \{ \alpha \in J : a_{n \alpha} \neq 1 \}$}}
    \step{<2>3}{\pick\ $\beta \in J \setminus \bigcup_{n=1}^\infty J_n$}
    \begin{proof}
      \pf\ This is possible because $J$ is uncountable.
    \end{proof}
    \step{<2>4}{$\vec{0} \in \inv{\pi_\beta}((-1, 1))$}
    \step{<2>5}{There is no $n$ such that $(a_{n\alpha})_\alpha \in \inv{\pi_\beta}((-1,1))$}
    \begin{proof}
      \pf\ $a_{n\beta} = 1$ for all $n$.
    \end{proof}
  \end{proof}
  \qed
\end{proof}
