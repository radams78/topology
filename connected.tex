\chapter{Connectedness}

\section{Connectedness}

\begin{df}[Connected]
  Let $X$ be a topological space. A \emph{separation} of $X$ is a pair of nonempty disjoint open sets $U$, $V$ such that $U \cup V = X$. The space $X$ is \emph{connected} iff it does not have a separation.
\end{df}

\begin{prop}
  A space $X$ is connected iff the only sets that are both open and closed are $\emptyset$ and $X$.
\end{prop}

\begin{proof}
  \pf\ Immediate from definitions. \qed
\end{proof}

\begin{prop}
  \label{prop:connected:subspace}
  Let $X$ be a topological space and $Y$ a subspace of $X$. Then a separation of $Y$ is a pair of disjoint nonempty sets $A$ and $B$ such that $A \cup B = Y$ and neither of $A$, $B$ contains a limit point of the other.
\end{prop}

\begin{proof}
  \pf\ If $A$ and $B$ are disjoint, nonempty, and $A \cup B = Y$, then we have
  \begin{align*}
    & \text{$A$ and $B$ form a separation of $Y$} \\
    \Leftrightarrow & \text{$A$ and $B$ are open in $Y$} \\
    \Leftrightarrow & \text{$A$ and $B$ are closed in $Y$} \\
    \Leftrightarrow & \text{$A$ and $B$ each contain all their limit points in $Y$} \\
    \Leftrightarrow & \text{neither of $A$, $B$ contains a point that is a limit point of the other in $Y$} \\
    \Leftrightarrow & \text{neither of $A$, $B$ contains a point that is a limit point of the other in $X$}
  \end{align*}
  \qed
\end{proof}

\begin{prop}
  \label{prop:connected:subset}
  If $C$ and $D$ form a separation of $X$ and $Y$ is a connected subspace of $X$ then $Y \subseteq C$ or $Y \subseteq D$.
\end{prop}

\begin{proof}
  \pf\ Otherwise $C \cap Y$ and $D \cap Y$ would form a separation of $Y$. \qed
\end{proof}

\begin{prop}
  Let $\mathcal{A}$ be a set of connected subspaces of a space $X$ and $B$ a connected subspace of $X$ such that every element of $\mathcal{A}$ intersects $B$. Then $\bigcup \mathcal{A} \cup B$ is connected.
\end{prop}

\begin{proof}
  \pf
  \step{<1>1}{\pflet{$\mathcal{A}$ be a set of connected subspaces of a space $X$ and $B$ a connected subspace of $X$ such that every element of $\mathcal{A}$ intersects $B$.}}
  \step{<1>2}{\assume{for a contradiction $C$ and $D$ form a separation of $\bigcup \mathcal{A} \cup B$}}
  \step{<1>3}{\assume{w.l.o.g.~$B \subseteq C$}}
  \begin{proof}
    \pf\ Proposition \ref{prop:connected:subset}, \stepref{<1>1}, \stepref{<1>2}.
  \end{proof}
  \step{<1>4}{For all $A \in \mathcal{A}$ we have $A \subseteq C$}
  \begin{proof}
    \pf\ Proposition \ref{prop:connected:subset}, \stepref{<1>1}, \stepref{<1>2}, \stepref{<1>3}.
  \end{proof}
  \step{<1>5}{$\bigcup \mathcal{A} \cup B \subseteq C$}
  \begin{proof}
    \pf\ \stepref{<1>3}, \stepref{<1>4}.
  \end{proof}
  \qedstep
  \begin{proof}
    \pf\ This contradicts the fact that $D$ is nonempty.
  \end{proof}
  \qed
\end{proof}

\begin{cor}
  \label{cor:connected:union}
  The union of a set of connected subspaces of a space $X$ that have a point in common is connected.
\end{cor}

\begin{prop}
  Let $(A_n)$ be a sequence of connected subspaces of $X$ such that $A_n$ intersects $A_{n+1}$ for all $n$. Then $\bigcup_n A_n$ is connected.
\end{prop}

\begin{proof}
  \pf
  \step{<1>1}{\pflet{$(A_n)$ be a sequence of connected subspaces of $X$ such that $A_n$ intersects $A_{n+1}$ for all $n$}}
  \step{<1>2}{\assume{for a contradiction $C$ and $D$ form a separation of $\bigcup_n A_n$}}
  \step{<1>3}{\assume{w.l.o.g.~$A_1 \subseteq C$}}
  \begin{proof}
    \pf\ Proposition \ref{prop:connected:subset}, \stepref{<1>1}, \stepref{<1>2}.
  \end{proof}
  \step{<1>4}{For all $n$, $A_n \subseteq C$}
  \begin{proof}
    \step{<2>1}{\assume{$A_n \subseteq C$}}
    \step{<2>2}{\pick\ $a \in A_n \cap A_{n+1}$}
    \begin{proof}
      \pf\ \stepref{<1>1}.
    \end{proof}
    \step{<2>3}{$a \in C$}
    \begin{proof}
      \pf\ \stepref{<2>1}, \stepref{<2>2}
    \end{proof}
    \step{<2>4}{$A_{n+1} \subseteq C$}
    \begin{proof}
      \pf\ Proposition \ref{prop:connected:subset}, \stepref{<1>1}, \stepref{<1>2}, \stepref{<2>2}, \stepref{<2>3}.
    \end{proof}
    \qedstep
    \begin{proof}
      \pf\ By induction using \stepref{<1>3}.
    \end{proof}
  \end{proof}
  \qedstep
  \begin{proof}
    \pf\ This contradicts the fact that $D$ is nonempty.
  \end{proof}
\end{proof}

\begin{prop}
  \label{prop:connected:closure}
  Let $A$ be a connected subspace of $X$. If $A \subseteq B \subseteq \overline{A}$ then $B$ is connected.
\end{prop}

\begin{proof}
  \pf
  \step{<1>1}{\assume{for a contradiction $C$ and $D$ form a separation of $B$}}
  \step{<1>2}{\assume{w.l.o.g.~$A \subseteq C$}}
  \begin{proof}
    \pf\ Using Proposition \ref{prop:connected:subset}.
  \end{proof}
  \step{<1>3}{\pick\ $x \in D$}
  \begin{proof}
    \pf\ Since $D$ is nonempty.
  \end{proof}
  \step{<1>4}{$x$ is a limit point of $A$}
  \begin{proof}
    \pf\ Since $x \in \overline{A}$ and $x \notin A$.
  \end{proof}
  \step{<1>5}{$x$ is a limit point of $C$}
  \begin{proof}
    \pf\ By \stepref{<1>2}
  \end{proof}
  \qedstep
  \begin{proof}
    \pf\ This contradicts Proposition \ref{prop:connected:subspace}.
  \end{proof}
  \qed
\end{proof}

\begin{cor}
  \label{cor:connected:closure}
  Let $X$ be a topological space. The closure of a connected subspace of $X$ is connected.
\end{cor}

\begin{prop}
  \label{prop:connected:image}
  The continuous image of a connected space is connected.
\end{prop}

\begin{proof}
  \pf\ Let $f : X \rightarrow Y$ be continuous and $X$ be connected. If $C$ and $D$ formed a separation of $f(X)$ then $\inv{f}(C)$ and $\inv{f}(D)$ would form a separation of $X$. \qed
\end{proof}

\begin{prop}
  \label{prop:connected:binary_product}
  The product of two connected spaces is connected.
\end{prop}

\begin{proof}
  \pf
  \step{<1>1}{\pflet{$X$ and $Y$ be connected.}}
  \step{<1>2}{\pick\ $(a,b) \in X \times Y$}
  \begin{proof}
    \pf\ If either $X$ or $Y$ is empty then $X \times Y$ is empty and hence connected.
  \end{proof}
  \step{<1>3}{$X \times \{ b \}$ is connected.}
  \begin{proof}
    \pf\ It is homeomorphic to $X$.
  \end{proof}
  \step{<1>4}{For all $x \in X$ we have $\{ x \} \times Y$ is connected.}
  \begin{proof}
    \pf\ It is homeomorphic to $Y$.
  \end{proof}
  \step{<1>5}{For $x \in X$ \pflet{$T_x = (X \times \{ b \}) \cup (\{x\} \times Y)$}}
  \step{<1>6}{For all $x \in X$ we have $T_x$ is connected.}
  \begin{proof}
    \pf\ Corollary \ref{cor:connected:union} since the two sets have $(x,b)$ is common.
  \end{proof}
  \step{<1>7}{$X \times Y = \bigcup_{x \in X} T_x$}
  \step{<1>8}{$X \times Y$ is connected.}
  \begin{proof}
    \pf\ Corollary \ref{cor:connected:union} since the sets all have $(a,b)$ in common.
  \end{proof}
  \qed
\end{proof}

\begin{prop}
  The product of a family of connected spaces is connected.
\end{prop}

\begin{proof}
  \pf
  \step{<1>1}{\pflet{$\{ X_\alpha \}_{\alpha \in J}$ be a family of connected spaces.}}
  \step{<1>2}{\pflet{$X = \prod_{\alpha \in J} X_\alpha$}}
  \step{<1>3}{\pick\ $(a_\alpha) \in X$}
  \begin{proof}
    \pf\ We may assume $X$ is nonempty as the empty space is trivially connected.
  \end{proof}
  \step{<1>4}{For $K \finsubseteq J$, \pflet{$X_K = \{ (x_\alpha) \in X : \forall \alpha \notin K. x_\alpha = a_\alpha \}$}}
  \step{<1>5}{For $K \finsubseteq J$, we have $X_K$ is connected.}
  \begin{proof}
    \pf\ This holds  using Proposition \ref{prop:connected:binary_product} because $X_K \cong \prod_{\alpha \in K} X_\alpha$.
  \end{proof}
  \step{<1>6}{\pflet{$Y = \bigcup_{K \finsubseteq J} X_K$}}
  \step{<1>7}{$Y$ is connected.}
  \begin{proof}
    \pf\ Corollary \ref{cor:connected:union}
  \end{proof}
  \step{<1>8}{$\overline{Y} = X$}
  \begin{proof}
    \step{<2>1}{\pflet{$(x_\alpha) \in X$}}
    \step{<2>2}{\pflet{$(x_\alpha) \in \prod_{\alpha \in J} U_\alpha$ where each $U_\alpha$ is open in $X_\alpha$ and $U_\alpha = X_\alpha$ for all but finitely many $\alpha$}}
    \step{<2>3}{\pflet{$K = \{ \alpha \in J : U_\alpha \neq X_\alpha \}$}}
    \step{<2>4}{\pflet{$(y_alpha)$ be the point with $y_\alpha = x_\alpha$ for $\alpha \in K$ and $y_\alpha = a_\alpha$ for all other $\alpha$}}
    \step{<2>5}{$(y_\alpha) \in \prod_{\alpha \in J} U_\alpha$}
    \step{<2>6}{$(y_\alpha) \in X_K$}
    \step{<2>7}{$\prod_{\alpha \in J} U_\alpha$ intersects $X_K$}
    \qedstep
    \begin{proof}
      \pf\ Proposition \ref{prop:closure:membership}.
    \end{proof}
    \qed
  \end{proof}
  \step{<1>9}{$X$ is connected.}
  \begin{proof}
    \pf\ Proposition \ref{prop:connected:closure}.
  \end{proof}
  \qed
\end{proof}

\begin{prop}
  Any linear continuum is connected under the order topology.
\end{prop}

\begin{proof}
  \pf
  \step{<1>1}{\pflet{$L$ be a linear continuum under the order topology.}}
  \step{<1>2}{\assume{for a contradiction $U$ and $V$ form a separation of $L$.}}
  \step{<1>3}{\pick\ $a \in U$ and $b \in V$}
  \step{<1>4}{\assume{w.l.o.g.~$a < b$}}
  \step{<1>5}{\pflet{$s = \sup \{ x \in U : a \leq x < b \}$}}
  \begin{proof}
    \pf\ $L$ is complete.
  \end{proof}
  \step{<1>6}{\case{$s \in U$}}
  \begin{proof}
    \step{<2>1}{$s < b$}
    \step{<2>2}{\pick\ $u > s$ such that $[s,u) \subseteq U$}
    \begin{proof}
      \pf\ Proposition \ref{prop:order:open_up}
    \end{proof}
    \step{<2>3}{\pick\ $t$ such that $s < t < u$}
    \begin{proof}
      \pf\ $L$ is dense.
    \end{proof}
    \step{<2>4}{$t \in U$ and $a \leq t < b$}
    \qedstep
    \begin{proof}
      \pf\ This contradicts \stepref{<1>5}.
    \end{proof}
  \end{proof}
  \step{<1>7}{\case{$s \in V$}}
  \begin{proof}
    \step{<2>1}{$a < s$}
    \step{<2>2}{\pick\ $u < s$ such that $(u,s] \subseteq V$}
    \begin{proof}
      \pf\ Proposition \ref{prop:order:open_down}
    \end{proof}
    \step{<2>3}{$u$ is an upper bound for $\{ x \in U : a \leq x < b \}$}
    \qedstep
    \begin{proof}
      \pf\ This contradicts \stepref{<1>5}.
    \end{proof}
  \end{proof}
  \qed
\end{proof}

\begin{cor}
  If $L$ is a linear continuum under the order topology, then every interval and ray in $L$ is connected.
\end{cor}

\begin{ex}
  \begin{enumerate}
    \item   The real line $\mathbb{R}$ is connected, and so is every interval and ray in $\mathbb{R}$.
    \item Any set with more than one point is disconnected under the discrete topology.
    \item Any set is connected under the indiscrete topology.
    \item Any infinite set is connected under the finite complement topology.
    \item Any uncountable set is connected under the countable complement topology.
    \item The space $\mathbb{R}_l$ is not connected because $(- \infty, 0)$ and $[0, + \infty)$ form a separation.
    \item The space $\mathbb{R}^\omega$ in the uniform topology is not connected. The set of bounded sequences and the set of unbounded sequences form a separation.
    \item The space $\mathbb{R}^\omega$ in the box topology is not connected because it is finer than the uniform topology.
    \item The ordered square is connected, because it is a linear continuum.
    \item The topologist's sine curve is connected. The set $\{ (x, \sin 1/x) : 0 < x \leq 1 \}$ is connected by Proposition \ref{prop:connected:image}, and so the topologist's sine curve is connected by Corollary \ref{cor:connected:closure}.
  \end{enumerate}
\end{ex}

\begin{prop}
  Let $X$ be a topological space, $C$ a connected subspace of $X$, and $A \subseteq X$. If $C$ intersects $A$ and $X \setminus A$ then $C$ intersects $\partial A$.
\end{prop}

\begin{proof}
  \pf\ Otherwise $C \cap \Int A$ and $C \cap \Int (X \setminus A)$ would form a separation of $C$. \qed
\end{proof}

\begin{thm}[Intermediate Value Theorem]
  Let $X$ be a connected space and $Y$ a linearly ordered set under the order topology. Let $f : X \rightarrow Y$ be continuous. Let $a,b \in X$ and $c \in Y$. If $f(a) < c < f(b)$, then there exists $x \in X$ such that $f(x) = c$.
\end{thm}

\begin{proof}
  \pf\ If not, then $(- \infty, c) \cap f(X)$ and $(c, +\infty) \cap f(X)$ would form a separation of $f(X)$, contradicting Proposition \ref{prop:connected:image}. \qed
\end{proof}

\section{Totally Disconnected Spaces}

\begin{df}[Totally Disconnected]
  A topological space $X$ is \emph{totally disconnected} iff the only nonempty connected subspaces are the one-point sets.
\end{df}

\begin{ex}
  \begin{enumerate}
    \item Any set under the discrete topology is totally disconnected.
    \item The space $\mathbb{Q}$ is totally disconnected.
  \end{enumerate}
\end{ex}

\section{Path Connected Spaces}

\begin{df}[Path]
  Let $X$ be a topological space and $a, b \in X$. A \emph{path} from $a$ to $b$ is a continuous function $f : [0,1] \rightarrow X$ such that $f(0) = a$ and $f(1) = b$.
\end{df}

\begin{df}[Path Connected]
  A topological space $X$ is \emph{path connected} iff, for any points $a$, $b$, there exists a path from $a$ to $b$.
\end{df}

\begin{prop}
  Every path connected space is connected.
\end{prop}

\begin{proof}
  \pf
  \step{<1>1}{\pflet{$X$ be a path connected space.}}
  \step{<1>2}{\assume{for a contradiction $U$ and $V$ form a separation of $X$.}}
  \step{<1>3}{\pick\ $a \in U$ and $b \in V$}
  \step{<1>4}{\pick\ a path $p : [0,1] \rightarrow X$ from $a$ to $b$}
  \step{<1>5}{$p([0,1]) \cap U$ and $p([0,1]) \cap V$ form a separation of $p([0,1])$}
  \qedstep
\end{proof}

\begin{prop}
  The continuous image of a path connected space is path connected.
\end{prop}

\begin{proof}
  \pf
  \step{<1>1}{\pflet{$X$ be path connected and $f : X \twoheadrightarrow Y$ surjective continuous.}}
  \step{<1>2}{\pflet{$a, b \in Y$.}}
  \step{<1>3}{\pick\ $x, y \in X$ such that $f(x) = a$ and $f(y) = b$.}
  \step{<1>4}{\pick\ a path $p : [0,1] \rightarrow X$ from $x$ to $y$.}
  \step{<1>5}{$f \circ p$ is a path from $a$ to $b$.}
  \qed
\end{proof}

\begin{ex}
  \begin{enumerate}
    \item For $n \geq 1$, the unit $n$-ball is path connected, hence connected. Given $a, b \in B^n$, the function $p : [0,1] \rightarrow B^n$ given by $p(t) = (1-t)a + tb$ is a path from $a$ to $b$.
    \item For $n > 1$, $n$-dimensional punctured Euclidean space is path connected, hence connected. Given $a, b \in \mathbb{R}^n \setminus \{0\}$, we can find a path from $a$ to $b$ that consists of at most two line segments.
    \item For $n \geq 1$, the unit $n$-sphere is path connected, hence connected. For the map $g : \mathbb{R}^{n+1} \setminus \{ 0 \} \twoheadrightarrow S^n$ given by $g(x_1, \ldots, x_n) = (x_1, \ldots, x_n) / \sqrt{x_1^2 + \cdots + x_n^2}$ is continuous.
    \item The ordered square is not path connected. If $p : [0,1] \rightarrow I_o^2$ is a path from $(0,0)$ to $(1,1)$ then $\{ \inv{p}(\{x\} \times (0,1)) : x \in [0,1] \}$ is a set of uncountably many disjoint open sets in $[0,1]$, which is impossible because each must contain a different rational.
  \end{enumerate}
\end{ex}

\begin{prop}[DC]
  The topologist's sine curve is not path connected.
\end{prop}

\begin{proof}
  \pf
  \step{<1>1}{\pflet{$S = \{ (x, \sin 1/x) : 0 < x \leq 1 \}$.}}
  \step{<1>2}{\assume{for a contradiction $p : [0,1] \rightarrow \overline{S}$ is a path from $(0,0)$ to $(1, \sin 1)$.}}
  \step{<1>3}{\pflet{$p(t) = (x(t), y(t))$ for $t \in [0,1]$}}
  \step{<1>4}{$x$ and $y$ are continuous.}
  \begin{proof}
    \pf\ Proposition \ref{prop:continuous:coordinate}, \stepref{<1>2}, \stepref{<1>3}.
  \end{proof}
  \step{<1>5}{$\{ t \in [0,1] : x(t) = 0 \}$ is closed.}
  \begin{proof}
    \pf\ From \stepref{<1>4} since the set is $\inv{x}(\{0\})$.
  \end{proof}
  \step{<1>6}{\pflet{$b$ be the greatest element of $[0,1]$ such that $x(b) = 0$ for some $y$.}}
  \begin{proof}
    \pf\ Proposition \ref{prop:order:closed_greatest}
  \end{proof}
  \step{<1>7}{There exists a sequence $(t_n)$ in $(b,1]$ such that $t_n \rightarrow b$ as $n \rightarrow \infty$ and $y(t_n) = (-1)^n$}
  \begin{proof}
    \step{<2>1}{\pflet{$U = 1 - b$}}
    \step{<2>2}{For all $n$, \pick\ $u_n$ with $0 < u_n < x(b + U/n)$ such that $\sin 1/u_n = (-1)^n$}
    \begin{proof}
      \pf\ This is possible since $\sin 1/x$ takes values 1 and -1 infinitely often in $(0, \epsilon)$ for any $\epsilon$.
    \end{proof}
    \step{<2>3}{For all $n$, \pick\ $t_n$ with $b < t_n < b + U/n$ such that $x(t_n) = u_n$}
    \begin{proof}
      \pf\ One such value must exist by the Intermediate Value Theorem.
    \end{proof}
    \step{<2>4}{$t_n \rightarrow b$ as $n \rightarrow \infty$}
    \step{<2>5}{$y(t_n) = (-1)^n$ for all $n$}
  \end{proof}
  \qedstep
  \begin{proof}
    \pf\ This contradicts Proposition \ref{prop:continuous:converge}.
  \end{proof}
\end{proof}
