\chapter{Functions Between Topological Spaces}

\section{Open Maps}

\begin{df}
  Let $X$ and $Y$ be topological spaces and $f : X \rightarrow Y$. Then $f$
  is
  an \emph{open map} iff, for all $U$ open in $X$, $f(U)$ is open in $Y$.
\end{df}

\begin{lm}
  \label{lm:topology:open_map:basis}
  Let $X$ and $Y$ be topological spaces and $f : X \rightarrow Y$. Let
  $\mathcal{B}$ be a basis for the topology on $X$. Then $f$ is an open map
  if
  and only if, for all $B \in \mathcal{B}$, $f(B)$ is open in $Y$.
\end{lm}

\begin{proof}
  \pf
  \step{<1>1}{If $f$ is an open map then, for all $B \in \mathcal{B}$, $f(B)$
    is
    open in $Y$.}
  \begin{proof}
    \pf\ Immediate from definitions.
  \end{proof}
  \step{<1>2}{If, for all $B \in \mathcal{B}$, $f(B)$ is open in $Y$, then
    $f$ is
    an open map.}
  \begin{proof}
    \step{<2>1}{\assume{For all $B \in \mathcal{B}$, $f(B)$ is open in $Y$.}}
    \step{<2>2}{\pflet{$U$ be open in $X$} \prove{$f(U)$ is open in $Y$}}
    \step{<2>3}{\pflet{$\mathcal{B}_0 \subseteq \mathcal{B}$ be such that $U
        =
        \bigcup \mathcal{B}_0$}}
    \step{<2>4}{$f(U) = \bigcup_{B \in \mathcal{B}_0} f(B)$}
    \begin{proof}
      \pf\ Set theory.
    \end{proof}
    \step{<2>5}{$f(U)$ is open in $Y$.}
    \begin{proof}
      \pf\ From \stepref{<2>1}, \stepref{<2>4} and the fact that the open
      sets
      are closed under union.
    \end{proof}
  \end{proof}
  \qed
\end{proof}

\begin{cor}
  \label{cor:topology:open_map:subbasis}
  Let $X$ and $Y$ be topological spaces and $f : X \rightarrow Y$. Let
  $\mathcal{S}$ be a subbasis for the topology on $X$. Then $f$ is an open
  map
  if
  and only if, for all $S \in \mathcal{S}$, $f(S)$ is open in $Y$.
\end{cor}

\begin{lm}[AC]
  Let $\{X_\alpha\}_{\alpha \in J}$ be a family of topological spaces. Then
  the projection $\pi_\alpha : \prod_{\alpha \in J} X_\alpha \rightarrow
  X_\alpha$ is an open map.
\end{lm}

\begin{proof}
  \pf
  \step{<1>1}{For $U$ open in $X_\alpha$, we have
    $\pi_\alpha(\pi_\alpha^{-1}(U))$ is open in $X_\alpha$}
  \begin{proof}
    \pf\ $\pi_\alpha(\pi_\alpha^{-1}(U)) = U$ if all the other $X_\alpha$ are
    nonempty, $\emptyset$ otherwise.
  \end{proof}
  \step{<1>2}{For $\beta \neq \alpha$ and $U$ open in $X_\beta$, we have
    $\pi_\alpha(\pi_\beta^{-1}(U))$ is open in $X_\alpha$}
  \begin{proof}
    \pf\ $\pi_\alpha(\pi_\beta^{-1}(U)) = X_\alpha$ if all the $X_\gamma$ are
    nonempty for $\gamma \neq \alpha$, $\emptyset$ otherwise.
  \end{proof}
  \qedstep
  \begin{proof}
    \pf\ By Corollary \ref{cor:topology:open_map:subbasis}.
  \end{proof}
\end{proof}

\section{Continuous Functions}

\begin{df}[Continuous]
  Let $X$ and $Y$ be topological spaces and $f : X \rightarrow Y$ a function.
  Then $f$ is \emph{continuous} if and only if, for every open set $U$ in
  $Y$,
  the set $f^{-1}(U)$ is open in $X$.
\end{df}

\begin{thm}
  \label{thm:topology:continuous:characterisation}
  Let $X$ and $Y$ be topological spaces and $f : X \rightarrow Y$. Then the
  following are equivalent.
  \begin{enumerate}
    \item $f$ is continuous.
    \item For every closed set $C$ in $Y$, the set $f^{-1}(C)$ is closed in
    $X$.
    \item For every set $A \subseteq X$, we have $f(\overline{A}) \subseteq
    \overline{f(A)}$.
  \end{enumerate}
\end{thm}

\begin{proof}
  \pf
  \step{<1>1}{$1 \Rightarrow 3$}
  \begin{proof}
    \step{<2>1}{\assume{$f$ is continuous.}}
    \step{<2>2}{\pflet{$A \subseteq X$}}
    \step{<2>3}{\pflet{$x \in \overline{A}$} \prove{$f(x) \in
        \overline{f(A)}$}}
    \step{<2>4}{\pflet{$V$ be a neighbourhood of $f(x)$}}
    \step{<2>5}{$f^{-1}(V)$ is a neighbourhood of $x$}
    \begin{proof}
      \pf\ \stepref{<2>1}, \stepref{<2>3}, \stepref{<2>4}
    \end{proof}
    \step{<2>6}{$f^{-1}(V)$ intersects $A$ in $a$, say.}
    \begin{proof}
      \pf\ \stepref{<2>3}, \stepref{<2>5}, Theorem
      \ref{thm:topology:closure:neighbourhoods}.
    \end{proof}
    \step{<2>7}{$V$ intersects $f(A)$ in $f(a)$.}
    \qedstep
    \begin{proof}
      \pf\ Theorem \ref{thm:topology:closure:neighbourhoods}.
    \end{proof}
  \end{proof}
  \step{<1>2}{$3 \Rightarrow 2$}
  \begin{proof}
    \step{<2>1}{\assume{3}}
    \step{<2>2}{\pflet{$C$ be a closed set in $Y$}}
    \step{<2>3}{$\overline{f^{-1}(C)} = f^{-1}(C)$}
    \begin{proof}
      \pf
      \begin{align*}
        f(\overline{f^{-1}(C)}) & \subseteq \overline{f(f^{-1}(C))} &
        (\text{\stepref{<2>1}}) \\
        & \subseteq \overline{C}
      \end{align*}
    \end{proof}
  \end{proof}
  \step{<1>3}{$2 \Rightarrow 1$}
  \begin{proof}
    \step{<2>1}{\assume{2}}
    \step{<2>2}{\pflet{$V$ be open in $Y$}}
    \step{<2>3}{$f^{-1}(Y \setminus V)$ is closed in $X$}
    \begin{proof}
      \pf\ By \stepref{<2>1}.
    \end{proof}
    \step{<2>4}{$f^{-1}(V)$ is open in $X$.}
    \begin{proof}
      \pf\ $f^{-1}(V) = X \setminus f^{-1}(Y \setminus V)$.
    \end{proof}
  \end{proof}
  \qed
\end{proof}

\begin{lm}
  \label{lm:topology:continuous:constant}
  If $f : X \rightarrow Y$ maps all of $X$ to the single point $y_0$ of $Y$,
  then $f$ is continuous.
\end{lm}

\begin{proof}
  \pf\ For $V$ open in $Y$, the set $f^{-1}(V)$ is either $X$ (if $y_0 \in
  V$) or $\emptyset$ (if $y_0 \notin V$).
\end{proof}

\begin{df}[Continuity at a Point]
  Let $X$ and $Y$ be topological spaces, $f : X \rightarrow Y$ a function,
  and
  $x \in X$. Then $f$ is \emph{continuous at $x$} if and only if, for every
  neighbourhood $V$ of $f(x)$, there exists a neighbourhood $U$ of $x$ such
  that
  $f(U) \subseteq V$.
\end{df}

\begin{thm}
  \label{thm:topology:continuous:at_every_point}
  Let $X$ and $Y$ be topological spaces and $f : X \rightarrow Y$. Then $f$
  is
  continuous if and only if $f$ is continuous at every point of $X$.
\end{thm}

\begin{proof}
  \pf
  \step{<1>1}{If $f$ is continuous then $f$ is continuous at every point of
    $X$.}
  \begin{proof}
    \step{<2>1}{\assume{$f$ is continuous}}
    \step{<2>2}{\pflet{$x \in X$}}
    \step{<2>3}{\pflet{$V$ be a neighbourhood of $f(x)$}}
    \step{<2>4}{$f^{-1}(V)$ is a neighbourhood of $x$}
    \step{<2>5}{$f(f^{-1}(V)) \subseteq V$}
  \end{proof}
  \step{<1>2}{If $f$ is continuous at every point of $X$ then $f$ is
    continuous.}
  \begin{proof}
    \step{<2>1}{\assume{$f$ is continuous at every point of $X$.}}
    \step{<2>2}{\pflet{$V$ be open in $Y$} \prove{$f^{-1}(V)$ is open in
        $X$.}}
    \step{<2>3}{\pflet{$x \in f^{-1}(V)$}}
    \step{<2>4}{$V$ is a neighbourhood of $f(x)$}
    \step{<2>5}{\pick\ a neighbourhood $U$ of $x$ such that $f(U) \subseteq
      V$}
    \begin{proof}
      \pf\ By \stepref{<2>1}.
    \end{proof}
    \step{<2>6}{$x \in U \subseteq f^{-1}(V)$}
    \qedstep
    \begin{proof}
      \pf\ By Proposition \ref{prop:topology:neighbourhood:open}.
    \end{proof}
  \end{proof}
  \qed
\end{proof}

\begin{lm}
  \label{lm:topology:continuous:basis}
  Let $X$ and $Y$ be topological spaces and $f : X \rightarrow Y$. Let
  $\mathcal{B}$ be a basis for the topology on $Y$. Then $f$ is continuous if
  and only if, for all $B \in \mathcal{B}$, the set $f^{-1}(B)$ is open in
  $X$.
\end{lm}

\begin{proof}
  \pf
  \step{<1>1}{If $f$ is continuous then, for all $B \in \mathcal{B}$, the set
    $f^{-1}(B)$ is open in $X$.}
  \begin{proof}
    \pf\ Immediate from definitions.
  \end{proof}
  \step{<1>2}{If, for all $B \in \mathcal{B}$, the set $f^{-1}(B)$ is open in
    $X$,
    then $f$ is continuous.}
  \begin{proof}
    \step{<2>1}{\assume{For all $B \in \mathcal{B}$, the set $f^{-1}(B)$ is
        open
        in $X$.}}
    \step{<2>2}{\pflet{$x \in X$}}
    \step{<2>3}{\pflet{$V$ be a neighbourhood of $f(x)$}}
    \step{<2>4}{\pick\ $B \in \mathcal{B}$ such that $f(x) \in B \subseteq V$}
    \step{<2>5}{$f^{-1}(B)$ is a neighbourhood of $x$}
    \begin{proof}
      \pf\ By \stepref{<2>1}.
    \end{proof}
    \step{<2>6}{$f(f^{-1}(B)) \subseteq B$}
    \begin{proof}
      \pf\ Set theory.
    \end{proof}
    \qedstep
    \begin{proof}
      \pf\ Theorem \ref{thm:topology:continuous:at_every_point}.
    \end{proof}
  \end{proof}
  \qed
\end{proof}

\begin{lm}
  \label{lm:topology:continuous:projections}
  The projections $\pi_1 : X \times Y \rightarrow X$ and $\pi_2 : X \times Y
  \rightarrow Y$ are continuous.
\end{lm}

\begin{proof}
  \pf Immediate from definitions. \qed
\end{proof}

\begin{thm}
  If $A$ is a subspace of $X$, the inclusion function $j : A \rightarrow
  X$
  is
  continuous.
\end{thm}

\begin{proof}
  \pf\ For $V$ open in $X$, the set $j^{-1}(V) = V \cap A$ is open in $A$.
\end{proof}

\begin{thm}
  \label{thm:topology:continuous:composite}
  If $f : X \rightarrow Y$ and $g : Y \rightarrow Z$ are continuous,
  then
  the map
  $g \circ f : X \rightarrow Z$ is continuous.
\end{thm}

\begin{proof}
  \pf
  \step{<1>1}{\pflet{$V$ be open in $Z$}}
  \step{<1>2}{$g^{-1}(V)$ is open in $Y$}
  \step{<1>3}{$f^{-1}(g^{-1}(V))$ is open in $X$}
  \qed
\end{proof}

\begin{prop}
  \label{continuous:restriction}
  If $f : X \rightarrow Y$ is continuous and if $A$ is a subspace of
  $X$,
  then
  the restricted function $f \restriction A : A \rightarrow Y$ is
  continuous.
\end{prop}

\begin{proof}
  \pf\ For $V$ open in $Y$, the set $(f \restriction A)^{-1}(V) = f^{-1}(V)
  \cap A$ is open in $A$. \qed
\end{proof}

\begin{thm}
  Let $f : X \rightarrow Y$ be continuous. If $Z$ is a subspace of $Y$
  that
  includes the range of $f$, then the function $g : X \rightarrow Z$
  obtained by
  restricting the codomain of $f$ is continuous. If $Z$ is a space having
  $Y$ as
  a subspace, then the function $h : X \rightarrow Z$ obtained by expanding
  the
  codomain of $f$ is continuous.
\end{thm}

\begin{proof}
  \pf
  \step{<1>1}{If $Z$ is a subspace of $Y$ that
    includes the range of $f$, then the function $g : X \rightarrow Z$
    obtained by
    restricting the codomain of $f$ is continuous.}
  \begin{proof}
    \step{<2>1}{\pflet{$V$ be open in $Z$}}
    \step{<2>2}{\pick\ $W$ open in $Y$ such that $V = W \cap Z$}
    \step{<2>3}{$f^{-1}(W)$ is open in $X$.}
    \step{<2>4}{$g^{-1}(V)$ is open in $X$.}
    \begin{proof}
      \pf\ $g^{-1}(V) = f^{-1}(W)$.
    \end{proof}
  \end{proof}
  \step{<1>2}{If $Z$ is a space having $Y$ as
    a subspace, then the function $h : X \rightarrow Z$ obtained by
    expanding the
    codomain of $f$ is continuous.}
  \begin{proof}
    \pf\ For $V$ open in $Z$, we have $h^{-1}(V) = f^{-1}(V \cap Y)$ is
    open
    in
    $X$.
  \end{proof}
  \qed
\end{proof}

\begin{thm}
  \label{thm:topology:continuous:convergence}
  Let $X$ and $Y$ be topological spaces and $f : X \rightarrow Y$.
  If $x_n \rightarrow x$ as $n \rightarrow \infty$ in $X$ and $f$ is
  continuous
  at $x$, then $f(x_n)
  \rightarrow f(x)$ as $n \rightarrow \infty$ in $Y$.
\end{thm}

\begin{proof}
  \pf
  \step{<1>1}{\assume{$x_n \rightarrow x$ as $n \rightarrow \infty$}}
  \step{<1>2}{\assume{$f$ is continuous at $x$}}
  \step{<1>3}{\pflet{$V$ be a neighbourhood of $f(x)$}}
  \step{<1>4}{\pick\ a neighbourhood $U$ of $x$ such that $f(U) \subseteq V$}
  \begin{proof}
    \pf\ By \stepref{<1>2}.
  \end{proof}
  \step{<1>5}{\pick\ $N$ such that, for all $n \geq N$, $x_n \in U$}
  \begin{proof}
    \pf\ By \stepref{<1>1}
  \end{proof}
  \step{<1>6}{For $n \geq N$, $f(x_n) \in V$}
  \begin{proof}
    \pf\ By \stepref{<1>4}.
  \end{proof}
  \qed
\end{proof}

\begin{cor}
  \label{cor:topology:continuous:product_converge}
  Let $\{ X_\alpha \}_{\alpha \in J}$ be a family of topological spaces and $(x_n)$ a family of points in $\prod_{\alpha \in J} X_\alpha$. We have $x_n \rightarrow l$ as $n \rightarrow \infty$ if and only if, for all $\alpha \in J$, $\pi_\alpha(x_n) \rightarrow \pi_\alpha(l)$ as $n \rightarrow \infty$.
\end{cor}

\begin{proof}
  \pf
  \step{<1>1}{If $x_n \rightarrow l$ as $n \rightarrow \infty$ then, for all $\alpha \in J$, $\pi_\alpha(x_n) \rightarrow \pi_\alpha(l)$ as $n \rightarrow \infty$}
  \begin{proof}
    \pf\ Theorem \ref{thm:topology:continuous:convergence} and Proposition \ref{lm:topology:continuous:projections}.
  \end{proof}
  \step{<1>2}{If, for all $\alpha \in J$,we have  $\pi_\alpha(x_n) \rightarrow \pi_\alpha(l)$ as $n \rightarrow \infty$, then $x_n \rightarrow l$ as $n \rightarrow \infty$}
  \begin{proof}
    \step{<2>1}{\assume{For all $\alpha \in J$,we have  $\pi_\alpha(x_n) \rightarrow \pi_\alpha(l)$ as $n \rightarrow \infty$}}
    \step{<2>2}{\pflet{$B = \prod_{\alpha \in J} U_\alpha$ be a basic open neighbourhood of $l$, where $U_\alpha = X_\alpha$ except for $\alpha = \alpha_1, \ldots, \alpha_k$}}
    \step{<2>3}{\pick\ $N$ such that, for all $n \geq N$ and $1 \leq i \leq k$, we have $\pi_i(x_n) \in U_{\alpha_i}$}
    \step{<2>4}{For $n \geq N$ we have $x_n \in B$}
  \end{proof}
  \qed
\end{proof}

\begin{thm}
  \label{thm:topology:continuous:local}
  Let $X$ and $Y$ be topological spaces.
  Let $f : X \rightarrow Y$. If there exists a set $\mathcal{A}$ of open
  sets in
  $X$ such that:
  \begin{itemize}
    \item $\bigcup \mathcal{A} = X$;
    \item for all $U \in \mathcal{A}$, the function $f \restriction U : U
    \rightarrow X$ is continuous;
  \end{itemize}
  then $f$ is continuous.
\end{thm}

\begin{proof}
  \pf
  \step{<1>1}{\pflet{$V$ be open in $Y$}}
  \step{<1>2}{For all $U \in \mathcal{A}$, the set $(f \restriction
    U)^{-1}(V)$
    is open in $X$.}
  \begin{proof}
    \step{<2>1}{\pflet{$U \in \mathcal{A}$}}
    \step{<2>2}{$(f \restriction U)^{-1}(V)$ is open in $U$}
    \begin{proof}
      \pf\ Since $f \restriction U : U \rightarrow X$ is continuous.
    \end{proof}
    \qedstep
    \begin{proof}
      \pf\ By Lemma \ref{lm:topology:subspace:open}.
    \end{proof}
  \end{proof}
  \qedstep
  \begin{proof}
    \pf\ Since $f^{-1}(V) = \bigcup_{U \in \mathcal{A}} (f \restriction
    U)^{-1}(V)$.
  \end{proof}
\end{proof}

\begin{thm}[The Pasting Lemma]
  Let $X = A \cup B$ where $A$ and $B$ are closed in $X$. Let $f : A
  \rightarrow
  Y$ and $g : B \rightarrow Y$ be continuous. If $f(x) = g(x)$ for every $x
  \in A
  \cap B$, then the function $h : X \rightarrow Y$ defined by
  \[ h(x) = \begin{cases}
    f(x) & \text{if } x \in A \\
    g(x) & \text{if } x \in B
  \end{cases} \]
  is continuous.
\end{thm}

\begin{proof}
  \pf
  \step{<1>1}{\pflet{$C$ be closed in $Y$}}
  \step{<1>2}{$f^{-1}(C)$ is closed in $A$}
  \begin{proof}
    \pf\ Theorem \ref{thm:topology:continuous:characterisation}.
  \end{proof}
  \step{<1>3}{$f^{-1}(C)$ is closed in $X$}
  \begin{proof}
    \pf\ Lemma \ref{cor:topology:subspace:closed}.
  \end{proof}
  \step{<1>4}{$g^{-1}(C)$ is closed in $B$}
  \begin{proof}
    \pf\ Theorem \ref{thm:topology:continuous:characterisation}.
  \end{proof}
  \step{<1>5}{$g^{-1}(C)$ is closed in $X$}
  \begin{proof}
    \pf\ Lemma \ref{cor:topology:subspace:closed}.
  \end{proof}
  \step{<1>6}{$h^{-1}(C)$ is closed in $X$}
  \begin{proof}
    \pf\ $h^{-1}(C) = f^{-1}(C) \cup g^{-1}(C)$
  \end{proof}
  \qedstep
  \begin{proof}
    \pf\ Theorem \ref{thm:topology:continuous:characterisation}.
  \end{proof}
  \qed
\end{proof}

\begin{thm}
  \label{thm:topology:continuous:product}
  Let $f : A \rightarrow \prod_{\alpha \in J} X_\alpha$ be given by the
  equation
  \[ f(a) = \{ f_\alpha(a) \}_{\alpha \in J} \enspace , \]
  where $f_\alpha : A \rightarrow X_\alpha$ for each $\alpha$. Let
  $\prod_{\alpha
    \in J} X_\alpha$ have the product topology. Then the function $f$ is
  continuous if and only if each function $f_\alpha$ is continuous.
\end{thm}

\begin{proof}
  \pf
  \step{<1>1}{If $f$ is continuous then each $f_\alpha$ is continuous.}
  \begin{proof}
    \pf\ This holds because $f_\alpha = \pi_\alpha \circ f$.
  \end{proof}
  \step{<1>2}{If every $f_\alpha$ is continuous then $f$ is continuous.}
  \begin{proof}
    \step{<2>1}{\assume{Every $f_\alpha$ is continuous.}}
    \step{<2>2}{\pflet{$\alpha \in J$ and $U$ be open in $X_\alpha$}}
    \step{<2>3}{$f^{-1}(\pi_\alpha^{-1}(U))$ is open in $A$}
    \begin{proof}
      \pf\ $f^{-1}(\pi_\alpha^{-1}(U)) = f_\alpha^{-1}(U)$.
    \end{proof}
  \end{proof}
  \qed
\end{proof}



\subsection{Homeomorphisms}

\begin{df}[Homeomorphism]
  Let $X$ and $Y$ be topological spaces and $f : X \rightarrow Y$. Then $f$
  is
  a
  \emph{homeomorphism} between $X$ and $Y$ iff $f$ is a bijection, and $f$
  and
  $f^{-1}$ are both continuous.
\end{df}

\begin{df}[Topological Property]
  A property $P$ of topological spaces is a \emph{topological property} iff,
  for any spaces $X$ and $Y$, if $X$ is homeomorphic to $Y$ then $P$ holds of
  $X$
  if and only if $P$ holds of $Y$.
\end{df}

\begin{df}[(Topological) Imbedding]
  Let $X$ and $Y$ be topological spaces and $f : X \rightarrow Y$. Then $f$
  is
  a
  \emph{(topological) imbedding} iff $f$ is a homeomorphism between $X$ and
  $\im f$.
\end{df}

\begin{df}[Homogeneous]
  A topological space $X$ is \emph{homogeneous} iff, for all $x, y \in X$,
  there exists a homeomorphism $f : X \cong X$ such that $f(x) = y$.
\end{df}


\subsection{Strongly Continuous Functions}

\begin{df}[Strongly Continuous]
  Let $X$ and $Y$ be topological spaces and $f : X \rightarrow Y$. Then $f$
  is
  \emph{strongly continuous} iff, for all $V \subseteq Y$, we have $V$ is
  open
  in $Y$ if and only if $f^{-1}(V)$ is open in $X$.
\end{df}

\begin{prop}
  \label{prop:topology:strongly_continuous:closed}
  Let $X$ and $Y$ be topological spaces and $f : X \rightarrow Y$. Then $f$
  is
  strongly continuous if and only if, for all $C \subseteq Y$, $C$ is closed
  in
  $Y$ if and only if $f^{-1}(C)$ is closed in $X$.
\end{prop}

\begin{proof}
  \pf
  \step{<1>1}{If $f$ is strongly continuous then, for all $C \subseteq Y$, we
    have $C$ is closed in $Y$ if and only if $f^{-1}(C)$ is closed in $X$.}
  \begin{proof}
    \pf
    \begin{align*}
      C \text{ is closed in } Y & \Leftrightarrow Y \setminus C \text{ is
        open in } Y \\
      & \Leftrightarrow f^{-1}(Y \setminus C) \text{ is open in } X \\
      & \Leftrightarrow X \setminus f^{-1}(C) \text{ is open in } X \\
      & \Leftrightarrow f^{-1}(C) \text{ is closed in } X
    \end{align*}
  \end{proof}
  \step{<1>2}{If, for all $C \subseteq Y$, we have $C$ is closed in $Y$ if
    and
    only if $f^{-1}(C)$ is closed in $X$, then $f$ is strongly continuous.}
  \begin{proof}
    \pf\ Similar.
  \end{proof}
  \qed
\end{proof}

\begin{prop}
  \label{prop:topology:strongly_continuous:composite}
  The composite of two strongly continuous functions is strongly continuous.
\end{prop}

\begin{proof}
  \pf
  \step{<1>1}{\pflet{$f : X \rightarrow Y$ and $g : Y \rightarrow Z$ be
      strongly
      continuous.}}
  \step{<1>2}{\pflet{$V \subseteq Z$}}
  \step{<1>3}{$V$ is open iff $f^{-1}(g^{-1}(V))$ is open}
  \begin{proof}
    \pf
    \begin{align*}
      V \text{ is open} & \Leftrightarrow g^{-1}(V) \text{ is open} &
      (\text{\stepref{<1>1}}) \\
      & \Leftrightarrow f^{-1}(g^{-1}(V)) \text{ is open} &
      (\text{\stepref{<1>1}})
    \end{align*}
  \end{proof}
  \qed
\end{proof}

\begin{prop}
  Let $X$, $Y$ and $Z$ be topological spaces.
  Let $f : X \rightarrow Y$ and $g : Y \rightarrow Z$. If $f$ is strongly
  continuous and $g \circ f$ is continuous, then $g$ is continuous.
\end{prop}

\begin{proof}
  \pf
  \step{<1>1}{\pflet{$V \subseteq Z$ be open in $Z$.}}
  \step{<1>2}{$f^{-1}(g^{-1}(V))$ is open in $X$.}
  \begin{proof}
    \pf\ $g \circ f$ is continuous.
  \end{proof}
  \step{<1>3}{$g^{-1}(V)$ is open in $Y$.}
  \begin{proof}
    \pf\ $f$ is strongly continuous.
  \end{proof}
  \qed
\end{proof}

\begin{prop}
  Let $X$, $Y$ and $Z$ be topological spaces. Let $f : X \rightarrow Y$ and
  $g
  : Y \rightarrow Z$. If $f$ and $g \circ f$ are strongly continuous, then
  $g$
  is
  strongly continuous.
\end{prop}

\begin{proof}
  \pf
  \step{<1>1}{\pflet{$U \subseteq Z$}}
  \step{<1>2}{$U$ is open in $Z$ iff $g^{-1}(U)$ is open in $Y$}
  \begin{proof}
    \pf
    \begin{align*}
      U \text{ is open in } Z & \Leftrightarrow f^{-1}(g^{-1}(U)) \text{ is
        open in } X & (g \circ f \text{ is strongly continuous}) \\
      & \Leftrightarrow g^{-1}(U) \text{ is open in } Y & (f \text{ is
        strongly continuous})
    \end{align*}
  \end{proof}
  \qed
\end{proof}

\section{Closed Maps}

\begin{df}[Closed Map]
  Let $X$ and $Y$ be topological spaces and $f : X \rightarrow Y$. Then $f$
  is
  a \emph{closed map} iff, for every closed set $C \subseteq X$, the
  set $f(C)$ is closed in $Y$.
\end{df}

\begin{lm}
  \label{lm:topology:closed_map:open}
  Let $p : X \rightarrow Y$ be a closed map. Let $B \subseteq Y$. Let $U$ be an open neighbourhood of $\inv{p}(B)$. Then there exists an open neighbourhood $V$ of $B$ such that $\inv{p}(V) \subseteq U$.
\end{lm}

\begin{proof}
  \pf
  \step{<1>1}{\pflet{$V = Y \setminus p(X \setminus U)$}}
  \step{<1>2}{$V$ is open}
  \step{<1>3}{$\inv{p}(V) \subseteq U$}
  \qed
\end{proof}

\section{Local Homeomorphism}

\begin{df}[Locally Homeomorphic]
  Let $X$ and $Y$ be topological spaces. Then $X$ is \emph{locally
    homeomorphic} to $Y$ iff every point in $X$ has an open neighborhood that
  is homeomorphic with an open set in $Y$.
\end{df}

\begin{prop}
  The long line is locally homeomorphic with $\mathbb{R}$.
\end{prop}

\begin{proof}
  \pf
  \step{<1>1}{\pflet{$x \in L$}}
  \step{<1>2}{\pick\ an ordinal $\alpha$ such that $x < (\alpha, 0)$.}
  \step{<1>3}{$(- \infty, (\alpha, 0))$ is an open neighbourhood of $x$ that
    is
    homeomorphic to $(0, 1)$.}
  \qed
\end{proof}

\section{Retracts}

 \begin{df}[Retract]
Let $Z$ be a topological space. If $Y$ is a subspace of $Z$, we say that $Y$
is a \emph{retract} of $Z$ iff there exists a continuous function $r : Z
\rightarrow Y$ such that $r(y) = y$ for all $y \in Y$.
\end{df}
