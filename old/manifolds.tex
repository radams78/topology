\chapter{Manifolds}

\section{Manifolds}

 \begin{df}[Manifold]
  Let $m \geq 1$. An \emph{$m$-manifold} is a second countable Hausdorff space
such that each point has a neighbourhood that is homeomorphic to an open subset
of $\mathbb{R}^m$.

A \emph{curve} is a 1-manifold and a \emph{surface} is a 2-manifold.
\end{df}

 \begin{thm}[Existence of Finite Partitions of Unity]
   \label{thm:topology:manifolds:partition}
  Let $X$ be a normal space.
  Let $\{ U_1, \ldots, U_n \}$ be a finite indexed open covering of $X$. Then
  there exists a partition of unity dominated by $\{ U_1, \ldots, U_n \}$.
\end{thm}

\begin{proof}
 \pf
 \step{<1>1}{For every finite indexed open covering $\{ U_1, \ldots, U_n \}$
of
   $X$, there exists a finite indexed open covering $\{ V_1, \ldots, V_n \}$
   such that $\overline{V_i} \subseteq U_i$}
 \begin{proof}
   \step{<2>1}{For $1 \leq k \leq n$, there exist open sets $V_1$, \ldots,
$V_k$
     such that $\overline{V_i} \subseteq U_i$ for all $i$ and $\{ V_1, \ldots,
     V_k, U_{k+1}, \ldots, U_n \}$ covers $X$}
   \begin{proof}
     \step{<3>1}{\assume{as an induction hypothesis that $0 leq k < k$ and
there
         exist open sets $V_1$, \ldots, $V_k$ such that $\overline{V_i}
         \subseteq U_i$ for all $i$  and $\{ V_1, \ldots, V_k, U_{k+1},
         \ldots, U_n \}$ covers $X$}}
     \step{<3>2}{\pflet{$A = X \setminus (V_1 \cup \cdots \cup V_k) \setminus
         (U_{k+2} \cup \cdots \cup U_n)$}}
     \step{<3>3}{$A$ is closed}
     \step{<3>4}{$A \subseteq U_{k+1}$}
     \begin{proof}
       \pf\ Since $\{ V_1, \ldots, V_k, U_{k+1}, \ldots, U_n \}$ covers $X$
     \end{proof}
     \step{<3>5}{\pick\ an open set $V_{k+1}$ such that $A \subseteq V_{k+1}$
and
       $\overline{V_{k+1}} \subseteq U_{k+1}$}
     \begin{proof}
       \pf\ By Proposition \ref{prop:topology:regular:closure}
     \end{proof}
     \step{<3>6}{$\{ V_1, \ldots, V_k, V_{k+1}, U_{k+2}, \ldots, U_n \}$
covers
       $X$}
   \end{proof}
 \end{proof}
 \step{<1>2}{\pick\ an open covering $\{ V_1, \ldots, V_n \}$ with
   $\overline{V_i} \subseteq U_i$ for all $i$}
 \begin{proof}
   \pf\ By \stepref{<1>1}.
 \end{proof}
 \step{<1>3}{\pick\ an open covering $\{ W_1, \ldots, W_n \}$ with
   $\overline{W_i} \subseteq V_i$ for all $i$}
 \begin{proof}
   \pf\ By \stepref{<1>1}.
 \end{proof}
 \step{<1>4}{For $1 \leq i \leq n$, \pick\ a continuous function $\psi_i : X
   \rightarrow [0,1]$ such that $\psi_i(\overline{W_i}) = \{ 1 \}$ and
   $\psi_i(X \setminus V_i) = \{ 0 \}$}
 \begin{proof}
   \pf\ By the Urysohn Lemma.
 \end{proof}
 \step{<1>5}{\pflet{$\Psi : X \rightarrow \mathbb{R}$ where $\Psi(x) =
     \sum_{i=1}^n \psi_i(x)$}}
 \step{<1>6}{$\Psi(x) > 0$ for all $x \in X$}
 \begin{proof}
   \step{<2>1}{\pflet{$x \in X$}}
   \step{<2>2}{\pick\ $i$ such that $x \in W_i$}
   \step{<2>3}{$\psi_i(x) = 1$}
 \end{proof}
 \step{<1>7}{For $1 \leq j \leq n$, \pflet{$\phi_j(x) =
     \frac{\psi_j(x)}{\Psi(x)}$}}
 \step{<1>8}{$\psi_1$, \ldots, $\psi_n$ are a partition of unity dominated by
   $\{ U_1, \ldots, U_n \}$}
 \begin{proof}
   \step{<2>1}{$\supp \psi_i \subseteq U_i$}
   \begin{proof}
     \step{<3>1}{$\inv{\psi_i}(\mathbb{R} \setminus \{ 0 \}) \subseteq V_i$}
     \begin{proof}
       \pf\ By \stepref{<1>4}
     \end{proof}
     \step{<3>2}{$\supp \psi_i \subseteq \overline{V_i}$}
     \begin{proof}
       \pf\ Proposition \ref{prop:topology:closure:monotone}
     \end{proof}
   \end{proof}
   \step{<2>2}{$\sum_{i=1}^n \psi_i(x) = 1$ for all $x \in X$}
 \end{proof}
 \qed
\end{proof}

\begin{thm}
  \label{thm:topology:manifolds:compact_Hausdorff}
 Let $X$ be a compact Hausdorff space. Suppose that, for every $x \in X$,
there exists a neighbourhood $U$ of $x$ and a positive integer $k$ such that
$U$ can be imbedded in $\mathbb{R}^k$. Then there exists a positive integer $N$
such that $X$ can be imbedded in $\mathbb{R}^N$.
\end{thm}

 \begin{proof}
 \pf
 \step{<1>1}{\pick\ a finite open covering $\{ U_1, \ldots, U_n \}$ of $X$
such
   that each $U_i$ can be imbedded in $\mathbb{R}^k$ for some $k$}
 \begin{proof}
   \pf\ Since $\{ U \text{ open in } X : U \text{ can be imbedded in }
   \mathbb{R}^k \text{ for some } k \}$ covers $X$.
 \end{proof}
 \step{<1>2}{For $1 \leq i \leq n$, \pick\ a positive integer $k_i$ and an
imbedding $g_i : U_i \rightarrow
\mathbb{R}^{k_i}$}
 \step{<1>3}{\pick\ a partition of unity $\phi_1$, \ldots, $\phi_n$ dominated
by
   $\{ U_1, \ldots, U_n \}$}
 \begin{proof}
   \step{<2>1}{$X$ is normal}
   \begin{proof}
     \pf\ By Lemma \ref{lm:topology:compact:normal}.
   \end{proof}
   \qedstep
   \begin{proof}
     \pf\ Theorem \ref{thm:topology:manifolds:partition}
   \end{proof}
 \end{proof}
 \step{<1>4}{For $1 \leq i \leq n$, \pflet{$A_i = \supp \phi_i$}}
 \step{<1>5}{For $1 \leq i \leq n$, \pflet{$h_i : X \rightarrow
     \mathbb{R}^{k_i}$
be
defined by
\[ h_i(x) = \begin{cases}
 \phi_i(x) g_i(x) & \text{for } x \in U_i \\
 \vec{0} & \text{for } x \in X \setminus A_i
\end{cases} \]}}
\begin{proof}
 \pf\ If $x \in U_i$ and $x \in X \setminus A_i$ then $x \notin \supp \phi_i$
so $\phi_i(x) = 0$
\end{proof}
\step{<1>6}{\pflet{$N = n + k_1 + \cdots + k_n$}}
\step{<1>7}{\pflet{$F : X \rightarrow \mathbb{R}^N$ be the function
   \[ F(x) = (\phi_1(x), \ldots, \phi_n(x), h_1(x), \ldots, h_n(x)) \]}}
\step{<1>8}{$F$ is an imbedding}
\begin{proof}
 \step{<2>1}{$F$ is continuous}
 \begin{proof}
   \pf\ Each $h_i$ is continuous by Theorem
   \ref{thm:topology:continuous:local}.
 \end{proof}
 \step{<2>2}{$F$ is injective}
 \begin{proof}
   \step{<3>1}{\assume{$F(x) = F(y)$}}
   \step{<3>2}{\pick\ $i$ such that $\phi_i(x) > 0$}
   \begin{proof}
     \pf\ Since $\sum_i \phi_i(x) = 1$ (\stepref{<1>3})
   \end{proof}
   \step{<3>3}{$\phi_i(y) = 0$}
   \begin{proof}
     \pf\ By \stepref{<3>1}
   \end{proof}
   \step{<3>4}{$x, y \in U_i$}
   \begin{proof}
     \pf\ Since $\supp \phi_i \subseteq U_i$
   \end{proof}
   \step{<3>5}{$h_i(x) = h_i(y)$}
   \begin{proof}
     \pf\ By \stepref{<3>1}
   \end{proof}
   \step{<3>6}{$g_i(x) = g_i(y)$}
   \begin{proof}
     \pf\ By \stepref{<1>5}
   \end{proof}
   \step{<3>7}{$x = y$}
   \begin{proof}
     \pf\ By \stepref{<1>2}
   \end{proof}
 \end{proof}
 \qedstep
 \begin{proof}
   \pf\ By Theorem \ref{thm:topology:compact:homeomorphism}
 \end{proof}
\end{proof}
\qed
\end{proof}

  \begin{cor}
  Every compact manifold can be imbedded in $\mathbb{R}^N$ for some $N$.
\end{cor}

\begin{prop}
 The line with two origins is a second countable $T_1$ space where every point
 has a neighbourhood that is homeomorphic to an open subset of $\mathbb{R}$,
but it is not a 1-manifold.
\end{prop}
