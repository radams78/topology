\chapter{Constructions of Topological Spaces}

\section{The Order Topology}

\begin{df}[Order Topology]
  Let $X$ be a linearly ordered set with more than one element. The
  \emph{order
    topology} on $X$ is the topology generated by the basis consisting of:
  \begin{itemize}
    \item all open intervals $(a, b)$
    \item all half-open intervals $(a, \top]$ where $\top$ is the greatest
    element of $X$, if there is one;
    \item all half-open intervals $[\bot, a)$ where $\bot$ is the least
    element of
    $X$, if there is one.
  \end{itemize}

  We prove this is a basis for a topology.
\end{df}

\begin{proof}
  \pf
  \step{<1>1}{\pflet{$\mathcal{B}$ be the set of all sets of these three
      forms.}}
  \step{<1>2}{$\bigcup \mathcal{B} = X$}
  \begin{proof}
    \step{<2>1}{\pflet{$x \in X$} \prove{There exists $B \in \mathcal{B}$
        such
        that $x \in B$}}
    \step{<2>2}{\case{$x$ is least in $X$}}
    \begin{proof}
      \step{<3>1}{\pick\ $a \in X$ such that $a > x$}
      \begin{proof}
        \pf\ $X$ has more than one element.
      \end{proof}
      \step{<3>2}{$x \in [x, a) \in \mathcal{B}$}
    \end{proof}
    \step{<2>3}{\case{$x$ is greatest in $X$}}
    \begin{proof}
      \step{<3>1}{\pick\ $a \in X$ such that $a < x$}
      \begin{proof}
        \pf\ $X$ has more than one element.
      \end{proof}
      \step{<3>2}{$x \in (a, x] \in \mathcal{B}$}
    \end{proof}
    \step{<2>4}{\case{$x$ is neither least nor greatest in $X$}}
    \begin{proof}
      \step{<3>1}{\pick\ $a, b \in X$ such that $a < x < b$}
      \step{<3>2}{$x \in (a, b) \in \mathcal{B}$}
    \end{proof}
  \end{proof}
  \step{<1>3}{For all $B_1, B_2 \in \mathcal{B}$ and $x \in B_1 \cap B_2$,
    there
    exists $B_3 \in \mathcal{B}$ such that $x \in B_3 \subseteq B_1 \cap B_2$}
  \begin{proof}
    \step{<2>1}{\pflet{$B_1, B_2 \in \mathcal{B}$ and $x \in B_1 \cap B_2$}}
    \step{<2>2}{\case{$B_1 = (a, b), B_2 = (c, d)$}}
    \begin{proof}
      \pf\ Take $B_3 = (\max(a, c), \min(b, d))$.
    \end{proof}
    \step{<2>3}{\case{$B_1 = (a, b), B_2 = (c, \top]$}}
    \begin{proof}
      \pf\ Take $B_3 = (\max(a, c), b)$.
    \end{proof}
    \step{<2>4}{\case{$B_1 = (a, b), B_2 = [\bot, d)$}}
    \begin{proof}
      \pf\ Take $B_3 = (a, \min(b, d))$.
    \end{proof}
    \step{<2>5}{\case{$B_1 = (a, \top], B_2 = (c, d)$}}
    \begin{proof}
      \pf\ Similar to \stepref{<2>3}.
    \end{proof}
    \step{<2>6}{\case{$B_1 = (a, \top], B_2 = (c, \top]$}}
    \begin{proof}
      \pf\ Take $B_3 = (\max(a, c), \top]$.
    \end{proof}
    \step{<2>7}{\case{$B_1 = (a, \top], B_2 = [\bot, d)$}}
    \begin{proof}
      \pf\ Take $B_3 = (a, d)$.
    \end{proof}
    \step{<2>8}{\case{$B_1 = [\bot, b), B_2 = (c, d)$}}
    \begin{proof}
      \pf\ Similar to \stepref{<2>4}.
    \end{proof}
    \step{<2>9}{\case{$B_1 = [\bot, b), B_2 = (c, \top]$}}
    \begin{proof}
      \pf\ Simlar to \stepref{<2>7}.
    \end{proof}
    \step{<2>10}{\case{$B_1 = [\bot, b), B_2 = [\bot, d)$}}
    \begin{proof}
      \pf\ Take $B_3 = [\bot, \min(b, d))$.
    \end{proof}
  \end{proof}
  \qedstep
  \begin{proof}
    \pf\ By Lemma \ref{lm:topology:basis:generate}.
  \end{proof}
  \qed
\end{proof}

    \begin{lm}
  \label{lm:topology:order:open}
  Let $X$ be a linearly ordered set, $U \subseteq X$ be open, and $a \in U$.
  Then:
  \begin{enumerate}
    \item Either $a$ is greatest in $X$, or there exists $a' > a$ such that
    $[a,
    a') \subseteq U$
    \item Either $a$ is least in $X$, or there exists $a' < a$ such that
    $(a',
    a]
    \subseteq U$.
  \end{enumerate}
\end{lm}

\begin{proof}
  \pf
  \step{<1>1}{Either $a$ is greatest in $X$, or there exists $a' > a$ such
    that
    $[a, a') \subseteq U$}
  \begin{proof}
    \step{<2>1}{\assume{$a$ is not greatest in $X$}}
    \step{<2>2}{\pick\ a basic open set $B$ such that $a \in B \subseteq U$}
    \step{<2>3}{\case{$B = (a'', a')$}}
    \begin{proof}
      \pf\ $a < a'$ and $[a, a') \subseteq B \subseteq U$
    \end{proof}
    \step{<2>4}{\case{$B = [\bot, a')$}}
    \begin{proof}
      \pf\ $a < a'$ and $[a, a') \subseteq B \subseteq U$
    \end{proof}
    \step{<2>5}{\case{$B = (a'', \top]$}}
    \begin{proof}
      \pf\ Pick any $a' > a$ (one exists by \stepref{<2>1}). Then $[a, a')
      \subseteq B \subseteq U$.S
    \end{proof}
  \end{proof}
  \step{<1>2}{Either $a$ is least in $X$, or there exists $a' < a$ such that
    $(a',
    a] \subseteq U$.}
  \begin{proof}
    \pf\ Similar.
  \end{proof}
  \qed
\end{proof}

\begin{lm}
  \label{lm:topology:order:subbasis}
  The open rays form a subbasis for the order topology.
\end{lm}

\begin{proof}
  \step{<1>1}{\pflet{$X$ be a linearly ordered set with more than one
      element.}}
  \step{<1>2}{The open rays form a subbasis for a topology.}
  \begin{proof}
    \step{<2>1}{\pflet{$x \in X$} \prove{$x$ is an element of an open ray.}}
    \step{<2>2}{\case{$x$ is greatest in $X$}}
    \begin{proof}
      \step{<3>1}{\pick\ $a \in X$ such that $a < x$}
      \begin{proof}
        \pf\ $X$ has more than one element (\stepref{<1>1}).
      \end{proof}
      \step{<3>2}{$x \in (a, +\infty)$}
    \end{proof}
    \step{<2>3}{\case{$x$ is not greatest in $X$}}
    \begin{proof}
      \step{<3>1}{\pick\ $a \in X$ such that $x < a$}
      \step{<3>2}{$x \in (-\infty, a)$}
    \end{proof}
    \qedstep
    \begin{proof}
      \pf\ By Lemma \ref{lm:topology:subbasis:generate}.
    \end{proof}
  \end{proof}
  \step{<1>3}{\pflet{$\mathcal{T}_o$ be the order topology and
      $\mathcal{T}_S$
      be
      the topology generated by the open rays.}}
  \step{<1>4}{$\mathcal{T}_o \subseteq \mathcal{T}_S$}
  \begin{proof}
    \step{<2>1}{Every open interval $(a, b)$ is open in $\mathcal{T}_S$}
    \begin{proof}
      \pf\ $(a, b) = (a, + \infty) \cap (-\infty, b)$.
    \end{proof}
    \step{<2>2}{If $\top$ is greatest then $(a, \top]$ is open in
      $\mathcal{T}_S$}
    \begin{proof}
      \pf\ $(a, \top] = (a, + \infty)$.
    \end{proof}
    \step{<2>3}{If $\bot$ is least then $[\bot, b)$ is open in
      $\mathcal{T}_S$}
    \begin{proof}
      \pf\ $[\bot, b) = [\bot, + \infty)$.
    \end{proof}
    \qedstep
    \begin{proof}
      \pf\ By Corollary \ref{cor:topology:basis:coarsest}.
    \end{proof}
  \end{proof}
  \step{<1>5}{$\mathcal{T}_S \subseteq \mathcal{T}_o$}
  \begin{proof}
    \step{<2>1}{For all $a \in X$, we have $(a, +\infty)$ is open in
      $\mathcal{T}_o$}
    \begin{proof}
      \step{<3>1}{\pflet{$x \in (a, +\infty)$} \prove{There exists a basis
          element
          $B$ such that $x \in B \subseteq (a, +\infty)$}}
      \step{<3>2}{\case{$x$ is greatest}}
      \begin{proof}
        \pf\ Take $B = (a, x]$
      \end{proof}
      \step{<3>3}{\case{$x$ is not greatest}}
      \begin{proof}
        \step{<4>1}{\pick\ $b > x$}
        \step{<4>2}{$x \in (a, b) \subseteq (a, +\infty)$}
      \end{proof}
    \end{proof}
    \step{<2>2}{For all $a \in X$, we have $(- \infty, a)$ is open in
      $\mathcal{T}_o$}
    \begin{proof}
      \pf\ Similar.
    \end{proof}
    \qedstep
    \begin{proof}
      \pf\ By Corollary \ref{cor:topology:subbasis:coarsest}.
    \end{proof}
  \end{proof}
  \qed
\end{proof}

\begin{lm}
  In a linearly ordered set $X$ under the order topology, the closed
  intervals
  and closed rays are closed.
\end{lm}

\begin{proof}
  \pf
  \begin{align*}
    X \setminus [a, b] & = (-\infty, a) \cup (b, +\infty) \\
    X \setminus (-\infty, a] & = (a, +\infty) \\
    X \setminus {[}a, +\infty) & = (-\infty, a) & \qed
  \end{align*}
\end{proof}

\begin{df}[Standard Topology on $\mathbb{R}$]
  The \emph{standard topology} on $\mathbb{R}$ is the order topology.
\end{df}

\begin{lm}
  The standard topology is strictly coarser than the lower limit topology.
\end{lm}

\begin{proof}
  \pf
  \step{<1>1}{The standard topology is coarser than the lower limit topology.}
  \begin{proof}
    \step{<2>1}{For every open interval $(a, b)$ and $x \in (a, b)$, there
      exists
      a half-open interval $[c, d)$ such that $x \in [c,d) \subseteq (a, b)$}
    \begin{proof}
      \pf\ Take $[c,d) = [x, b)$.
    \end{proof}
    \qedstep
    \begin{proof}
      \pf\ By Lemma \ref{lm:topology:basis:finer}.
    \end{proof}
  \end{proof}
  \step{<1>2}{There exists a set $U$ open in the lower limit topology that is
    not
    open in the standard topology.}
  \begin{proof}
    \pf\ Take $U = [0,1)$.
  \end{proof}
  \qed
\end{proof}

\begin{lm}
  The standard topology is strictly coarser than the $K$-topology.
\end{lm}

\begin{proof}
  \pf
  \step{<1>1}{The standard topology is coarser than the $K$-topology.}
  \begin{proof}
    \pf\ Every open interval is open in the $K$-topology.
  \end{proof}
  \step{<1>2}{There exists a set $U$ open in the $K$-topology that is
    not
    open in the standard topology.}
  \begin{proof}
    \pf\ Take $U = (-1, 1) \setminus K$. Then $0 \in U$ but there is no open
    interval $(a, b)$ such that $0 \in (a, b) \subseteq U$.
  \end{proof}
  \qed
\end{proof}

\begin{df}[Ordered Square]
  The \emph{ordered square} $I_o^2$ is the topological space $[0,1]^2$ under
  the order topology induced by the lexicographic order.
\end{df}

\begin{lm}
  \label{lm:topology:continuum:closed}
  Let $L$ be a linear continuum with a greatest element. Then every
  non-empty closed set in $L$ has a greatest element.
\end{lm}

\begin{proof}
  \pf
  \step{<1>1}{\pflet{$C$ be a non-empty closed set in $L$}}
  \step{<1>2}{\pflet{$u$ be the supremum of $C$}}
  \step{<1>3}{$u \in C$}
  \begin{proof}
    \step{<2>1}{\assume{w.l.o.g~$u$ is not least in $L$}}
    \begin{proof}
      \pf\ If $u$ is least then $C = \{ u \}$.
    \end{proof}
    \step{<2>2}{\pflet{$U$ be any open neighbourhood of $u$}}
    \step{<2>3}{\pick\ $v < u$ such that $(v, u] \subseteq U$}
    \begin{proof}
      \pf\ By Lemma \ref{lm:topology:order:open}.
    \end{proof}
    \step{<2>4}{\pick\ $x \in C$ such that $v < x$}
    \begin{proof}
      \pf\ $v$ is not an upper bound for $C$ (\stepref{<1>2}).
    \end{proof}
    \step{<2>5}{$U$ intersects $C$ in $v$}
    \qedstep
    \begin{proof}
      \pf\ By Theorem \ref{thm:topology:closure:neighbourhoods}.
    \end{proof}
  \end{proof}
  \qed
\end{proof}

\begin{df}[Long Line]
  The \emph{long line} is $(S_\Omega \times [0, 1)) \setminus \{(0, 0)\}$
  under the dictionary
  order, where $S_\Omega$ is the first uncountable ordinal under the order
  topology.
\end{df}

\section{The Product Topology}

\begin{df}[Product Topology]
  Let $\{ X_\alpha \}_{\alpha \in J}$ be a family of topological spaces. The
  \emph{product topology} on $\prod_{\alpha \in J} X_\alpha$ is the topology
  generated by the subbasis consisting of all sets of the form
  $\pi_\alpha^{-1}(U)$ where $\alpha \in J$ and $U$ is open in $X_\alpha$.
  The \emph{product space} of $\{ X_\alpha \}_{\alpha \in J}$ is
  $\prod_{\alpha \in J} X_\alpha$ under the product topology.
\end{df}

\begin{lm}
  Let $\{ X_\alpha \}_{\alpha \in J}$ be a family of topological spaces and
  $A_\alpha$ be closed in $X_\alpha$ for all $\alpha$. Then $\prod_{\alpha
    \in J} A_\alpha$ is closed in $\prod_{\alpha \in J} X_\alpha$.
\end{lm}

\begin{proof}
  \pf\ This holds because $\prod_{\alpha \in J} X_\alpha \setminus
  \prod_{\alpha \in J} A_\alpha = \bigcup_{\alpha \in J}
  \pi_\alpha^{-1}(X_\alpha \setminus A_\alpha)$. \qed
\end{proof}

\begin{thm}
  \label{thm:topology:product:basis}
  Let $\{ X_\alpha \}_{\alpha \in J}$ be a family of topological spaces.
  The set of all sets of the form $\prod_{\alpha \in J} U_\alpha$ where each
  $U_\alpha$ is open in $X_\alpha$, and $U_\alpha = X_\alpha$ for all but
  finitely many $\alpha$, is a basis for the product topology on
  $\prod_{\alpha
    \in J} X_\alpha$.
\end{thm}

\begin{proof}
  \pf\ By Lemma \ref{lm:topology:subbasis:generate}. \qed
\end{proof}

\begin{thm}
  Let $\{X_\alpha\}_{\alpha \in J}$ be a family of topological spaces, and
  let $\mathcal{B}_\alpha$ be a basis for the topology on $X_\alpha$ for each
  $\alpha$. Then
  \begin{align*}
    \mathcal{B} & = \{ \prod_{\alpha \in J} U_\alpha : \text{for finitely
      many
    } \alpha \in J, U_\alpha \in \mathcal{B}_\alpha,  \\
    &  \text{ and } U_\alpha =
    X_\alpha \text{ for all other values of } \alpha \}
  \end{align*}
  is a basis for the product topology on $\prod_{\alpha \in J} X_\alpha$.
\end{thm}

\begin{proof}
  \pf
  \step{<1>1}{Every member of $\mathcal{B}$ is open in the product topology.}
  \begin{proof}
    \pf\ Immediate from definitions.
  \end{proof}
  \step{<1>2}{For every open set  $U$ and $\{x_\alpha\}_{\alpha \in J} \in
    U$,
    there exists $B \in \mathcal{B}$ such that $\{ x_\alpha \}_{\alpha \in J}
    \in B \subseteq U$.}
  \begin{proof}
    \step{<2>1}{\pflet{$U$ be open and $\{x_\alpha\}_{\alpha \in J} \in U$}}
    \step{<2>2}{\pick\ $U_\alpha$ open in $X_\alpha$ for each $\alpha$ such
      that
      $\{ x_\alpha \}_{\alpha \in J} \in \prod_{\alpha \in J} U_\alpha
      \subseteq U$ and $U_\alpha = X_\alpha$ for all $\alpha$ except
      $\alpha_1$, \ldots, $\alpha_n$.}
    \begin{proof}
      \pf\ By Theorem \ref{thm:topology:product:basis}.
    \end{proof}
    \step{<2>3}{\pick\ $B_{\alpha_i} \in \mathcal{B}_{\alpha_i}$ such that
      $x_\alpha \in
      B_{\alpha_i} \subseteq U_{\alpha_i}$ for $i=1, \ldots, n$}
    \step{<2>4}{$\{ x_\alpha \}_{\alpha \in J} \in \prod_{\alpha \in J}
      V_\alpha
      \subseteq U$ where $V_{\alpha_i} = B_{\alpha_i}$ for $i = 1, \ldots,
      n$,
      and
      $V_\alpha = X_\alpha$ for all other $\alpha$.}
  \end{proof}
  \qed
\end{proof}

\begin{thm}[AC]
  \label{thm:topology:product:closure}
  Let $\{ X_\alpha \}_{\alpha \in J}$ be a family of topological spaces and
  $A_\alpha \subseteq X_\alpha$ for all $\alpha$. If $\prod_{\alpha \in J}
  X_\alpha$ is given the product topology, then
  \[ \prod_{\alpha \in J} \overline{A_\alpha} = \overline{\prod_{\alpha \in
      J}
    A_\alpha} \enspace . \]
\end{thm}

\begin{proof}
  \pf
  \step{<1>1}{$\prod_{\alpha \in J} \overline{A_\alpha} \subseteq
    \overline{\prod_{\alpha \in J} A_\alpha}$}
  \begin{proof}
    \step{<2>1}{\pflet{$\{ x_\alpha \}_{\alpha \in J} \in \prod_{\alpha \in
          J}
        \overline{A_\alpha}$}}
    \step{<2>2}{\pflet{$\prod_{\alpha \in J} U_\alpha$ be a basic
        neighbourhood
        of $\{ x_\alpha \}_{\alpha \in J}$, where each $U_\alpha$ is open in
        $X_\alpha$, and $U_\alpha = X_\alpha$ except for $\alpha = \alpha_1,
        \ldots,  \alpha_n$.}}
    \step{<2>3}{For $\alpha \in J$, \pick\ $a_\alpha \in A_\alpha
      \cap         U_\alpha$.}
    \begin{proof}
      \pf\ By Theorem \ref{thm:topology:closure:neighbourhoods}, using the
      Axiom of Choice.
    \end{proof}
    \step{<2>4}{$\{ a_\alpha \}_{\alpha \in J} \in \prod_{\alpha \in J}
      A_\alpha \cap \prod_{\alpha \in J} U_\alpha$}
    \qedstep
    \begin{proof}
      \pf\ By Theorem \ref{thm:topology:closure:neighbourhoods}.
    \end{proof}
  \end{proof}
  \step{<1>2}{$\overline{\prod_{\alpha \in
        J} A_\alpha} \subseteq \prod_{\alpha \in J} \overline{A_\alpha}$}
  \begin{proof}
    \step{<2>1}{\pflet{$\{ x_\alpha \}_{\alpha \in J} \in
        \overline{\prod_{\alpha \in J} A_\alpha}$}}
    \step{<2>2}{\pflet{$\alpha \in J$} \prove{$x_\alpha \in
        \overline{A_\alpha}$}}
    \step{<2>3}{\pflet{$U$ be a neighbourhood of $x_\alpha$ in $X_\alpha$}}
    \step{<2>4}{$\pi_\alpha^{-1}(U)$ is a neighbourhood of $\{ x_\alpha
      \}_{\alpha \in J}$}
    \step{<2>5}{\pick\ $\{ a_\alpha \}_{\alpha \in J} \in \pi_\alpha^{-1}(U)
      \cap \prod_{\alpha \in J} A_\alpha$}
    \begin{proof}
      \pf\ By Theorem \ref{thm:topology:closure:neighbourhoods}.
    \end{proof}
    \step{<2>6}{$a_\alpha \in U \cap A_\alpha$}
    \qedstep
    \begin{proof}
      \pf\ By Theorem \ref{thm:topology:closure:neighbourhoods}.
    \end{proof}
  \end{proof}
  \qed
\end{proof}

\begin{df}[Standard Topology on $\mathbb{R}^J$]
  For $J$ a set, the \emph{standard topology} on $\mathbb{R}^J$ is the
  product topology where $\mathbb{R}$ is given the standard topology.
\end{df}

\begin{df}[Closed Unit Ball]
  The \emph{closed unit ball} $B^2$ is $\{ (x, y) \in \mathbb{R}^2 : x^2 +
  y^2 \leq 1 \}$ as a subset of $\mathbb{R}^2$.
\end{df}

 \begin{df}[Sorgenfrey Plane]
 The \emph{Sorgenfrey plane} is $\mathbb{R}_l^2$.
\end{df}

\section{The Subspace Topology}

\begin{df}[Subspace Topology]
  Let $X$ be a topological space and $Y \subseteq X$. The \emph{subspace
    topology} on $Y$ is $\{ Y \cap U : U \text{ open in } X \}$. With this
  topology, $Y$ is a \emph{subspace} of $X$.

  We prove this is a topology.
\end{df}

\begin{proof}
  \pf
  \step{<1>1}{\pflet{$\mathcal{T} = \{ Y \cap U : U \text{ open in } X \}$}}
  \step{<1>2}{$Y \in \mathcal{T}$}
  \begin{proof}
    \pf\ $Y = Y \cap X$
  \end{proof}
  \step{<1>3}{$\mathcal{T}$ is closed under union.}
  \begin{proof}
    \step{<2>1}{\pflet{$\mathcal{A} \subseteq \mathcal{T}$} \prove{$\bigcup
        \mathcal{A} = Y \cap \bigcup \{ U \text{ open in } X : Y \cap U \in
        \mathcal{A} \}$}}
    \step{<2>2}{$\bigcup
      \mathcal{A} \subseteq Y \cap \bigcup \{ U \text{ open in } X : Y \cap U
      \in \mathcal{A} \}$}
    \begin{proof}
      \step{<3>1}{\pflet{$x \in \bigcup \mathcal{A}$}}
      \step{<3>2}{\pick\ $V \in \mathcal{A}$ such that $x \in V$}
      \step{<3>3}{\pick\ $U$ open in $X$ such that $V = Y \cap U$}
      \begin{proof}
        \pf\ By the definition of $\mathcal{T}$ (\stepref{<1>1},
        \stepref{<2>1},
        \stepref{<3>2})
      \end{proof}
      \step{<3>4}{$x \in Y \cap \bigcup \{ U \text{ open in } X : Y \cap U
        \in
        \mathcal{A} \}$}
    \end{proof}
    \step{<2>3}{$Y \cap \bigcup \{ U \text{ open in } X : Y \cap U \in
      \mathcal{A} \} \subseteq \bigcup \mathcal{A}$}
    \begin{proof}
      \pf\ Set theory.
    \end{proof}
  \end{proof}
  \step{<1>4}{$\mathcal{T}$ is closed under binary intersection.}
  \begin{proof}
    \pf\ This holds because $(Y \cap U) \cap (Y \cap V) = Y \cap (U \cap V)$.
  \end{proof}
  \qed
\end{proof}

\begin{lm}
  Let $X$ be a topological space, $Y \subseteq X$, and $A \subseteq Y$. Then
  the
  topology $A$ inherits as a subspace of $X$ is the same as the topology $A$
  inherits as a subspace of $Y$.
\end{lm}

\begin{proof}
  \pf
  \begin{align*}
    & \text{topology as a subspace of } Y \\
    = & \{ V \cap A : V \text{ open in } Y \} \\
    = & \{ V \cap A : \exists U \text{ open in } X. V = U \cap Y \} \\
    = & \{ U \cap Y \cap A : U \text{ open in } X \} \\
    = & \{ U \cap A : U \text{ open in } X \} \\
    = & \text{topology as a subspace of } X \qed
  \end{align*}
\end{proof}

\begin{lm}
  \label{lm:topology:subspace:open}
  Let $Y$ be a subspace of $X$. If $U$ is open in $Y$ and $Y$ is open in $X$
  then $U$ is open in $X$.
\end{lm}

\begin{proof}
  \pf
  \step{<1>1}{\pick\ $V$ open in $X$ such that $U = Y \cap V$}
  \step{<1>2}{$U$ is open in $X$}
  \begin{proof}
    \pf\ The open sets in $X$ are closed under binary intersection.
  \end{proof}
  \qed
\end{proof}

\begin{thm}
  \label{thm:topology:subspace:closure}
  Let $Y$ be a subspace of $X$. Let $A \subseteq Y$. Let $\overline{A}$ be
  the
  closure of $A$ in $X$. Then the closure of $A$ in $Y$ is $\overline{A} \cap
  Y$.
\end{thm}

\begin{proof}
  \pf
  \step{<1>1}{$\overline{A} \cap Y$ is a closed set in $Y$ that includes $A$.}
  \begin{proof}
    \step{<2>1}{$\overline{A} \cap Y$ is closed in $Y$.}
    \begin{proof}
      \pf\ By Lemma \ref{cor:topology:subspace:closed}.
    \end{proof}
    \step{<2>2}{$A \subseteq \overline{A} \cap Y$.}
  \end{proof}
  \step{<1>2}{If $C$ is any closed set in $Y$ that includes $A$ then
    $\overline{A}
    \cap Y \subseteq C$.}
  \begin{proof}
    \step{<2>1}{\pflet{$C$ be a closed set in $Y$ that includes $A$.}}
    \step{<2>2}{\pick\ $D$ closed in $X$ such that $C = D \cap Y$.}
    \begin{proof}
      \pf\ By Lemma \ref{cor:topology:subspace:closed}.
    \end{proof}
    \step{<2>3}{$\overline{A} \subseteq D$}
    \step{<2>4}{$\overline{A} \subseteq C$}
  \end{proof}
  \qed
\end{proof}

\begin{cor}
  \label{cor:topology:subspace:closed}
  Let $Y$ be a subspace of $X$. Then a set $A \subseteq Y$ is closed in $Y$
  if
  and only if it is the intersection of a closed set in $X$ with $Y$.
\end{cor}

\begin{cor}
  \label{cor:topology:subspace:closed2}
  Let $Y$ be a subspace of $X$. If $A$ is closed in $Y$ and $Y$ is closed in
  $X$
  then $A$ is closed in $X$.
\end{cor}

\begin{lm}
  \label{lm:topology:subspace:basis}
  Let $X$ be a topological space and $Y \subseteq X$. If $\mathcal{B}$ is a
  basis for the topology on $X$ then $\{ B \cap Y : B \in \mathcal{B} \}$ is
  a
  basis for the subspace topology on $Y$.
\end{lm}

\begin{proof}
  \pf
  \step{<1>1}{For all $B \in \mathcal{B}$, we have $B \cap Y$ is open in $Y$.}
  \begin{proof}
    \pf\ Immediate from definitions.
  \end{proof}
  \step{<1>2}{For every $V$ open in $Y$ and $y \in V$, there exists $B \in
    \mathcal{B}$ such that $y \in B \cap Y \subseteq V$.}
  \begin{proof}
    \step{<2>1}{\pflet{$V$ be open in $Y$ and $y \in V$}}
    \step{<2>2}{\pick\ $U$ open in $X$ such that $V = Y \cap U$}
    \step{<2>3}{\pick\ $B \in \mathcal{B}$ such that $y \in B \subseteq U$}
    \step{<2>4}{$y \in B \cap Y \subseteq V$}
  \end{proof}
  \qed
\end{proof}

\begin{lm}
  \label{lm:topology:subspace:subbasis}
  Let $X$ be a topological space and $Y \subseteq X$. If $\mathcal{S}$ is a
  subbasis for the topology on $X$ then $\{ S \cap Y : S \in \mathcal{S} \}$
  is
  a subbasis for the subspace topology on $Y$.
\end{lm}

\begin{proof}
  \pf
  \step{<1>1}{For all $S \in \mathcal{S}$, we have $S \cap Y$ is open in $Y$.}
  \begin{proof}
    \pf\ Immediate from definitions.
  \end{proof}
  \step{<1>2}{For every $V$ open in $Y$ and $y \in V$, there exist $S_1,
    \ldots,
    S_n \in \mathcal{S}$ such that $y \in (S_1 \cap Y) \cap \cdots \cap (S_n
    \cap Y) \subseteq V$}
  \begin{proof}
    \step{<2>1}{\pflet{$V$ be open in $Y$ and $y \in V$}}
    \step{<2>2}{\pick\ $U$ open in $X$ such that $V = U \cap Y$}
    \step{<2>3}{\pick\ $S_1, \ldots, S_n \in \mathcal{S}$ such that $y \in
      S_1
      \cap \cdots \cap S_n \subseteq U$}
    \step{<2>4}{$y \in (S_1 \cap Y) \cap \cdots \cap (S_n \cap Y) \subseteq
      V$}
  \end{proof}
  \qed
\end{proof}

\begin{thm}
  Let $X$ be a linearly ordered set in the order topology. Let $Y \subseteq
  X$
  be convex. Then the order topology on $Y$ is the same as the subspace
  topology.
\end{thm}

\begin{proof}
  \pf
  \step{<1>1}{\pflet{$\mathcal{T}_o$ be the order topology and
      $\mathcal{T}_s$
      be
      the subspace topology.}}
  \step{<1>2}{$\mathcal{T}_o \subseteq \mathcal{T}_s$}
  \begin{proof}
    \step{<2>1}{For all $a \in Y$, we have $\{ y \in Y : a < y \} \in
      \mathcal{T}_s$}
    \begin{proof}
      \pf\ $\{ y \in Y : a < y \} = \{ x \in X : a < x \} \cap Y$
    \end{proof}
    \step{<2>2}{For all $a \in Y$, we have $\{ y \in Y : y < a \} \in
      \mathcal{T}_s$}
    \begin{proof}
      \pf\ Similar.
    \end{proof}
    \qedstep
    \begin{proof}
      \pf\ Lemma \ref{lm:topology:order:subbasis} and Corollary
      \ref{cor:topology:subbasis:coarsest}.
    \end{proof}
  \end{proof}
  \step{<1>3}{$\mathcal{T}_s \subseteq \mathcal{T}_o$}
  \begin{proof}
    \step{<2>1}{The sets $(a, +\infty) \cap Y$ and $(-\infty, a) \cap Y$ for
      $a
      \in X$ form a subbasis for $\mathcal{T}_s$}
    \begin{proof}
      \pf\ Lemma \ref{lm:topology:subspace:subbasis}, Lemma
      \ref{lm:topology:order:subbasis}.
    \end{proof}
    \step{<2>2}{For all $a \in X$, we have $(a, +\infty) \cap Y \in
      \mathcal{T}_o$}
    \begin{proof}
      \step{<3>1}{\pflet{$a \in X$}}
      \step{<3>2}{\case{$a \in Y$}}
      \begin{proof}
        \pf\ In this case, $(a, +\infty) \cap Y$ is an open ray in $Y$.
      \end{proof}
      \step{<3>3}{\case{For all $y \in Y$ we have $a < y$}}
      \begin{proof}
        \pf\ In this case, $(a, +\infty) \cap Y = Y$.
      \end{proof}
      \step{<3>4}{\case{For all $y \in Y$ we have $y < a$}}
      \begin{proof}
        \pf\ In this case, $(a, +\infty) \cap Y = \emptyset$.
      \end{proof}
      \qedstep
      \begin{proof}
        \pf\ These are the only cases because $Y$ is convex.
      \end{proof}
    \end{proof}
    \step{<2>3}{For all $a \in X$, we have $(-\infty, a) \cap Y \in
      \mathcal{T}_o$}
    \begin{proof}
      \pf\ Similar.
    \end{proof}
    \qedstep
    \begin{proof}
      \pf\ Corollary \ref{cor:topology:subbasis:coarsest}.
    \end{proof}
  \end{proof}
  \qed
\end{proof}

\begin{thm}
  Let $\{X_\alpha\}_{\alpha \in J}$ be a family of topological spaces, and
  let $A_\alpha$ be a subspace of $X_\alpha$ for all $\alpha$. Then the
  product
  topology on $\prod_{\alpha \in J} A_\alpha$ is the same as the topology it
  inherits as a subspace of $\prod_{\alpha \in J} X_\alpha$.
\end{thm}

\begin{proof}
  \pf\ Each is the topology generated by the subbasis consisting of
  $\pi_\alpha^{-1}(U) \cap \prod_{\alpha \in J} A_\alpha = \pi_\alpha^{-1}(U
  \cap A_\alpha)$ where $\alpha \in J$ and $U$ is open in $X_\alpha$, using
  Lemma
  \ref{lm:topology:subspace:subbasis}. \qed
\end{proof}

\begin{df}[Unit Circle]
  The \emph{unit circle} $S^1$ is $\{ (x,y) \in \mathbb{R}^2 : x^2 + y^2 = 1
  \}$ as a subspace of $\mathbb{R}^2$.
\end{df}

\begin{prop}
  \label{prop:topology:subspace:limit_point}
  Let $Y$ be a subspace of $X$, $A \subseteq Y$, and $a \in Y$. Then $a$ is a
  limit point of $A$ in the subspace topology on $Y$ if and only if $a$ is a
  limit point of $A$ is the topology of $X$.
\end{prop}

\begin{proof}
  \pf
  \begin{align*}
    & a \text{ is a limit point of } A \text{ in } Y \\
    \Leftrightarrow & \forall U \text{ open in } Y( a \in U \Rightarrow U
    \text{ intersects } A \text{ outside } a) \\
    \Leftrightarrow & \forall V \text{ open in } X( a \in V \cap Y
    \Rightarrow
    V \cap Y \text{ intersects } A \text{ outside } a) \\
    \Leftrightarrow & \forall V \text{ open in } X( a \in V \Rightarrow V
    \text{ intersects } A \text{ outside } a) \\
    & \qquad (a \in Y, A \subseteq Y) \\
    \Leftrightarrow & a \text{ is a limit point of } A \text{ in } X & \qed
  \end{align*}
\end{proof}

\section{The Box Topology}

\begin{df}[Box Topology]
  Let $\{ X_\alpha \}_{\alpha \in J}$ be a family of topological spaces. The
  \emph{box topology} on $\prod_{\alpha \in J} X_\alpha$ is the topology
  generated by the basis consisting of all sets of the form $\prod_{\alpha
    \in J} U_\alpha$, where each $U_\alpha$ is open in $X_\alpha$.

  We prove this is a basis.
\end{df}

\begin{proof}
  \pf
  \step{<1>1}{\pflet{$\mathcal{B}$ be the set of all sets of the form
      $\prod_{\alpha \in J} U_\alpha$, where each $U_\alpha$ is open in
      $X_\alpha$.}}
  \step{<1>2}{$\bigcup \mathcal{B} = \prod_{\alpha \in J} X_\alpha$}
  \begin{proof}
    \pf\ This holds because $\prod_{\alpha \in J} X_\alpha \in \mathcal{B}$.
  \end{proof}
  \step{<1>3}{$\mathcal{B}$ is closed under binary intersection.}
  \begin{proof}
    \pf\ $\prod_{\alpha \in J} U_\alpha \cap \prod_{\alpha \in J} V_\alpha =
    \prod_{\alpha \in J} (U_\alpha \cap V_\alpha)$.
  \end{proof}
  \qedstep
  \begin{proof}
    \pf\ Corollary \ref{cor:topology:basis:generate}.
  \end{proof}
\end{proof}

\begin{thm}[AC]
  Let $\{X_\alpha\}_{\alpha \in J}$ be a family of topological spaces, and
  let $\mathcal{B}_\alpha$ be a basis for the topology on $X_\alpha$ for each
  $\alpha$. Then
  \[ \mathcal{B} = \{ \prod_{\alpha \in J} B_\alpha : \forall \alpha \in J.
  B_\alpha \in \mathcal{B}_\alpha \} \]
  is a basis for the box topology on $\prod_{\alpha \in J} X_\alpha$.
\end{thm}

\begin{proof}
  \pf
  \step{<1>1}{Every member of $\mathcal{B}$ is open in the box topology.}
  \begin{proof}
    \pf\ Immediate from definitions.
  \end{proof}
  \step{<1>2}{For every open set $U$ and $\{x_\alpha\}_{\alpha \in J} \in U$,
    there exists $B \in \mathcal{B}$ such that $\{ x_\alpha \}_{\alpha \in J}
    \in B \subseteq U$.}
  \begin{proof}
    \step{<2>1}{\pflet{$U$ be open and $\{x_\alpha\}_{\alpha \in J} \in U$}}
    \step{<2>2}{\pick\ $U_\alpha$ open in $X_\alpha$ for each $\alpha$ such
      that
      $\{ x_\alpha \}_{\alpha \in J} \in \prod_{\alpha \in J} U_\alpha
      \subseteq U$.}
    \step{<2>3}{\pick\ $B_\alpha \in \mathcal{B}_\alpha$ such that $x_\alpha
      \in
      B_\alpha \subseteq U_\alpha$ for each $\alpha$}
    \begin{proof}
      \pf\ Using the Axiom of Choice.
    \end{proof}
    \step{<2>4}{$\{ x_\alpha \}_{\alpha \in J} \in \prod_{\alpha \in J}
      B_\alpha
      \subseteq U$}
  \end{proof}
  \qed
\end{proof}

\begin{thm}
  Let $\{X_\alpha\}_{\alpha \in J}$ be a family of topological spaces, and
  let $A_\alpha$ be a subspace of $X_\alpha$ for all $\alpha$. Let
  $\prod_{\alpha \in J} X_\alpha$ be given the box topology. Then the box
  topology on $\prod_{\alpha \in J} A_\alpha$ is the same as the topology it
  inherits as a subspace of $\prod_{\alpha \in J} X_\alpha$.
\end{thm}

\begin{proof}
  \pf\ Each is the topology generated by the basis \\$\{ \prod_{\alpha \in J}
  (U_\alpha \cap A_\alpha) : U_\alpha \text{ is open in } X_\alpha \}$, using
  Lemma \ref{lm:topology:subspace:basis}. \qed
\end{proof}

\begin{thm}
  Let $\{ X_\alpha \}_{\alpha \in J}$ be a family of Hausdorff spaces. Then
  $\prod_{\alpha \in J} X_\alpha$ is Hausdorff under the box topology.
\end{thm}

\begin{proof}
  \pf
  \step{<1>1}{\pflet{$\{x_\alpha\}_{\alpha \in J}, \{y_\alpha\}_{\alpha \in
        J}
      \in \prod_{\alpha \in J} X_\alpha$ with $\{x_\alpha\}_{\alpha \in J}
      \neq \{y_\alpha\}_{\alpha \in J}$}}
  \step{<1>2}{\pick\ $\alpha \in J$ such that $x_\alpha \neq y_\alpha$}
  \step{<1>3}{\pick\ disjoint neighbourhoods $U$ of $x_\alpha$ and $V$ of
    $y_\alpha$.}
  \step{<1>4}{$\pi_\alpha^{-1}(U)$ and $\pi_\alpha^{-1}(V)$ are disjoint
    neighbourhoods of $\{x_\alpha\}_{\alpha \in J}$ and $\{y_\alpha\}_{\alpha
      \in
      J}$}
  \qed
\end{proof}

\begin{cor}
  The space $\mathbb{R}^\omega$ under the box topology is Hausdorff.
\end{cor}

\begin{thm}[AC]
  Let $\{ X_\alpha \}_{\alpha \in J}$ be a family of topological spaces and
  $A_\alpha \subseteq X_\alpha$ for all $\alpha$. If $\prod_{\alpha \in J}
  X_\alpha$ is given the box topology, then
  \[ \prod_{\alpha \in J} \overline{A_\alpha} = \overline{\prod_{\alpha \in
      J}
    A_\alpha} \enspace . \]
\end{thm}

\begin{proof}
  \pf
  \step{<1>1}{$\prod_{\alpha \in J} \overline{A_\alpha} \subseteq
    \overline{\prod_{\alpha \in J} A_\alpha}$}
  \begin{proof}
    \step{<2>1}{\pflet{$\{ x_\alpha \}_{\alpha \in J} \in \prod_{\alpha \in
          J}
        \overline{A_\alpha}$}}
    \step{<2>2}{\pflet{$\prod_{\alpha \in J} U_\alpha$ be a basic
        neighbourhood
        of $\{ x_\alpha \}_{\alpha \in J}$, where each $U_\alpha$ is open in
        $X_\alpha$.}}
    \step{<2>3}{For $\alpha \in J$, \pick\ $a_\alpha \in A_\alpha
      \cap         U_\alpha$.}
    \begin{proof}
      \pf\ By Theorem \ref{thm:topology:closure:neighbourhoods}, using the
      Axiom of Choice.
    \end{proof}
    \step{<2>4}{$\{ a_\alpha \}_{\alpha \in J} \in \prod_{\alpha \in J}
      A_\alpha \cap \prod_{\alpha \in J} U_\alpha$}
    \qedstep
    \begin{proof}
      \pf\ By Theorem \ref{thm:topology:closure:neighbourhoods}.
    \end{proof}
  \end{proof}
  \step{<1>2}{$\overline{\prod_{\alpha \in
        J} A_\alpha} \subseteq \prod_{\alpha \in J} \overline{A_\alpha}$}
  \begin{proof}
    \step{<2>1}{\pflet{$\{ x_\alpha \}_{\alpha \in J} \in
        \overline{\prod_{\alpha \in J} A_\alpha}$}}
    \step{<2>2}{\pflet{$\alpha \in J$} \prove{$x_\alpha \in
        \overline{A_\alpha}$}}
    \step{<2>3}{\pflet{$U$ be a neighbourhood of $x_\alpha$ in $X_\alpha$}}
    \step{<2>4}{$\pi_\alpha^{-1}(U)$ is a neighbourhood of $\{ x_\alpha
      \}_{\alpha \in J}$}
    \step{<2>5}{\pick\ $\{ a_\alpha \}_{\alpha \in J} \in \pi_\alpha^{-1}(U)
      \cap \prod_{\alpha \in J} A_\alpha$}
    \begin{proof}
      \pf\ By Theorem \ref{thm:topology:closure:neighbourhoods}.
    \end{proof}
    \step{<2>6}{$a_\alpha \in U \cap A_\alpha$}
    \qedstep
    \begin{proof}
      \pf\ By Theorem \ref{thm:topology:closure:neighbourhoods}.
    \end{proof}
  \end{proof}
  \qed
\end{proof}

\section{The Quotient Topology}

\begin{df}[Quotient Map]
  Let $X$ and $Y$ be topological spaces. Let $p : X \twoheadrightarrow Y$ be
  a
  surjective map. Then $p$ is a \emph{quotient map} iff, for all $U \subseteq
  Y$, we have $U$ is open in $Y$ iff $p^{-1}(U)$ is open in $X$.
\end{df}

\begin{lm}
  \label{lm:topology:quotient:saturated}
  Let $X$ and $Y$ be topological spaces and $p : X \twoheadrightarrow Y$ be
  surjective and continuous. Then the following are equivalent.
  \begin{enumerate}
    \item $p$ is a quotient map.
    \item $p$ maps saturated open sets to open sets.
    \item $p$ maps saturated closed sets to closed sets.
  \end{enumerate}
\end{lm}

\begin{proof}
  \pf
  \step{<1>1}{$1 \Rightarrow 2$}
  \begin{proof}
    \step{<2>1}{\assume{$p$ is a quotient map.}}
    \step{<2>2}{\pflet{$U \subseteq X$ be a saturated open set.}}
    \step{<2>3}{$U = p^{-1}(p(U))$}
    \begin{proof}
      \step{<3>1}{$U \subseteq p^{-1}(p(U))$}
      \begin{proof}
        \pf\ Set theory.
      \end{proof}
      \step{<3>2}{$p^{-1}(p(U)) \subseteq U$}
      \begin{proof}
        \step{<4>1}{\pflet{$x \in p^{-1}(p(U))$}}
        \step{<4>2}{\pick\ $y \in U$ such that $p(x) = p(y)$}
        \step{<4>3}{$x \in U$}
        \begin{proof}
          \pf\ \stepref{<2>2}, \stepref{<4>2}.
        \end{proof}
      \end{proof}
    \end{proof}
    \step{<2>4}{$p(U)$ is open}
    \begin{proof}
      \pf\ \stepref{<2>1}, \stepref{<2>3}.
    \end{proof}
  \end{proof}
  \step{<1>2}{$2 \Rightarrow 3$}
  \begin{proof}
    \step{<2>1}{\assume{$p$ maps saturated open sets to open sets}}
    \step{<2>2}{\pflet{$C \subseteq X$ be a saturated closed set.}}
    \step{<2>3}{$X  \setminus C$ is a saturated open set.}
    \begin{proof}
      \step{<3>1}{\pflet{$x \in X \setminus C$ and $x' \in X$ be such that
          $p(x)
          = p(x')$}}
      \step{<3>2}{$x' \notin C$}
      \begin{proof}
        \pf\ If $x' \in C$ then $x \in C$ since $C$ is saturated.
      \end{proof}
    \end{proof}
    \step{<2>4}{$p(X \setminus C)$ is open.}
    \begin{proof}
      \pf\ By \stepref{<2>1} and \stepref{<2>3}.
    \end{proof}
    \step{<2>5}{$p(X \setminus C) = Y \setminus p(C)$}
    \begin{proof}
      \step{<3>1}{$p(X \setminus C) \subseteq Y \setminus p(C)$}
      \begin{proof}
        \step{<4>1}{\pflet{$x \in X \setminus C$}}
        \step{<4>2}{\assume{for a contradiction $p(x) \in p(C)$}}
        \step{<4>3}{\pick\ $x' \in C$ such that $p(x) = p(x')$}
        \qedstep
        \begin{proof}
          \pf\ We have $x \notin C$, $x' \in C$ and $p(x) = p(x')$,
          contradicting \stepref{<2>2}.
        \end{proof}
      \end{proof}
      \step{<3>2}{$Y \setminus p(C) \subseteq p(X \setminus C)$}
      \begin{proof}
        \step{<4>1}{\pflet{$y \notin p(C)$}}
        \step{<4>2}{\pick\ $x \in X$ such that $p(x) = y$}
        \begin{proof}
          \pf\ $p$ is surjective.
        \end{proof}
        \step{<4>3}{$x \notin C$}
      \end{proof}
    \end{proof}
  \end{proof}
  \step{<1>3}{$3 \Rightarrow 1$}
  \begin{proof}
    \step{<2>1}{\assume{$p$ maps saturated closed sets to closed sets}}
    \step{<2>2}{\pflet{$C \subseteq Y$ be such that $p^{-1}(Y)$ is closed}}
    \step{<2>3}{$p^{-1}(C)$ is saturated} % TODO Extract lemma
    \begin{proof}
      \step{<3>1}{\pflet{$x \in p^{-1}(C)$, $x' \in X$ and $p(x) = p(x')$}}
      \step{<3>2}{$x' \in p^{-1}(C)$}
    \end{proof}
    \step{<2>4}{$p(p^{-1}(C))$ is closed}
    \begin{proof}
      \pf\ By \stepref{<2>1} and \stepref{<2>3}.
    \end{proof}
    \step{<2>5}{$C = p(p^{-1}(C))$}
    \begin{proof}
      \pf\ By set theory, since $p$ is surjective.
    \end{proof}
  \end{proof}
  \qed
\end{proof}

\begin{cor}
  If $p : X \twoheadrightarrow Y$ is a surjective continuous map that is
  either an open map or a closed map, then $p$ is a quotient map.
\end{cor}

\begin{df}[Quotient Topology]
  Let $X$ be a topological space, $A$ a set, and $p : X \twoheadrightarrow A$
  a surjective map. Then the \emph{quotient topology} on $A$ induced by $p$ is
  \[ \{ U \subseteq A : p^{-1}(U) \text{ is open in } X \} \enspace . \]

  It is easy to check this is a topology.
\end{df}

\begin{lm}
  Let $X$ be a topological space, $A$ a set, and $p : X \twoheadrightarrow A$
  a surjective map. Then the quotient topology induced by $p$ is the
  unique topology on $A$ such that $p$ is a quotient map.
\end{lm}

\begin{proof}
  \pf\ Immediate from definitions. \qed
\end{proof}

\begin{df}[Quotient Space]
  Let $X$ be a topological space and $X^*$ a partition of $X$. Let $p : X
  \twoheadrightarrow X^*$ be the canonical map. Then $X^*$ under the quotient
  topology induced by $p$ is called a \emph{quotient space} of $X$.
\end{df}

\begin{prop}
  Let $p : X \twoheadrightarrow Y$ be a quotient map. Let $A \subseteq X$ be
  open and saturated. Then $p \restriction_A : A \twoheadrightarrow p(A)$ is
  a
  quotient map.
\end{prop}

\begin{proof}
  \pf
  \step{<1>1}{\pflet{$q = p \restriction_A : A \twoheadrightarrow p(A)$}}
  \step{<1>2}{For all $V \subseteq p(A)$, we have $q^{-1}(V) = p^{-1}(V)$}
  \begin{proof}
    \step{<2>1}{$q^{-1}(V) \subseteq p^{-1}(V)$}
    \begin{proof}
      \pf\ Trivial.
    \end{proof}
    \step{<2>2}{$p^{-1}(V) \subseteq q^{-1}(V)$}
    \begin{proof}
      \step{<3>1}{\pflet{$x \in p^{-1}(V)$}}
      \step{<3>2}{\pick\ $x' \in A$ such that $p(x') = p(x)$}
      \begin{proof}
        \pf\ One exists because $p(x) \in V \subseteq p(A)$.
      \end{proof}
      \step{<3>3}{$x \in A$}
      \begin{proof}
        \pf\ This holds because $A$ is saturated.
      \end{proof}
      \step{<3>4}{$x \in q^{-1}(V)$}
      \begin{proof}
        \pf\ From \stepref{<3>1} and \stepref{<3>3}.
      \end{proof}
    \end{proof}
  \end{proof}
  \step{<1>3}{For all $U \subseteq X$, we have $p(U \cap A) = p(U) \cap p(A)$}
  \step{<1>4}{\pflet{$V \subseteq p(A)$ be such that $q^{-1}(V)$ is open in
      $A$.} \prove{$V$ is open in $p(A)$.}}
  \step{<1>5}{$q^{-1}(V)$ is open in $X$}
  \step{<1>6}{$p^{-1}(V)$ is open in $X$}
  \step{<1>7}{$V$ is open in $Y$}
  \step{<1>8}{$V$ is open in $p(A)$}
  \qed
\end{proof}

\begin{prop}
  Let $p : X \twoheadrightarrow Y$ be a quotient map. Let $A \subseteq X$ be
  closed and saturated. Then $p \restriction_A : A \twoheadrightarrow p(A)$
  is
  a
  quotient map.
\end{prop}

\begin{proof}
  \pf\ Similar. \qed
\end{proof}

\begin{prop}
  Let $p : X \twoheadrightarrow Y$ be an open quotient map. Let $A \subseteq
  X$ be
  saturated. Then $p \restriction_A : A \twoheadrightarrow p(A)$ is a
  quotient map.
\end{prop}

\begin{proof}
  \pf
  \step{<1>1}{\pflet{$q = p \restriction_A : A \twoheadrightarrow p(A)$}}
  \step{<1>2}{For all $V \subseteq p(A)$, we have $q^{-1}(V) = p^{-1}(V)$}
  \begin{proof}
    \step{<2>1}{$q^{-1}(V) \subseteq p^{-1}(V)$}
    \begin{proof}
      \pf\ Trivial.
    \end{proof}
    \step{<2>2}{$p^{-1}(V) \subseteq q^{-1}(V)$}
    \begin{proof}
      \step{<3>1}{\pflet{$x \in p^{-1}(V)$}}
      \step{<3>2}{\pick\ $x' \in A$ such that $p(x') = p(x)$}
      \begin{proof}
        \pf\ One exists because $p(x) \in V \subseteq p(A)$.
      \end{proof}
      \step{<3>3}{$x \in A$}
      \begin{proof}
        \pf\ This holds because $A$ is saturated.
      \end{proof}
      \step{<3>4}{$x \in q^{-1}(V)$}
      \begin{proof}
        \pf\ From \stepref{<3>1} and \stepref{<3>3}.
      \end{proof}
    \end{proof}
  \end{proof}
  \step{<1>3}{For all $U \subseteq X$, we have $p(U \cap A) = p(U) \cap p(A)$}
  \begin{proof}
    \step{<2>1}{$p(U \cap A) \subseteq p(U) \cap p(A)$}
    \begin{proof}
      \pf\ Set theory.
    \end{proof}
    \step{<2>2}{$p(U) \cap p(A) \subseteq p(U \cap A)$}
    \begin{proof}
      \step{<3>1}{\pflet{$x \in U$, $y \in A$, $p(x) = p(y)$} \prove{$p(x)
          \in
          p(U \cap A)$}}
      \step{<3>2}{$x \in A$}
      \begin{proof}
        \pf\ $A$ is saturated.
      \end{proof}
      \step{<3>3}{$x \in U \cap A$}
    \end{proof}
  \end{proof}
  \step{<1>4}{\pflet{$V \subseteq p(A)$ be such that $q^{-1}(V)$ is open in
      $A$.} \prove{$V$ is open in $p(A)$.}}
  \step{<1>5}{$p^{-1}(V)$ is open in $A$}
  \begin{proof}
    \pf\ By \stepref{<1>2}
  \end{proof}
  \step{<1>6}{\pick\ $U$ open in $X$ such that $p^{-1}(V) = U \cap A$}
  \step{<1>7}{$V = p(U) \cap p(A)$}
  \begin{proof}
    \pf
    \begin{align*}
      V & = p(p^{-1}(V)) & (p \text{ is surjective}) \\
      & = p(U \cap A) & (\text{\stepref{<1>6}}) \\
      & = p(U) \cap p(A) & (\text{\stepref{<1>3}})
    \end{align*}
  \end{proof}
  \step{<1>8}{$p(U)$ is open in $Y$}
  \begin{proof}
    \pf\ \stepref{<1>6}, $p$ is an open map.
  \end{proof}
  \step{<1>9}{$V$ is open in $p(A)$}
  \begin{proof}
    \pf\ \stepref{<1>7}, \stepref{<1>8}
  \end{proof}
  \qed
\end{proof}

\begin{prop}
  Let $p : X \twoheadrightarrow Y$ be a closed quotient map. Let $A \subseteq
  X$ be
  saturated. Then $p \restriction_A : A \twoheadrightarrow p(A)$ is a
  quotient map.
\end{prop}

\begin{proof}
  \pf\ Similar. \qed
\end{proof}

\begin{prop}
  \label{prop:topology:quotient:composite}
  The composite of two quotient maps is a quotient map.
\end{prop}

\begin{proof}
  \pf\ From Proposition \ref{prop:topology:strongly_continuous:composite}.
  \qed
\end{proof}

\begin{prop}
  Let $X^*$ be a quotient space of $X$. If every element of $X^*$ is closed
  in
  $X$, then $X^*$ is $T_1$.
\end{prop}

\begin{proof}
  \pf
  \step{<1>1}{\pflet{$C \in X^*$}}
  \step{<1>2}{$p^{-1}(\{C\}) = C$}
  \begin{proof}
    \pf\ Definition of $p$.
  \end{proof}
  \step{<1>3}{$p^{-1}(\{C\})$ is closed in $X$}
  \begin{proof}
    \pf\ By hypothesis.
  \end{proof}
  \step{<1>4}{$\{C\}$ is closed in $X^*$.}
  \begin{proof}
    \pf\ By Proposition \ref{prop:topology:strongly_continuous:closed}.
  \end{proof}
  \qed
\end{proof}
