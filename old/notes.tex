\documentclass{book}

\usepackage{amsmath}
\usepackage{amssymb}
\usepackage{amsthm}
\let\proof\relax
\let\endproof\relax
\let\qed\relax
\usepackage{pf2}
\usepackage{tikz-cd}

\title{Understand the Universe}
\author{Robin Adams}

\newtheorem{lm}{Lemma}[chapter]
\newtheorem{thm}[lm]{Theorem}
\newtheorem{cor}{Corollary}[lm]
\newtheorem{ax}{Axiom}
\theoremstyle{definition}
\newtheorem{df}[lm]{Definition}

\newcommand{\id}[1]{\ensuremath{\mathrm{id}_{#1}}}
\newcommand{\inv}[1]{\ensuremath{{#1}^{-1}}}
\newcommand{\Set}{\ensuremath{\mathbf{Set}}}
\newcommand{\op}[1]{\ensuremath{{#1}^{\mathrm{op}}}}
\newcommand{\Top}{\ensuremath{\mathbf{Top}}}
\newcommand{\im}{\ensuremath{\operatorname{im}}}

\newcommand{\mc}[1]{\ensuremath{\mathcal{#1}}}
\begin{document}
  
  \maketitle
  
  \chapter{Set Theory}
  
  \section{Primitive Terms}
  
  Let there be \emph{sets}.
  
  Given two sets $A$, $B$, let there be \emph{functions} from $A$ to $B$. We 
  write $f : A \rightarrow B$ iff $f$ is a function from $A$ to $B$, and call 
  $A$ 
  the \emph{domain} of $f$ and $B$ the \emph{codomain}.
  
  Given functions $f : A \rightarrow B$ and $g : B \rightarrow C$, let there be 
  a 
  function $g \circ f : A \rightarrow C$, the \emph{composite} of $f$ and $g$.
  
  \begin{ax}[Associativity]
    For any functions $f : A \rightarrow B$, $g : B \rightarrow C$ and $h : C 
    \rightarrow D$, we have $h \circ (g \circ f) = (h \circ g) \circ f : A 
    \rightarrow D$.
  \end{ax}
  
  Henceforth, we write $f_1 \circ \cdots \circ f_n$ for $((\cdots(f_1 \circ 
  f_2) 
  \circ f_3) \circ \cdots ) \circ f_n$.
  
  \section{Identity Mapping}
  
  \begin{ax}[Identity Mapping]
    For every set $A$, there exists a function $\id{A} : A \rightarrow A$, the 
    \emph{identity mapping} on $A$, such that:
    \begin{description}
      \item[Left Unit Law] for every set $B$ and function $f : A \rightarrow B$, 
      we 
      have $f \circ 
      i = f : A \rightarrow B$
      \item[Right Unit Law] for every set $B$ and function $f : B \rightarrow 
      A$, we 
      have $i \circ f 
      = f : B \rightarrow A$.
    \end{description}
  \end{ax}
  
  \begin{lm}
    The identity mapping on any set is unique.
  \end{lm}
  
  \begin{proof}
    \pf
    \step{1}{\pflet{$A$ be a set.}}
    \step{2}{\pflet{$i, j : A \rightarrow A$ be identity mappings on $A$}}
    \step{3}{$i = j$}
    \begin{proof}
      \pf
      \begin{align*}
        i & = i \circ j & (\text{\stepref{2}}) \\
        & = j & (\text{\stepref{2}}) & \qed
      \end{align*}
    \end{proof}
  \end{proof}
  
  \section{Sections and Retractions}
  
  \begin{df}[Section, Retraction]
    Let $r : A \rightarrow B$ and $s : B \rightarrow A$. Then $r$ is a 
    \emph{retraction} of $s$, and $s$ is a \emph{section} of $r$, iff $r \circ s 
    = 
    \id{B}$.
  \end{df}
  
  \begin{lm}
    \label{lm:section_retraction:equal}
    Let $f : A \rightarrow B$.  If $r : B \rightarrow A$ is a retraction of $f$ 
    and $s : B \rightarrow A$ is a section of $f$, then $r = s$.
  \end{lm}
  
  \begin{proof}
    \pf
    \begin{align*}
      r & = r \circ \id{B} & (\text{Right Unit Law}) \\
      & = r \circ f \circ s & (\text{$s$ is a section of $f$}) \\
      & = \id{A} \circ s & (\text{$r$ is a retraction of $f$}) \\
      & = s & (\text{Left Unit Law}) & \qed
    \end{align*}
  \end{proof}
  
  \section{Bijections}
  
  \begin{df}[Bijection]
    A function $f : A \rightarrow B$ is a \emph{bijection}, $f : A \cong B$, 
    iff 
    there exists a function $\inv{f} : B \rightarrow A$, the \emph{inverse} of 
    $f$, 
    such that $\inv{f}$ is a section of $f$ and a retraction of $f$.
    
    Two sets $A$ and $B$ are \emph{equinumerous}, $A \cong B$, iff there exists 
    a 
    bijection $f : A \cong B$.
  \end{df}
  
  \begin{lm}
    The inverse of any bijection is unique.
  \end{lm}
  
  \begin{proof}
    \pf\ Immediate from Lemma \ref{lm:section_retraction:equal}. \qed
  \end{proof}
  
  \section{Terminal Set}
  
  \begin{ax}[Terminal Set]
    There exists a set $1$, the \emph{terminal} set, such that, for every set 
    $A$, 
    there exists a unique function $A \rightarrow 1$.
  \end{ax}
  
  \begin{thm}
    The terminal set is unique up to unique bijection.
  \end{thm}
  
  \begin{proof}
    \step{1}{\pflet{$T$ and $T'$ be terminal sets.}}
    \step{2}{There exists a bijection $b : T \cong T'$}
    \begin{proof}
      \step{2a}{\pflet{$b$ be the unique function $T \rightarrow T'$}}
      \step{2b}{\pflet{$\inv{b}$ be the unique function $T' \rightarrow T$}}
      \step{2c}{$\inv{b} \circ b = \id{T}$}
      \begin{proof}
        \pf\ By the uniqueness of the function $T \rightarrow T$.
      \end{proof}
      \step{2d}{$b \circ \inv{b} = \id{T'}$}
      \begin{proof}
        \pf\ By the uniqueness of the function $T' \rightarrow T'$.
      \end{proof}
    \end{proof}
    \step{3}{If $b, b' : T \cong T'$ are bijections then $b = b'$.}
    \begin{proof}
      \pf\ By the uniqueness of the function $T \rightarrow T'$.
    \end{proof}
    \qed
  \end{proof}
  
  \begin{df}[Element]
    An \emph{element} of a set $A$ is a function $1 \rightarrow A$.
    
    We write $a \in A$ for $a : 1 \rightarrow A$.  We write $f(a)$ for $f \circ 
    a$ 
    when $a \in A$ and $f : A \rightarrow B$.
  \end{df}
  
  \begin{ax}[Extensionality]
    Let $f, g : A \rightarrow B$. If, for all $x \in A$, we have $f(x) = g(x)$, 
    then $f = g$.
  \end{ax}
  
  \section{Surjective Functions}
  
  \begin{df}[Surjective]
    A function $f : A \rightarrow B$ is \emph{surjective}, $f : A 
    \twoheadrightarrow B$, iff, for all $b \in B$, there exists $a \in A$ such 
    that 
    $f(a) = b$.
  \end{df}
  
  \section{Injective Functions}
  
  \begin{df}[Injective]
    A function $f : A \rightarrow B$ is \emph{injective}, $f : A 
    \rightarrowtail 
    B$, iff, for all $x, y \in A$, if $f(x) = f(y)$ then $x = y$.
  \end{df}
  
  \begin{lm}
    A function is a bijection iff it is injective and surjective.
  \end{lm}
  
  %TODO
  
  \section{The Empty Set}
  
  \begin{ax}[Empty Set]
    There exists an \emph{empty} set $\emptyset$ such that, for every set $A$, 
    there exists a unique function $\emptyset \rightarrow A$.
  \end{ax}
  
  \begin{ax}[Nondegeneracy of the Category of Sets]
    $0 \not\cong 1$
  \end{ax}
  
  \chapter{Category Theory}
  
  \begin{df}[Category]
    A \emph{category} $\mathcal{C}$ consists of the following.
    \begin{itemize}
      \item A class $|\mathcal{C}|$ of \emph{objects}. We write $A \in
      \mathcal{C}$ for $A \in |\mathcal{C}|$.
      \item For any objects $A, B \in \mathcal{C}$, a set $\mathcal{C}[A,B]$ of
      \emph{morphisms} from $A$ to $B$. We write $f:A\rightarrow B$ for
      $f\in\mathcal{C}[A,B]$.
      \item For any object $A \in \mathcal{C}$, a morphism $\id{A} : A
      \rightarrow A$, the \emph{identity} morphism on $A$.
      \item For any morphisms $f : A \rightarrow B$ and $g : B \rightarrow C$, a
      morphism $g \circ f : A \rightarrow C$, the \emph{composite} of $g$ and 
      $f$.
    \end{itemize}
    such that:
    \begin{description}
      \item[Associativity]
      For any $f : A \rightarrow B$, $g : B \rightarrow C$ and $h : C 
      \rightarrow
      D$, we have
      \[ h \circ (g \circ f) = (h \circ g) \circ f \enspace . \]
      \item[Left Unit Law]
      For any $f : A \rightarrow B$, we have
      \[ \id{B} \circ f = f \enspace . \]
      \item[Right Unit Law]
      For any $f : A \rightarrow B$, we have
      \[ f \circ \id{A} = f \enspace . \]
    \end{description}
  \end{df}
  
  \begin{df}[Category of Sets]
    The \emph{category of sets} $\Set$ is the category with objects all
    sets and morphisms all functions.
  \end{df}
  
  \begin{df}[Isomorphism]
    A morphism $f : A \rightarrow B$ is an \emph{isomorphism}, $f : A \cong B$,
    iff there exists a morphism $\inv{f} : B \rightarrow A$, the \emph{inverse} 
    of
    $f$, such that
    \[ \inv{f} \circ f = \id{A}, \qquad f \circ \inv{f} = \id{B} \enspace . \]
  \end{df}
  
  \section{Opposite Category}
  
  \begin{df}
    Let $\mathcal{C}$ be a category. The \emph{opposite category} 
    $\mathcal{C}^{\mathrm{op}}$ is defined by:
    \begin{itemize}
      \item for every object $A$ of $\mathcal{C}$, there is an object 
      $A^{\mathrm{op}}$ of $\mathcal{C}^{\mathrm{op}}$
      \item for every morphism $f : A \rightarrow B : \mathcal{C}$, there is a 
      morphism $f^{\mathrm{op}} : B^{\mathrm{op}} \rightarrow A^{\mathrm{op}} : 
      \mathcal{C}^{\mathrm{op}}$
      \item $\id{A^{\mathrm{op}}} = \id{A}^{\mathrm{op}}$
      \item $f^{\mathrm{op}} \circ g^{\mathrm{op}} = (g \circ f)^{\mathrm{op}}$
    \end{itemize}
    
  \end{df}
  
  \section{Functors}
  
  \begin{df}
    Let $\mathcal{C}$ and $\mathcal{D}$ be categories. A \emph{functor} $F : 
    \mathcal{C} \rightarrow \mathcal{D}$ consists of:
    \begin{itemize}
      \item for every object $A \in \mathcal{C}$, an object $FA \in 
      \mathcal{D}$;
      \item for every morphism $f : A \rightarrow B : \mathcal{C}$, a morphism 
      $Ff : 
      FA \rightarrow FB : \mathcal{D}$
    \end{itemize}
    such that:
    \begin{itemize}
      \item for every object $A \in \mathcal{C}$, we have $F \id{A} = \id{FA} : 
      FA 
      \rightarrow FA : \mathcal{D}$;
      \item for any morphisms $f : A \rightarrow B, g : B \rightarrow C : 
      \mathcal{C}$, we have $F(g \circ f) = Fg \circ Ff : FA \rightarrow FC : 
      \mathcal{D}$
    \end{itemize}
  \end{df}
  
  \begin{df}[Hom-functor]
    For any category $\mathcal{C}$, the \emph{hom-functor} $\mathcal{C}[-, +] : 
    \mathcal{C}^{\mathrm{op}} \times \mathcal{C} \rightarrow \Set$ is defined 
    by:
    \begin{itemize}
      \item $\mathcal{C}[A^{\mathrm{op}}, B]$ is the set of morphisms $A 
      \rightarrow 
      B$
      \item Given $f : A \rightarrow A'$ and $g : B \rightarrow B'$, define 
      $\mathcal{C}[f^{\mathrm{op}}, g] : \mathcal{C}[(A')^{\mathrm{op}}, B] 
      \rightarrow \mathcal{C}[A^{\mathrm{op}}, B']$ by
      \[ \mathcal{C}[f^{\mathrm{op}}, g](h) = g \circ h \circ f \]
    \end{itemize}
  \end{df}
  
  \section{Natural Transformations}
  
  \begin{df}[Natural Transformation]
    Let $F, G : \mathcal{C} \rightarrow \mathcal{D}$ be functors. A 
    \emph{natural 
      transformation} $\tau : F \Rightarrow G$ consists of:
    \begin{itemize}
      \item for every object $A \in \mathcal{C}$, a morphism $\tau_A : FA 
      \rightarrow G A : \mathcal{D}$, the \emph{component} of $\tau$ at $A$
    \end{itemize}
    such that:
    \begin{itemize}
      \item for any morphism $f : A \rightarrow B : \mathcal{C}$, the following 
      diagram commutes:
      \[ \begin{tikzcd}
        FA \ar[r, "Ff"] \ar[d, "\tau_A"'] & FB \ar[d, "\tau_B"] \\
        GA \ar[r, "Gf"'] & GB
      \end{tikzcd} \]
    \end{itemize}
  \end{df}
  
  \begin{df}[Natural Isomorphism]
    A natural transformation $\tau : F \Rightarrow G : \mathcal{C} \rightarrow 
    \mathcal{D}$ is a \emph{natural isomorphism}, $\tau : F \cong G$, iff, for 
    all 
    $A \in \mathcal{C}$, we have $\tau_A : FA \cong GA : \mathcal{D}$ is an 
    isomorphism.
  \end{df}
  
  \begin{lm}
    If $\tau : F \rightarrow G$ is a natural isomorphism, then the inverses 
    $\inv{\tau_A} : GA \cong FA$ form a natural isomorphism $\inv{\tau} : G 
    \cong 
    F$.
  \end{lm}
  
  \section{Product Categories}
  
  \begin{df}[Product Category]
    Let $\mathcal{C}$ and $\mathcal{D}$ be categories. The \emph{product 
      category} 
    $\mathcal{C} \times \mathcal{D}$ is the category with:
    \begin{itemize}
      \item objects all pairs $(A, B)$ with $A \in \mathcal{C}$ and $B \in 
      \mathcal{D}$;
      \item morphisms $(A, B) \rightarrow (A', B')$ all pairs $(f, g)$ with $f : 
      A 
      \rightarrow A' : \mathcal{C}$ and $g : B \rightarrow B' : \mathcal{D}$
      \item $\id{(A, B)} = (\id{A}, \id{B})$
      \item $(f', g') \circ (f, g) = (f' \circ f, g' \circ g)$
    \end{itemize}
  \end{df}
  
  \section{Limits}
  
  \begin{df}[Cone]
    Let $D : \mathcal{J} \rightarrow \mathcal{C}$ be a functor. A \emph{cone} 
    $tau$ over $D$ with \emph{vertex} $C \in \mathcal{C}$ is a natural 
    transformation $\tau : K C \Rightarrow D : \mathcal{J} \rightarrow 
    \mathcal{C}$. That is, it consists of a
    morphism $\tau_J : C \rightarrow D J$ for all $J \in \mathcal{J}$ such that,
    for every morphism $t : I \rightarrow J : \mathcal{J}$, the following 
    diagram 
    commutes.
    \[ \begin{tikzcd}
      & C \ar[dl, "\tau_I"'] \ar[dr, "\tau_J"] & \\
      DI \ar[rr, "Dt"'] & & DJ
    \end{tikzcd} \]
  \end{df}
  
  \begin{df}[Limit]
    Let $D : \mathcal{J} \rightarrow \mathcal{C}$ be a functor. A \emph{limit} 
    of 
    $D$ consists of:
    \begin{itemize}
      \item an object $L \in \mathcal{C}$
      \item a cone $\nu_J : L \rightarrow D J$ over $D$, the \emph{universal 
        cone}
    \end{itemize}
    such that, for every cone $\tau_J : X \rightarrow D J$ over $D$, there 
    exists a 
    unique morphism $\overline{\tau} : X \rightarrow L$, the \emph{mediating} 
    morphism, such that
    \[ \nu_J \circ \overline{\tau} = \tau_J : X \rightarrow D J \]
    for all $J \in \mathcal{J}$.
  \end{df}
  
  \begin{df}[Preserves Limits]
    A functor $F : \mathcal{C} \rightarrow \mathcal{D}$ \emph{preserves limits} 
    of shape $\mathcal{J}$ iff, for any functor $D : \mathcal{J} \rightarrow 
    \mathcal{C}$, if $L$ is a limit of $D$ with universal cone $\nu_J : L 
    \rightarrow D J$, then $FL$ is a limit of $F \circ D$ with universal cone 
    $F 
    \nu_J : F L \rightarrow F (D J)$.
  \end{df}
  
  \begin{df}[Reflects Limits]
    A functor $F : \mathcal{C} \rightarrow \mathcal{D}$ \emph{reflects limits} 
    of shape $\mathcal{J}$ iff, for any functor $D : \mathcal{J} \rightarrow 
    \mathcal{C}$ and cone $\nu_J : L \rightarrow D J$ over $D$, if $F L$ is a 
    limit 
    of $F \circ D$ with universal cone $F \nu_J$, then $L$ is a limit of $D$ 
    with 
    universal cone $\nu$.
  \end{df}
  
  \begin{df}[Creates Limits]
    A functor $F : \mathcal{C} \rightarrow \mathcal{D}$ \emph{reflects limits} 
    of shape $\mathcal{J}$ iff $F$ preserves and reflects limits of shape 
    $\mathcal{J}$ and, for any functor $D : \mathcal{J} \rightarrow 
    \mathcal{C}$, if $F \circ D$ has a limit then $D$ has a limit.
  \end{df}
  
  \section{Terminal Objects}
  
  \begin{df}[Terminal Object]
    An object $1$ is \emph{terminal} iff, for every object $X$, there exists a 
    unique morphism $X \rightarrow 1$.
  \end{df}
  
  \section{Products}
  
  \begin{df}
    Let $\{X_j\}_{j \in J}$ be a family of objects in $\mathcal{C}$. A 
    \emph{product} of $\{X_j\}_{j \in J}$ is a limit of the functor from the 
    discrete category $J$ to $\mathcal{C}$ that maps $j$ to $X_j$. We write the 
    limiting object as $\prod_{j \in J} X_j$, and call the components of the 
    universal cone \emph{projections} $\pi_j : \prod_{j \in J} X_j \rightarrow 
    X_j$.
    
    Given an object $A$ and morphisms $f_j : A \rightarrow X_j$ for all $j \in 
    J$, 
    we write $\langle f_j \mid j \in J \rangle$ for the mediating morphism
    \begin{align*}
      \langle f_j \mid j \in J \rangle & : A \rightarrow \prod_{j \in J} X_j \\
      \pi_j \circ \langle f_j \mid j \in J \rangle & = f_j & (j \in J)
    \end{align*}
  \end{df}
  
  \begin{thm}
    A category has finite products iff it has a terminal object and binary 
    products.
  \end{thm}
  
  \section{Multiplicative Categories}
  
  \begin{df}[Multiplicative Category]
    A \emph{multiplicative category} $\mathcal{C}$ consists of a category 
    $\mathcal{C}$ and a functor $\otimes : \mathcal{C}^2 \rightarrow 
    \mathcal{C}$, 
    the \emph{tensor product}.
  \end{df}
  
  \begin{df}[Multiplicative Subcategory]
    Let $(\mathcal{C}, \otimes_{\mathcal{C}})$ be a multiplicative category. A 
    \emph{multiplicative subcategory} $(\mathcal{D}, \otimes_{\mathcal{D}})$ is 
    a 
    multiplicative category such that $\mathcal{D}$ is a subcategory of 
    $\mathcal{C}$, and $\otimes_{\mathcal{D}}$ is the restriction of 
    $\otimes_{\mathcal{C}}$ to $\mathcal{D}$. 
  \end{df}
  
  \begin{df}
    Let $\mathcal{C}$ be a multiplicative category, and $\mathcal{D}$ a 
    subcategory 
    of $\mathcal{C}$. The multiplicative subcategory \emph{generated} by 
    $\mathcal{D}$ is the multiplicative subcategory $\langle \mathcal{D} 
    \rangle$ 
    defined by:
    \begin{itemize}
      \item the objects of $\langle \mathcal{D} \rangle$ are defined inductively 
      by:
      \begin{itemize}
        \item every object of $\mathcal{D}$ is an object of $\langle 
        \mathcal{D} 
        \rangle$
        \item $I \in \langle \mathcal{D} \rangle$
        \item if $A, B \in \langle \mathcal{D} \rangle$ then $A \otimes B \in 
        \langle 
        \mathcal{D} \rangle$
      \end{itemize}
      \item the morphisms of $\langle \mathcal{D} \rangle$ are defined 
      inductively 
      by:
      \begin{itemize}
        \item every morphism in $\mathcal{D}$ is a morphism in $\langle 
        \mathcal{D} 
        \rangle$
        \item for all $A \in \langle \mathcal{D} \rangle$ we have $\id{A} : A 
        \rightarrow A : \langle \mathcal{D} \rangle$
        \item for all $f : A \rightarrow B, g : B \rightarrow C : \langle 
        \mathcal{D} \rangle$ we have $g \circ f : A \rightarrow C : \langle 
        \mathcal{D} 
        \rangle$
        \item for all $f : A \rightarrow B, g : C \rightarrow D : \langle 
        \mathcal{D} 
        \rangle$ we have $f \otimes g : A \otimes C \rightarrow B \otimes D : 
        \langle 
        \mathcal{D} \rangle$.
      \end{itemize}
    \end{itemize}
  \end{df}
  
  \begin{df}
    Let $\mathcal{C}$ be a multiplicative category and $f : A \rightarrow B : 
    \mathcal{C}$. The set of \emph{expansions} of $f$ is defined inductively by:
    \begin{itemize}
      \item $f$ is an expansion of $f$
      \item If $g : C \rightarrow D$ is an expansion of $f$ and $E \in 
      \mathcal{C}$ 
      then $\id{E} \otimes g : E \otimes C \rightarrow E \otimes D$ and $g 
      \otimes 
      \id{E} : C \otimes E \rightarrow D \otimes E$ are expansions of $f$.
    \end{itemize}
  \end{df}
  
  \begin{lm}
    The class of morphisms of $\langle \mathcal{D} \rangle$ is the class 
    consisting of all the identity morphisms on objects of $\langle \mathcal{D} 
    \rangle$, and all the composites of expansions in $\langle \mathcal{D} 
    \rangle$ 
    of morphisms in $\mathcal{D}$.
  \end{lm}
  
  \begin{df}
    Given multiplicative categories $\mathcal{C}$, $\mathcal{D}$ and functors 
    $F, 
    G : \mathcal{C} \rightarrow \mathcal{D}$, define the functor $F \otimes G : 
    \mathcal{C} \rightarrow \mathcal{D}$ by:
    \begin{align*}
      (F \otimes G) C & = FC \otimes GC (C \in \mathcal{C}) \\
      (F \otimes G) f & = Ff \otimes Gf : FC \otimes GC \rightarrow FD \otimes 
      GD (f 
      : C \rightarrow D : \mathcal{C})
    \end{align*}
    Given natural transformations $\sigma : F \Rightarrow F' : \mathcal{C} 
    \rightarrow \mathcal{D}$ and $\tau : G \Rightarrow G' : \mathcal{C} 
    \rightarrow 
    \mathcal{D}$, define the natural transformation $\sigma \otimes \tau : F 
    \otimes G \Rightarrow F' \otimes G' : \mathcal{C} \rightarrow \mathcal{D}$ 
    by
    \[ (\sigma \otimes \tau)_C = \sigma_C \otimes \tau_C : FC \otimes F'C 
    \rightarrow GC \otimes G'C \]
  \end{df}
  
  \begin{df}[Iterate]
    Let $\mathcal{C}$ be a multiplicative category. The set of \emph{iterates} 
    of 
    $\otimes$ is the set of functors $\mathcal{C}^n \rightarrow \mathcal{C}$ 
    for some $n$ defined inductively by:
    \begin{itemize}
      \item $\id{\mathcal{C}}$ is an iterate of $\otimes$
      \item If $F$ and $G$ are iterates of $\otimes$ then so is $F \otimes G$
    \end{itemize}
    
    Let $It_\otimes(\mathcal{C})$ be the multiplicative category whose objects 
    are 
    the iterates of $\otimes$, and whose morphisms are the natural 
    transformations.
  \end{df}
  
  \section{Semigroupal Categories}
  
  \begin{df}[Semigroupal Category]
    A \emph{semigroupal category} $\mathcal{C}$ consists of:
    \begin{itemize}
      \item a multiplicative category $\mathcal{C}$
      \item a natural isomorphism $\alpha_{XYZ} : X \otimes (Y \otimes Z) \cong 
      (X 
      \otimes Y) \otimes Z$, the \emph{associator}
    \end{itemize}
    such that, for any objects $W, X, Y, Z \in \mathcal{C}$, the following 
    diagram 
    (the \emph{pentagon equation}) commutes:
    \[ \begin{tikzcd}
      & ((W \otimes X) \otimes Y) \otimes Z \ar[ddl] \ar[dr] \\
      & & (W \otimes (X \otimes Y)) \otimes Z \ar[dd] \\
      (W \otimes X) \otimes (Y \otimes Z) \ar[ddr] \\
      & & W \otimes ((X \otimes Y) \otimes Z) \ar[dl] \\
      & W \otimes (X \otimes (Y \otimes Z))
    \end{tikzcd} \]
  \end{df}
  
  \begin{df}[Iterates of $\alpha$]
    The \emph{iterates} of $\alpha$ are the morphisms of the multiplicative 
    subcategory of $It_\otimes(\mathcal{C})$ generated by all natural 
    isomorphisms
    $\alpha(F \times G \times H) : F \otimes (G \otimes 
    H) \cong (F \otimes G) \otimes H : \mathcal{C}^{m + n + k} \rightarrow 
    \mathcal{C}$, where $F : \mathcal{C}^m \rightarrow \mathcal{C}$, $G : 
    \mathcal{C}^n 
    \rightarrow \mathcal{C}$, $H : \mathcal{C}^k \rightarrow \mathcal{C}$ are 
    iterates of $\otimes$.
  \end{df}
  
  \section{Monoidal Categories}
  
  \begin{df}[Monoidal Category]
    A \emph{monoidal category} $\mathcal{C}$ consists of:
    \begin{itemize}
      \item a semigroupal category $\mathcal{C}$;
      \item an item $I \in \mathcal{C}$, the \emph{unit};
      \item a natural isomorphism $l_X : I \otimes X \cong X$, the \emph{left 
        unitor};
      \item a natural isomorphism $r_X : X \otimes I \cong X$, the \emph{right 
        unitor};
    \end{itemize}
    such that:
    for any objects $X, Y \in \mathcal{C}$, the following diagram (the 
    \emph{triangle equation}) commutes:
    \[ \begin{tikzcd}
      X \otimes (I \otimes Y) \ar[rr, "\alpha_{XIY}"] \ar[dr, "\id{X} \otimes 
      l_Y"] 
      & & (X \otimes I) \otimes Y \ar[dl, "r_X \otimes \id{Y}"] \\
      & X \otimes Y
    \end{tikzcd} \]
  \end{df}
  
  To read: S. Mac Lane, Natural associativity and commutativity, Rice Univ. 
  Stud. 
  49 (1963) 28–46.
  
  \begin{thm}
    Every category with chosen finite products is a monoidal category with 
    tensor 
    product $\times$ and unit the terminal object.
  \end{thm}
  
  \begin{df}[Cartesian Monodial Category]
    A \emph{Cartesian} monoidal category is a monoidal category such that $A 
    \otimes B$ is the product of $A$ and $B$, and the unit is terminal, with 
    the 
    associators and unitors being the canonical morphisms.
  \end{df}
  
  \begin{df}[Clique]
    A \emph{clique} in a category $\mathcal{C}$ consists of:
    \begin{itemize}
      \item a nonempty family $\{ C_i \}_{i \in I}$ of objects of $\mathcal{C}$;
      \item for all $i, j \in I$, a morphism $\mu_{ij} : C_i \rightarrow C_j$
    \end{itemize}
    such that:
    \begin{itemize}
      \item for all $i \in I$, we have $\mu_{ii} = \id{C_i}$
      \item for all $i, j, k \in I$, we have $\mu_{jk} \circ \mu_{ij} = 
      \mu_{ik}$
    \end{itemize}
    
    The \emph{category of cliques in $\mathcal{C}$} $\operatorname{clq} 
    \mathcal{C}$ has objects all cliques, with a morphism $f : (\{ C_i \}_{i 
      \in 
      I}, \{ \mu_{ij} \}_{i,j \in I}) \rightarrow (\{ D_k \}_{k \in K}, \{ 
    \nu_{kl} 
    \}_{k,l \in K})$ being a family of morphisms
    \[ f_{ik} : C_i \rightarrow D_k \]
    such that, for all $i, j \in I$ and $k, l \in K$, the following diagram 
    commutes
    \[ \begin{tikzcd}
      C_i \ar[r, "\mu_{ij}"] \ar[d, "f_{ik}"] & C_j \ar[d, "f_{jl}"] \\
      D_k \ar[r, "\nu_{kl}"] & D_l
    \end{tikzcd} \]
  \end{df}
  
  \begin{lm}
    In any clique, every morphism $\mu_{ij}$ is an isomorphism with inverse 
    $\inv{\mu_{ij}} = \mu_{ji}$.
  \end{lm}
  
  \begin{proof}
    \pf\ $\mu_{ji} \circ \mu_{ij} = \mu_{ii} = \id{C_i}$
  \end{proof}
  
  \begin{thm}[Coherence Theorem]
    Let $\mathcal{C}$ be a monoidal category and $n \in \mathbb{N}$. 
  \end{thm}
  
  \section{Braided Monoidal Categories}
  
  \begin{df}[Braided Monoidal Category]
    A \emph{braided monoidal category} $\mathcal{C}$ consists of:
    \begin{itemize}
      \item a monoidal category $\mathcal{C}$
      \item a natural isomorphism $b_{XY} : X \otimes Y \cong Y \otimes X$, the 
      \emph{braiding}
    \end{itemize}
    such that, for all $X, Y, Z \in \mathcal{C}$, the following \emph{hexagon 
      equations} hold:
    \[ \begin{tikzcd}
      X \otimes (Y \otimes Z) \ar[r, "\alpha_{XYZ}"] \ar[d, "b_{X,Y \otimes Z}"] 
      &
      (X \otimes Y) \otimes Z \ar[r, "b_{XY} \otimes \id{Z}"] &
      (Y \otimes X) \otimes Z \ar[d, "\inv{\alpha_{YXZ}}"] \\
      (Y \otimes Z) \otimes X &
      Y \otimes (Z \otimes X) \ar[l, "\alpha_{YZX}"] &
      Y \otimes (X \otimes Z) \ar[l, "\id{Y} \otimes b_{XZ}"]
    \end{tikzcd} \]
    \[ \begin{tikzcd}
      (X \otimes Y) \otimes Z \ar[r, "\inv{\alpha_{XYZ}}"] \ar[d, "b_{X 
        \otimes Y, Z}"] &
      X \otimes (Y \otimes Z) \ar[r, "\id{X} \otimes b_{YZ}"] &
      X \otimes (Z \otimes Y) \ar[d, "\alpha_{XZY}"] \\
      Z \otimes (X \otimes Y) &
      (Z \otimes X) \otimes Y \ar[l, "\inv{\alpha_{ZXY}}"] &
      (X \otimes Z) \otimes Y \ar[l, "b_{XZ} \otimes \id{Y}"]
    \end{tikzcd} \]
  \end{df}
  
  To read: A. Joyal and R. Street, Braided monoidal categories, Macquarie Math 
  Reports 860081 (1986). Available
  at http://rutherglen.ics.mq.edu.au/∼street/JS86.pdf.
  A. Joyal and R. Street, Braided tensor categories, Adv. Math. 102 (1993), 
  20–78.
  
  \begin{thm}
    Every Cartesian monoidal category is braided with $b_{XY} = \langle \pi_2, 
    \pi_1 \rangle$.
  \end{thm}
  
  \section{Symmetric Monoidal Categories}
  
  \begin{df}
    A \emph{symmetric monoial category} is a braided monoidal category such 
    that 
    $b_{XY} = \inv{b_{YX}}$ for all objects $X$, $Y$.
  \end{df}
  
  \begin{thm}
    Every Cartesian monoidal category is symmetric monoidal.
  \end{thm}
  
  \section{Closed Monoidal Categories}
  
  \begin{df}[Left Closed Monoidal Categories]
    A \emph{left closed} monoidal category $\mathcal{C}$ consists of:
    \begin{itemize}
      \item a monoidal category $\mathcal{C}$;
      \item a functor $\multimap : \op{\mathcal{C}} \times \mathcal{C} 
      \rightarrow 
      \mathcal{C}$, the \emph{internal hom functor};
      \item a natural isomorphism $c_{XYZ} : \mathcal{C}[X \otimes Y, Z] \cong 
      \mathcal{C}[X, Y \multimap Z]$, called \emph{currying}
    \end{itemize}
  \end{df}
  
  \begin{df}[Right Closed Monoidal Categories]
    A \emph{right closed} monoidal category $\mathcal{C}$ consists of:
    \begin{itemize}
      \item a monoidal category $\mathcal{C}$;
      \item a functor $\multimap : \op{\mathcal{C}} \times \mathcal{C} 
      \rightarrow 
      \mathcal{C}$, the \emph{internal hom functor};
      \item a natural isomorphism $c_{XYZ} : \mathcal{C}[X \otimes Y, Z] \cong 
      \mathcal{C}[Y, X \multimap Z]$, called \emph{currying}
    \end{itemize}
  \end{df}
  
  \begin{lm}
    A braided monoidal category is left closed iff it is right closed.
  \end{lm}
  
  \begin{lm}
    $\Set$ is closed.
  \end{lm}
  
  \begin{df}[Name]
    In a closed braided monoidal category, the \emph{name} of a morphism $f : X 
    \rightarrow Y$ is the morphism
    \[ \ulcorner f \urcorner : I \rightarrow X \multimap Y \]
    defined by
    \[ \ulcorner f \urcorner = c(f \circ r_X) \enspace . \]
  \end{df}
  
  \section{Compact Monoidal Categories}
  
  \begin{df}[Right Dual]
    Let $\mathcal{C}$ be a monoidal category and $X \in \mathcal{C}$. A 
    \emph{right dual} $X^*$ of $X$ consists of:
    \begin{itemize}
      \item an object $X^*$
      \item a morphism $i_X : I \rightarrow X^* \otimes X$, the \emph{unit}
      \item a morphism $e_X : X \otimes X^* \rightarrow I$, the \emph{counit}
    \end{itemize}
    such that the following diagrams (the \emph{zig-zag equations}) commute:
    \[ \begin{tikzcd}
      X \otimes I \ar[r, "\id{X} \otimes i_X"] \ar[d, "r_X"] &
      X \otimes (X^* \otimes X) \ar[r, "\alpha_{XX^*X}"] &
      (X \otimes X^*) \otimes X \ar[d, "e_X \otimes \id{X}"] \\
      X & & I \otimes X \ar[ll, "l_X"]
    \end{tikzcd} \]
    \[ \begin{tikzcd}
      I \otimes X^* \ar[d, "l_{X^*}"] \ar[r, "i_X \otimes \id{X^*}"] &
      (X^* \otimes X) \otimes X^* \ar[r, "\inv{\alpha{X^*XX^*}}"] &
      X^* \otimes (X \otimes X^*) \ar[d, "\id{X^*} \otimes e_X"] \\
      X^* & & X^* \otimes I \ar[ll, "r_{X^*}"]
    \end{tikzcd}\]
    
  \end{df}
  
  \begin{df}[Left Dual]
    Let $\mathcal{C}$ be a monoidal category and $X \in \mathcal{C}$. A 
    \emph{left dual} $X_*$ of $X$ consists of:
    \begin{itemize}
      \item an object $X_*$
      \item a morphism $i_X : I \rightarrow X \otimes X_*$, the \emph{unit}
      \item a morphism $e_X : X_* \otimes X \rightarrow I$, the \emph{counit}
    \end{itemize}
    such that the following diagrams (the \emph{zig-zag equations}) commute:
    \[ \begin{tikzcd}
      X_* \otimes I \ar[r, "\id{X_*} \otimes i_X"] \ar[d, "r_{X_*}"] &
      X_* \otimes (X \otimes X_*) \ar[r, "\alpha_{X_*XX_*}"] &
      (X_* \otimes X) \otimes X_* \ar[d, "e_X \otimes \id{X_*}"] \\
      X_* & & I \otimes X_* \ar[ll, "l_{X_*}"]
    \end{tikzcd} \]
    \[ \begin{tikzcd}
      I \otimes X \ar[d, "l_X"] \ar[r, "i_X \otimes \id{X}"] &
      (X \otimes X_*) \otimes X \ar[r, "\inv{\alpha{XX_*X}}"] &
      X \otimes (X_* \otimes X) \ar[d, "\id{X} \otimes e_X"] \\
      X_* & & X_* \otimes I \ar[ll, "r_{X_*}"]
    \end{tikzcd}\]
    
  \end{df}
  
  \begin{df}[Compact Monoidal Category]
    A monoidal category is \emph{compact} iff every object has a left dual and 
    a 
    right dual.
  \end{df}
  
  \begin{thm}
    Every compact monoidal category is closed with $X \multimap Y = X^* \otimes 
    Y$.
  \end{thm}
  
  \section{Dagger Categories}
  
  \begin{df}[Dagger Category]
    A \emph{dagger category} $\mathcal{C}$ consists of:
    \begin{itemize}
      \item a category $\mathcal{C}$
      \item for every morphism $f : X \rightarrow Y$, a morphism $f^\dagger : Y 
      \rightarrow X$
    \end{itemize}
    such that:
    \begin{itemize}
      \item for every morphism $f : X \rightarrow Y$, we have 
      $(f^\dagger)^\dagger = 
      f$
      \item for any morphisms $f : X \rightarrow Y$ and $g : Y \rightarrow Z$, 
      we 
      have $(g \circ f)^\dagger = f^\dagger \circ g^\dagger$
    \end{itemize}
  \end{df}
  
  \begin{thm}
    $\Set$ is not a dagger category.
  \end{thm}
  
  \begin{proof}
    \pf\ There is no morphism $1 \rightarrow \emptyset$.
  \end{proof}
  
  \chapter{Topology}
  
  \section{Topological Spaces}
  
  \begin{df}[Topology]
    A \emph{topology} on a set $X$ is a set $\mathcal{T} \subseteq \mathcal{P}
    X$, whose elements we call \emph{open} sets, such that:
    \begin{enumerate}
      \item $\emptyset, X \in \mathcal{T}$
      \item For all $\mathcal{A} \subseteq \mathcal{T}$, we have $\bigcup
      \mathcal{A} \in \mathcal{T}$
      \item For all $A, B \in \mathcal{T}$, we have $A \cap B \in \mathcal{T}$.
    \end{enumerate}
    A \emph{topological space} consists of a set $X$ and a topology on $X$.
  \end{df}
  
  \begin{lm}
    \label{lm:topology:open:membership}
    Let $X$ be a topological space and $U \subseteq X$. Then $U$ is open if and 
    only if, for all $x \in U$, there exists $V$ open such that $x \in V 
    \subseteq 
    U$.
  \end{lm}
  
  \begin{proof}
    \pf
    \step{1}{If $U$ is open then, for all $x \in U$, there exists $V$ open such 
      that $x \in V \subseteq U$}
    \begin{proof}
      \pf\ Trivial, taking $V = U$.
    \end{proof}
    \step{2}{If, for all $x \in U$, there exists $V$ open such that $x \in V 
      \subseteq U$, then $U$ is open.}
    \begin{proof}
      \pf\ This follows because $U = \bigcup \{ V \text{ open} : V \subseteq U 
      \}$.
    \end{proof}
  \end{proof}
  
  \begin{df}[Neighbourhood]
    A \emph{neighbourhood} of a point $x$ is an open set that contains $x$.
  \end{df}
  
  \begin{df}[Discrete Topology]
    For any set $X$, the \emph{discrete} topology on $X$ is $\mathcal{P} X$.
  \end{df}
  
  \begin{df}[Indiscrete Topology]
    For any set $X$, the \emph{indiscrete} or \emph{trivial} topology on $X$ is
    $\{ \emptyset, X \}$.
  \end{df}
  
  \begin{df}[Finite Complement Topology]
    For any set $X$, the \emph{finite complement topology} on $X$ is $\{ X
    \setminus F : F \subseteq X \text{ is finite} \}$.
  \end{df}
  
  \begin{df}[Finer, Coarser]
    Let $\mathcal{T}$ and $\mathcal{T}'$ be topologies on $X$. Then:
    \begin{itemize}
      \item
      $\mathcal{T}'$ is \emph{finer} than $\mathcal{T}$, and $\mathcal{T}$ is
      \emph{coarser} than $\mathcal{T}'$, iff $\mathcal{T} \subseteq 
      \mathcal{T}'$
      \item
      $\mathcal{T}'$ is \emph{strictly} finer than $\mathcal{T}$, and 
      $\mathcal{T}$
      is
      \emph{strictly} coarser than $\mathcal{T}'$, iff $\mathcal{T} \subseteq
      \mathcal{T}'$
      \item
      $\mathcal{T}$ and $\mathcal{T}'$ are \emph{comparable} iff one is finer 
      than
      the other.
    \end{itemize}
  \end{df}
  
  \begin{lm}
    The discrete topology is the finest topology on a set. The indiscrete 
    topology 
    is the coarsest topology.
  \end{lm}
  
  \subsection{Closed Sets}
  
  \begin{df}[Closed Set]
    Let $X$ be a topological space. A set $C \subseteq X$ is \emph{closed} iff 
    $X 
    \setminus C$ is open.
  \end{df}
  
  \begin{thm}
    Let $X$ be a topological space and $\mathcal{C}$ the set of closed sets. 
    Then:
    \begin{enumerate}
      \item $\emptyset, X \in \mathcal{C}$
      \item $\mathcal{C}$ is closed under arbitrary intersection.
      \item $\mathcal{C}$ is closed under finite union.
    \end{enumerate}
    
    Conversely, if $\mathcal{C}$ satisfies 1--3, then there exists a unique 
    topology $\mathcal{T}$ such that $\mathcal{C}$ is the set of closed sets 
    under 
    $\mathcal{T}$, namely
    \[ \mathcal{T} = \{ X \setminus C : C \in \mathcal{C} \} \enspace . \]
  \end{thm}
  
  \subsection{Closure}
  
  \begin{df}[Closure]
    Let $X$ be a topological space and $A \subseteq X$. Then the \emph{closure} 
    of 
    $A$, $\overline{A}$, is the intersection of all the closed sets that 
    include 
    $A$.
  \end{df}
  
  \begin{thm}
    \label{thm:topology:closure:membership}
    A point $x$ is in $\overline{A}$ if and only if every open set that 
    contains 
    $x$ intersects $A$.
  \end{thm}
  
  \begin{proof}
    \step{1}{If $x \in \overline{A}$ then every open set that contains $x$ 
      intersects $A$.}
    \begin{proof}
      \step{1a}{\assume{$x \in \overline{A}$}}
      \step{1b}{\pflet{$U$ be an open set that contains $x$}}
      \step{1c}{$x \notin X \setminus U$}
      \step{1d}{$X \setminus U$ does not include $A$.}
      \step{1e}{$U$ intersects $A$}
    \end{proof}
    \step{2}{If every open set that contains $x$ intersects $A$ then $x \in 
      \overline{A}$.}
    \begin{proof}
      \step{xxx}{\assume{Every open set that contains $x$ intersects $A$}}
      \step{2a}{$X \setminus \overline{A}$ does not intersect $A$}
      \step{2b}{$X \setminus \overline{A}$ does not contain $x$}
      \step{2c}{$x \in \overline{A}$}
    \end{proof}
  \end{proof}
  
  \begin{thm}[Kuratowski Closure Axioms]
    Let $X$ be a topological space and $A, B \subseteq X$. Then:
    \begin{enumerate}
      \item $\overline{\emptyset} = \emptyset$
      \item $A \subseteq \overline{A}$
      \item $\overline{\overline{A}} = \overline{A}$
      \item $\overline{A \cup B} = \overline{A} \cup \overline{B}$
    \end{enumerate}
    
    Conversely, let $X$ be a set and $\overline{\ } : \mathcal{P} X \rightarrow 
    \mathcal{P} X$ an operation that satisfies 1--4. Then there exists a unique
    topology on $X$ with respect to which $\overline{ }$ is the closure 
    operator.
  \end{thm}
  
  \begin{lm}
    \label{lm:topology:closure:closed}
    Let $C \subseteq X$. The following are equivalent.
    \begin{enumerate}
      \item $C$ is closed
      \item There exists a set $A$ such that $C = \overline{A}$
      \item $\overline{C} = C$
      \item $\overline{C} \subseteq C$
    \end{enumerate}
  \end{lm}
  
  \subsection{Interior}
  
  \begin{df}[Interior]
    Let $X$ be a topological space and $A \subseteq X$. The \emph{interior} of 
    $A$, $A^\circ$, is the union of all the open subsets of $A$.
  \end{df}
  
  \begin{thm}[Kuratowski Interior Axioms]
    Let $X$ be a topological space and $A, B \subseteq X$. Then:
    \begin{enumerate}
      \item $X^\circ = X$
      \item $A^\circ \subseteq A$
      \item $(A^\circ)^\circ = A^\circ$
      \item $(A \cap B)^\circ = A^\circ \cap B^\circ$
    \end{enumerate}
    
    Conversely, let $X$ be a set and $^\circ : \mathcal{P} X \rightarrow 
    \mathcal{P} X$ an operation that satisfies 1--4. Then there exists a unique
    topology on $X$ with respect to which $^\circ$ is the interior operator.
  \end{thm}
  
  \begin{lm}
    Let $X$ be a topological space and $U \subseteq X$. Then the following are 
    equivalent:
    \begin{enumerate}
      \item $U$ is open.
      \item There exists a set $A$ such that $U = A^\circ$
      \item $U = U^\circ$
      \item $U \subseteq U^\circ$
    \end{enumerate}
  \end{lm}
  
  \subsection{Limit Points}
  
  \begin{df}[Limit Point]
    Let $X$ be a topological space and $A \subseteq X$. A point $l \in X$ is a 
    \emph{limit point} of $A$ iff every neighbourhood of $l$ intersects $A$ in 
    a 
    point other than $l$.
  \end{df}
  
  \begin{thm}
    Let $A$ be a subspace of the topological space $X$. Let $A'$ be the set of 
    limit points of $A$. Then
    \[ \overline{A} = A \cup A' \]
  \end{thm}
  
  \begin{proof}
    \step{1}{If $x \in \overline{A}$ then $x \in A$ or $x \in A'$}
    \begin{proof}
      \step{1a}{\assume{$x \in \overline{A}$ and $x \notin A$}}
      \step{1b}{Every open set that contains $x$ intersects $A$}
      \begin{proof}
        \pf\ Theorem \ref{thm:topology:closure:membership}.
      \end{proof}
      \step{1c}{Every open set that contains $x$ intersects $A$ in a point 
        other 
        than $x$}
      \step{1d}{$x \in A'$}
    \end{proof}
    \step{2}{$A \subseteq \overline{A}$}
    \begin{proof}
      \pf\ By definition of $\overline{A}$
    \end{proof}
    \step{3}{$A' \subseteq \overline{A}$}
    \begin{proof}
      \pf\ From Theorem \ref{thm:topology:closure:membership}.
    \end{proof}
  \end{proof}
  
  \begin{cor}
    A set is closed if and only if it contains its limit points.
  \end{cor}
  
  \subsection{Convergence}
  
  \begin{df}[Converge]
    Let $X$ be a topological space. Let $(a_n)$ be a sequence of points in $X$, 
    and $l \in X$. Then $a_n$ \emph{converges} to the \emph{limit} $l$, $a_n 
    \rightarrow l$ as $n \rightarrow \infty$, iff, for every neighbourhood $U$ 
    of 
    $l$, there exists $N$ such that
    \[ \forall n > N. a_n \in U \enspace . \]
  \end{df}
  
  \begin{lm}
    Let $X$ be a topological space and $A \subseteq X$. If $(a_n)$ is a 
    sequence 
    of points in $A$ and $a_n \rightarrow l$ as $n \rightarrow \infty$, then $l 
    \in 
    \overline{A}$.
  \end{lm}
  
  \begin{proof}
    \pf
    \step{1}{\pflet{$U$ be a neighbourhood of $l$}}
    \step{2}{\pick\ $N$ such that, for all $n \geq N$, we have $a_n \in U$}
    \step{3}{$a_N \in A \cap U$}
    \qedstep
    \begin{proof}
      \pf\ By Theorem \ref{thm:topology:closure:membership}.
    \end{proof}
  \end{proof}
  
  \subsection{Continuous Functions}
  
  \begin{df}[Continuous]
    Let $X$ and $Y$ be topological spaces. A function $f : X \rightarrow Y$ is 
    \emph{continuous} iff, for every open $V \subseteq Y$, we have that 
    $f^{-1}(V)$ 
    is open in $X$.
  \end{df}
  
  \begin{lm}
    \label{lm:topology:continuous:criteria}
    Let $X$ and $Y$ be topological spaces and $f : X \rightarrow Y$. The 
    following 
    are equivalent.
    \begin{enumerate}
      \item $f$ is continuous
      \item For every set $A \subseteq X$, we have $f(\overline{A}) \subseteq 
      \overline{f(A)}$.
      \item For every closed $C \subseteq Y$, we have that $f^{-1}(C)$ is closed 
      in 
      $X$.
      \item For every point $x \in X$ and neighbourhood $V$ of $f(x)$, there 
      exists a 
      neighbourhood $U$ of $x$ such that $f(U) \subseteq V$.
    \end{enumerate}
  \end{lm}
  
  \begin{proof}
    \step{1}{$1 \Rightarrow 2$}
    \begin{proof}
      \step{xxx}{\assume{$f$ is continuous}}
      \step{1a}{\pflet{$A \subseteq X$}}
      \step{1b}{\pflet{$x \in \overline{A}$} \prove{$f(x) \in \overline{f(A)}$}}
      \step{1c}{\pflet{$V$ be a neighbourhood of $f(x)$} \prove{$V$ intersects 
          $f(A)$}}
      \step{1d}{$f^{-1}(V)$ is a neighbourhood of $x$}
      \begin{proof}
        \pf\ Using \stepref{xxx}
      \end{proof}
      \step{1e}{\pick\ $y \in f^{-1}(V) \cap A$}
      \begin{proof}
        \pf\ Using \stepref{1b}
      \end{proof}
      \step{1f}{$f(y) \in V \cap f(A)$}
    \end{proof}
    \step{2}{$2 \Rightarrow 3$}
    \begin{proof}
      \step{2a}{\assume{2 is true}}
      \step{2b}{\pflet{$C \subseteq Y$ be closed}}
      \step{2c}{\pflet{$A = f^{-1}(C)$} \prove{$\overline{A} \subseteq A$}}
      \step{2d}{$f(\overline{A}) \subseteq C$}
      \begin{proof}
        \pf \begin{align*}
          f(\overline{A}) & \subseteq \overline{f(A)} & (\text{\stepref{2a}}) \\
          & \subseteq \overline{C} & (\text{lm:topology:closure:monotone}) \\
          & = C & (\text{\stepref{2b}})
        \end{align*}
      \end{proof}
      \step{2e}{$\overline{A} \subseteq A$}
      \begin{proof}
        \pf\ $\overline{A} \subseteq f^{-1}(C) = A$
      \end{proof}
      \qedstep
      \begin{proof}
        \pf\ By Lemma \ref{lm:topology:closure:closed}
      \end{proof}
    \end{proof}
    \step{3}{$3 \Rightarrow 1$}
    \begin{proof}
      \step{3a}{\assume{3 is true}}
      \step{3b}{\pflet{$V \subseteq Y$ be open}}
      \step{3c}{$f^{-1}(Y \setminus V)$ is closed}
      \begin{proof}
        \pf\ By \stepref{3a}
      \end{proof}
      \step{3d}{$X \setminus f^{-1}(V)$ is closed}
      \step{3e}{$f^{-1}(V)$ is open}
    \end{proof}
    \step{4}{$1 \Rightarrow 4$}
    \begin{proof}
      \pf\ Take $U = f^{-1}(V)$
    \end{proof}
    \step{5}{$4 \Rightarrow 1$}
    \begin{proof}
      \step{5a}{\assume{4 is true}}
      \step{5b}{\pflet{$V \subseteq Y$ be open}}
      \step{5c}{\pflet{$x \in f^{-1}(V)$}}
      \step{5d}{There is a neighbourhood $U$ of $x$ such that $U \subseteq 
        f^{-1}(V)$}
      \begin{proof}
        \pf\ By \stepref{5a}
      \end{proof}
      \step{5e}{$f^{-1}(V)$ is open}
      \begin{proof}
        \pf\ By Lemma \ref{lm:topology:open:membership}.
      \end{proof}
    \end{proof}
  \end{proof}
  
  \begin{lm}
    For any topological space $X$, the identity function on $X$ is continuous.
  \end{lm}
  
  \begin{proof}
    \pf\ For any open set $U$, we have that $\inv{\id{X}}(U) = U$ is open.
  \end{proof}
  
  \begin{lm}
    \label{lm:topology:continuous:composite}
    The composite of two continuous functions is continuous.
  \end{lm}
  
  \begin{proof}
    \step{1}{\pflet{$f : X \rightarrow Y$ and $g : Y \rightarrow Z$ be 
        continuous.}}
    \step{2}{\pflet{$U \subseteq Z$ be open.}}
    \step{3}{$\inv{g}(U)$ is open.}
    \step{4}{$\inf{f}(\inv{g}(U))$ is open.}<
  \end{proof}
  
  \begin{lm}
    Let $f : X \rightarrow Y$ be continuous at $l$. If $x_n \rightarrow l$ as 
    $n 
    \rightarrow \infty$ in $X$, then $f(x_n) \rightarrow f(l)$ as $n 
    \rightarrow 
    \infty$ in $Y$.
  \end{lm}
  
  \begin{proof}
    \pf
    \step{1}{\pflet{$V$ be a neighbourhood of $f(l)$}}
    \step{2}{$\inv{f}(V)$ is a neighbourhood of $l$}
    \begin{proof}
      \pf\ $f$ is continuous.
    \end{proof}
    \step{3}{\pick\ $N$ such that, for all $n \geq N$, $x_n \in \inf{f}(V)$}
    \begin{proof}
      \pf\ $x_n \rightarrow l$ as $n \rightarrow \infty$.
    \end{proof}
    \step{4}{For all $n \geq N$, $f(x_n) \in V$}
    \begin{proof}
      \pf\ Immediate from \stepref{3}.
    \end{proof}
    \qed
  \end{proof}
  
  \begin{df}[Category of Topological Spaces]
    The \emph{category of topological spaces} $\Top$ has objects all 
    topological spaces and morphisms all continuous functions.
  \end{df}
  
  \begin{df}[Homeomorphism]
    A \emph{homeomorphism} is an isomorphism in $\Top$.
    
    Two topological spaces are \emph{homeomorphic} iff they are isomorphic in 
    $\Top$.
  \end{df}
  
  \begin{lm}
    Let $X$ and $Y$ be topological spaces.
    Every constant function $X \rightarrow Y$ is continuous.
  \end{lm}
  
  \begin{proof}
    \step{1}{\pflet{$f : X \rightarrow Y$ be the constant function with value 
        $b$.}}
    \step{2}{\pflet{$U \subseteq Y$ be open}}
    \step{3}{$\inv{f}(U)$ is either $X$ or $\emptyset$}
    \begin{proof}
      \pf\ $\inv{f}(U)$ is $X$ if $b \in U$, and $\emptyset$ if $b \notin U$.
    \end{proof}
  \end{proof}
  
  \subsection{Open and Closed Maps}
  
  \begin{df}[Open Map]
    Let $X$ and $Y$ be topological spaces. A function $f : X \rightarrow Y$ is 
    an 
    \emph{open map} iff, for every open $U \subseteq X$, $f(U)$ is open in $Y$.
  \end{df}
  
  \begin{df}[Closed Map]
    Let $X$ and $Y$ be topological spaces. A function $f : X \rightarrow Y$ is 
    an 
    \emph{closed map} iff, for every closed $C \subseteq X$, $f(C)$ is closed 
    in 
    $Y$.
  \end{df}
  
  \subsection{Bases for Topologies}
  
  \begin{df}
    Let $X$ be a set. A \emph{basis for a topology} on $X$ is a set $\mathcal{B}
    \subseteq \mathcal{P} X$, whose elements we call \emph{basis} elements, such
    that
    \begin{enumerate}
      \item For each $x \in X$, there exists $B \in \mathcal{B}$ such that $x 
      \in B$.
      \item For all $B_1, B_2 \in \mathcal{B}$ and $x \in B_1 \cap B_2$, there
      exists $B_3 \in \mathcal{B}$ such that $x \in B_3 \subseteq B_1 \cap B_2$.
    \end{enumerate}
    
    The \emph{topology generated by $\mathcal{B}$} is then
    \[ \{ U \subseteq X : \forall x \in U. \exists B \in \mathcal{B}. x \in B
    \subseteq U \} \enspace . \]
  \end{df}
  
  \begin{lm}
    The topology generated by a basis is a topology.
  \end{lm}
  
  \begin{proof}
    Let $\mathcal{B}$ be a basis for a topology on $X$, and let $\mathcal{T}$ be
    the topology generated by $\mathcal{B}$.
    \begin{enumerate}
      \item For all $x \in X$, there exists $B \in \mathcal{B}$ such that $x \in 
      B
      \subseteq X$. Thus, $X \in \mathcal{T}$.
      
      Vacuously, for all $x \in \emptyset$, there exists $B \in \mathcal{B}$ 
      such
      that $x \in B \subseteq \emptyset$. Thus, $\emptyset \in \mathcal{T}$.
      \item Let $\mathcal{A} \subseteq \mathcal{T}$. Let $x \in \bigcup 
      \mathcal{A}$.
      Pick $U \in \mathcal{A}$ such that $x \in U$. Pick $B \in \mathcal{B}$ 
      such
      that $x \in B \subseteq U$. Then $x \in B \subseteq \bigcup \mathcal{A}$.
      \item Let $U, V \in \mathcal{T}$. Let $x \in U \cap V$. Pick $B_1, B_2 \in
      \mathcal{B}$ such that $x \in B_1 \subseteq U$ and $x \in B_2 \subseteq 
      V$.
      Pick $B_3 \in \mathcal{B}$ such that $x \in B_3 \subseteq B_1 \cap B_2$. 
      Then
      $x \in B_3 \subseteq U \cap V$.
    \end{enumerate}
  \end{proof}
  
  \begin{lm}
    \label{lm:topology_basis:unions_of_subsets}
    Let $X$ be a set. Let $\mathcal{B}$ be a basis for a topology on $X$. Then 
    the
    topology generated by $\mathcal{B}$ is the set of all unions of subsets of
    $\mathcal{B}$.
  \end{lm}
  
  \begin{proof}
    Let $\mathcal{T}$ be the topology generated by $\mathcal{B}$.
    
    Let $U \in \mathcal{T}$. Let $\mathcal{A} = \{ B \in \mathcal{B} : B 
    \subseteq
    U \}$. Clearly $\bigcup \mathcal{A} \subseteq U$. Also, for all $x \in U$,
    there exists $B \in \mathcal{B}$ such that $x \in B \subseteq U$. We have $B
    \in \mathcal{A}$ hence $U \subseteq \bigcup \mathcal{A}$. Thus, $U = \bigcup
    \mathcal{A}$.
    
    Conversely, let $\mathcal{A} \subseteq \mathcal{B}$. Let $x \in \bigcup
    \mathcal{A}$. Then there exists $B \in \mathcal{A}$ such that $x \in B$, 
    hence
    there exists $B \in \mathcal{B}$ such that $x \in B \subseteq \bigcup
    \mathcal{A}$. Thus, $\bigcup \mathcal{A} \in \mathcal{T}$.
  \end{proof}
  
  \begin{lm}
    \label{lm:topology:basis:criterion}
    Let $X$ be a topological space. Suppose $\mathcal{B}$ is a set of open sets 
    such that, for
    every open set $U$ and point $x \in U$, there exists $B \in \mathcal{B}$ 
    such 
    that $x \in B \subseteq U$.
    Then $\mathcal{B}$ is a basis for the topology on $X$.
  \end{lm}
  
  \begin{proof}
    \step{1}{$\mathcal{B}$ is the basis for a topology on $X$.}
    \begin{proof}
      \step{1a}{For all $x \in X$, there exists $B \in \mathcal{B}$ such that 
        $x 
        \in B$.}
      \begin{proof}
        \pf\ This follows from the hypothesis, since $X$ is open.
      \end{proof}
      \step{1b}{For all $B_1, B_2 \in \mathcal{B}$ and $x \in B_1 \cap B_2$, 
        there 
        exists $B_3 \in \mathcal{B}$ such that $x \in B_3 \subseteq B_1 \cap 
        B_2$.}
      \begin{proof}
        \pf\ This follows from the hypothesis, since $B_1 \cap B_2$ is open.
      \end{proof}
    \end{proof}
    \step{2}{The topology on $X$ is the one generated by $\mathcal{B}$.}
    \begin{proof}
      \step{2a}{For every open set $U$, we have $\forall x \in U. \exists B \in 
        \mathcal{B}. x \in B \subseteq U$}
      \begin{proof}
        \pf\ By the hypothesis.
      \end{proof}
      \step{2b}{For any set $U$, if $\forall x \in U. \exists B \in 
        \mathcal{B}. x 
        \in B \subseteq U$ then $U$ is open.}
      \begin{proof}
        \step{i}{\pflet\ $U$ be a set such that $\forall x \in U. \exists B \in 
          \mathcal{B}. x \in B \subseteq U$}
        \step{ii}{$U = \bigcup \{ B \in \mathcal{B}: B \subseteq U \}$}
        \begin{proof}
          \step{a}{$\forall x \in U. \exists B \in \mathcal{B}. x \in B 
            \subseteq 
            U$}
          \begin{proof}
            \pf\ By \stepref{i}.
          \end{proof}
          \step{b}{If $x \in B \in \mathcal{B}$ and $B \subseteq U$ then $x \in 
            U$.}
          \begin{proof}
            \pf\ This holds just by the definition of $\subseteq$.
          \end{proof}
        \end{proof}
        \qedstep
        \begin{proof}
          \pf\ $U$ is the union of a subset of $\mathcal{B}$, and every element 
          of 
          $\mathcal{B}$ is open.
        \end{proof}
      \end{proof}
    \end{proof}
  \end{proof}
  
  \begin{lm}
    \label{lm:topology:basis:criterion2}
    Let $\mathcal{B}$ be a basis for the topology on $X$. Let $\mathcal{B}'$ be 
    a 
    set of open sets such that, for all $B \in \mathcal{B}$ and $x \in B$, 
    there 
    exists $B' \in \mathcal{B}'$ such that $x \in B' \subseteq B$. Then 
    $\mathcal{B}'$ is also a basis for the topology on $X$.
  \end{lm}
  
  \begin{proof}
    \step{1}{\pflet{$U$ be an open set and $x \in U$}}
    \step{2}{\pick\ $B \in \mathcal{B}$ such that $x \in B \subseteq U$}
    \step{3}{\pick\ $B' \in \mathcal{B}'$ such that $x \in B' \subseteq B$}
    \step{4}{$x \in B' \subseteq U$}
    \qedstep
    \begin{proof}
      \pf\ By Lemma \ref{lm:topology:basis:criterion}.
    \end{proof}
  \end{proof}
  
  \begin{lm}
    \label{lm:topology_basis:finer}
    Let $\mathcal{B}$ and $\mathcal{B}'$ be bases for the topologies 
    $\mathcal{T}$ 
    and $\mathcal{T}'$ on $X$, respectively. Then $\mathcal{T}'$ is finer than 
    $\mathcal{T}$ if and only if, for all $B \in \mathcal{B}$ and $x \in B$, 
    there 
    exists $B' \in \mathcal{B}'$ such that $x \in B' \subseteq B$.
  \end{lm}
  
  \begin{proof}
    \step{1}{If $\mathcal{T}$ is finer than $\mathcal{T}'$ then, for all $B \in 
      \mathcal{B}$ and $x \in B$, there exists $B' \in \mathcal{B}'$ such that 
      $x  
      \in B' \subseteq B$.}
    \begin{proof}
      \step{1a}{\assume{$\mathcal{T}'$ is finer than $\mathcal{T}$}}
      \step{1b}{\pflet{$B \in \mathcal{B}$ and $x \in B$}}
      \step{1c}{$B \in \mathcal{T}$}
      \begin{proof}
        \pf\ From \stepref{1b} and Lemma 
        \ref{lm:topology_basis:unions_of_subsets}.
      \end{proof}
      \step{1d}{$B \in \mathcal{T}'$}
      \begin{proof}
        \pf\ By \stepref{1a} and \stepref{1c}
      \end{proof}
      \step{1e}{There exists $B' \in \mathcal{B}'$ such that $x \in B' 
        \subseteq 
        B$}
      \begin{proof}
        \pf\ By \stepref{1d} and the definition of $\mathcal{T}'$.
      \end{proof}
    \end{proof}
    \step{2}{If, for all $B \in \mathcal{B}$ and $x \in B$, there exists $B' 
      \in 
      \mathcal{B}'$ such that $x \in B' \subseteq B$, then $\mathcal{T}'$ is 
      finer 
      than $\mathcal{T}$}
    \begin{proof}
      \step{2a}{\assume{For all $B \in \mathcal{B}$ and $x \in B$, there exists 
          $B'     \in \mathcal{B}'$ such that $x \in B' \subseteq B$}}
      \step{2b}{\pflet{$U \in \mathcal{T}$} \prove{$U \in \mathcal{T}'$}}
      \step{2c}{For all $x \in U$, there exists $B' \in \mathcal{B}'$ such that 
        $x     \in B' \subseteq U$}
      \begin{proof}
        \step{i}{\pflet{$x \in U$}}
        \step{ii}{\pick\ $B \in \mathcal{B}$ such that $x \in B \subseteq U$}
        \begin{proof}
          \pf\ From \stepref{2b} and \stepref{i}
        \end{proof}
        \step{iii}{\pick\ $B' \in \mathcal{B}'$ such that $x \in B' \subseteq 
          B$}
        \begin{proof}
          \pf\ From \stepref{2a} and \stepref{ii}
        \end{proof}
        \step{iv}{$B' \subseteq U$}
        \begin{proof}
          \pf\ From \stepref{ii} and \stepref{iii}
        \end{proof}
      \end{proof}
      \qedstep
      \begin{proof}
        \pf\ By the definition of $\mathcal{T}'$
      \end{proof}
    \end{proof}
  \end{proof}
  
  \begin{lm}
    \label{lm:topology:closure:membership2}
    Let $\mathcal{B}$ be a basis. Then $x \in \overline{A}$ if and only if 
    every 
    basis element $B \in \mathcal{B}$ that contains $x$ intersects $A$.
  \end{lm}
  
  \begin{proof}
    \step{1}{If $x \in \overline{A}$ then every basis element $B \in 
      \mathcal{B}$ 
      that contains $x$ intersects $A$.}
    \begin{proof}
      \pf\ Immediate from Theorem \ref{thm:topology:closure:membership}.
    \end{proof}
    \step{2}{If every basis element $B \in \mathcal{B}$ that contains $x$ 
      intersects $A$ then $x \in \overline{A}$.}
    \begin{proof}
      \step{2a}{\assume{Every basis element $B \in \mathcal{B}$ that contains 
          $x$ 
          intersects $A$.}}
      \step{2b}{\pflet{$U$ be a neighbourhood of $x$.}}
      \step{2c}{\pick\ $B \in \mathcal{B}$ such that $x \in B \subseteq U$}
      \step{2d}{$B$ intersects $A$.}
      \begin{proof}
        \pf\ By \stepref{2a}
      \end{proof}
      \step{2e}{$U$ intersects $A$.}
      \qedstep
      \begin{proof}
        \pf\ Be Theorem \ref{thm:topology:closure:membership}.
      \end{proof}
    \end{proof}
  \end{proof}
  
  \begin{df}[Lower Limit Topology on $\mathbb{R}$]
    The \emph{lower limit topology} on $\mathbb{R}$ is the topology generated 
    by 
    the half-open intervals of the form $[a,b)$.
    
    We write $\mathbb{R}_l$ for the topological space consisting of 
    $\mathbb{R}$ 
    under the lower limit topology.
    
    We prove that these intervals form a basis for a topology.
  \end{df}
  
  \begin{proof}
    \step{1}{For all $x \in \mathbb{R}$, there exist $a$, $b$ such 
      that $x \in [a,b)$.}
    \begin{proof}
      \pf\ We have $x \in [x,x+1)$.
    \end{proof}
    \step{2}{For any $a$, $b$, $c$, $d$ and $x \in [a, b) \cap 
      [c, 
      d)$, there exist $e$, $f$ such that $x \in [e,f) \subseteq [a, b) \cap 
      [c, d)$.}
    \begin{proof}
      \pf\ Take $e = \max(a,c)$ and $f= \min(b,d)$.
    \end{proof}
  \end{proof}
  
  \begin{df}[K-topology on $\mathbb{R}$]
    The \emph{K-topology} on $\mathbb{R}$ is the topology generated by 
    the open intervals $(a,b)$, together with all sets of the form $(a,b) 
    \setminus 
    K$ where $K = \{ 1/n : n \in \mathbb{Z}^+ \}$.
    
    We write $\mathbb{R}_K$ for the topological space consisting of 
    $\mathbb{R}$ 
    under the K-topology.
    
    We prove that these sets form a basis for a topology.
  \end{df}
  
  \begin{proof}
    \step{1}{\pflet{$\mathcal{B} = \{ (a,b) : a < b \} \cup \{ (a,b) \setminus K 
        : 
        a < b \}$}}
    \step{2}{For all $x \in \mathbb{R}$, there exists $B \in \mathcal{B}$ such 
      that $x \in B$.}
    \begin{proof}
      \pf\ Take $B = (x-1,x+1)$.
    \end{proof}
    \step{2}{For any $B_1, B_2 \in \mathcal{B}$ and $x \in B_1 \cap B_2$, there 
      exists $B_3 \in \mathcal{B}$ such that $x \in B_3 \subseteq B_1 \cap 
      B_2$.}
    \begin{proof}
      \step{2a}{\case{$B_1 = (a,b)$, $B_2 = (c,d)$}}
      \begin{proof}
        \pf\ Take $B_3 = (\max(a,c), \min(b,d))$
      \end{proof}
      \step{2b}{\case{$B_1 = (a,b) \setminus K$, $B_2 = (c,d)$}}
      \begin{proof}
        \pf\ Take $B_3 = (\max(a,c), \min(b,d)) \setminus K$
      \end{proof}
      \step{2c}{\case{$B_1 = (a,b)$, $B_2 = (c,d) \setminus K$}}
      \begin{proof}
        \pf\ Take $B_3 = (\max(a,c), \min(b,d)) \setminus K$
      \end{proof}
      \step{2b}{\case{$B_1 = (a,b) \setminus K$, $B_2 = (c,d) \setminus K$}}
      \begin{proof}
        \pf\ Take $B_3 = (\max(a,c), \min(b,d)) \setminus K$
      \end{proof}
    \end{proof}
  \end{proof}
  
  \begin{lm}
    The lower limit topology and the K-topology are incomparable.
  \end{lm}
  
  \begin{proof}
    \step{1}{$[0,1)$ is not open in the K-topology.}
    \begin{proof}
      \pf\ There is no open interval $(a,b)$ such that $0 \in (a,b) \setminus 
      [0,1)$ or $0 \in (a,b) \setminus K \subseteq [0,1)$.
    \end{proof}
    \step{2}{$(-1,1) \setminus K$ is not open in the lower limit topology.}
    \begin{proof}
      \pf\ There is no interval $[a,b)$ such that $0 \in [a,b) \subseteq (-1,1) 
      \setminus K$, as any such interval would have to include elements of $K$.
    \end{proof}
  \end{proof}
  
  \begin{lm}
    \label{lm:topology:basis:continuous}
    Let $X$ and $Y$ be topological spaces and
    let $f : X \rightarrow Y$. Let $\mathcal{B}$ be a basis for $Y$. Then $f$ 
    is 
    continuous if and only if, for all $B \in \mathcal{B}$, we have $\inv{f}(B)$ 
    is 
    open in $X$.
  \end{lm}
  
  \begin{proof}
    \step{1}{If $f$ is continuous then, for all $B \in \mathcal{B}$, we have 
      $\inv{f}(B)$ is open.}
    \begin{proof}
      \pf\ This holds because every $B \in \mathcal{B}$ is open by Lemma 
      \ref{lm:topology_basis:unions_of_subsets}.
    \end{proof}
    \step{2}{If, for all $B \in \mathcal{B}$, we have $\inv{f}(B)$ is open, 
      then 
      $f$ is continuous.}
    \begin{proof}
      \step{2a}{\assume{for all $B \in \mathcal{B}$, we have $\inv{f}(B)$ is 
          open.}}
      \step{2b}{\pflet{$V \subseteq Y$ be open}}
      \step{2c}{\pflet{$x \in \inv{f}(V)$}}
      \step{2d}{\pick\ $B \in \mathcal{B}$ such that $f(x) \in B \subseteq V$}
      \step{2e}{$x \in \inv{f}(B) \subseteq \inv{f}(V)$}
      \step{2f}{$\inv{f}(B)$ is open}
      \begin{proof}
        \pf\ By \stepref{2a}
      \end{proof}
      \qedstep
      \begin{proof}
        \pf\ By Lemma \ref{lm:topology:open:membership}.
      \end{proof}
    \end{proof}
  \end{proof}
  
  \subsection{Subbases}
  
  \begin{df}[Subbasis]
    Let $X$ be a set. A \emph{subbasis for a topology} on $X$ is a set 
    $\mathcal{S} \subseteq \mathcal{P} X$ such that $\bigcup \mathcal{S} = X$.
    
    The \emph{topology generated by $\mathcal{S}$} is then the set of all unions 
    of 
    finite intersections of elements of $\mathcal{S}$.
  \end{df}
  
  \begin{lm}
    The topology generated by a subbasis $\mathcal{S}$ is a topology.
  \end{lm}
  
  \begin{proof}
    \step{1}{\pflet{$\mathcal{B}$ be the set of all finite intersections of 
        elements of $\mathcal{S}$.}}
    \step{2}{$\mathcal{B}$ is a basis for a topology on $X$.}
    \begin{proof}
      \step{2a}{For all $x \in X$, there exists $B \in \mathcal{B}$ such that 
        $x 
        \in B$.}
      \begin{proof}       
        \step{i}{\pflet{$x \in X$}}
        \step{ii}{\pick\ $S in \mathcal{S}$ such that $x \in S$}
        \step{iii}{$S \in \mathcal{B}$}
      \end{proof}
      \step{2b}{Given $B_1, B_2 \in \mathcal{B}$ and $x \in B_1 \cap B_2$, 
        there 
        exists $B_3 \in \mathcal{B}$ such that $x \in B_3 \subseteq B_1 \cap 
        B_2$.}
      \begin{proof}
        \pf\ Take $B_3 = B_1 \cap B_2$, which is in $\mathcal{B}$ because the 
        intersection of two finite intersections of elements of $\mathcal{S}$ is 
        a 
        finite intersection of elements of $\mathcal{S}$.
      \end{proof}
    \end{proof}
    \qedstep
    \begin{proof}
      \pf\ By Lemma \ref{lm:topology_basis:unions_of_subsets}.
    \end{proof}
  \end{proof}
  
  \begin{lm}
    \label{lm:topology:subbasis:basis}
    Let $\mathcal{S}$ be a subbasis for the topology $\mathcal{T}$ on $X$. Then 
    the set of all finite intersections of elements of $\mathcal{S}$ is a basis 
    for 
    $\mathcal{T}$.
  \end{lm}
  
  \begin{proof}
    \step{1}{\pflet{$\mathcal{B}$ be the set of all finite intersections of 
        elements of $\mathcal{S}$}}
    \step{2}{$\mathcal{B}$ is a basis for a topology on $X$.}
    \begin{proof}
      \step{2a}{For all $x \in X$, there exists $B \in \mathcal{B}$ such that 
        $x 
        \in B$.}
      \begin{proof}
        \step{i}{\pick\ $S \in \mathcal{S}$ such that $x \in S$}
        \begin{proof}
          \pf\ Since $X = \bigcup \mathcal{S}$
        \end{proof}
        \step{ii}{Take $B = S$}
      \end{proof}
      \step{2b}{For all $B_1, B_2 \in \mathcal{B}$ and $x \in B_1 \cap B_2$, 
        there exists $B_3 \in \mathcal{B}$ such that $x \in B_3 \subseteq B_1 
        \cap 
        B_2$.}
      \begin{proof}
        \pf\ Take $B_3 = B_1 \cap B_2$
      \end{proof}
    \end{proof}
    \step{3}{$\mathcal{T}$ is the set of all unions of elements of 
      $\mathcal{B}$.}
    \begin{proof}
      \pf\ Immediate from definitions.
    \end{proof}
  \end{proof}
  
  \subsection{Local Bases}
  
  \begin{df}[Local Basis]
    Let $X$ be a topological space and $x \in X$. A \emph{(local) basis at the 
      point $x$} $\mathcal{B}$ is a set of neighbourhoods of $x$ such that 
    every 
    neighbourhood of $x$ includes a member of $\mathcal{B}$.
  \end{df}
  
  \subsection{Order Topology}
  
  \begin{df}[Order Topology]
    Let $X$ be a linearly ordered set with more than one element. The 
    \emph{order 
      topology} is the topology generated by the basis $\mathcal{B}$ consisting 
    of:
    \begin{itemize}
      \item all open intervals $(a,b)$ with $a < b$
      \item all intervals of the form $[\bot, b)$ where $\bot$ is the least 
      element 
      of $X$, if any
      \item all intervals of the form $(a, \top]$ where $\top$ is the greatest 
      element of $X$, if any
    \end{itemize}
    
    We prove that $\mathcal{B}$ is a basis for a topology.
  \end{df}
  
  \begin{proof}
    \step{1}{For all $x \in X$, there exists $B \in \mathcal{B}$ such that $x 
      \in 
      B$}
    \begin{proof}
      \step{1a}{If $x$ is the least element, then $x \in [x, b)$ where $b$ is 
        any 
        other element of $X$.}
      \step{1b}{If $x$ is the greatest element, then $x \in (a, x]$ where $a$ 
        is 
        any other element of $X$.}
      \step{1c}{If $x$ is neither least nor greatest, then $x \in (a, b)$ where 
        $a$ is any element such that $a < x$ and $b$ is any element such that $x 
        < b$.}
    \end{proof}
    \step{2}{For all $B_1, B_2 \in \mathcal{B}$ and $x \in B_1 \cap B_2$, there 
      exists $B_3 \in \mathcal{B}$ such that $x \in B_3 \subseteq B_1 \cap B_2$}
    \begin{proof}
      \step{2a}{\case{$B_1 = (a,b)$, $B_2 = (c,d)$}}
      \begin{proof}
        \pf\ Take $B_3 = (\max(a,c), \min(b,d))$
      \end{proof}
      \step{2b}{\case{$B_1 = (a,b)$, $B_2 = [\bot, d)$}}
      \begin{proof}
        \pf\ Take $B_3 = (a, \min(b,d))$
      \end{proof}
      \step{2c}{\case{$B_1 = (a,b)$, $B_2 = (c, \top]$}}
      \begin{proof}
        \pf\ Take $B_3 = (\max(a,c), b)$
      \end{proof}
      \step{2d}{\case{$B_1 = [\bot, b)$, $B_2 = [\bot, d)$}}
      \begin{proof}
        \pf\ Take $B_3 = [\bot, \min(b,d))$
      \end{proof}
      \step{2e}{\case{$B_1 = [\bot, b)$, $B_2 = (c, \top]$}}
      \begin{proof}
        \pf\ Take $B_3 = (c, b)$
      \end{proof}
      \step{2f}{\case{$B_1 = (a, \top]$, $B_2 = (c, \top]$}}
      \begin{proof}
        \pf\ Take $B_3 = (\max(a,c), \top]$
      \end{proof}
    \end{proof}
  \end{proof}
  
  \begin{lm}
    \label{lm:topology:order:rays}
    The open rays $(a, +\infty)$ and $(-\infty, a)$ form a subbasis for the 
    order 
    topology.
  \end{lm}
  
  \begin{proof}
    \step{1}{Every open ray is open in the order topology.}
    \begin{proof}
      \step{1a}{For all $a$, we have $(a, +\infty)$ is open.}
      \begin{proof}
        \step{i}{If there is a greatest element, then $(a, +\infty) = (a, 
          \top]$}
        \step{ii}{If not, then $(a, +\infty) = \bigcup \{ (a, b) : b > a \}$}
      \end{proof}
      \step{1b}{For all $a$, we have $(-\infty, a)$ is open.}
      \begin{proof}
        \pf\ Similar.
      \end{proof}
    \end{proof}
    \step{2}{Every basis element of the order topolgy is a finite intersection 
      of
      open rays.}
    \begin{proof}
      \step{i}{$(a, b) = (a, +\infty) \cap (-\infty, b)$}
      \step{ii}{$(a, \top]= (a, +\infty)$}
      \step{iii}{$[\bot, b) = (-\infty, b)$}
    \end{proof}
  \end{proof}
  
  \begin{df}[Standard Topology on $\mathbb{R}$]
    The \emph{standard topology} on $\mathbb{R}$ is the order topology.
  \end{df}
  
  \begin{lm}
    The lower limit topology is strictly finer than the standard topology.
  \end{lm}
  
  \begin{proof}
    \step{1}{The lower limit topology is finer than the standard topology.}
    \begin{proof}
      \step{1a}{\pflet{$x \in (a,b)$}}
      \step{1b}{$x \in [x,b) \subseteq (a,b)$}
      \qedstep
      \begin{proof}
        \pf\ By Lemma \ref{lm:topology_basis:finer}.
      \end{proof}
    \end{proof}
    \step{2}{$[0,1)$ is not open in the standard topology.}
    \begin{proof}
      \pf\ There is no open interval $(a,b)$ such that $0 \in (a,b) \subseteq 
      [0,1)$.
    \end{proof}
  \end{proof}
  
  \begin{lm}
    The K-topology is strictly finer than the standard topology.
  \end{lm}
  
  \begin{proof}
    \step{1}{The K-topology is finer than the standard topology.}
    \begin{proof}
      \pf\ Every open interval is open in the K-topology.
    \end{proof}
    \step{2}{$(-1,1) \setminus K$ is not open in the standard topology.}
    \begin{proof}
      \pf\ There is no open interval $I$ such that $0 \in I \subseteq (-1, 1) 
      \setminus K$, because any such $I$ would have to include elements of $K$.
    \end{proof}
  \end{proof}
  
  \subsection{Product Topology}
  
  \begin{df}
    Let $\{ X_j \}_{j \in J}$ be a family of topological spaces. For $j \in J$, 
    let $\mathcal{S}_j = \{ \inv{\pi_j}(U) : U \text{ open in } X_j \}$ and 
    $\mathcal{S} = \bigcup_{j \in J} \mathcal{S}_j$. Then the \emph{product 
      topology} on $\prod_{j \in J} X_j$ is the topology generated by the 
    subbasis 
    $\mathcal{S}$. The \emph{product space} $\prod_{j \in J} X_j$ is the set 
    $\prod_{j \in J} X_j$ under the product topology.
    
    We prove this is a subbasis.
  \end{df}
  
  \begin{proof}
    \step{1}{For all $x \in \prod_{j \in J} X_j$, there exists $S \in 
      \mathcal{S}$ 
      such that $x \in S$}
    \begin{proof}
      \step{1a}{\pflet{$x \in \prod_{j \in J} X_j$}}
      \step{1b}{\pick\ $j \in J$}
      \step{1c}{$x \in \inv{\pi_j}(X_j) \in \mathcal{S}$}
    \end{proof}
  \end{proof}
  
  \begin{thm}
    $\prod_{j \in J} X_j$ under the product topology is the product of the 
    family 
    $X_j$ in $\Top$ with projections $\pi_j : \prod_{j \in J} X_j \rightarrow 
    X_j$ 
    the same as the projections in $\Set$.
  \end{thm}
  
  \begin{proof}
    \step{1}{The projections $\pi_j$ are continuous.}
    \begin{proof}
      \pf\ By construction, if $U$ is open in $X_j$ then $\inv{\pi_j}(U)$ is 
      open.
    \end{proof}
    \step{2}{Given continuous functions $f_j : A \rightarrow X_j$ for each $j$, 
      the function $\langle f_j \mid j \in J \rangle : A \rightarrow \prod_{j 
        \in J} 
      X_j$ is continuous.}
    \begin{proof}
      \pf\ For $U$ open in $X_j$, we have
      $\inv{(\langle f_j \mid j \in J \rangle)}(\inv{\pi_j}(U)) = \inv{f_j}(U)$ 
      is open in $A$.
    \end{proof}
  \end{proof}
  
  \begin{cor}
    The forgetful functor $\Top \rightarrow \Set$ preserves products.
  \end{cor}
  
  \begin{cor}
    \label{cor:topology:product:continuous}
    Let $A$ be a topological space and $f : A \rightarrow \prod_{j \in J} X_j$. 
    Then $f$ is continuous if and only if $\pi_j \circ f$ is continuous for all 
    $j 
    \in J$.
  \end{cor}
  
  \begin{thm}
    \label{thm:topology:product:basis1}
    The product topology on $\prod_{j \in J} X_j$ has as basis all sets of the 
    form $\prod_{j \in J} U_j$, where each $U_j$ is open in $X_j$ and $U_j = 
    X_j$ 
    except for finitely many values of $j$.
  \end{thm}
  
  \begin{proof}
    \step{1}{The finite intersections of elements of the canonical subbasis 
      $\mathcal{S}$ are exactly the sets of the form $\prod_{j \in J} U_j$, 
      where each 
      $U_j$ is open in $X_j$ and $U_j = X_j$ 
      except for finitely many values of $j$.}
    \begin{proof}
      \step{1a}{Every element of $\mathcal{S}$ has this form.}
      \begin{proof}
        \pf\ $\inv{\pi_j}(U) = \prod_{j' \in J} U_{j'}$, where $U_j = U$ and 
        $U_{j'} = X_{j'}$ for $j' \neq j$.
      \end{proof}
      \step{1b}{The intersection of two sets of this form has this form.}
      \begin{proof}
        \step{i}{\pflet{$U = \prod_j U_j$ and $V = \prod_j V_j$ be of this 
            form.}}
        \step{ii}{$U \cap V = \prod_j (U_j \cap V_j)$}
        \step{iii}{$U_j \cap V_j = X_j$ for all but finitely many values of 
          $j$.}
        \begin{proof}
          \pf\ If $U_j = X_j$ and $V_j = X_j$ then $U_j \cap V_j = X_j$.
        \end{proof}
      \end{proof}
      \step{1c}{Every set of this form is a finite intersection of elements of 
        $\mathcal{S}$.}
      \begin{proof}
        \step{i}{\pflet{$U = \prod_j U_j$ where $U_j = X_j$ except for $j = 
            j_1, 
            \ldots, j_n$.}}
        \step{ii}{$U = \inv{\pi_{j_1}}(U_{j_1}) \cap \cdots \cap 
          \inv{\pi_{j_n}}(U_{j_n})$}
      \end{proof}
    \end{proof}
    \qedstep
    \begin{proof}
      \pf\ By Lemma \ref{lm:topology:subbasis:basis}.
    \end{proof}
  \end{proof}
  
  \begin{thm}
    \label{thm:topology:product:basis}
    Let $\{ X_j \}_{j \in J}$ be a family of topological spaces and 
    $\mathcal{B}_j$ 
    be a basis for $X_j$ for all $j$. Then
    \[ \mathcal{B} = \left\{ \prod_{j \in J} B_j : \forall j \in J. B_j \in 
    \mathcal{B}_j, B_j = X_j \text{ for all but finitely many } j \right\} \]
    is a basis for the product topology on $\prod_{j \in J} X_j$.
  \end{thm}
  
  \begin{proof}
    \step{1}{Every element of $\mathcal{B}$ is open in the product topology.}
    \begin{proof}
      \pf\ This holds because if $B_j \in \mathcal{B}_j$ then $B_j$ is open in 
      $X_j$.
    \end{proof}
    \step{2}{If $U_j$ is open in $X_j$ for all $j$, $U_j = X_j$ for all but 
      finitely many $j$, and $x \in \prod_{j \in J} U_j$, then there exists $B 
      \in 
      \mathcal{B}$ such that $x \in B \subseteq \prod_{j \in J} U_j$}
    \begin{proof}
      \step{2a}{For $j \in J$ such that $U_j \neq X_j$, \pick\ $B_j \in 
        \mathcal{B}_j$ such that $x_j \in B_j \subseteq U_j$}
      \step{2b}{For $j \in J$ such that $U_j = X_j$, let $B_j = X_j$}
      \step{2c}{Take $B = \prod_{j \in J} B_j$}
    \end{proof}
    \qedstep
    \begin{proof}
      \pf\ By Lemma \ref{lm:topology:basis:criterion2}.
    \end{proof}
  \end{proof}
  
  \begin{thm}
    Let $\{ X_j \}_{j \in J}$ be a family of topological spaces. Let $A_j 
    \subseteq X_j$ for all $j \in J$. Then
    \[ \prod_{j \in J} \overline{A_j} = \overline{\prod_{j \in J} A_j} \]
  \end{thm}
  
  \begin{proof}
    \step{1}{$\prod_{j \in J} \overline{A_j} = \overline{\prod_{j \in J} A_j}$}
    \begin{proof}
      \step{1a}{\pflet{$x \in \prod_{j \in J} \overline{A_j}$}}
      \step{1b}{\pflet{$x \in \prod_{j \in J} U_j$ where each $U_j$ is open in 
          $X_j$ and $U_j = X_j$ for all but finitely many $j$, say $j = j_1, 
          \ldots, 
          j_n$}}
      \step{1c}{For $j \in j$, \pick\ $y_j \in U_j \cap A_j$}
      \begin{proof}
        \pf\ Possible as $x_j \in \overline{A_j} \cap U_j$. We do not need the 
        Axiom of Choice as we can take $y_j = x_j$ for $j \neq j_1, \ldots, 
        j_n$.
      \end{proof}
      \step{1d}{$y \in \prod_{j \in J} U_j$ and $y \in \prod_{j \in J} A_j$}
      \qedstep
      \begin{proof}
        \pf\ By Lemma \ref{lm:topology:closure:membership2}.
      \end{proof}
    \end{proof}
    \step{2}{$\overline{\prod_{j \in J} A_j} \subseteq \prod_{j \in J} 
      \overline{A_j}$}
    \begin{proof}
      \step{2a}{\pflet{$x \in \overline{\prod{j \in J} A_j}$}}
      \step{2b}{\pflet{$j \in J$} \prove{$x_j \in \overline{A_j}$}}
      \step{2c}{\pflet{$U$ be a neighbourhood of $x_j$}}
      \step{2d}{$\inv{\pi_j}(U)$ is a neighbourhood of $x$}
      \step{2e}{$\inv{\pi_j}(U)$ intersects $\prod_{j \in J} A_j$ in $y$, say}
      \step{2f}{$y_j \in U$ and $y_j \in A_j$}
    \end{proof}
  \end{proof}
  
  \subsection{Subspace Topology}
  
  \begin{df}[Subspace Topology]
    Let $X$ be a topological space and $Y \subseteq X$. The \emph{subspace 
      topology} on $Y$ is $\mathcal{T} = \{ U \cap Y : U \text{ open in } X \}$.
    
    A \emph{subspace} of $X$ is a subset of $X$ under the subspace topology.
    
    We prove that the subspace topology is a topology.
  \end{df}
  
  \begin{proof}
    \step{1}{$\emptyset \in \mathcal{T}$}
    \begin{proof}
      \pf\ $\emptyset = \emptyset \cap Y$
    \end{proof}
    \step{2}{$Y \in \mathcal{T}$}
    \begin{proof}
      \pf\ $Y = X \cap Y$
    \end{proof}
    \step{3}{$\mathcal{T}$ is closed under finite intersection}
    \begin{proof}
      \pf\ $(U \cap Y) \cap (V \cap Y) = (U \cap V) \cap Y$
    \end{proof}
    \step{4}{$\mathcal{T}$ is closed under union}
    \begin{proof}
      \pf\ For $\mathcal{A} \subseteq \mathcal{T}$, we have $\bigcup \mathcal{A} 
      = 
      (\bigcup \{ U : U \text{ open in } X, U \cap Y \in \mathcal{A} \}) \cap Y$
    \end{proof}
  \end{proof}
  
  \begin{lm}
    \label{lm:topology:subspace:basis}
    Let $Y \subseteq X$. If $\mathcal{B}$ is a basis for $X$. then $\{ B \cap Y 
    : 
    B \in \mathcal{B} \}$ is a basis for $Y$.
  \end{lm}
  
  \begin{proof}
    \step{1}{For all $B \in \mathcal{B}$, we have $B \cap Y$ is open in $Y$.}
    \step{2}{For all $U$ open in $Y$ and $y \in U$, there exists $B \in 
      \mathcal{B}$ such that $y \in B \cap Y \subseteq U$.} 
    \begin{proof}
      \step{2a}{\pflet{$U$ be open in $Y$ and $y \in U$}}
      \step{2b}{\pick\ $V$ open in $X$ such that $U = V \cap Y$}
      \step{2c}{\pick\ $B \in \mathcal{B}$ such that $y \in B \subseteq V$}
      \step{2d}{$y \in B \cap Y \subseteq U$}
    \end{proof}
  \end{proof}
  
  \begin{lm}
    \label{lm:topology:subspace:subbasis}
    Let $Y \subseteq X$. If $\mathcal{S}$ is a subbasis for $X$. then $\{ S \cap 
    Y 
    : 
    S \in \mathcal{S} \}$ is a subbasis for $Y$.
  \end{lm}
  
  \begin{proof}
    \step{1}{For all $y \in Y$, there exists $S \in \mathcal{S}$ such that $y 
      \in 
      S \cap Y$}
    \step{2}{The set of all finite intersections of sets of the form $S \cap Y$ 
      for $S \in \mathcal{S}$ is a basis for $Y$.}
    \begin{proof}
      \pf\ This is the same as $\{ B \cap Y : B \text{ a finite intersection of 
        elements of } \mathcal{S} \}$, which is a basis by Lemma 
      \ref{lm:topology:subspace:basis}.
    \end{proof}
  \end{proof}
  
  \begin{lm}
    Let $Y$ be a subspace of $X$. If $U$ is open in $Y$ and $Y$ is open in $X$ 
    then $U$ is open in $X$.
  \end{lm}
  
  \begin{proof}
    \pf\ Let $V$ be open in $X$ such that $U = V \cap Y$. Then $U$ is open in 
    $X$, since it is the intersection of two open sets.
  \end{proof}
  
  \begin{thm}
    Let $\{ X_j \}_{j \in J}$ be a family of topological spaces and $A_j 
    \subseteq 
    X_j$ a subspace for all $j \in J$. Then the product topology on $\prod_{j 
      \in 
      J} A_j$ is the same as its topology as a subspace of $\prod_{j \in J} 
    X_j$.
  \end{thm}
  
  \begin{proof}
    \step{0}{\pflet{$\pi_j : \prod_{j \in J} X_j \rightarrow X_j$ and $p_j : 
        \prod_{j \in J} A_j \rightarrow A_j$ be the projections.}}
    \step{1}{The subspace topology is finer than the product topology.}
    \begin{proof}
      \step{1a}{\pflet{$U$ be open in $A_j$} \prove{$\inv{p_j}(U)$ is open in 
          the subspace topology.}}
      \step{1b}{$\inv{p_j}(U) = \inv{\pi_j}(U) \cap \prod_{j \in J} A_j$}
    \end{proof}
    \step{2}{The product topology is finer than the subspace topology.}
    \begin{proof}
      \step{2a}{$\{ \inv{\pi_j}(U) \cap \prod_{j \in J} A_j : j \in J, U \text{ 
          open in } X_j \}$ is a subbasis for the subspace topology.}
      \begin{proof}
        \pf\ By Lemma \ref{lm:topology:subspace:subbasis}.
      \end{proof}
      \step{2b}{For $j \in J$ and $U$ open in $X_j$, we have $\inv{\pi_j}(U) 
        \cap 
        \prod_{j \in J} A_j$ is open in the product topology.}
      \begin{proof}
        \pf\ $\inv{\pi_j}(U) \cap \prod_{j \in J} A_j = \inv{p_j}(U \cap A_j)$
      \end{proof}
    \end{proof}
  \end{proof}
  
  \begin{thm}
    Let $X$ be a linearly ordered set under the order topology, and $I 
    \subseteq 
    X$ an interval. Then the order topology on $I$ is the same as the subspace 
    topology.
  \end{thm}
  
  \begin{proof}
    \step{1}{For $x \in I$, we have $\{ z \in I : z < x \}$ is open 
      in the subspace topology.}
    \begin{proof}
      \pf\ $\{ z \in I : z < x \} = (-\infty, x) \cap I$
    \end{proof}
    \step{2}{For $x \in I$, we have $\{ z \in I : x < z \}$ is open in the 
      subspace topology.} 
    \begin{proof}
      \pf\ $\{ z \in I : x < z \} = (x, +\infty) \cap I$
    \end{proof}
    \step{3}{For $x \in X$, we have $(-\infty, x) \cap I$ is open in the order 
      topology.}
    \begin{proof}
      \step{3a}{\case{$x \in I$}}
      \begin{proof}
        \pf\ In this case, $(-\infty, x) \cap I = \{ z \in I : z < x \}$.
      \end{proof}
      \step{3b}{\case{$x \notin I$}}
      \begin{proof}
        \pf\ In this case, $(-\infty, x) \cap I$ is either $I$ or $\emptyset$.
      \end{proof}
    \end{proof}
    \step{4}{For $x \in X$, we have $(x, +\infty) \cap I$ is open in the order 
      topology.}
    \begin{proof}
      \pf\ Similar.
    \end{proof}
  \end{proof}
  
  \begin{lm}
    \label{lm:topology:subspace:closed}
    Let $Y$ be a subspace of $X$. Then a set $A \subseteq Y$ is closed in $Y$ 
    iff 
    it is the intersection of a closed set in $X$ with $Y$.
  \end{lm}
  
  \begin{proof}
    \pf\ \begin{align*}
      A \text{ is closed in } Y
      & \Leftrightarrow \exists U \text{ open in } Y. A = Y \setminus \\
      & \Leftrightarrow \exists V \text{ open in } X. A = Y \setminus (Y \cap 
      V) \\
      & \Leftrightarrow \exists V \text{ open in } X. A = Y \cap (X \setminus 
      V)
    \end{align*}
  \end{proof}
  
  \begin{thm}
    \label{thm:topology:subspace:closed}
    Let $Y$ be a subspace of $X$. If $A$ is closed in $Y$ and $Y$ is closed in 
    $X$ 
    then $A$ is closed in $X$.
  \end{thm}
  
  \begin{proof}
    \step{1}{$Y \setminus A$ is open in $Y$}
    \step{2}{\pick\ $U$ open in $X$ such that $Y \setminus A = U \cap Y$}
    \step{3}{$X \setminus A = (X \setminus Y) \cup U$}
  \end{proof}
  
  \begin{thm}
    Let $Y$ be a subspace of $X$ and $A \subseteq Y$. Then the closure of $A$ 
    in 
    $Y$ is the intersection of the closure of $A$ in $X$ with $Y$.
  \end{thm}
  
  \begin{proof}
    \step{1}{\pflet{$\overline{A}$ be the closure of $A$ in $X$}}
    \step{2}{$\overline{A} \cap Y$ is closed in $Y$ and includes $A$}
    \step{3}{If $C$ is any set closed in $Y$ that includes $A$ then 
      $\overline{A} 
      \cap Y \subseteq C$}
    \begin{proof}
      \step{3a}{\pflet{$C$ be a set closed in $Y$ that includes $A$}}
      \step{3b}{\pick\ $D$ closed in $X$ such that $C = D \cap Y$}
      \begin{proof}
        \pf\ By Lemma \ref{lm:topology:subspace:closed}.
      \end{proof}
      \step{3c}{$\overline{A} \subseteq D$}
      \step{3d}{$\overline{A} \cap Y \subseteq C$}
    \end{proof}
  \end{proof}
  
  \begin{lm}
    If $Y$ is a subspace of $X$, then the inclusion $i : Y \rightarrowtail X$ 
    is 
    continuous.
  \end{lm}
  
  \begin{proof}
    \pf\ For any $U$ open in $X$, we have $\inv{i}(U) = U \cap Y$ is open in 
    $Y$.
  \end{proof}
  
  \begin{lm}
    Let $f : X \rightarrow Y$ be continuous. Then $f$ is continuous considered 
    as 
    a map $X \rightarrow \im f$.
  \end{lm}
  
  \begin{proof}
    \step{1}{\pflet{$U$ be open in $\im f$}}
    \step{2}{\pick\ $V$ open in $Y$ such that $U = V \cap \im f$}
    \step{3}{$\inv{f}(U) = \inv{f}(V)$}
    \step{4}{$\inv{f}(U)$ is open in $X$}
  \end{proof}
  
  \begin{lm}[Local Formulation of Continuity]
    Let $X$ and $Y$ be topological spaces and $f : X \rightarrow Y$. Then $f$ 
    is 
    continuous if and only if there exists a set of open sets $\mathcal{U}$ 
    such 
    that:
    \begin{enumerate}
      \item $X = \bigcup \mathcal{U}$
      \item For all $U \in \mathcal{U}$, we have $f \restriction U$ is 
      continuous.
    \end{enumerate}
  \end{lm}
  
  \begin{proof}
    \step{1}{If $f$ is continuous then there exists a set $\mathcal{U}$ 
      satisfying 1 and 2.}
    \begin{proof}
      \pf\ Take $\mathcal{U} = \{ X \}$.
    \end{proof}
    \step{2}{If there exists a set $\mathcal{U}$ satisfying 1 and 2, then $f$ 
      is 
      continuous.}
    \begin{proof}
      \step{2a}{\pflet{$\mathcal{U}$ be a set satisfying 1 and 2}}
      \step{2b}{\pflet{$V \subseteq Y$ be open}}
      \step{2c}{For all $U \in \mathcal{U}$, we have $\inv{f}(V) \cap U$ is 
        open.}
      \begin{proof}
        \pf\ $\inv{f}(V) \cap U = \inv{(f \restriction U)}(V)$
      \end{proof}
      \step{2d}{$\inf{f}(V) = \bigcup_{U \in \mathcal{U}} (\inv{f}(V) \cap U)$}
      \step{2e}{$\inf{f}(V)$ is open}
    \end{proof}
  \end{proof}
  
  \begin{thm}[Pasting Lemma]
    Let $X = A \cup B$, where $A$ and $B$ are closed in $X$. Let $f : A 
    \rightarrow Y$ and $g : B \rightarrow Y$ be f. If $f(x) = g(x)$ for 
    all $x \in A \cap B$, then the function $h : X \rightarrow Y$ defined by
    \[ h(x) = \begin{cases}
      f(x) & \text{if } x \in A \\
      g(x) & \text{if } x \in B
    \end{cases} \]
    is continuous.
  \end{thm}
  
  \begin{proof}
    \step{1}{\pflet{$C$ be closed in $Y$}}
    \step{2}{$\inv{h}(C) = \inv{f}(C) \cup \inv{g}(C)$}
    \step{3}{$\inv{f}(C)$ is closed in $X$}
    \begin{proof}
      \step{3a}{$\inv{f}(C)$ is closed in $A$}
      \begin{proof}
        \pf\ Lemma \ref{lm:topology:continuous:criteria}.
      \end{proof}
      \qedstep
      \begin{proof}
        \pf\ Theorem \ref{thm:topology:subspace:closed}.
      \end{proof}
    \end{proof}
    \step{4}{$\inv{g}(C)$ is closed in $X$}
    \begin{proof}
      \pf\ Similar.
    \end{proof}
    \step{5}{$\inv{h}(C)$ is closed in $X$}
    \qedstep
    \begin{proof}
      \pf\ By Lemma \ref{lm:topology:continuous:criteria}.
    \end{proof}
  \end{proof}
  
  \begin{df}[Imbedding]
    Let $f : X \rightarrow Y$ be continuous. Then $f$ is an \emph{imbedding} 
    iff 
    $f$ is a homeomorphism between $X$ and $\im f$, considered as a subspace of 
    $Y$.
  \end{df}
  
  \subsection{Box Topology}
  
  \begin{df}[Box Topology]
    Let $\{ X_j \}_{j \in J}$ be a family of topological spaces. The \emph{box 
      topology} on $\prod_{j \in J} X_j$ is the topology generated by the basis 
    $\mathcal{B} = \{ \prod_{j \in J} U_j: \forall j \in J. U_j \text{ open in 
    } 
    X_j \}$.
    
    We prove this is a basis.
  \end{df}
  
  \begin{proof}
    \step{1}{For all $x \in \prod_{j \in J} X_j$, there exists $B \in 
      \mathcal{B}$ 
      such that $x \in B$}
    \begin{proof}
      \pf\ Take $B = \prod_{j \in J} X_j$.
    \end{proof}
    \step{2}{For all $B_1, B_2 \in \mathcal{B}$ and $x \in B_1 \cap B_2$, there 
      exists $B_3 \in \mathcal{B}$ such that $x \in B_3 \subseteq B_1 \cap 
      B_2$.}
    \begin{proof}
      \pf\ If $B_1 = \prod_{j \in J} U_j$ and $B_2 = \prod_{j \in J} V_j$, take 
      $B_3 = \prod_{j \in J} (U_j \cap V_j)$.
    \end{proof}
  \end{proof}
  
  \begin{thm}[AC]
    Let $\{ X_j \}_{j \in J}$ be a family of topological spaces, and let 
    $\mathcal{B}_j$ be a basis for $X_j$ for each $j$. Then
    \[ \mathcal{B} = \{ \prod_{j \in J} B_j : \forall j \in J. B_j \in 
    \mathcal{B}_j \} \]
    is a basis for the box topology on $\prod_{j \in J} X_j$.
  \end{thm}
  
  \begin{proof}
    \step{1}{Every element of $\mathcal{B}$ is open in the box topology.}
    \begin{proof}
      \pf\ This holds since, if $B_j \in \mathcal{B}_j$, then $B_j$ is open in 
      $X_j$.
    \end{proof}
    \step{2}{If $U_j$ is open in $X_j$ for all $j$ and $x \in \prod_{j \in J} 
      U_j$, then there exists $B \in \mathcal{B}$ such that $x \in B \subseteq 
      \prod_{j \in J} U_j$}
    \begin{proof}
      \step{2a}{\pflet{$U_j$ be open in $X_j$ for all $j$ and $x \in \prod_{j 
            \in 
            J} U_j$}}
      \step{2b}{\pick\ $B_j \in \mathcal{B}_j$ such that $x_j \in B_j \subseteq 
        U_j$ for all $j \in J$}
      \begin{proof}
        \pf\ Using the Axiom of Choice.
      \end{proof}
      \step{2c}{Take $B = \prod_{j \in J} B_j$}
    \end{proof}
    \qedstep
    \begin{proof}
      \pf\ By Lemma \ref{lm:topology:basis:criterion2}.
    \end{proof}
  \end{proof}
  
  \begin{thm}[AC]
    Let $\{ X_j \}_{j \in J}$ be a family of topological spaces, and $A_j 
    \subseteq X_j$ be a subspace for all $j \in J$. Then the box topology on 
    $\prod_{j \in J} A_j$ is the same as its topology as a subspace of 
    $\prod_{j 
      \in J} X_j$ under the box topology.
  \end{thm}
  
  \begin{proof}
    \step{1}{The subspace topology is finer than the box topology.}
    \begin{proof}
      \step{1a}{\pflet{$U_j$ be open in $A_j$ for all $j \in J$.} 
        \prove{$\prod_{j 
            \in J} U_j$ is open in the subspace topology.}}
      \step{1b}{For $j \in j$, \pick\ $V_j$ open in $X_j$ such that $U_j = V_j 
        \cap 
        A_j$}
      \begin{proof}
        \pf\ Using the Axiom of Choice.
      \end{proof}
      \step{1c}{$\prod_{j \in J} U_j = \prod_{j \in J} V_j \cap \prod_{j \in J} 
        A_j$}
    \end{proof}
    \step{2}{The box topology is finer than the subspace topology.}
    \begin{proof}
      \step{2a}{$\{ \prod_{j \in J} U_j \cap \prod_{j \in J} A_j : \forall j \in 
        J. 
        U_j \text{ open in } x_j \}$ is a basis for the subspace topology.}
      \begin{proof}
        \pf\ By Lemma \ref{lm:topology:subspace:basis}.
      \end{proof}
      \step{2b}{\pflet{$U_j$ be open in $X_j$ for all $j$} \prove{$\prod_{j \in 
            J} 
          U_j \cap \prod_{j \in J} A_j$ is open in the box topology.}}
      \step{2c}{$\prod_{j \in J} 
        U_j \cap \prod_{j \in J} A_j = \prod_{j \in J} (U_j \cap A_j)$}
    \end{proof}
  \end{proof}
  
  \begin{thm}[AC]
    Let $\{ X_j \}_{j \in J}$ be a family of topological spaces. Let $A_j 
    \subseteq X_j$ for all $j \in J$. Then
    \[ \prod_{j \in J} \overline{A_j} = \overline{\prod_{j \in J} A_j} \]
    in $\prod_{j \in J} X_j$ under the box topology.
  \end{thm}
  
  \begin{proof}
    \step{1}{$\prod_{j \in J} \overline{A_j} = \overline{\prod_{j \in J} A_j}$}
    \begin{proof}
      \step{1a}{\pflet{$x \in \prod_{j \in J} \overline{A_j}$}}
      \step{1b}{\pflet{$x \in \prod_{j \in J} U_j$ where each $U_j$ is open in 
          $X_j$}}
      \step{1c}{For $j \in j$, \pick\ $y_j \in U_j \cap A_j$}
      \begin{proof}
        \pf\ Possible as $x_j \in \overline{A_j} \cap U_j$, using the Axiom of 
        Choice.
      \end{proof}
      \step{1d}{$y \in \prod_{j \in J} U_j$ and $y \in \prod_{j \in J} A_j$}
      \qedstep
      \begin{proof}
        \pf\ By Lemma \ref{lm:topology:closure:membership2}.
      \end{proof}
    \end{proof}
    \step{2}{$\overline{\prod_{j \in J} A_j} \subseteq \prod_{j \in J} 
      \overline{A_j}$}
    \begin{proof}
      \step{2a}{\pflet{$x \in \overline{\prod{j \in J} A_j}$}}
      \step{2b}{\pflet{$j \in J$} \prove{$x_j \in \overline{A_j}$}}
      \step{2c}{\pflet{$U$ be a neighbourhood of $x_j$}}
      \step{2d}{$\inv{\pi_j}(U)$ is a neighbourhood of $x$}
      \step{2e}{$\inv{\pi_j}(U)$ intersects $\prod_{j \in J} A_j$ in $y$, say}
      \step{2f}{$y_j \in U$ and $y_j \in A_j$}
    \end{proof}
  \end{proof}
  
  \subsection{Quotient Spaces}
  
  \begin{df}[Quotient Map]
    Let $X$ and $Y$ be topological spaces. A function $q : X \rightarrow Y$ is 
    a 
    \emph{quotient map} iff $q$ is surjective and, for $U \subseteq Y$, $U$ is 
    open if and only if $\inv{q}(U)$ is open in $X$.
  \end{df}
  
  \begin{df}[Saturated Set]
    Let $p : X \rightarrow Y$ be surjective and $C \subseteq X$. Then $C$ is 
    \emph{saturated} with respect to $p$ if and only if, for all $x, y \in X$, 
    if 
    $x \in C$ and $p(x) = p(y)$ then $y \in C$.
  \end{df}
  
  \begin{lm}
    Let $X$ and $Y$ be topological spaces.
    Let $q : X \rightarrow Y$ be surjective. Then the following are equivalent.
    \begin{enumerate}
      \item $q$ is a quotient map.
      \item $q$ is continuous and maps saturated open sets to open sets.
      \item $q$ is continuous and maps saturated closed sets to closed sets.
    \end{enumerate}
  \end{lm}
  
  \begin{cor}
    Let $q : X \rightarrow Y$ be surjective. If $q$ is continuous and either an 
    open map or a closed map, then $q$ is a quotient map.
  \end{cor}
  
  \begin{lm}
    Every project $\pi_j : \prod_{j \in J} X_j \rightarrow X_j$ is a quotient 
    map.
  \end{lm}
  
  \begin{lm}
    Let $X$ be a topological space, $A$ a set, and $q : X \rightarrow A$ a 
    surjective function. Then there exists a unique topology on $A$ such that 
    $q$ 
    is a quotient map.
  \end{lm}
  
  \begin{proof}
    \pf\ The topology is defined by: $U \subseteq A$ is open in $A$ iff 
    $\inv{q}(U)$ is open in $X$. \qed
  \end{proof}
  
  \begin{df}[Quontient Topology]
    Let $X$ be a topological space, $A$ a set, and $q : X \rightarrow A$ a 
    surjective function. The \emph{quotient topology} induced by $q$ is the 
    unique 
    topology on $A$ such that $q$ is a quotient map.
  \end{df}
  
  \begin{df}[Quotient Space]
    Let $X$ be a topological space and $X^*$ a partition of $X$. Then $X^*$ 
    under 
    the quotient topology induced by the canonical map $X \rightarrow X^*$ is a 
    \emph{quotient space} of $X$.
  \end{df}
  
  \section{$T_1$ Spaces}
  
  \begin{df}[$T_1$ Space]
    A topological space $X$ satisfies the \emph{$T_1$ axiom} iff every finite 
    set 
    is closed.
  \end{df}
  
  \begin{thm}
    Let $X$ be a $T_1$ space and $A \subseteq X$. Then $x$ is a limit point of 
    $A$ 
    if and only if every neighbourhood of $x$ contains infinitely many points 
    of 
    $A$.
  \end{thm}
  
  \begin{proof}
    \step{1}{If $x$ is a limit point of $A$ then every neighbourhood of $x$ 
      contains infinitely many points of $A$.}
    \begin{proof}
      \step{1a}{\assume{$x$ is a limit point of $A$.}}
      \step{1b}{\pflet{$U$ be a neighbourhood of $x$.}}
      \step{1c}{\assume{for a contradiction $U \cap A$ is finite.}}
      \step{1d}{$X \setminus ((U \cap A) \setminus \{x\})$ is a neighbourhood 
        of 
        $x$.}
      \step{1e}{$X \setminus ((U \cap A) \setminus \{x\})$ intersects $A$ in a 
        point other than $x$.}
      \step{1f}{This is a contradiction.}
    \end{proof}
    \step{2}{If every neighbourhood of $x$ contains infinitely many points of 
      $A$ 
      then $x$ is a limit point of $A$.}
    \begin{proof}
      \pf\ Immediate from the definition of limit point.
    \end{proof}
  \end{proof}
  
  \section{Hausdorff Spaces}
  
  \begin{df}[Hausdorff Space]
    A topological space $X$ is a \emph{Hausdorff space} iff, for any points $x, 
    y 
    \in X$ with $x \neq y$, there exist disjoint neighbourhoods $U$ of $x$ and 
    $V$ 
    of $y$.
  \end{df}
  
  \begin{thm}
    Every Hausdorff space is $T_1$.
  \end{thm}
  
  \begin{proof}
    \step{1}{\pflet{$X$ be a Hausdorff space and $a \in X$}
      \prove{$X \setminus \{ a \}$ is open}}
    \step{2}{\pflet{$b \in X \setminus \{a\}$}}
    \step{3}{\pick\ disjoint neighbourhoods $U$ of $a$ and $V$ of $b$}
    \step{4}{$b \in V \subseteq X \setminus \{a\}$}
    \step{5}{$X \setminus \{a\}$ is open.}
    \begin{proof}
      \pf\ By Lemma \ref{lm:topology:open:membership}.
    \end{proof}
  \end{proof}
  
  \begin{thm}
    In a Hausdorff space, a sequence has at most one limit.
  \end{thm}
  
  \begin{proof}
    \step{1}{\assume{for a contradiction $a_n \rightarrow l$ and $a_n 
        \rightarrow 
        m$ as $n \rightarrow \infty$, and $l \neq m$.}}
    \step{2}{\pick\ disjoint neighbourhoods $U$ of $l$ and $V$ of $m$.}
    \step{3}{\pick\ numbers $M$ and $N$ such that $\forall n > M. a_n \in U$ 
      and 
      $\forall n > N. a_n \in V$}
    \step{4}{$a_{\max(M,N) + 1}$ is in both $U$ and $V$.}
    \step{5}{This contradicts the fact that $U$ and $V$ are disjoint.}
  \end{proof}
  
  \begin{thm}
    Every linearly ordered set is a Hausdorff space under the order topology.
  \end{thm}
  
  \begin{proof}
    \step{1}{\pflet{$X$ be a linearly ordered set and $a,b \in X$ with $a<b$}}
    \step{2}{\assume{w.l.o.g. $a < b$}}
    \step{3}{\case{There exists $c$ such that $a < c < b$}}
    \begin{proof}
      \pf\ In this case, $a \in (-\infty, c)$ and $b \in (c, +\infty)$.
    \end{proof}
    \step{4}{\case{There does not exist $c$ such that $a < c < b$}}
    \begin{proof}
      \pf\ In this case, $a \in (-\infty, b)$ and $b \in (a, +\infty)$.
    \end{proof}
  \end{proof}
  
  \begin{thm}
    The product of a family of Hausdorff spaces is Hausdorff.
  \end{thm}
  
  \begin{proof}
    \step{1}{\pflet{$\{ X_j \}_{j \in J}$ be a family of Hausdorff spaces.}}
    \step{2}{\pflet{$x, y \in \prod_{j \in J} X_j$ with $x \neq y$}}
    \step{3}{\pick\ $j \in J$ such that $x_j \neq y_j$}
    \step{4}{\pick\ $U$, $V$ disjoint neighbourhoods of $x_j$, $y_j$ 
      respectively.}
    \step{5}{$\inv{\pi_j}(U)$, $\inv{\pi_j}(V)$ are disjoint neighbourhoods of 
      $x$ and $y$}
  \end{proof}
  
  \begin{thm}
    A subspace of a Hausdorff space is Hausdorff.
  \end{thm}
  
  \begin{proof}
    \step{1}{\pflet{$X$ be a Hausdorff space and $Y \subseteq X$}}
    \step{2}{\pflet{$a, b$ be distinct points in $Y$.}}
    \step{3}{\pick\ disjoint neighbourhoods $U$ of $a$ and $V$ of $b$ in $X$.}
    \step{4}{We have $a \in U \cap Y$ and $b \in V \cap Y$}
  \end{proof}
  
  \begin{thm}
    Let $\{ X_j \}_{j \in J}$ be a family of Hausdorff spaces. Then $\prod_{j 
      \in 
      J} X_j$ is Hausdorff under the box topology.
  \end{thm}
  
  \begin{proof}
    \step{1}{\pflet{$\{ X_j \}_{j \in J}$ be a family of Hausdorff spaces.}}
    \step{2}{\pflet{$x, y \in \prod_{j \in J} X_j$ with $x \neq y$}}
    \step{3}{\pick\ $j \in J$ such that $x_j \neq y_j$}
    \step{4}{\pick\ $U$, $V$ disjoint neighbourhoods of $x_j$, $y_j$ 
      respectively.}
    \step{5}{$\inv{\pi_j}(U)$, $\inv{\pi_j}(V)$ are disjoint neighbourhoods of 
      $x$ and $y$}
  \end{proof}
  
  \section{First Countability Axiom}
  
  \begin{lm}
    \label{lm:topology:countabilty:decreasing_basis}
    If there is a point $x$ has a countable local basis, then it has a 
    countable 
    local basis $\{ U_n : n \in \mathbb{N} \}$ such that $U_0 \supseteq U_1 
    \supseteq U_2 \supseteq \cdots$. 
  \end{lm}
  
  \begin{proof}
    \pf\ Let $\{ V_n : n \in \mathbb{N} \}$ be a countable local basis. Take
    \[ U_n = V_0 \cap V_1 \cap \cdots \cap V_n \enspace . \]
  \end{proof}
  
  \begin{lm}[Sequence Lemma (CC)]
    Let $X$ be a topological space and $A \subseteq X$. If $l \in \overline{A}$ 
    and there is a countable basis at $l$, then there exists a sequence $(a_n)$ 
    in 
    $A$ such that $a_n \rightarrow l$ as $n \rightarrow \infty$.
  \end{lm}
  
  \begin{proof}
    \step{1}{\pick\ a countable basis $\{ U_n : n \in \mathbb{N} \}$ at $l$ 
      such 
      that $U_0 \supseteq U_1 \supseteq \cdots$}
    \begin{proof}
      \pf\ By Lemma \ref{lm:topology:countabilty:decreasing_basis}.
    \end{proof}
    \step{2}{For $n \in \mathbb{N}$, \pick\ a point $a_n \in U_n \cap A$}
    \begin{proof}
      \pf\ A point exists by Theorem \ref{thm:topology:closure:membership}. 
      Apply 
      the Axiom of Countable Choice.
    \end{proof}
    \step{3}{$a_n \rightarrow l$ as $n \rightarrow \infty$}
    \begin{proof}
      \pf
      \step{3a}{\pflet{$U$ be a neighbourhood of $l$}}
      \step{3b}{\pick\ $N$ such that $U_N \subseteq U$}
      \begin{proof}
        \pf\ By \stepref{1}
      \end{proof}
      \step{3c}{For $n \geq N$, we have $a_n \in U$}
      \begin{proof}
        \pf
        \begin{align*}
          a_n & \in U_n & (\text{\stepref{2}}) \\
          & \subseteq U_N & (\text{\stepref{1}}) \\
          & \subseteq U & (\text{\stepref{3b}})
        \end{align*}
      \end{proof}
    \end{proof}
    \qed
  \end{proof}
  
  \begin{lm}[CC]
    Let $X$ and $Y$ be topological spaces and $f : X \rightarrow Y$. Suppose 
    there 
    is a countable local basis at $l$ and, for every sequence $(x_n)$ in $X$ 
    such 
    that $x_n \rightarrow l$ as $n \rightarrow \infty$, we have $f(x_n) 
    \rightarrow 
    f(l)$ as $n \rightarrow \infty$. Then $f$ is continuous at $l$.
  \end{lm}
  
  \begin{proof}
    \step{1}{\pick\ a basis $\{ U_n : n \in \mathbb{N} \}$ at $l$ such that 
      $U_0 
      \supseteq U_1 \supseteq \cdots$.}
    \step{2}{\pflet{$V$ be a neighbourhood of $f(l)$}}
    \step{3}{\assume{for a contradiction $\inv{f}(V)$ is not open}}
    \step{4}{For $n \in \mathbb{N}$, \pick\ $x_n \in U_n$ such that $x_n \notin 
      \inv{f}(V)$}
    \begin{proof}
      \pf\ Using the Axiom of Countable Choice.
    \end{proof}
    \step{5}{$x_n \rightarrow l$ as $n \rightarrow \infty$}
    \begin{proof}
      \pf
      \step{5a}{\pflet{$U$ be a neighbourhood of $l$}}
      \step{5b}{\pick\ $N$ such that $U_N \subseteq U$}
      \begin{proof}
        \pf\ By \stepref{1}
      \end{proof}
      \step{5c}{For $n \geq N$, we have $x_n \in U$}
      \begin{proof}
        \pf
        \begin{align*}
          x_n & \in U_n & (\text{\stepref{4}}) \\
          & \subseteq U_N & (\text{\stepref{1}}) \\
          & \subseteq U & (\text{\stepref{5b}})
        \end{align*}
      \end{proof}
    \end{proof}
    \step{6}{$f(x_n) \rightarrow f(l)$ as $n \rightarrow \infty$}
    \begin{proof}
      \pf\ From \stepref{5} and hypothesis.
    \end{proof}
    \step{7}{\pick\ $N$ such that, for all $n \geq N$, $f(x_n) \in V$}
    \qedstep
    \begin{proof}
      \pf\ This is a contradiction as $x_N \notin \inv{f}(V)$ by \stepref{4}.
    \end{proof}
  \end{proof}
  
  \begin{df}[First Countability Axiom]
    A topological space $X$ satisfies the \emph{first countability axiom}, or 
    is 
    \emph{first countable}, iff every point has a countable local basis.
  \end{df}
  
  \begin{lm}
    $\mathbb{R}^\omega$ under the box topology is not first countable.
  \end{lm}
  
  \begin{proof}
    \step{1}{\assume{for a contradiction $\{ U_n : n \in \mathbb{N} \}$ is a 
        countable basis at $\vec{0}$.}}
    \step{2}{For $i, n \in \mathbb{N}$, \pick\ $a_{in}, b_{in}$ such that 
      $\vec{0} \in \prod_{i=0}^\infty (a_{in}, b_{in}) \subseteq U_n$}
    \step{3}{\pflet{$U = \prod_{n=0}^\infty (a_{nn}/2, b_{nn}/2)$}}
    \step{4}{$\vec{0} \in U$}
    \step{5}{There is no $n$ such that $U_n \subseteq U$}
  \end{proof}
  
  \begin{lm}
    If $J$ is uncountable then $\mathbb{R}^J$ is not first countable.
  \end{lm}
  
  \begin{proof}
    \step{1}{\assume{for a contradiction $\{ U_n : n \in \mathbb{N} \}$ is a 
        countable basis for $\vec{0}$}}
    \step{2}{For $n \in \mathbb{N}$, let $\vec{0} \in \prod_{j \in J} V_{nj} 
      \subseteq U_n$, where $V_{nj} = \mathbb{R}$ for all but finitely many 
      $j$.}
    \step{3}{\pick\ $j_0 \in J$ such that $V_{nj_0} = \mathbb{R}$ for all $n$.}
    \step{4}{\pflet{$W = \prod_{j \in J} W_j$ where $W_{j_0} = (-1, 1)$ and 
        $W_j 
        = \mathbb{R}$ for all other $j$}}
    \step{5}{$\vec{0} \in W$}
    \step{6}{$W$ is open}
    \step{7}{There is no $n$ such that $U_n \subseteq W$}
  \end{proof}
  
  \section{Metric Spaces}
  
  \begin{df}[Metric]
    A \emph{metric} on a set $X$ is a function $d : X^2 \rightarrow \mathbb{R}$ 
    such that:
    \begin{enumerate}
      \item For all $x, y \in X$, we have $d(x,y) \geq 0$
      \item For all $x, y \in X$, we have $d(x, y) = 0$ if and only if $x = y$
      \item For all $x, y \in X$, we have $d(x, y) = d(y, x)$
      \item \textbf{Triangle Inequality} For all $x, y, z \in X$, we have $d(x, 
      z) 
      \leq d(x, y) + d(y, z)$
    \end{enumerate}
    
    A \emph{metric space} consists of a set and a metric on that set. We call 
    $d(x, 
    y)$ the \emph{distance} between $x$ and $y$.
  \end{df}
  
  \begin{df}[Ball]
    Let $(X, d)$ be a metric space. Let $\epsilon > 0$ and $x \in X$. The 
    \emph{$\epsilon$-ball} with \emph{centre} $x$ is the set
    \[ B_d(x, \epsilon) = \{ y \in X : d(x, y) < \epsilon \} \enspace . \]
  \end{df}
  
  \begin{df}[Induced Topology]
    Let $(X, d)$ be a metric space. The topology \emph{induced} by $d$ is the 
    topology generated by the basis consisting of all the balls.
    
    We prove this is a basis for a topology.
  \end{df}
  
  \begin{proof}
    \step{1}{For all $x \in X$, there exists a ball $B$ such that $x \in B$}
    \begin{proof}
      \pf\ Take $B = B(x, 1)$
    \end{proof}
    \step{2}{For any balls $B_1$, $B_2$ and point $x \in B_1 \cap B_2$, there 
      exists a ball $B_3$ such that $x \in B_3 \subseteq B_1 \cap B_2$}
    \begin{proof}
      \step{2a}{\pflet{$B_1 = B(a_1, \epsilon_1)$ and $B_2 = B(a_2, 
          \epsilon_2)$}}
      \step{2b}{\pflet{$\epsilon = \min(\epsilon_1 - d(x, a_1), \epsilon_2 - 
          d(x, 
          a_2))$}}
      \step{2c}{$\epsilon > 0$}
      \step{2d}{$B(x, \epsilon) \subseteq B_1$}
      \begin{proof}
        \step{i}{\pflet{$y \in B(x, \epsilon)$}}
        \step{ii}{$d(y, a_1) < \epsilon_1$}
        \begin{proof}
          \pf\ \begin{align*}
            d(y, a_1) & \leq d(y, x) + d(x, a_1) & (\text{Triangle 
              Inequality}) \\
            & < \epsilon + d(x, a_1) & (\text{\stepref{i}}) \\
            & \leq \epsilon_1 & (\text{\stepref{2b}})
          \end{align*}
        \end{proof}
      \end{proof}
      \step{2e}{$B(x, \epsilon) \subseteq B_2$}
      \begin{proof}
        \pf\ Similar
      \end{proof}
    \end{proof}
  \end{proof}
  
  \begin{lm}
    Let $(X, d)$ be a metric space and $Y \subseteq X$. Then the restriction of 
    $d$ to $Y$ is a metric on $Y$ that induces the subspace topology on $Y$.
  \end{lm}
  
  \begin{proof}
    \pf\ The metric topology and the subspace topology are both the topology 
    induced by $\{ B_{d'}(y, \epsilon) : y \in Y \} = \{ B_d(y, \epsilon) \cap Y 
    : 
    y \in Y \}$, where $d' = d \restriction Y^2$.
  \end{proof}
  
  \begin{thm}
    Every metric space is Hausdorff.
  \end{thm}
  
  \begin{proof}
    \step{1}{\pflet{$(X, d)$ be a metric space.}}
    \step{2}{\pflet{$x, y \in X$ with $x \neq y$.}}
    \step{3}{\pflet{$\epsilon = d(x,y)$}}
    \step{4}{$\epsilon > 0$}
    \begin{proof}
      \pf\ Axiom 2 of metric spaces (\stepref{1}, \stepref{2}, \stepref{3}).
    \end{proof}
    \step{5}{\pflet{$U = B(x, \epsilon / 2)$, $V = B(y, \epsilon / 2)$} 
      \prove{$U 
        \cap V = \emptyset$}}
    \step{6}{\assume{for a contradiction $z \in U \cap V$}}
    \step{7}{$d(x, y) < \epsilon$}
    \begin{proof}
      \pf\ \begin{align*}
        d(x, y) & \leq d(x, z) + d(z, y) & (\text{Triangle Inequality}) \\
        & < \epsilon / 2 + \epsilon / 2 & (\text{\stepref{5}, \stepref{6}}) \\
        & = \epsilon
      \end{align*}
    \end{proof}
    \qedstep
    \begin{proof}
      \pf\ This contradicts \stepref{3}.
    \end{proof}
  \end{proof}
  
  \begin{thm}
    Every metric space is first countable.
  \end{thm}
  
  \begin{proof}
    \pf\ For any point $x$, we have $\{ B(x, 1/n) : n \in \mathbb{Z}^+ \}$ is a 
    local basis at $x$. \qed
  \end{proof}
  
  \begin{lm}
    Let $X$ and $Y$ be metric spaces and $f : X \rightarrow Y$. Then $f$ is 
    continuous if and only if, for all $\epsilon > 0$ and $x \in X$, there 
    exists 
    $\delta > 0$ such that, for all $x' \in X$, if $d(x, x') < \delta$ then 
    $d(f(x), f(x')) < \epsilon$.
  \end{lm}
  
  \begin{proof}
    \pf
    \step{1}{If $f$ is continuous then, for all $\epsilon > 0$ and $x \in X$, 
      there exists $\delta > 0$ such that, for all $x' \in X$, if $d(x, x') < 
      \delta$ then $d(f(x), f(x')) < \epsilon$.}
    \begin{proof}
      \pf
      \step{1a}{\assume{$f$ is continuous}}
      \step{1b}{\pflet{$\epsilon > 0$}}
      \step{1c}{$x \in X$}
      \step{1d}{$\inv{f}(B(f(x), \epsilon))$ is open in $U$}
      \begin{proof}
        \pf\ Using \stepref{1a}.
      \end{proof}
      \step{1e}{$x \in \inv{f}(B(f(x), \epsilon)$}
      \begin{proof}
        \pf\ Since $d(f(x), f(x)) = 0$.
      \end{proof}
      \step{1f}{\pick\ $\delta > 0$ such that $B(x, \delta) \subseteq 
        \inv{f}(B(f(x), \epsilon))$}
      \begin{proof}
        \pf\ From \stepref{1d} and \stepref{1e}.
      \end{proof}
      \step{1g}{\pflet{$x' \in X$ with $d(x, x') < \delta$}}
      \step{1h}{$x' \in B(x, \delta)$}
      \begin{proof}
        \pf\ From \stepref{1g}.
      \end{proof}
      \step{1i}{$x' \in \inv{f}(B(f(x), \epsilon))$}
      \begin{proof}
        \pf\ From \stepref{1f} and \stepref{1h}.
      \end{proof}
      \step{1j}{$d(f(x), f(x')) < \epsilon$}
      \begin{proof}
        \pf\ From \stepref{1i}
      \end{proof}
    \end{proof}
    \step{2}{If, for all $\epsilon > 0$ and $x \in X$, there exists $\delta > 
      0$ 
      such that, for all $x' \in X$, if $d(x, x') < \delta$ then $d(f(x), 
      f(x')) 
      < \epsilon$.}
    \begin{proof}
      \pf
      \step{2a}{\assume{For all $\epsilon > 0$ and $x \in X$, there exists 
          $\delta > 0$ such that, for all $x' \in X$, if $d(x, x') < \delta$ 
          then 
          $d(f(x), f(x')) < \epsilon$}}
      \step{2b}{\pflet{$V$ be open in $Y$} \prove{$\inv{f}(V)$ is open in $X$}}
      \step{2c}{\pflet{$x \in \inv{f}(V)$} \prove{There exists $\delta > 0$ 
          such 
          that $B(x, \delta) \subseteq \inv{f}(V)$}}
      \step{2d}{\pick\ $\epsilon > 0$ such that $B(f(x), \epsilon) \subseteq V$}
      \begin{proof}
        \pf\ Such an $\epsilon$ exists because $f(x) \in V$ (\stepref{2c}) and 
        $V$ is open (\stepref{2b}).
      \end{proof}
      \step{2e}{\pick\ $\delta > 0$ such that, for all $x' \in X$, if $d(x, x') 
        < 
        \delta$ then $d(f(x), f(x')) < \epsilon$ \prove{$B(x, \delta) \subseteq 
          \inv{f}(V)$}}
      \begin{proof}
        \pf\ By \stepref{2a}
      \end{proof}
      \step{2f}{\pflet{$x' \in B(x, \delta)$} \prove{$x' \in \inv{f}(V)$}}
      \step{2g}{$d(x, x') < \delta$}
      \begin{proof}
        \pf\ From \stepref{2f}
      \end{proof}
      \step{2h}{$d(f(x), f(x')) < \epsilon$}
      \begin{proof}
        \pf\ From \stepref{2e} and \stepref{2g}.
      \end{proof}
      \step{2i}{$f(x') \in V$}
      \begin{proof}
        \pf\ From \stepref{2h} and \stepref{2d}.
      \end{proof}
    \end{proof}
    \qed
  \end{proof}
  
  \begin{lm}
    Let $(X, d)$ be a metric space. Then $x_n \rightarrow l$ as $n \rightarrow 
    \infty$ in $X$ if and only if, for all $\epsilon > 0$, there exists $N$ 
    such 
    that, for all $n \geq N$, $d(x_n, l) < \epsilon$.
  \end{lm}
  
  \begin{df}[Metrizable]
    A topological space is \emph{metrizable} iff there exists a metric that 
    induces its topology.
  \end{df}
  
  \begin{df}[Bounded]
    Let $X$ be a metric space and $A \subseteq X$. Then $A$ is \emph{bounded} 
    iff 
    $\{ d(x,y) : x,y \in A \}$ is bounded above, in which case the 
    \emph{diameter} 
    of $A$ is its supremum.
  \end{df}
  
  \begin{df}[Standard Bounded Metric]
    \label{df:topology:metric:standard_bounded}
    Let $(X, d)$ be a metric space. The \emph{standard bounded metric} 
    corresponding to $d$ is
    \begin{align*}
      \overline{d} & : X^2 \rightarrow \mathbb{R} \\
      \overline{d}(x, y) & = \min(d(x, y), 1)
    \end{align*}
  \end{df}
  
  \begin{lm}
    \label{lm:topology:metric:bounded}
    The standard bounded metric corresponding to $d$ is a bounded metric that 
    induces the 
    same topology as $d$.
  \end{lm}
  
  \begin{proof}
    \step{1}{$\overline{d}$ is a metric on $X$}
    \begin{proof}
      \step{1a}{$\overline{d}(x, y) \geq 0$}
      \begin{proof}
        \pf\ \begin{align*}
          \overline{d}(x, y) & = \min(d(x, y), 1) & (\text{Definition 
            \ref{df:topology:metric:standard_bounded}}) \\
          & \geq 0
        \end{align*}
        since $d(x, y) \geq 0$ because $d$ is a metric.
      \end{proof}
      \step{1b}{$\overline{d}(x, y) = 0$ iff $x = y$}
      \begin{proof}
        \pf\ \begin{align*}
          \overline{d}(x, y) = 0 & \Leftrightarrow \min(d(x, y), 1) = 0 & 
          (\text{Definition \ref{df:topology:metric:standard_bounded}}) \\
          & \Leftrightarrow d(x, y) = 0 \\
          & \Leftrightarrow x = y & (d \text{ is a metric})
        \end{align*}
      \end{proof}
      \step{1c}{$\overline{d}(x, y) = \overline{d}(y, x)$}
      \begin{proof}
        \pf\ \begin{align*}
          \overline{d}(x, y) & = \min(d(x, y), 1) & (\text{Definition 
            \ref{df:topology:metric:standard_bounded}}) \\
          & = \min(d(y, x), 1) & (d \text{ is a metric}) \\
          & = \overline{d}(y, x) & (\text{Definition 
            \ref{df:topology:metric:standard_bounded}})
        \end{align*}
      \end{proof}
      \step{1d}{$\overline{d}(x, z) \leq \overline{d}(x, y) + \overline{d}(y, 
        z)$}
      \begin{proof}
        \step{i}{\case{$d(x, y) \leq 1, d(y, z) \leq 1$}}
        \begin{proof}
          \pf\ In this case,
          \begin{align*}
            \overline{d}(x, z) & \leq d(x, z) & (\text{Definition 
              \ref{df:topology:metric:standard_bounded}}) \\
            & \leq d(x, y) + d(y, z) & (d \text{ is a metric}) \\
            & = \overline{d}(x, y) + \overline{d}(y, z) & (\text{Definition 
              \ref{df:topology:metric:standard_bounded}})
          \end{align*}
        \end{proof}
        \step{ii}{\case{$d(y, z) > 1$}}
        \begin{proof}
          \pf\ In this case, we have
          \begin{align*}
            \overline{d}(x, z) & \leq 1 & (\text{Definition 
              \ref{df:topology:metric:standard_bounded}}) \\
            & \leq \overline{d}(x, y) + 1 & (\text{\stepref{1a}}) \\
            & = \overline{d}(x, y) + \overline{d}(y, z) & (\text{Definition 
              \ref{df:topology:metric:standard_bounded}})
          \end{align*}
        \end{proof}
        \step{iii}{\case{$d(x, y) > 1$}}
        \begin{proof}
          \pf\ Similar.
        \end{proof}
      \end{proof}
    \end{proof}
    \step{2}{$\overline{d}$ induces the same topology as $d$}
    \begin{proof}
      \step{2a}{\pflet{$x \in X$ and $\epsilon > 0$}}
      \step{2b}{$B_{\overline{d}}(x, \epsilon)$ is open in the $d$-topology.}
      \begin{proof}
        \step{i}{\case{$\epsilon > 1$}}
        \begin{proof}
          \pf\ In this case, $B_{\overline{d}}(x, \epsilon) = X$.
        \end{proof}
        \step{ii}{\case{$\epsilon \leq 1$}}
        \begin{proof}
          \pf\ In this case, $B_{\overline{d}}(x, \epsilon) = B_d(x, \epsilon)$.
        \end{proof}
      \end{proof}
      \step{2c}{$B_d(x, \epsilon)$ is open in the $\overline{d}$-topology.}
      \begin{proof}
        \step{i}{\pflet{$y \in B_d(x, \epsilon0$}}
        \step{ii}{$B_{\overline{d}}(y, \min(\epsilon, 1/2)) \subseteq B_d(x, 
          \epsilon)$}
      \end{proof}
    \end{proof}
  \end{proof}
  
  \begin{df}[Square Metric]
    The \emph{square metric} on $\mathbb{R}^n$ is defined by
    \[ \rho(\vec{x}, \vec{y}) = \max(|x_1 - y_1|, \ldots, |x_n - y_n|) \]
  \end{df}
  
  \begin{lm}
    \label{lm:topology:metric:open}
    Let $X$ be a metric space and $U \subseteq X$. Then $U$ is open if and only 
    if, for all $x \in U$, there exists $\epsilon > 0$ such that $B(x, 
    \epsilon) 
    \subseteq U$.
  \end{lm}
  
  \begin{proof}
    \step{1}{If $U$ is open then, for all $x \in U$, there exists $\epsilon > 
      0$ 
      such that $B(x, \epsilon) \subseteq U$}
    \begin{proof}
      \step{1a}{\assume{$U$ is open}}
      \step{1b}{\pflet{$x \in U$}}
      \step{1c}{\pick\ $y \in X$ and $\epsilon > 0$ such that $x \in B(y, 
        \epsilon) \subseteq U$
        \prove{$B(x, \epsilon - d(x, y)) \subseteq U$}}
      \step{1d}{\pflet{$z \in B(x, \epsilon - d(x, y))$}}
      \step{1e}{$d(z, y) < \epsilon$}
      \begin{proof}
        \pf\ \begin{align*}
          d(z, y) & \leq d(z, x) + d(x, y) & (\text{Triangle Inequality}) \\
          & < \epsilon & (\text{\stepref{1d}})
        \end{align*}
      \end{proof}
      \qedstep
      \begin{proof}
        \pf\ From \stepref{1c} and \stepref{1e}
      \end{proof}
    \end{proof}
    \step{2}{If, for all $x \in U$, there exists $\epsilon > 0$ such that $B(x, 
      \epsilon) \subseteq U$, then $U$ is open.}
    \begin{proof}
      \pf\ Immediate from the definition of the topology.
    \end{proof}
  \end{proof}
  
  \begin{lm}
    Let $d$ and $d'$ be two metrics on the set $X$. Let $\mathcal{T}_d$ and 
    $\mathcal{T}_{d'}$ be the induced topologies. Then $\mathcal{T}_{d'}$ is 
    finer 
    than $\mathcal{T}_d$ if and only if, for all $x \in X$ and $\epsilon > 0$, 
    there exists $\delta > 0$ such that
    \[ B_{d'}(x, \delta) \subseteq B_d(x, \epsilon) \enspace . \]
  \end{lm}
  
  \begin{proof}
    \step{1}{If $\mathcal{T}_{d'}$ is finer than $\mathcal{T}_d$ then, for all 
      $x 
      \in X$ and $\epsilon > 0$, 
      there exists $\delta > 0$ such that $B_{d'}(x, \delta) \subseteq B_d(x, 
      \epsilon)$.}
    \begin{proof}
      \step{1a}{\assume{$\mathcal{T}_{d'}$ is finer than $\mathcal{T}_d$}}
      \step{1b}{\pflet{$x \in X$ and $\epsilon > 0$}}
      \step{1c}{$B_d(x, \epsilon)$ is open in $\mathcal{T}_{d'}$}
      \begin{proof}
        \pf\ From \stepref{1a}.
      \end{proof}
      \step{1d}{There exists $\delta > 0$ such that $B_{d'}(x, \delta) 
        \subseteq 
        B_d(x, \epsilon)$}
      \begin{proof}
        \pf\ By Lemma \ref{lm:topology:metric:open}.
      \end{proof}
    \end{proof}
    \step{2}{If, for all $x \in X$ and $\epsilon > 0$, 
      there exists $\delta > 0$ such that $B_{d'}(x, \delta) \subseteq B_d(x, 
      \epsilon)$, then $\mathcal{T}_{d'}$ is finer than $\mathcal{T}_d$.}
    \begin{proof}
      \pf\ From Lemma \ref{lm:topology_basis:finer}
    \end{proof}
  \end{proof}
  
  \begin{thm}
    \label{thm:topology:metric:square_product}
    The topology induced by the square metric is the same as the product 
    topology 
    on $\mathbb{R}^n$.
  \end{thm}
  
  \begin{proof}
    \step{1}{Any set of the form $(a_1, b_1) \times \cdots \times (a_n, b_n)$ 
      is 
      open under the square metric.}
    \begin{proof}
      \step{1a}{\pflet{$\vec{x} \in (a_1, b_1) \times \cdots \times (a_n, 
          b_n)$}}
      \step{1b}{\pflet{$\epsilon = \min(x_1 - a_1, b_1 - x_1, \cdots, x_n - 
          a_n, 
          b_n - x_n)$}}
      \step{1c}{$B_\rho(\vec{x}, \epsilon) \subseteq (a_1, b_1) \times \cdots 
        \times (a_n, b_n)$}
    \end{proof}
    \step{2}{Any ball $B_\rho(\vec{x}, \epsilon)$ is open in the product 
      topology.}
    \begin{proof}
      \pf\ $B_\rho(\vec{x}, \epsilon) = (x_1 - \epsilon, x_1 + \epsilon) \times 
      \cdots \times (x_n - \epsilon, x_n + \epsilon)$.
    \end{proof}
  \end{proof}
  
  \begin{thm}
    The product of a countable family of metrizable spaces is metrizable.
  \end{thm}
  
  \begin{proof}
    \step{1}{\pflet{$(X_n)_{n \in \mathbb{N}}$ be a family of metric spaces.} 
      \pflet{$P = \prod_{n=0}^\infty X_n$}}
    \step{2}{\pflet{$d_n$ be a metric that induces the topology on $X_n$ and is 
        bounded by 1.}}
    \begin{proof}
      \pf\ Such a metric always exists by Lemma 
      \ref{lm:topology:metric:bounded}.
    \end{proof}
    \step{3}{\pflet{$D : (\prod_{n=0}^\infty X_n)^2 \rightarrow \mathbb{R}$ be 
        defined by
        \[ D(\vec{x}, \vec{y}) = \sup_{n \in \mathbb{N}} \frac{d_n(x_n, 
          y_n)}{n+1} \]}}
    \step{4}{$D$ is a metric on $\prod_{n = 0}^\infty X_n$}
    \begin{proof}
      \step{1a}{For all $\vec{x}, \vec{y} \in P$, we have 
        $D(\vec{x}, \vec{y}) \geq 0$}
      \step{1b}{For all $\vec{x}, \vec{y} \in P$, we have 
        $D(\vec{x}, \vec{y}) = 0$ if and only if $\vec{x} = \vec{y}$.}
      \step{1c}{$D(\vec{x}, \vec{y}) = D(\vec{y}, \vec{x})$}
      \step{1d}{$D(\vec{x}, \vec{z}) \leq D(\vec{x}, \vec{y}) + D(\vec{y}, 
        \vec{z})$}
      \begin{proof}
        \step{i}{\pflet{$i \geq 0$}}
        \step{ii}{$\frac{d_i(x_i, z_i)}{i+1} \leq \frac{d_i(x_i, 
            y_1)}{i+1} + \frac{d_i(y_i, z_i)}{i+1}$}
        \step{iii}{$\frac{d_i(x_i, z_i)}{i+1} \leq D(\vec{x}, \vec{y}) + 
          D(\vec{y}, \vec{z})$}
      \end{proof}
    \end{proof}
    \step{5}{Every $D$-ball is open in the product topology}
    \begin{proof}
      \step{5a}{\pflet{$\vec{x} \in P$ and $\epsilon > 0$}}
      \step{5b}{\pick\ $N$ such that $1/N < \epsilon$}
      \step{5c}{$B(\vec{x}, \epsilon) = \prod_{i=0}^N (x_i - (i+1)\epsilon, x_i 
        + i 
        \epsilon) \times \prod_{i=N+1}^\infty X_i$}
    \end{proof}
    \step{6}{Every subbasis element for the product topology is open under the 
      $D$-metric.}
    \begin{proof}
      \step{3a}{\pflet{$U = \prod_{i=0}^\infty U_i$ where $U_{i_0} = B(a, 
          \epsilon)$ and $U_i = X_i$ for all other $i$}}
      \step{3b}{\pflet{$\vec{x} \in U$}}
      \step{3d}{$B(\vec{x}, \epsilon / (i_0 + 1)) \subseteq U$}
    \end{proof}
  \end{proof}
  
  \begin{thm}
    Let $(X_1, d_1)$, \ldots, $(X_n, d_n)$ be metric spaces. Define $d : (X_1 
    \times \cdots \times X_n)^2 \rightarrow \mathbb{R}$ by
    \[ d(\vec{x}, \vec{y}) = \max(d_1(x_1, y_1), \ldots, d_n(x_n, y_n)) \]
    Then $d$ is a metric on $X_1 \times \cdots \times X_n$ that induces the 
    product topology.
  \end{thm}
  
  \subsection{Real Analysis}
  
  \begin{lm}
    Addition is a continuous function $\mathbb{R}^2 \rightarrow \mathbb{R}$.
  \end{lm}
  
  \begin{proof}
    \step{1}{\pflet{$a, b \in \mathbb{R}$ and $\epsilon > 0$}}
    \step{2}{\pflet{$\delta = \epsilon / 2$}}
    \step{3}{\pflet{$(x, y) \in B((a, b), \delta)$ under the metric 
        $\lambda ((x_1, y_1), (x_2, y_2)). \max(|x_1 - y_1|, |x_2 - y_2|)$}}
    \step{4}{$|(x + y) - (a + b)| < \epsilon$}
    \begin{proof}
      \pf
      \begin{align*}
        |(x + y) - (a + b)| & \leq | x - a | + | y - b | \\
        & < 2 \delta & (\text{\stepref{3}}) \\
        & = \epsilon & (\text{\stepref{2}})
      \end{align*}
    \end{proof}
  \end{proof}
  
  \begin{lm}
    Mulitplication is a continuous function $\mathbb{R}^2 \rightarrow 
    \mathbb{R}$.
  \end{lm}
  
  \begin{proof}
    \step{1}{\pflet{$a, b \in \mathbb{R}$ and $\epsilon > 0$}}
    \step{2}{\pflet{$\delta = \min(\epsilon / (|a| + |b| + 1), 1)$}}
    \step{3}{\pflet{$(x, y) \in B((a, b), \delta)$ under the metric 
        $\lambda ((x_1, y_1), (x_2, y_2)). \max(|x_1 - y_1|, |x_2 - y_2|)$}}
    \step{4}{$|xy - ab| < \epsilon$}
    \begin{proof}
      \pf
      \begin{align*}
        |xy - ab| & = |(x-a)(y-b) + a(y-b) + (x-a)b| \\
        & \leq |x-a||y-b| + |a||y-b| + |x-a||b| \\
        & < \delta^2 + |a| \delta + |b| \delta \\
        & \leq \delta + |a| \delta + |b| \delta \\
        & \leq \epsilon
      \end{align*}
    \end{proof}
  \end{proof}
  
  \begin{lm}
    The function $i = \lambda x. x^{-1}$ is a continuous function $\mathbb{R} 
    \setminus \{0\} \rightarrow \mathbb{R}$.
  \end{lm}
  
  \begin{proof}
    \step{1}{\pflet{$a, b \in \mathbb{R}$ with $a < b$} \prove{$\inv{i}((a, 
        b))$ 
        is open}}
    \step{2}{\case{$0 < a < b$}}
    \begin{proof}
      \pf\ In this case, we have $\inv{i}((a, b)) = (\inv{b}, \inv{a})$.
    \end{proof}
    \step{3}{\case{$a = 0$}}
    \begin{proof}
      \pf\ In this case, we have $\inv{i}((a, b)) = (\inv{b}, +\infty)$.
    \end{proof}
    \step{4}{\case{$a < 0 < b$}}
    \begin{proof}
      \pf\ In this case, we have $\inv{i}((a, b)) = (- \infty, \inv{a}) \cup 
      (\inv{b}, + \infty)$.
    \end{proof}
    \step{5}{\case{$b = 0$}}
    \begin{proof}
      \pf\ In this case, we have $\inv{i}((a, b)) = (- \infty, \inv{a})$.
    \end{proof}
    \step{6}{\case{$a < b < 0$}}
    \begin{proof}
      \pf\ In this case, we have $\inv{i}((a, b)) = (\inv{b}, \inv{a})$.
    \end{proof}
    \qed
  \end{proof}
  
  \begin{lm}
    \label{lm:analysis:increasing_converge}
    Every increasing sequence of real numbers bounded above converges to its 
    supremum.
  \end{lm}
  
  \begin{proof}
    \step{1}{\pflet{$(s_n)_{n \in \mathbb{N}}$ be an increasing sequence of 
        real 
        numbers bounded above with supremum $u$.}}
    \step{2}{\pflet{$\epsilon > 0$}}
    \step{3}{$u - \epsilon$ is not an upper bound for $s$}
    \step{4}{\pick $N$ such that $s_N > u - \epsilon$}
    \step{5}{For $n \geq N$, we have $u - \epsilon < s_n \leq u$}
  \end{proof}
  
  \begin{df}[Infinite Series]
    Let $a$ be a sequence of real numbers. The \emph{infinite series} 
    $\sum_{i=0}^\infty a_0$ \emph{converges} to the \emph{sum} $s$,
    \[ \sum_{i=0}^\infty a_i = s \]
    if and only if the sequence of \emph{partial sums}
    \[ \sum_{i=0}^n a_i \]
    converges to $s$ as $n \rightarrow \infty$.
  \end{df}
  
  \begin{lm}
    \label{lm:analysis:series_linear}
    If $\sum_{i=0}^\infty a_i = s$ and $\sum_{i=0}^\infty b_i = t$ then 
    $\sum_{i=0}^\infty (c a_i + b_i) = cs + t$.
  \end{lm}
  
  \begin{proof}
    \begin{align*}
      \sum_{i=0}^n (c a_i + b_i) & = c \sum_{i=0}^n a_i + \sum_{i=0}^n b_i \\
      & \rightarrow c s + t & \text{as } n \rightarrow \infty
    \end{align*}
  \end{proof}
  
  \begin{thm}[Comparison Test]
    If $|a_i| \leq b_i$ for all $i$ and $\sum_{i=0}^\infty b_i$ converges then 
    $\sum_{i=0}^\infty a_i$ converges.
  \end{thm}
  
  \begin{proof}
    \pf
    \step{1}{$\sum_{i=0}^\infty |a_i|$ converges}
    \begin{proof}
      \pf\ By Lemma \ref{lm:analysis:increasing_converge}, since $\sum_{i=0}^n 
      |a_i|$ is an increasing sequence bounded above by $\sum_{i=0}^\infty b_i$.
    \end{proof}
    \step{2}{$\sum_{i=0}^\infty (|a_i| + a_i)$ converges}
    \begin{proof}\
      \pf\ By Lemma \ref{lm:analysis:increasing_converge}, since $\sum_{i=0}^n 
      (|a_i| + a_i)$ is an increasing sequence bounded above by $2 
      \sum_{i=0}^\infty 
      b_i$.
    \end{proof}
    \step{3}{$\sum_{i=0}^\infty a_i$ converges}
    \begin{proof}
      \pf\ From \stepref{1}, \stepref{2} and Lemma 
      \ref{lm:analysis:series_linear}.
    \end{proof}
  \end{proof}
  
  \begin{thm}[Weierstrass $M$-test]
    Let $f_n : X \rightarrow \mathbb{R}$ for all $n \in \mathbb{N}$. Let
    \[ s_n(x) = \sum_{i=0}^n f_i(x) \qquad (n \in \mathbb{N}, x \in X) \enspace 
    . 
    \]
    Suppose $|f_i(x)| \leq M_i$ for all $i \in \mathbb{N}$ and $x \in X$, and 
    $\sum_{i=0}^\infty M_i$ converges. Then
    \[ s_n(x) \rightarrow s(x) = \sum_{i=0}^\infty f_i(x) \]
    uniformly as $n \rightarrow \infty$.
  \end{thm}
  
  \begin{proof}
    \pf
    \step{1}{For all $x \in X$, we have $\sum_{i=0}^\infty f_i(x)$ converges.}
    \begin{proof}
      \pf\ By the Comparison Test.
    \end{proof}
    \step{2}{Let $r_n = \sum_{i=n+1}^\infty M_i$ for $n \in \mathbb{N}$.}
    \begin{proof}
      \pf\ The series converges because
      \[ \sum_{i=n+1}^m M_i = \sum_{i=0}^m M_i - \sum_{i=0}^n M_i \rightarrow 
      \sum_{i=0}^\infty M_i - \sum_{i=0}^n M_i \]
      as $m \rightarrow \infty$.
    \end{proof}
    \step{2.5}{$r_n \rightarrow 0$ as $n \rightarrow \infty$}
    \begin{proof}
      \pf\ \begin{align*}
        r_n & = \sum_{i=0}^\infty M_i - \sum_{i=0}^n M_i \\
        & \rightarrow \sum_{i=0}^\infty M_i - \sum_{i=0}^\infty M_i & \text{as } 
        n 
        \rightarrow \infty \\
        & = 0
      \end{align*}
    \end{proof}
    \step{3}{For $k > n$, we have $|s_k(x) - s_n(x)| \leq r_n$ for all $x \in 
      X$}
    \begin{proof}
      \pf\ \begin{align*}
        |s_k(x) - s_n(x)| & = |\sum_{i=n+1}^k f_i(x)| \\
        & \leq \sum_{i=n+1}^k |f_i(x)| \\
        & = \sum_{i=n+1}^k M_i \\
        & \leq \sum_{i=n+1}^\infty M_i \\
        & = r_n
      \end{align*}
    \end{proof}
    \step{4}{For all $n \in \mathbb{N}$, we have $|s(x) - s_n(x)| \leq r_n$}
    \begin{proof}
      \pf\ \[  |s(x) - s_n(x)| = \lim_{k \rightarrow \infty} |s_k(x) - s_n(x)| 
      \leq r_n \]
    \end{proof}
    \step{5}{$s \rightarrow s_n$ uniformly as $n \rightarrow \infty$}
    \begin{proof}
      \step{5a}{\pflet{$\epsilon > 0$}}
      \step{5b}{\pick\ $N$ such that $r_n < \epsilon$ for all $n \geq N$}
      \begin{proof}
        \pf\ By \stepref{2.5}
      \end{proof}
      \step{5c}{For $n \geq N$ and all $x \in X$, we have $|s(x) - s_n(x)| < 
        \epsilon$}
      \begin{proof}
        \pf\ From \stepref{4}
      \end{proof}
    \end{proof}
  \end{proof}
  
  \subsection{The Uniform Metric}
  
  \begin{df}[Uniform Metric]
    Let $J$ be a nonempty set.
    The \emph{uniform metric} on $\mathbb{R}^J$ is defined by
    \[ \overline{\rho}(\vec{x}, \vec{y}) = \sup_{j \in J} \overline{d}(x_j, y_j) 
    \]
    where $\overline{d}$ is the standard bounded metric on $\mathbb{R}$.
    
    We prove this is a metric.
  \end{df}
  
  \begin{proof}
    \step{1}{For all $x, y \in \mathbb{R}^J$, we have $\overline{\rho}(x, y) 
      \geq 
      0$}
    \step{2}{For all $x, y \in \mathbb{R}^J$, we have $\overline{\rho}(x, y) = 
      0$ 
      if and only if $x = y$}
    \step{3}{For all $x, y \in \mathbb{R}^J$, we have $\overline{\rho}(x, y) = 
      \overline{\rho}(y, x)$}
    \step{4}{For all $x, y, z \in \mathbb{R}^J$, we have $\overline{\rho}(x, z) 
      \leq \overline{\rho}(x, y) + \overline{\rho}(y, z)$}
    \begin{proof}
      \step{4a}{\pflet{$x, y, z \in \mathbb{R}^J$}}
      \step{4b}{For all $j \in J$, we have $\overline{d}(x_j, z_j) \leq 
        \overline{\rho}(x, y) + \overline{\rho}(y, z)$}
      \begin{proof}
        \pf\ \begin{align*}
          \overline{d}(x_j, z_j) & \leq \overline{d}(x_j, y_j) + 
          \overline{d}(y_j, z_j) \\
          & \leq \overline{\rho}(x, y) + \overline{\rho}(y, z)
        \end{align*}
      \end{proof}
      \step{4c}{$\overline{\rho}(x, z) \leq \overline{\rho}(x, y) + 
        \overline{\rho}(y, z)$}
    \end{proof}
  \end{proof}
  
  \begin{df}[Uniform Topology]
    The \emph{uniform topology} on $\mathbb{R}^J$ is the topology induced by 
    the 
    uniform metric.
  \end{df}
  
  \begin{thm}[CC]
    The uniform topology on $\mathbb{R}^J$ is finer than the product topology 
    and 
    coarser than the box topology. These three topologies are all different if 
    and 
    only if $J$ is infinite.
  \end{thm}
  
  \begin{proof}
    \step{1}{The uniform topology is finer than the product topology.}
    \begin{proof}
      \step{1a}{Let $U_\alpha = (a_\alpha, b_\alpha)$ for $\alpha = \alpha_1, 
        \ldots, \alpha_n \in J$ and $U_\alpha = \mathbb{R}$ for all other 
        $\alpha$.}
      \step{1b}{\pflet{$x \in \prod_{\alpha \in J} U_\alpha$}}
      \step{1c}{\pflet{$\epsilon = \min(x_{\alpha_1} - a_{\alpha_1}, 
          b_{\alpha_1} - 
          x_{\alpha_1}, \ldots, x_{\alpha_n} - a_{\alpha_n}, b_{\alpha_n} - 
          x_{\alpha_n}, 
          1/2)$}}
      \step{1d}{$B_{\overline{\rho}}(x, \epsilon) \subseteq \prod_{\alpha \in 
          J} 
        U_\alpha$}
    \end{proof}
    \step{2}{The uniform topology is coarser than the box topology.}
    \begin{proof}
      \pf\ We have $B(x, \epsilon) = \prod_{\alpha \in J} (x_\alpha - \epsilon, 
      x_\alpha + \epsilon)$ if $\epsilon < 1$, and $B(x, \epsilon) = 
      \mathbb{R}^J$ if 
      $\epsilon \geq 1$.
    \end{proof}
    \step{3}{If $J$ is finite, then the three topologies are equal.}
    \step{4}{If $J$ is infinite, then the uniform topology is strictly finer 
      than the product topology.}
    \begin{proof}
      \pf\ $B(0, 1/2)$ is not open in the product topology.
    \end{proof}
    \step{5}{If $J$ is infinite then the uniform topology is strictly coarser 
      than 
      the box topology.}
    \begin{proof}
      \step{5a}{\pick\ a countably infinite sequence $\alpha_1$, $\alpha_2$, 
        \ldots in $J$.}
      \begin{proof}
        \pf\ This requires Countable Choice.
      \end{proof}
      \step{5b}{\pflet{$U = \prod_{\alpha \in J} U_\alpha$ where $U_{\alpha_i} 
          = 
          (-1/i, 1/i)$ for $i = 1, 2, \ldots$ and $U_\alpha = \mathbb{R}$ for 
          all other 
          $\alpha$}}
      \step{5c}{$U$ is not open in the uniform topology.}
      \begin{proof}
        \pf\ There is no $\epsilon > 0$ such that $B(\vec{0}, \epsilon) 
        \subseteq 
        U$.
      \end{proof}
    \end{proof}
  \end{proof}
  
  \begin{df}[Uniform Convergence]
    Let $X$ be a set and $Y$ a metric space. Let $f_n : X 
    \rightarrow Y$ for $n \in \mathbb{N}$ and $f : X \rightarrow Y$. Then $f_n 
    \rightarrow f$ as $n \rightarrow \infty$ uniformly iff $f_n \rightarrow f$ 
    as 
    $n \rightarrow \infty$ in the space $Y^X$ under the uniform topology.
  \end{df}
  
  \begin{thm}
    Let $X$ be a set and $Y$ a metric space. Let $f_n : X 
    \rightarrow Y$ for $n \in \mathbb{N}$ and $f : X \rightarrow Y$. Then $f_n 
    \rightarrow f$ as $n \rightarrow \infty$ uniformly if 
    and only if, for all $\epsilon > 0$, there exists $N$ such that, for all $x 
    \in 
    X$ and $n \geq N$,
    \[ d(f_n(x), f(x)) < \epsilon \enspace . \]
  \end{thm}
  
  \begin{proof}
    \step{1}{If $f_n \rightarrow f$ as $n \rightarrow \infty$ uniformly then, 
      for 
      all $\epsilon > 0$, there exists $N$ such that, for all $x \in X$ and $n 
      \geq 
      N$, $d(f_n(x), f(x)) < \epsilon$.}
    \begin{proof}
      \step{1a}{\assume{$f_n \rightarrow f$ as $n \rightarrow \infty$}}
      \step{1b}{\pflet{$\epsilon > 0$}}
      \step{1c}{\pick\ $N$ such that, for all $n \geq N$, $\overline{\rho}(f_n, 
        f) 
        < \epsilon$}
      \step{1d}{For all $x \in X$ and $n \geq N$, $d(f_n(x), f(x)) < \epsilon$.}
    \end{proof}
    \step{2}{If, for 
      all $\epsilon > 0$, there exists $N$ such that, for all $x \in X$ and $n 
      \geq 
      N$, $d(f_n(x), f(x)) < \epsilon$, then $f_n \rightarrow f$ as $n 
      \rightarrow 
      \infty$ uniformly.}
    \begin{proof}
      \step{2a}{\assume{for 
          all $\epsilon > 0$, there exists $N$ such that, for all $x \in X$ and 
          $n \geq 
          N$, $d(f_n(x), f(x)) < \epsilon$}}
      \step{2b}{\pflet{$\epsilon > 0$}}
      \step{2c}{\pick\ $N$ such that, for all $x \in X$ and $n \geq N$, 
        $d(f_n(x), 
        f(x)) < \epsilon / 2$}
      \step{2d}{For $n \geq N$, $\overline{\rho}(f_n, f) < \epsilon$}
      \begin{proof}
        \pf\ $\overline{\rho}(f_n, f) \leq \epsilon / 2$.
      \end{proof}
    \end{proof}
    \qed
  \end{proof}
  
  \begin{thm}[Uniform Limit Theorem]
    Let $X$ be a topological space and $Y$ a metric space. Let $f_n : X 
    \rightarrow Y$ for $n \in \mathbb{N}$ and $f : X \rightarrow Y$. If every 
    $f_n$ is continuous and $f_n \rightarrow f$ as $n \rightarrow \infty$ 
    uniformly, then $f$ is continuous.
  \end{thm}
  
  \begin{proof}
    \step{1}{\pflet{$V$ be open in $Y$}}
    \step{2}{\pflet{$x_0 \in \inv{f}(V)$} \prove{There exists a neighbourhood 
        $U$ 
        of $x_0$ such that $U \subseteq \inv{f}(V)$}}
    \step{3}{\pflet{$y_0 = f(x_0)$}}
    \step{4}{\pick\ $\epsilon > 0$ such that $B(y_0, \epsilon) \subseteq V$}
    \step{5}{\pick\ $N$ such that, for all $n \geq N$ and $x \in X$, $d(f_n(x), 
      f(x)) < \epsilon / 3$}
    \step{6}{\pick\ a neighbourhood $U$ of $x_0$ such that $U \subseteq 
      \inv{f_N}(B(f_N(x_0), \epsilon / 3))$ \prove{$U \subseteq \inv{f}(V)$}}
    \step{7}{\pflet{$x' \in U$}}
    \step{8}{$d(f(x'), y_0) < \epsilon$}
    \begin{proof}
      \pf
      \begin{align*}
        d(f(x'), y_0) & \leq d(f(x'), f_N(x')) + d(f_N(x'), f_N(x_0)) + 
        d(f_N(x_0), y_0) & (\text{Triangle Inequality}) \\
        & < \epsilon / 3 + \epsilon / 3 + \epsilon / 3 & (\text{\stepref{5}, 
          \stepref{6}}) \\
        & = \epsilon
      \end{align*}
    \end{proof}
    \step{9}{$f(x') \in V$}
    \begin{proof}
      \pf\ By \stepref{4}.
    \end{proof}
    \qed
  \end{proof}
  
  \chapter{Algebra}
  
  \section{Group Theory}
  
  \begin{df}[Action]
    An \emph{action} of a group $G$ is a functor $G \rightarrow \Set$.
  \end{df}
  
  \section{Polynomials}
  
  \begin{lm}
    \[ d(P + Q) \leq \max(d(P), d(Q)) \]
  \end{lm}
  
  \begin{lm}
    \[ d(PQ) = d(P) + d(Q) \]
  \end{lm}
  
  \begin{lm}[Division Algorithm for Polynomials]
    For any two polynomials $P$ and $D$ with $D \neq 0$, there exist unique 
    polynomials $Q$ and $R$ such that $P = DQ + R$ and $d(R) < d(D)$ or $R = 0$.
  \end{lm}
  
  \begin{proof}
    \pf
    \step{A}{\pflet{$D$ be a non-zero polynomial.}}
    \step{B}{For any polynomial $P$, there exist polynomials $Q$ and $R$ such 
      that $P = DQ + R$ and $d(R) < d(D)$ or $R = 0$}
    \begin{proof}
      \step{2}{There exist polynomials $Q$ and $R$ such that $0 = DQ + R$ and 
        $d(R)     < d(D)$ or $R = 0$}
      \begin{proof}
        \pf\ Take $Q = R = 0$.
      \end{proof}
      \step{3}{For every integer $n$ and every polynomial $P$ with $d(P) = n$, 
        there exist polynomials $Q$ and $R$ such that $P = DQ + R$ and $d(R) < 
        d(D)$   or $R = 0$}
      \begin{proof}
        \step{3a}{\assume{for all $k < n$ and every polynomial $P$ with $d(P) = 
            k$,         there exist polynomials $Q$ and $R$ such that $P = DQ + 
            R$ and $d(R) <         d(D)$ or  $R = 0$.}}
        \step{3b}{\pflet{$P$ be a polynomial of degree $n$}}
        \step{3c}{\case{$n < d(D)$}}
        \begin{proof}
          \pf\ Take $Q = 0$ and $R = P$.
        \end{proof}
        \step{3d}{\case{$n \geq d(D)$}}
        \begin{proof}
          \step{i}{\pflet{the leading term of $P$ be $ax^n$ and the leading term 
              of           $D$ be $bx^m$}}
          \step{ii}{\pflet{$P' = P - (a/b)x^{n-m}D$}}
          \step{iii}{$d(P') < n$ or $P' = 0$}
          \step{iv}{\pick{polynomials $Q'$ and $R'$ such that $P' = DQ' + R'$ 
              and           $d(R') < d(D)$ or $R' = 0$}}
          \step{v}{$P = D(Q' + (a/b)x^{n-m}) + R'$}
        \end{proof}
      \end{proof}
    \end{proof}
    \step{C}{If $DQ + R = DQ' + R'$ and $d(R) < d(D)$ or $R = 0$, and $d(R') < 
      d(D)$ or $R' = 0$, then $Q = Q'$ and $R = R'$}
    \begin{proof}
      \step{1}{$D(Q - Q') = R - R'$}
      \step{2}{$d(D(Q - Q')) < d(D)$ or $D(Q - Q') = 0$}
      \step{3}{$d(Q - Q') < 1$ or $D(Q - Q') = 0$}
      \step{4}{$Q - Q'$ is a constant $k$, say}
      \step{5}{$kD = R - R'$}
      \step{6}{$k = 0$}
      \step{7}{$Q = Q'$ and $R = R'$}
    \end{proof}
    \qed
  \end{proof}
  
  \begin{lm}
   For any polynomial $P$ and number $t$, there exists a unique polynomial $Q$ 
such that
\[ P(x) = (x - t) Q(x) + P(t) \]
  \end{lm}

  \begin{proof}
   \pf
    \step{1}{\pflet{$Q$ and $R$ be the unique polynomials such that
        \[ P(x) = (x - t) Q(x) + R \]
        and $R$ is constant}}
    \step{2}{$R = P(t)$}
    \begin{proof}
      \pf\ Take $x = t$ on both sides
    \end{proof}
    \qed
  \end{proof}

  \begin{cor}
   For any polynomial $P$ and number $t$, $x - t$ divides $P$ if and only if 
$P(t) = 0$.
  \end{cor}

  \begin{lm}
   Any two non-zero polynomials $P$ and $Q$ have a greatest common divisor $D$. 
Further, there exist polynomials $A$ and $B$ such that $D = AP + BQ$.
  \end{lm}

  \begin{proof}
    \step{1}{\pick{$A_0$ and $B_0$ such that $A_0 P + B_0 Q$ has minimal degree 
        out of all the non-zero polynomials of the form $AP+BQ$}}
    \step{2}{\pflet{$D = A_0 P + B_0 Q$}}
    \step{3}{$D \mid P$}
    \begin{proof}
      \step{3a}{\pflet{$E$, $R$ be the polynomials such that $P = ED + R$ and 
          $d(R) < d(D)$ or $R = 0$}}
      \step{3b}{$R = (1 - A_0 E)P - E B_0 Q$}
      \step{3c}{$R = 0$}
      \begin{proof}
        \pf\ By minimality of $d(D)$ in \stepref{1}.
      \end{proof}
    \end{proof}
    \step{4}{$D \mid Q$}
    \begin{proof}
      \pf\ Similar.
    \end{proof}
    \step{5}{If $R \mid P$ and $R \mid Q$ then $R \mid D$}
    \begin{proof}
      \pf\ Immediate from \stepref{2}.
    \end{proof}
    \qed
  \end{proof}

  \chapter{Linear Algebra}
  
  \section{Vector Spaces}
  
  \begin{df}[Vector Space]
    Let $K$ be a field. A \emph{vector space} $V$ over $K$ consists of:
    \begin{itemize}
      \item a set $V$ whose elements are called \emph{vectors};
      \item a vector $0 \in V$, the \emph{zero vector};
      \item a function $- : V \rightarrow V$;
      \item a function $+ : V^2 \rightarrow V$, called \emph{addition};
      \item a function $\cdot : K \times V \rightarrow V$, called \emph{scalar 
        multiplication}
    \end{itemize}
    such that, for all $u, v, w \in V$ and $\alpha, \beta \in K$:
    \begin{itemize}
      \item $u + (v + w) = (u + v) + w$
      \item $u + v = v + u$
      \item $v + 0 = v$
      \item $v + (-v) = 0$
      \item $\alpha (\beta v) = (\alpha \beta) v$
      \item $(\alpha + \beta) v = \alpha v + \beta v$
      \item $\alpha (u + v) = \alpha u + \alpha v$
      \item $1 v = v$
    \end{itemize}
  \end{df}
  
  \begin{lm}
    Let $V$ be a vector space over $K$ and $W \subseteq V$. If $W$ is a closed 
    under the operations of $V$, then $W$ is a vector space under the 
    restrictions 
    of these operations.
  \end{lm}
  
  \begin{df}[Subspace]
    Let $V$ be a vector space over $K$. A \emph{subspace} of $V$ is a vector 
    space $W$ such that $W \subseteq V$ and the operations of $W$ are the 
    restrictions of the operations of $V$.
  \end{df}
  
  \begin{thm}
    Let $V$ be a vector space over $K$ and $A \subseteq V$. The set of all 
    linear 
    combinations of the elements of $A$ is a subspace of $V$.
  \end{thm}
  
  \begin{df}[Span]
    Let $V$ be a vector space over $K$ and $A \subseteq V$. The \emph{(linear) 
      span} of $A$, $\langle A \rangle$, is the subspace of $V$ consisting of 
    all 
    linear 
    combinations of the elements of $A$.
  \end{df}
  
  \begin{df}[Linear Transformation]
    Let $V$ and $W$ be vector spaces over $K$ and $T : V \rightarrow W$. Then 
    $T$ 
    is a \emph{linear transformation} or \emph{linear map} iff:
    \begin{itemize}
      \item for all $u, v \in V$, we have $T(u + v) = T(u) + T(v)$
      \item for all $\alpha \in K$ and $v \in V$, we have $T(\alpha v) = \alpha 
      T(v)$
    \end{itemize}
  \end{df}
  
  \begin{lm}
    The composition of two linear transformations is linear.
  \end{lm}
  
  \begin{df}[Unit Point]
    The \emph{unit point} $e_i \in K^n$ is the vector with $i$th coordinate 1, 
    and all other coordinates 0.
  \end{df}
  
  \begin{lm}
    Let $V$ be a vector space over $K$ and $T : K^n \rightarrow V$. Then $T$ is 
    a 
    linear transformation iff
    \[ T(x_1, \ldots, x_n) = \sum_{i=1}^n x_i \beta_i \]
    where $\beta_i = T(e_i)$.
    
    This defines a bijection from $V^n$ to the set of all linear 
    transformations 
    $K^n \rightarrow V$.
  \end{lm}
  
  \begin{thm}
    A function $T : K^n \rightarrow K^m$ is linear if and only if
    \[ T(x_1, \ldots, x_n) = (\sum_{j=1}^n t_{1j} x_j, ldots, \sum_{j=1}^n 
    t_{mj} 
    x_j) \]
    where the values $t_{ij} \in K$ are given by:
    \[ T(e_j) = (t_{1j}, \ldots, t_{mj}) \]
    
    This defines a bijection from the set of $m \times n$-matrices over $K$ to 
    the 
    set of all linear transformations $K^n \rightarrow K^m$.
  \end{thm}
  
  \begin{thm}
    If $T : V \rightarrow W$ is linear then $T(\langle A \rangle) = \langle 
    T(A) 
    \rangle$.
  \end{thm}
  
  \begin{thm}
    If $T : V \rightarrow W$ is linear and $A$ is a subspace of $V$ then $T(A)$ 
    is a subspace of $W$.
  \end{thm}
  
  \begin{thm}
    If $T : V \rightarrow W$ is linear and $Y$ is a subspace of $W$ then 
    $\inv{T}(Y)$ is a subspace of $V$.
  \end{thm}
  
  \begin{df}[Kernel]
    The \emph{kernel} of a linear transformation $T : V \rightarrow W$ is $\ker 
    T 
    = \{ v \in V : T(v) = 0 \}$.
  \end{df}
  
  \begin{lm}
    \label{lm:linear:injective_kernel}
    A linear transformation $T$ is injective iff $\ker T = \{ 0 \}$.
  \end{lm}
  
  \begin{lm}
    Let $\{X_j\}_{j \in J}$ be a family of vector spaces over $K$. There exists 
    a 
    unique way to make $\prod_{j \in J} X_j$ into a vector space such that the 
    projections $\pi_j : \prod_{j \in J} X_j \rightarrow X_j$ are all linear, 
    namely:
    \begin{align*}
      0 & = (0)_{j \in J} \\
      -(x_j)_{j \in J} & = (-x_j)_{j \in J} \\
      (x_j)_{j \in J} + (y_j)_{j \in J} & = (x_j + y_j)_{j \in J} \\
      \alpha (x_j)_{j \in J} & = (\alpha x_j)_{j \in J}
    \end{align*}
  \end{lm}
  
  \begin{df}[Direct Sum]
    The \emph{direct sum} $\bigoplus_{j \in J} X_j$ is the vector space in the 
    previous lemma.
    
    The $j$th \emph{injection} $\theta_j : X_j \rightarrow \prod_{j \in J} X_j$ 
    is defined by: $\theta_j(x)$ is the element whose $j$th component is $x$, 
    and 
    whose other components are all zero.
  \end{df}
  
  \begin{lm}
    The injections are all linear.
  \end{lm}
  
  \begin{lm}
    \[ \pi_j \circ \theta_j = \id{X_j} \]
  \end{lm}
  
  \begin{lm}
    \[ \pi_j \circ \theta_i = 0 \text{ if } i \neq j \]
  \end{lm}
  
  \begin{lm}
    If $J$ is finite then \[ \sum_{j \in J} \theta_j \circ \pi_j = \id{\prod_{j 
        \in J} X_j} \]
  \end{lm}
  
  \begin{thm}
    If $T_j : V \rightarrow W_j$ is linear for each $j$ in the set $J$, 
    then there exists a unique linear transformation $T : V \rightarrow 
    \prod_{j 
      \in J} W_j$ such that $T_j = \pi_j \circ T$ for all $j \in J$.
  \end{thm}
  
  \begin{thm}
    If $T_j : V_j \rightarrow W$ is linear for each $j$ in the finite set $J$, 
    then there exists a unique linear transformation $T : \prod_{j \in J} V_j 
    \rightarrow W$ such that $T_j = T \circ \theta_j$ for all $j \in J$.
  \end{thm}
  
  \begin{df}
    Subspaces $V_1$, \ldots, $V_n$ of $V$ are \emph{independent} iff the 
    map $\bigoplus_{i=1}^n V_i \rightarrow V$ defined by
    \[ v_1 \oplus \cdots \oplus v_n \mapsto v_1 + \cdots + v_n \]
    is injective.
  \end{df}
  
  \begin{lm}
    $V_1$, \ldots, $V_n$ are independent if and only if, for all $v_1 \in V_1$, 
    \ldots, $v_n \in V_n$, if $v_1 + \cdots + v_n = 0$ then $v_1 = \cdots = v_n 
    = 
    0$.
  \end{lm}
  
  \begin{proof}
    \pf
    \step{1}{\pflet{$T : V_1 \oplus \cdots \oplus V_n \rightarrow V$ be the 
        function $T(v_1 \oplus \cdots \oplus v_n) = v_1 + \cdots + v_n$}}
    \step{2}{$T$ is injective if and only if, for all $v_1 \in V_1$, \ldots, 
      $v_n 
      \in V_n$, if $v_1 + \cdots + v_n = 0$ then $v_1 = \cdots = v_n = 0$}
    \begin{proof}
      \pf\ From Lemma \ref{lm:linear:injective_kernel}.
    \end{proof}
    \qed
  \end{proof}
  
  \begin{cor}
    Two subspaces $M$, $N$ are independent if and only if $M \cap N = \{ 0 \}$.
  \end{cor}
  
  \begin{proof}
    \pf
    \step{1}{If $M$ and $N$ are independent then $M \cap N = \{ 0 \}$.}
    \begin{proof}
      \step{1a}{\assume{$M$ and $N$ are independent.}}
      \step{1b}{\pflet{$v \in M \cap N$}}
      \step{1c}{$v - v = 0$}
      \step{1d}{$v = 0$}
      \begin{proof}
        \pf\ By the lemma.
      \end{proof}
    \end{proof}
    \step{2}{If $M \cap N = \{ 0 \}$ then $M$ and $N$ are independent.}
    \begin{proof}
      \step{2a}{\assume{$M \cap N = \{ 0 \}$}}
      \step{2b}{\pflet{$m \in M$, $n \in N$} \assume{$m + n = 0$}}
      \step{2c}{$m = -n$}
      \step{2d}{$m \in M \cap N$}
      \step{2e}{$m = 0$}
      \step{2f}{$n = 0$}
    \end{proof}
    \qed
  \end{proof}
  
  \begin{cor}
    Let $M$ and $N$ be subspaces of $V$. Then $V \cong M \oplus N$ via the 
    canonical isomorphism if and only if $V = M + N$ and $M \cap N = \{ 0 \}$.
  \end{cor}
  
  \begin{df}
    In this case, we say $M$ and $N$ are \emph{complementary} subspaces, and 
    each 
    is a \emph{complement} of the other.
  \end{df}
  
  \begin{lm}
    If $V_1$, \ldots, $V_n$ are independent subspaces of $V$, and $U_{i1}$, 
    \ldots, $U_{ir_i}$ are independent subspaces of $V_i$, then $U_{11}$, 
    \ldots, 
    $U_{nr_n}$ are independent subspaces of $V$.
  \end{lm}
  
  \begin{thm}
    \label{thm:linear:direct_sum}
    Let $P_1, \ldots, P_n : V \rightarrow V$ satisfy $\sum_{i=1}^n P_i = I$ and 
    $P_i \circ P_j = 0$ for $i \neq j$. Let $V_i = \im P_i$. Then $V \cong 
    \bigoplus_{i=1}^n V_i$ and $P_i$ is the projection onto $V_i$.
  \end{thm}
  
  \begin{proof}
    \pf
    \step{1}{$V = V_1 + \cdots + V_n$}
    \begin{proof}
      $v = P_1(v) + \cdots + P_n(v)$.
    \end{proof}
    \step{2}{If $v \in V_i$ then $P_i(v) = v$}
    \begin{proof}
      \pf $v = P_1(v) + \cdots + P_n(v) = P_i(v)$.
    \end{proof}
    \step{3}{$V_1$, \ldots, $V_n$ are independent.}
    \begin{proof}
      \step{2a}{\pflet{$v_1 \in V_1$, \ldots, $v_n \in V_n$}}
      \step{2b}{\pflet{$v_1 = P_1(u_1)$, \ldots, $v_n = P_n(u_n)$}}
      \step{2c}{\assume{$v_1 + \cdots + v_n = 0$}}
      \step{2d}{For each $i$, $v_i = 0$}
      \begin{proof}
        \pf
        \begin{align*}
          v_i & = P_i(v_i) & (\text{\stepref{2}}) \\
          & = P_i(v_1 + \cdots + v_n) \\
          & = P_i(0) \\
          & = 0
        \end{align*}
      \end{proof}
    \end{proof}
    \qed
  \end{proof}
  
  \begin{thm}
    Let $P : V \rightarrow V$ satisfy $P \circ P = P$. Then $V \cong \im P 
    \oplus 
    \ker P$ and $P$ is the projection $V \cong \im P \oplus \ker P \rightarrow 
    \im 
    P$.
  \end{thm}
  
  \begin{proof}
    \step{1}{\pflet{$Q = I - P$}}
    \step{2}{$PQ = 0$}
    \begin{proof}
      \pf $PQ = P-P^2 = 0$
    \end{proof}
    \step{3}{$QP = 0$}
    \begin{proof}
      \pf\ Similar.
    \end{proof}
    \step{4}{$V \cong \im P \oplus \im Q$}
    \begin{proof}
      \pf\ By Theorem \ref{thm:linear:direct_sum}.
    \end{proof}
    \step{5}{$\im Q = \ker P$}
    \begin{proof}
      \step{5a}{For all $v \in V$ we have $Q(v) \in \ker P$}
      \begin{proof}
        \pf\ By \stepref{2}.
      \end{proof}
      \step{5b}{For all $v \in \ker P$ we have $v \in \im Q$}
      \begin{proof}
        \pf $Q(v) = v - P(v) = v$.
      \end{proof}
    \end{proof}
    \qed
  \end{proof}
  
  \begin{thm}
    If $V$ and $W$ are vector spaces over $K$, then $\mathbf{Vect}_K[V, W]$ is 
    a 
    subspace of $W^V$.
  \end{thm}
  
  \begin{thm}
    Composition is a bilinear map $\mathbf{Vect}_K[W, X] \times 
    \mathbf{Vect}_K[V, W] \rightarrow \mathbf{Vect}_K[V, X]$.
  \end{thm}
  
  \begin{thm}
    Let $T$ be a surjective linear map from the vector space $V$ to the vector 
    space $W$, and let $N$ be its kernel. Then a subspace $M$ is a complement 
    of $N$ if and only if the restriction of $T$ to $M$ is an isomorphism from 
    $M$ to $W$. The mapping $M \mapsto \inv{(T \restriction M)}$ is a bijection 
    from the set of all such complementary subspaces of $M$ to the set of all 
    linear right inverses of $T$.
  \end{thm}
  
  \begin{proof}
    \pf\ 
    The inverse of the mapping takes a right inverse $S$ to $\im S$. \qed
  \end{proof}
  
  
  \subsection{Affine Subspaces}
  
  \begin{df}[Affine Subspace]
    Let $V$ be a vector space over $K$, $N$ a subspace of $V$, and $\alpha \in 
    V$.
    Then the \emph{coset} of $N$ containing $\alpha$, or the \emph{affine 
      subspace} of $V$ through $\alpha$ and parallel to $N$, or the 
    \emph{translate} of $N$ through $\alpha$, is the set
    \[ N + \alpha = \{ \xi + \alpha : \xi \in N \} \enspace . \]
  \end{df}
  
  \begin{lm}
    \label{lm:linear:affine:coset}
    If $\gamma \in N + \alpha$ then $N + \gamma = N + \alpha$.
  \end{lm}
  
  \begin{proof}
    \pf
    \step{1}{\assume{$\gamma = n + \alpha$}}
    \step{2}{$N + \gamma \subseteq N + \alpha$}
    \begin{proof}
      \pf\ For all $n' \in N$, we have $n' + \gamma = (n' + n) + \alpha$.
    \end{proof}
    \step{3}{$N + \alpha \subseteq N + \gamma$}
    \begin{proof}
      \pf\ For all $n' \in N$, we have $n' + \alpha = (n' - n) + \gamma$.
    \end{proof}
    \qed
  \end{proof}
  
  \begin{lm}
    The cosets of $N$ partition $V$.
  \end{lm}
  
  \begin{proof}
    \pf\ If $N + \alpha$ and $N + \beta$ are not disjoint, say $\gamma \in N + 
    \alpha \cap N + \beta$, then $N + \alpha = N + \gamma = N + \beta$. \qed
  \end{proof}
  
  \begin{lm}
    Given a family of cosets $\{ N_i + \alpha_i \}_{i \in I}$, we have
    $\bigcap_{i \in I} (N_i + \alpha_i)$ is either empty or a coset of 
    $\bigcap_{i \in I} N_i$.
  \end{lm}
  
  \begin{proof}
    \pf
    \step{1}{\assume{$\beta \in \bigcap_{i \in I} (N_i + \alpha_i)$}}
    \step{2}{$\bigcap_{i \in I} (N_i + \alpha_i) = \bigcap_{i \in I} N_i + 
      \beta$}
    \begin{proof}
      \pf
      \begin{align*}
        \bigcap_{i \in I} (N_i + \alpha_i) & = \bigcap_{i \in I} (N_i + \beta) 
        & 
        (\text{Lemma \ref{lm:linear:affine:coset}}) \\
        & = \bigcap_{i \in I} N_i + \beta
      \end{align*}
    \end{proof}
  \end{proof}
  
  \begin{lm}
    The sum of two cosets is a coset.
  \end{lm}
  
  \begin{proof}
    \pf\ $(M + \alpha) + (N + \beta) = (M + N) + (\alpha + \beta)$. \qed
  \end{proof}
  
  \begin{lm}
    Let $T : V \rightarrow W$ be a linear transformation. The image under $T$ of 
    a 
    coset in $V$ is a coset in $W$.
  \end{lm}
  
  \begin{proof}
    \pf\ $T(N + \alpha) = T(N) + T(\alpha)$. \qed
  \end{proof}
  
  \begin{cor}
    For any coset $A = N + \alpha$ and $\lambda \in K$, we have that $\lambda 
    A$ 
    is a coset, namely $\lambda A = N + \lambda \alpha$.
  \end{cor}
  
  \begin{lm}
    Let $T : V \rightarrow W$ be a linear transformation and $A$ a coset in 
    $W$. 
    Then $\inv{T}(A)$ is either empty or a coset in $V$.
  \end{lm}
  
  \begin{proof}
    \pf
    \step{1}{\pflet{$\alpha \in \inv{T}(A)$}}
    \step{2}{\pflet{$A = N + T(\alpha)$}}
    \step{3}{$\inv{T}(A) = \inv{T}(N) + \alpha$}
    \qed
  \end{proof}
  
  \begin{df}[Affine Transformation]
    Let $V$ and $W$ be vector spaces over $K$. An \emph{affine transformation} 
    $S 
    : V \rightarrow W$ is a function of the form
    \[ S(v) = T(v) + \alpha \]
    where $T : V \rightarrow W$ is a linear transformation, and $\alpha \in W$.
  \end{df}
  
  \begin{lm}
    The image of a coset under an affine transformation is a coset.
  \end{lm}
  
  \begin{proof}
    \pf\ Let $S(v) = T(v) + \alpha$. Then $S(N + \beta) = T(N) + (T(\beta) + 
    \alpha)$. \qed
  \end{proof}
  
  \begin{thm}
    Let $N$ be a subspace of $V$. The cosets of $N$ form a vector space $V / N$ 
    under 
    \begin{align*}
      (N + \alpha) + (N + \beta) & = N + (\alpha + \beta)
      \lambda (N + \alpha) & = N + \lambda \alpha
    \end{align*}
    The mapping $\pi : V \rightarrow V / N$
    \[ \pi(v) = N + v \]
    is a surjective linear transformation with kernel $N$.
    
    For any vector space $W$ and linear transformation $T : V \rightarrow W$ 
    such 
    that $N \subseteq \ker T$, there exists a unique linear transformation 
    $\overline{T} : V / N \rightarrow W$ such that the following diagram 
    commutes.
    \[ \begin{tikzcd}
      V \ar[r, "\pi"] \ar[dr, "T"] & V / N \ar[d, "\overline{T}"] \\
      & W
    \end{tikzcd} \]
  \end{thm}
  
  \begin{thm}
    Let $T : V \rightarrow V$ and $N$ be a subspace of $V$. If $T(N) \subseteq 
    N$, 
    then there exists a unique $S : V / N \rightarrow V / N$ such that $S \circ 
    \pi 
    = \pi \circ T$.
  \end{thm}
  
  \begin{thm}[Second Isomorphism Theorem]
    Let $M$ and $N$ be subspaces of $V$. Then $(M+N)/N \cong M/(M\cap N)$.
  \end{thm}
  
  \begin{thm}[Third Isomorphism Theorem]
    Let $N$ be a subspace of $V$ and $M$ a subspace of $N$. Then $M/N$ is a 
    subspace of $V/N$, and
    \[ (V/N)/(M/N) \cong V/M \]
  \end{thm}
  
  \begin{proof}
    \step{1}{\pflet{$T : V / N \rightarrow V/M$ be the unique linear 
        transformation such that $T \circ \pi_N = \pi_M$}}
    \step{2}{\pflet{$S : (V/N)/(M/N) \rightarrow V/M$ be the unique linear 
        transformation such that $S \circ \pi = T$}}
    \step{3}{$S$ is injective}
    \step{4}{$S$ is surjective}
  \end{proof}
  
  \section{Normed Spaces}
  
  \begin{df}[Norm]
    Let $V$ be a vector space over $K = \mathbb{R}$ or $K = \mathbb{C}$. A 
    \emph{norm} on $V$ is a function $\| \
    \| : V \rightarrow \mathbb{R}$ such that, for all $x, y \in V$ and $\lambda 
    \in
    K$,
    \begin{enumerate}
      \item $\| x \| \geq 0$
      \item If $\| x \| = 0$ then $x=0$
      \item $\| \lambda x \| = | \lambda | \| x \|$
      \item (\emph{Triangle Inequality}) $\| x + y \| \leq \| x \| + \| y \|$
    \end{enumerate}
    A \emph{normed space} consists of a complex vector space and a norm.
  \end{df}
  
  \begin{df}
    In a normed space, the metric \emph{induced} by the norm is $d(x, y) = \| x 
    - 
    y \|$.
    
    We prove this is a metric.
  \end{df}
  
  \begin{proof}
    \step{1}{$\| x - y \| \geq 0$}
    \begin{proof}
      \pf\ This follows from the positivity of the norm.
    \end{proof}
    \step{2}{If $\| x - y \| = 0$ then $x = y$}
    \begin{proof}
      \pf\ If $\| x - y \| = 0$ then $x - y = 0$ so $x = y$.
    \end{proof}
    \step{3}{$\| x - y \| = \| y - x \|$}
    \begin{proof}
      \pf\ \begin{align*}
        \| x - y \| & = | -1 | \| y - x \| \\
        & = \| y - x \|
      \end{align*}
    \end{proof}
    \step{4}{$\| x - z \| \leq \| x - y \| + \| y - z \|$}
    \begin{proof}
      \pf\ Immediate from the triangle inequality in the definition of norm.
    \end{proof}
  \end{proof}
  
  \begin{df}[Cauchy Sequence]
    Let $V$ be a normed space. A sequence $(a_n)$ in $V$ is a \emph{Cauchy
      sequence} iff, for all $\epsilon > 0$, there exists $N$ such that, for all 
    $m,
    n > N$, we have $\| a_m - a_n \| < \epsilon$.
  \end{df}
  
  \begin{df}[Converge]
    Let $V$ be a normed space. A sequence $(a_n)$ \emph{converges} to the
    \emph{limit} $l$, $a_n \rightarrow l$ as $n \rightarrow \infty$, iff, for 
    all
    $\epsilon > 0$, there exists $N$ such that, for all $n > N$, we have $\| x_n 
    -
    l \| < \epsilon$.
  \end{df}
  
  \begin{df}[Complete]
    A normed space is \emph{complete} or a \emph{Banach space} iff every Cauchy
    sequence converges.
  \end{df}
  
  \begin{df}[Bounded]
    Let $U$ and $V$ be normed spaces and $T : U \rightarrow V$ be a linear
    transformation. Then $T$ is \emph{bounded} iff there exists $M > 0$ such 
    that,
    for all $x \in U$,
    \[x \| T(x) \| \leq M \| x \| \]
  \end{df}
  
  \begin{lm}
    For any normed space $V$, the identity linear transformation on $V$ is
    bounded.
  \end{lm}
  
  \begin{proof}
    We have $\| id{V}(x) \| = 1 \| x \|$.
  \end{proof}
  
  \begin{lm}
    The composite of two bounded linear transformations is bounded.
  \end{lm}
  
  \begin{proof}
    Let $S : U \rightarrow V$ and $T : V \rightarrow W$ satisfy
    \[ \| S(x) \| \leq M \| x \|, \qquad \| T(y) \| \leq N \| y \| \]
    for $x \in U$ and $y \in V$, where $M, N > 0$. Then
    \[ \| T(S(x)) \| \leq MN \| x \| \]
    for $x \in U$.
  \end{proof}
  
  \section{Inner Product Spaces}
  
  \begin{df}[Inner Product]
    Let $V$ be a vector space over $K = \mathbb{R}$ or $K = \mathbb{C}$. An 
    \emph{inner product} on $V$ is
    a function $\langle \ , \ \rangle : V^2 \rightarrow K$ such that, for
    all $x,y,z \in V$ and $\alpha, \beta \in K$:
    \begin{enumerate}
      \item $\langle x,y \rangle = \overline{\langle y,x \rangle}$
      \item $\langle \alpha x + \beta y, z \rangle = \alpha \langle x,z \rangle 
      +
      \beta \langle y, z \rangle$
      \item $\langle x,x \rangle \geq 0$
      \item If $\langle x,x \rangle = 0$ then $x = 0$.
    \end{enumerate}
    An \emph{inner product space} consists of a vector space with an inner
    product.
  \end{df}
  
  \begin{df}
    We make $\mathbb{R}^n$ into an inner product space by defining
    \[ \langle x, y \rangle = x_1 y_1 + \cdots + x_n y_n \]
    We prove this is an inner product.
  \end{df}
  
  \begin{proof}
    \step{1}{$\langle x,y \rangle = \langle y,x \rangle$}
    \begin{proof}
      \pf\ \begin{align*}
        \langle x,y \rangle & = x_1 y_1 + \cdots + x_n y_n \\
        & = y_1 x_1 + \cdots + y_n x-n \\
        & = \langle y, x \rangle
      \end{align*}
    \end{proof}
    \step{2}{$\langle \alpha x + \beta y, z \rangle = \alpha \langle x,z \rangle 
      +
      \beta \langle y, z \rangle$}
    \begin{proof}
      \pf\ \begin{align*}
        \langle \alpha x + \beta y, z \rangle & = (\alpha x_1 + \beta y_1) 
        z_1 + \cdots + (\alpha x_n + \beta y_n) z_n \\
        & = \alpha x_1 z-1 + \cdots + \alpha x_n z_n + \beta y_1 z_1 + \cdots + 
        \beta 
        y_n z_n \\
        & = \alpha \langle x, z \rangle + \beta \langle y, z \rangle
      \end{align*}
    \end{proof}
    \step{3}{$\langle x, x \rangle \geq 0$}
    \begin{proof}
      \pf\ $\langle x, x \rangle = x_1^2 + \cdots + x_n^2 \geq 0$
    \end{proof}
    \step{4}{If $\langle x,x \rangle = 0$ then $x = 0$}
    \begin{proof}
      \pf\ If $\langle x, x \rangle = 0$ then $x_1^2 + \cdots + x_n^2 = 0$ so 
      $x_1 = \cdots = x_n = 0$.
    \end{proof}
  \end{proof}
  
  \begin{lm}
    \[ \langle x, \alpha y + \beta z \rangle = \overline{\alpha} \langle x,z
    \rangle + \overline{\beta} \langle x, z \rangle \]
  \end{lm}
  
  \begin{proof}
\begin{align*}
  \langle x, \alpha y + \beta z \rangle & = \overline{\langle \alpha y + \beta
z, x \rangle} \\
& = \overline{\alpha} \overline{\langle y, x \rangle} + \overline{\beta}
\overline{\langle z, x \rangle} \\
& = \overline{\alpha} \langle x,y \rangle + \overline{\beta} \langle x, z
\rangle
\end{align*}
\end{proof}

\begin{df}
  Given an inner product space $V$, define the \emph{norm} $\| \ \| : V
\rightarrow \mathbb{R}$ by
\[ \| x \| = \sqrt{\langle x,x \rangle} \]
\end{df}

\begin{lm}[Schwarz Inequality]
  \[ |\langle x,y \rangle| \leq \| x \| \| y \| \]
\end{lm}

\begin{proof}
Let $\lambda = - \langle x,y \rangle / \| y \|^2$. Then
  \begin{align*}
  0 & \leq \| x + \lambda y \|^2 \\
  & = \langle x + \lambda y, x + \lambda y \rangle \\
  & = \| x \|^2 + \lambda \langle y,x \rangle + \overline{\lambda} \langle x,y
\rangle + |\lambda|^2 \| y \|^2 \\
& =\| x \|^2 - |\langle x,y \rangle|^2/\|y\|^2
  \end{align*}
and the desired inequality follows.
\end{proof}

\begin{lm}
  The norm on an inner product space is a norm.
\end{lm}

\begin{proof}
  \begin{enumerate}
  \item If $\| x \| = 0$ then $\langle x,x \rangle = 0$ and so $x = 0$.
  \item \begin{align*}
          \| \lambda x \|^2 & = \langle \lambda x, \lambda x \rangle \\
          & = \lambda \overline{\lambda} \langle x,x \rangle \\
          & = | \lambda |^2 \| x \|^2
        \end{align*}
  \item \begin{align*}
          \| x + y \|^2 & = \langle x + y, x + y \rangle \\
          & = \| x \|^2 + \langle x,y \rangle + \overline{\langle x,y \rangle}
+ \| y \|^2 \\
& = \| x \|^2 + 2 \Re \langle x,y \rangle + \| y \|^2 \\
& \leq \| x \|^2 + 2 |\langle x,y \rangle| + \| y \|^2 \\
& \leq \| x \|^2 + 2 \| x \| \| y \| + \| y \|^2 & (\text{Schwarz inequality})
\\
& = (\| x \| + \| y \|)^2
        \end{align*}
  \end{enumerate}
\end{proof}

\begin{df}[Euclidean Metric]
The \emph{Euclidean metric} on $\mathbb{R}^n$ is the metric induced by the 
inner product.
\end{df}

\begin{lm}
The topology induced by the Euclidean metric is the same as that induced by 
the square metric.
\end{lm}

\begin{proof}
  \step{0}{\pflet{$d$ be the Euclidean metric and $\rho$ the square metric on 
$\mathbb{R}^n$}}
\step{1}{For all $\vec{x} \in \mathbb{R}^n$ and $\epsilon > 0$, we have 
$B_\rho(\vec{x}, \epsilon / \sqrt{n}) \subseteq B_d(\vec{x}, \epsilon)$}
  \begin{proof}
    \step{i}{$d(\vec{x}, \vec{y}) \leq \sqrt{n} \rho(\vec{x}, \vec{y})$}
    \begin{proof}
      \pf\ \begin{align*}
            d(\vec{x}, \vec{y})^2 & = (y_1 - x_1)^2 + \cdots + (y_n - x_n)^2
            & \leq n \max((y_1 - x_1)^2, \ldots, (y_n - x_n)^2) \\
            & = n \rho(\vec{x}, \vec{y})^2
          \end{align*}
    \end{proof}
  \end{proof}
\step{2}{For all $\vec{x} \in \mathbb{R}^n$ and $\epsilon > 0$, we have 
$B_d(\vec{x}, \epsilon) \subseteq B_\rho(\vec{x}, \epsilon)$}
  \begin{proof}
    \step{i}{$\rho(\vec{x}, \vec{y}) \leq d(\vec{x}, \vec{y})$}
    \begin{proof}
      \pf\ \begin{align*}
            \rho(\vec{x}, \vec{y})
            & = \max(|y_1 - x_1|, \ldots, |y_n - x_n|) \\
            & \leq \sqrt((y_1 - x_1)^2 + \cdots + (y_n - x_n)^2) \\
            & = d(\vec{x}, \vec{y})
          \end{align*}
    \end{proof}
  \end{proof}
\end{proof}

\begin{cor}
The topology induced by the Euclidean metric is the product topology on 
$\mathbb{R}^n$.
\end{cor}

\begin{proof}
From Theorem \ref{thm:topology:metric:square_product}.
\end{proof}

\begin{df}[Category of Hilbert Spaces]
  The \emph{category of Hilbert spaces} $\mathbf{Hilb}$ has objects all Hilbert
spaces and morphisms all bounded linear transformations.
\end{df}

\begin{thm}
$\mathbf{Hilb}$ is a compact symmetric monoidal category under $\otimes$ with 
unit $\mathbb{C}$
\end{thm}

\begin{thm}
$\mathbf{Hilb}$ is a dagger category.
\end{thm}

\begin{proof}
\pf\ Given $f : U \rightarrow V$, let $f^\dagger : V \rightarrow U$ be the 
unique linear transformation such that
\[ \langle f^\dagger(v), u \rangle = \langle v, f(u) \rangle \]
for all $u \in U$ and $v \in V$.
\end{proof}

\begin{thm}
$\mathbf{Hilb}$ is a Cartesian monoidal category under $\oplus$ with unit 
$\{0\}$.
\end{thm}

\begin{df}[Representation]
A \emph{representation} of a group $G$ on a Hilbert space is a functor $G 
\rightarrow \mathbf{Hilb}$.

An \emph{intertwining operator} between representations $A, B : G \rightarrow 
\mathbf{Hilb}$ is a natural transformation $A \Rightarrow B$.
\end{df}

\begin{thm}
$\mathbf{Hilb}$ is not Cartesian.
\end{thm}

To read:

J. S. Bell, On the Einstein-Podolsky-Rosen paradox, Physics 1 (1964), 195–200.

W. K. Wootters and W. H. Zurek, A single quantum cannot be cloned, Nature 299 
(1982), 802–803.

\subsection{The $l^2$-metric}

\begin{lm}
Let $\vec{x}$ and $\vec{y}$ be two sequences of real numbers such that 
$\sum_{i=0}^\infty x_i^2 < \infty$ and $\sum_{i=0}^\infty y_i^2 < \infty$. Then
$\sum_{i=0}^\infty |x_i y_i| < \infty$.
\end{lm}

\begin{proof}
\pf
\step{1}{$\{ \sum_{i=0}^n |x_i y_i| : n \in \mathbb{N} \}$ is bounded above by 
$\sqrt{\sum_{i=0}^\infty x_i^2 \sum_{i=0}^\infty y_i^2}$}
\begin{proof}
\step{1a}{\pflet{$n \in \mathbb{N}$}}
\step{1b}{\pflet{$\vec{a} = (|x_0|, \ldots, |x_n|), \vec{b} = (|y_0|, \ldots, 
|y_n|)$}}
\step{1c}{$\sum_{i=0}^n |x_i y_i| \leq \sqrt{\sum_{i=0}^\infty x_i^2 
\sum_{i=0}^\infty y_i^2}$}
\begin{proof}
\pf
\begin{align*}
  \sum_{i=0}^n |x_i y_i| & = \langle \vec{a}, \vec{b} \rangle \\
  & \leq \| \vec{a} \| \| \vec{b} \| & (\text{Schwarz Inequality}) \\
  & \leq \sqrt{\sum_{i=0}^\infty x_i^2} \sqrt{\sum_{i=0}^\infty y_i^2}
\end{align*}
\end{proof}
\end{proof}
\qedstep
\begin{proof}
\pf\ By Lemma \ref{lm:analysis:increasing_converge}.
\end{proof}
\end{proof}

\section{Algebras}

\begin{df}[Algebra]
  Let $K$ be a field. An \emph{algebra} $V$ over $K$ consists of:
  \begin{itemize}
  \item a vector space $V$ over $K$
  \item a function $\cdot : V^2 \rightarrow V$, the \emph{multiplication}
  \end{itemize}
such that, for all $x,y,z \in V$ and $\alpha \in K$,
\begin{itemize}
\item $x(yz) = (xy)z$
\item $x(y+z) = xy+xz$
\item $(x+y)z = xz+yz$
\item $\alpha(xy) = (\alpha x)y = x (\alpha y)$
\end{itemize}
\end{df}

\begin{thm}
  If $V$ is a vector space over $K$, then $\mathbf{Vect}_K[V, V]$ is an algebra 
over $K$ under composition.
\end{thm}

\begin{lm}
For any set $A$, the vector space $K^A$ is an algebra over $K$ under pointwise 
multiplication.
\end{lm}

\begin{lm}
  The vector space $\mathcal{C}([0,1])$ is a real algebra under pointwise 
multiplication.
\end{lm}

\section{Cobordisms}

\begin{df}
Let $n\mathbf{Cob}$ be the category whose objects are 
$(n-1)$-dimensional manifolds, and morphisms are $n$-dimensional cobordisms.
\end{df}

To read: S. Sawin, Links, quantum groups and TQFTs, Bull. Amer. Math. Soc. 33 
(1996), 413-445. Also available
as arXiv:q-alg/9506002

\begin{thm}
$n\mathbf{Cob}$ is a compact symmetric monoidal category, with $\otimes$ given 
by 
disjoint union.
\end{thm}

\begin{thm}
$n\mathbf{Cob}$ is a dagger category, with $f^\dagger : Y \rightarrow X$ the 
reflection of $f : X \rightarrow Y$.
\end{thm}

\section{Tangles}

\begin{df}
Let $\mathbf{Tang}_k$ be the category whose morphisms are $k$-codimensional 
tangles.
\end{df}

To read:
P. Freyd and D. Yetter, Braided compact monoidal categories with applications 
to 
low dimensional topol-
ogy, Adv. Math. 77 (1989), 156–182.

C. Kassel, Quantum Groups, Springer, Berlin, 1995.

S. Sawin, Links, quantum groups and TQFTs, Bull. Amer. Math. Soc. 33 (1996), 
413-445. Also available
as arXiv:q-alg/9506002.

M.-C. Shum, Tortile tensor categories, Jour. Pure Appl. Alg. 93 (1994), 57–110.

V. G. Turaev, Quantum Invariants of Knots and 3-Manifolds, de Gruyter, Berlin, 
1994.

D. N. Yetter, Functorial Knot Theory: Categories of Tangles, Coherence, 
Categorical Deformations, and
Topological Invariants, World Scientific, Singapore, 2001.

J. Baez and J. Dolan, Higher-dimensional algebra and topological quantum field 
theory, Jour. Math. Phys.
36 (1995), 6073–6105. Also available as arXiv:q-alg/9503002.

M.-C. Shum, Tortile tensor categories, Jour. Pure Appl. Alg. 93 (1994), 57–110.

\begin{thm}
The category $\mathbf{Tang}_k$ is a dagger category with $f^\dagger : Y 
\rightarrow X$ being the reflection of $f : X \rightarrow Y$.
\end{thm}

\begin{thm}
If $k \geq 1$ then $\mathbf{Tang}_k$ is a compact monoidal category, with 
$\otimes$ given by disjoint 
union.
\end{thm}

\begin{thm}
If $k \geq 2$ then $\mathbf{Tang}_k$ is a braided monoidal category.
\end{thm}

\begin{thm}
If $k \geq 3$ then $\mathbf{Tang}_k$ is a symmetric monoidal category.
\end{thm}

\begin{thm}
For all $m, n \geq 3$, the categories $\mathbf{Tang}_m$ andn $\mathbf{Tang}_n$ 
are equivalent.
\end{thm}

Proof: See J. Baez and J. Dolan, Higher-dimensional algebra and topological 
quantum field theory, Jour. Math. Phys.
36 (1995), 6073–6105. Also available as arXiv:q-alg/9503002.

\chapter{Knot Theory}

\begin{df}[Unit Ball]
  For $n \in \mathbb{N}$, the \emph{unit $n$-ball} is the subspace
  \[ D^n = \{ x \in \mathbb{R}^n : |x| \leq 1 \} \]
\end{df}

\begin{df}[Unit Sphere]
  For $n \in \mathbb{N}$, the \emph{unit $n$-sphere} is the subspace
  \[ S^{n-1} = \{ x \in \mathbb{R}^n : |x| = 1 \} \]
\end{df}

\begin{df}[Knot]
Let $X$ be a topological space.
  A \emph{knot} in $X$ is a subspace of $X$ that is homeomorphic with $S^p$ for
some $p$.

A \emph{link} is a sequence of disjoint knots.

Two knots $K$, $K'$ are \emph{equivalent} iff there exists an automorphism
$\phi : X \cong X$ such that $\phi(K) = K'$.

Two links of the same length $(K_1, \ldots, K_n)$ and $(K_1', \ldots, K_n')$
are \emph{equivalent} iff there exists an automorphism $\phi : X \cong X$ such
that $\phi(K_i) = K_i'$ for all $i$.

A \emph{knot type} is an equivalence class of knots.

A \emph{link type} is an equivalence class of links.
\end{df}

\end{document}
