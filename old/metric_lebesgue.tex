\section{Lebesgue Numbers}

\begin{df}[Lebesgue Number]
  Let $X$ be a metric space and $\mathcal{A}$ an open covering of $X$. A
  \emph{Lebesgue number} for $\mathcal{A}$ is a real $\delta > 0$ such that,
  for every nonempty set $A \subseteq X$ of diameter $< \delta$, there exists
  $U \in \mathcal{A}$ such that $A \subseteq U$.
\end{df}

\begin{lm}[Lebesgue Number Lemma]
  In a compact metric space, every open covering has a Lebesgue number.
\end{lm}

\begin{proof}
  \pf
  \step{<1>1}{\pflet{$X$ be a compact metric space and $\mathcal{A}$ an open
  covering of $X$} \prove{There exists a Lebesgue number $\delta$ for
  $\mathcal{A}$.}}
  \step{<1>2}{\assume{w.l.o.g.~$X \notin \mathcal{A}$}}
  \begin{proof}
    \pf\ If $X \in \mathcal{A}$ then we can take $\delta = 1$.
  \end{proof}
  \step{<1>3}{\pick\ a finite subcovering $\{ U_1, \ldots, U_n \} \subseteq
  \mathcal{A}$ that covers $X$}
  \step{<1>4}{For $1 \leq i \leq n$, \pflet{$C_i = X \setminus U_i$}}
  \step{<1>5}{\pflet{$f : X \rightarrow \mathbb{R}$ be defined by
  \[ f(x) = 1/n \sum_{i=1}^n d(x, C_i) \enspace . \]}}
  \begin{proof}
    \pf\ Each $C_i$ is nonempty by \stepref{<1>2}.
  \end{proof}
  \step{<1>6}{For all $x \in X$ we have $f(x) > 0$}
  \begin{proof}
    \step{<2>1}{\pflet{$x \in X$}}
    \step{<2>2}{\pick\ $i$ such that $x \in U_i$}
    \begin{proof}
      \pf\ By \stepref{<1>3}.
    \end{proof}
    \step{<2>3}{\pick\ $\epsilon > 0$ such that $B(x, \epsilon) \subseteq
    U_i$}
    \begin{proof}
      \pf\ By Lemma \ref{lm:topology:metric:open}.
    \end{proof}
    \step{<2>4}{$d(x, C_i) \geq \epsilon$}
  \end{proof}
  \step{<1>7}{$f$ is continuous}
  \begin{proof}
    \pf\ From Lemma \ref{lm:topology:metric:dist_continuous}.
  \end{proof}
  \step{<1>8}{\pflet{$\delta = \min f(X)$} \prove{For every nonempty set $A
  \subseteq X$        with diameter $< \delta$, there exists $U \in
  \mathcal{A}$ such that $A        \subseteq U$}}
  \begin{proof}
    \pf\ $f(X)$ has a minimum by the Extreme Value Theorem.
  \end{proof}
  \step{<1>9}{\pflet{$A \subseteq X$ be nonempty with $\diam A < \delta$}}
  \step{<1>10}{\pick\ $x_0 \in A$}
  \step{<1>11}{\pflet{$i$ be such that $d(x_0, C_i)$ is greatest among
  $d(x_0,
  C_1)$, \ldots, $d(x_0, C_n)$}}
  \step{<1>12}{$\delta \leq d(x_0, C_i)$}
  \begin{proof}
    \pf
    \begin{align*}
      \delta & \leq f(x_0) & (\text{\stepref{<1>8}}) \\
      & = 1/n \sum_{j=1}^n d(x_0, C_j) & (\text{\stepref{<1>5}}) \\
      & \leq 1/n \sum_{j=1}^n d(x_0, C_i) & (\text{\stepref{<1>11}}) \\
      & = d(x_0, C_i)
    \end{align*}
  \end{proof}
  \step{<1>13}{$x_0 \in U_i$}
  \begin{proof}
    \pf\ $x_0 \notin C_i$ because $d(x_0, C_i) > 0$.
  \end{proof}
  \qed
\end{proof}

\begin{thm}[DC]
  \label{thm:topology:metric:compact}
  Let $X$ be a metrizable space. Then the following are equivalent:
  \begin{enumerate}
    \item $X$ is compact.
    \item $X$ is limit point compact.
    \item $X$ is sequentially compact.
  \end{enumerate}
\end{thm}

\begin{proof}
  \pf
  \step{<1>1}{$1 \Rightarrow 2$}
  \begin{proof}
    \pf\ Theorem \ref{thm:topology:compact:limit_point_compact}.
  \end{proof}
  \step{<1>2}{$2 \Rightarrow 3$}
  \begin{proof}
    \step{<2>1}{\assume{$X$ is limit point compact.}}
    \step{<2>2}{\pflet{$(x_n)$ be a sequence in $X$} \prove{$(x_n)$ has a
    convergent subsequence.}}
    \step{<2>3}{\case{$\{x_n : n \in \mathbb{Z}^+ \}$ is finite.}}
    \begin{proof}
      \pf\ In this case, $(x_n)$ has a constant subsequence.
    \end{proof}
    \step{<2>4}{\case{$\{x_n : n \in \mathbb{Z}^+ \}$ is infinite.}}
    \begin{proof}
      \step{<3>1}{\pick\ a limit point $l$ of $\{ x_n : n \in \mathbb{Z}^+
      \}$}
      \step{<3>2}{For every poisitive integer $r$, \pick\ $n_r$ such that
      $n_r
      >
      n_{r-1}$ and $d(x_{n_r}, l) < 1/r$}
      \begin{proof}
        \pf\ There always exists such an $n_r$ since $B(l, 1/r)$ intersects
        $\{ x_n : n \in \mathbb{Z}^+ \}$ in infinitely many points by
        Theorem \ref{thm:topology:T1:limit_point}.
      \end{proof}
      \step{<3>3}{$x_{n_r} \rightarrow l$ as $r \rightarrow \infty$}
    \end{proof}
  \end{proof}
  \step{<1>3}{$3 \Rightarrow 1$}
  \begin{proof}
    \step{<2>1}{\assume{$X$ is sequentially compact.}}
    \step{<2>2}{Every open covering of $X$ has a Lebesgue number.}
    \begin{proof}
      \step{<3>1}{\pflet{$\mathcal{A}$ be an open covering of $X$.}}
      \step{<3>2}{\assume{for a contradiction that, for all $\delta > 0$,
      there
      exists a set $C \subseteq X$ with $\diam C < \delta$ such that
      there
      is no $U \in \mathcal{A}$ such that $C \subseteq U$}}
      \step{<3>3}{For $n \geq 1$, \pick\ $C_n \subseteq X$ with $\diam C_n <
      1/n$
      such that there is no $U \in \mathcal{A}$ such that $C_n \subseteq U$}
      \step{<3>4}{For $n \geq 1$, \pick\ $x_n \in C_n$}
      \step{<3>5}{\pick\ a convergent subsequence $(x_{n_r})$ of $(x_n)$}
      \begin{proof}
        \pf\ By \stepref{<2>1}.
      \end{proof}
      \step{<3>6}{\pflet{$x_{n_r} \rightarrow l$ as $r \rightarrow \infty$}}
      \step{<3>7}{\pick\ $U \in \mathcal{A}$ with $l \in U$}
      \begin{proof}
        \pf\ By \stepref{<3>1}
      \end{proof}
      \step{<3>8}{\pick\ $\epsilon > 0$ such that $B(l, \epsilon) \subseteq
      U$}
      \begin{proof}
        \pf\ By Lemma \ref{lm:topology:metric:open}.
      \end{proof}
      \step{<3>9}{\pick\ $R$ such that $1/n_R < \epsilon / 2$ and $d(x_{n_R},
      l) <
      \epsilon / 2$}
      \begin{proof}
        \pf\ By \stepref{<3>6}
      \end{proof}
      \step{<3>10}{$C_{n_R} \subseteq U$}
      \begin{proof}
        \pf
        \begin{align*}
          C_{n_R} & \subseteq B(x_{n_R}, 1/n_R) & (\text{\stepref{<3>3},
          \stepref{<3>4}}) \\
          & \subseteq B(x_{n_R}, \epsilon / 2) & (\text{\stepref{<3>9}}) \\
          & \subseteq B(l, \epsilon) & (\text{\stepref{<3>9}}) \\
          & \subseteq U & (\text{\stepref{<3>8}})
        \end{align*}
      \end{proof}
      \qedstep
      \begin{proof}
        \pf\ This contradicts \stepref{<3>3}.
      \end{proof}
    \end{proof}
    \step{<2>3}{For all $\epsilon > 0$, there exists a finite covering of $X$
    by
    $\epsilon$-balls.}
    \begin{proof}
      \step{<3>1}{\pflet{$\epsilon > 0$}}
      \step{<3>2}{\assume{for a contradiction there is no finite covering of
      $X$
      by $\epsilon$-balls.}}
      \step{<3>3}{\pick\ a sequence $(x_n)$ in $X$ such that, for all $n$,
      \[ x_n \notin B(x_1, \epsilon) \cup \cdots \cup B(x_{n-1}, \epsilon)
      \enspace . \]}
      \step{<3>4}{For all $m$, $n$ with $m > n$ we have $d(x_m, x_n) \geq
      \epsilon$}
      \step{<3>5}{Any $\epsilon / 2$-ball contains at most one element of
      $(x_n)$.}
      \step{<3>6}{$(x_n)$ has no convergent subsequence.}
      \qedstep
      \begin{proof}
        \pf\ This contradicts \stepref{<2>1}.
      \end{proof}
    \end{proof}
    \step{<2>4}{\pflet{$\mathcal{A}$ be an open covering of $X$}}
    \step{<2>5}{\pick\ a Lebesgue number $\delta$ for $\mathcal{A}$}
    \begin{proof}
      \pf\ By \stepref{<2>2}.
    \end{proof}
    \step{<2>6}{\pick\ a finite covering $\{ B_1, \ldots, B_n \}$ of $X$ by
    $\delta / 3$-balls.}
    \begin{proof}
      \pf\ By \stepref{<2>3}.
    \end{proof}
    \step{<2>7}{For $1 \leq i \leq n$, \pick\ $U_i \in \mathcal{A}$ such that
    $B_i \subseteq U_i$}
    \step{<2>8}{$\{ U_1, \ldots, U_n \}$ covers $X$.}
  \end{proof}
  \qed
\end{proof}

\begin{cor}
  $S_\Omega$ is not metrizable.
\end{cor}

\begin{proof}
  \pf\ It is limit point compact (Corollary
  \ref{cor:topology:limit_point_compact:S_omega}) but not compact (Proposition
  \ref{prop:topology:compact:S_omega}). \qed
\end{proof}

\begin{cor}
  The space $\mathbb{R}^\omega$ is not limit point compact.
\end{cor}
