\chapter{Set Theory}

\section{Primitive Notions}

Let there be \emph{sets}.

Given sets $A$ and $B$, let there be \emph{functions} from $A$ to $B$. We
write $f : A \rightarrow B$ iff $f$ is a function from $A$ to $B$.

Given functions $f : A \rightarrow B$ and $g : B \rightarrow C$, let there be
a function $g \circ f : A \rightarrow C$, the \emph{composite} of $f$ and $g$.

\section{The Axiom of Associativity}

\begin{ax}[Associativity]
  Let $f : A \rightarrow B$, $g : B \rightarrow C$ and $h : C \rightarrow D$.
  Then $h \circ (g \circ f) = (h \circ g) \circ f : A \rightarrow D$.
\end{ax}

From now on we write $h \circ g \circ f$ for the composite of $f$, $g$ and
$h$, and similarly for more than three functions.

\section{Injective Functions}

\begin{df}[Injective]
  A function $f : A \rightarrow B$ is \emph{injective}, $f : A
  \rightarrowtail B$, iff, for every set $X$ and functions $g, h : X
  \rightarrow
  A$, if $f \circ g = f \circ h$ then $g = h$.
\end{df}

\begin{prop}
\label{lm:sets:injective:comp}
 Let $f : A \rightarrow B$ and $g : B \rightarrow C$. If $f$ and $g$ are
injective then $g \circ f$ is injective.
\end{prop}

\begin{proof}
 \pf
 \step{<1>1}{\assume{$f$ is injective.}}
 \step{<1>2}{\assume{$g$ is injective.}}
 \step{<1>3}{\pflet{$X$ be a set and $x, y : X \rightarrow A$.}}
 \step{<1>4}{\assume{$g \circ f \circ x = g \circ f \circ y$.} \prove{$x = y$}}
 \step{<1>5}{$f \circ x = f \circ y$}
 \begin{proof}
   \pf\ \stepref{<1>2}, \stepref{<1>4}
 \end{proof}
 \step{<1>6}{$x = y$}
 \begin{proof}
   \pf\ \stepref{<1>1}, \stepref{<1>5}
 \end{proof}
 \qed
\end{proof}

\begin{lm}
  \label{lm:sets:injective:compr}
 Let $f : A \rightarrow B$ and $g : B \rightarrow C$. If $g \circ f$ is
injective then $f$ is injective.
\end{lm}

\begin{proof}
 \pf
 \step{<1>1}{\assume{$g \circ f$ is injective.}}
 \step{<1>2}{\pflet{$X$ be any set and $x, y : X \rightarrow A$}}
 \step{<1>3}{\assume{$f \circ x = f \circ y$}}
 \step{<1>4}{$g \circ f \circ x = g \circ f \circ y$}
 \begin{proof}
   \pf\ \stepref{<1>3}
 \end{proof}
 \step{<1>5}{$x = y$}
 \begin{proof}
   \pf\ \stepref{<1>1}, \stepref{<1>4}.
 \end{proof}
 \qed
\end{proof}

\section{Surjective Functions}

\begin{df}[Surjective]
  Let $f : A \rightarrow B$. Then $f$ is \emph{surjective}, $f : A
\twoheadrightarrow B$, iff, for any set
  $X$ and functions $g, h : B \rightarrow X$, if $g \circ f = h \circ f$ then
  $g
  = h$.
\end{df}

\begin{lm}
 Let $f : A \rightarrow B$ and $g : B \rightarrow C$. If $f$ and $g$ are
surjective then $g \circ f$ is surjective.
\end{lm}

\begin{proof}
 \pf\ Dual to Lemma \ref{lm:sets:injective:comp}. \qed
\end{proof}

\begin{lm}
 Let $f : A \rightarrow B$ and $g : B \rightarrow C$. If $g \circ f$ is
surjective then $g$ is surjective.
\end{lm}

\begin{proof}
 \pf\ Dual to Lemma \ref{lm:sets:injective:compr}. \qed
\end{proof}

\section{Retractions and Sections}

\begin{df}[Retraction, Section]
  Let $r : A \rightarrow B$ and $s : B \rightarrow A$. Then $r$ is a
  \emph{retraction} of $s$, and $s$ is a \emph{section} of $r$, iff $r \circ
  s = \id{B}$.
\end{df}

\begin{prop}
\label{prop:sets:retraction:comp}
If $r_1 : A \rightarrow B$ is a retraction of $s_1 : B \rightarrow A$ and $r_2 : B \rightarrow C$ is a retraction of $s_2 : C \rightarrow B$ then $r_2 \circ r_1$ is a retraction of $s_1 \circ s_2$.
\end{prop}

\begin{proof}
\pf
\begin{align*}
r_2 \circ r_1 \circ s_1 \circ s_2 & = r_2 \circ \id{B} \circ s_2 & (\text{$r_1$ is a retraction of $s_1$}) \\
& = r_2 \circ s_2 & (\text{Unit Laws}) \\
& = \id{C} & (\text{$r_2$ is a retraction of $s_2$})
\end{align*}
\qed
\end{proof}

\begin{prop}
\label{prop:sets:section:injective}
Every section is injective.
\end{prop}

\begin{proof}
  \pf
  \step{<1>1}{\pflet{$s : A \rightarrow B$ be a section of $r : B \rightarrow A$}}
  \step{<1>2}{\pflet{$x, y : X \rightarrow A$ satisfy $s \circ x = s \circ y$}}
  \step{<1>3}{$x = y$}
  \begin{proof}
    \pf
    \begin{align*}
      x & = \id{A} \circ x & (\text{Left Unit Law}) \\
      & = r \circ s \circ x & (\text{\stepref{<1>1}}) \\
      & = r \circ s \circ y & (\text{\stepref{<1>2}}) \\
      & = \id{A} \circ y & (\text{\stepref{<1>1}}) \\
      & = y & (\text{Left Unit Law})
    \end{align*}
  \end{proof}
  \qed
\end{proof}

\begin{prop}
  \label{prop:sets:retraction:surjective}
  Every retraction is surjective.
\end{prop}

\begin{proof}
  \pf\ Dual. \qed
\end{proof}

\section{Identity Functions}

\begin{ax}[Identity Function]
  For any set $A$, there exists a function $\id{A} : A \rightarrow A$, the
  \emph{identity} function on $A$, such that:
  \begin{description}
    \item[Left Unit Law] for every set $B$ and function $f : B \rightarrow A$
    we have $\id{A} \circ f = f : B \rightarrow A$;
    \item[Right Unit Law] for every set $B$ and function $f : A \rightarrow
    B$ we have $f \circ \id{A} = f : A \rightarrow B$.
  \end{description}
\end{ax}

\begin{prop}
  The identity function on a set is unique.
\end{prop}

\begin{proof}
  \pf If $i, j : A \rightarrow A$ are both identity functions, then
  \begin{align*}
    i & = i \circ j & (\text{Right Unit Law for } j) \\
    & = j & (\text{Left Unit Law for } i) \\
    & : A \rightarrow A & \qed
  \end{align*}
\end{proof}

\begin{prop}
\label{prop:sets:retraction:id}
Every identity function is a retraction of itself.
\end{prop}

\begin{proof}
\pf\ Immediate from the Unit Laws. \qed
\end{proof}

\begin{prop}
  \label{prop:sets:identity:injective}
 Every identity function is injective.
\end{prop}

\begin{proof}
  \pf\ From Proposition \ref{prop:sets:section:injective} and
  \ref{prop:sets:retraction:id}. \qed
\end{proof}

\begin{prop}
Every identity function is surjective.
\end{prop}

\begin{proof}
  \pf\ From Proposition \ref{prop:sets:retraction:surjective} and
  \ref{prop:sets:retraction:id}. \qed
\end{proof}

\begin{prop}
\label{prop:sets:retraction:ret_sect_unique}
If $r : B \rightarrow A$ is a retraction of $f : A \rightarrow B$ and $s$ is a section of $f$  then $ r = s$.
\end{prop}

\begin{proof}
\pf
\begin{align*}
r & = r \circ \id{B} & (\text{Right Unit Law}) \\
& = r \circ f \circ s & (\text{$s$ is a section of $f$}) \\
& = \id{A} \circ s & (\text{$r$ is a retraction of $f$}) \\
& = s & (\text{Left Unit Law})
\end{align*}
\end{proof}

\subsection{Isomorphisms}

\begin{df}[Isomorphism]
Let $A$ and $B$ be sets. A function $i : A \rightarrow B$  is an
\emph{isomorphism} between $A$ and $B$, $i : A \cong B$, iff there exists a
function $\inv{i} : B \rightarrow A$, the \emph{inverse} to $i$, that is a
section and a retraction of $i$.
\end{df}

\begin{prop}
The inverse of an isomorphism is unique.
\end{prop}

\begin{proof}
\pf\ Immediate from Proposition \ref{prop:sets:retraction:ret_sect_unique}. \qed
\end{proof}

\begin{prop}
Every isomorphism is injective.
\end{prop}

\begin{proof}
\pf\ Immediate from Proposition \ref{prop:sets:section:injective}. \qed
\end{proof}

\begin{prop}
Every isomorphism is surjective.
\end{prop}

\begin{proof}
\pf\ Immediate from Proposition \ref{prop:sets:retraction:surjective}. \qed
\end{proof}

\begin{prop}
\label{prop:sets:isomorphism:identity}
Every identity function is an isomorphism and is its own inverse.
\end{prop}

\begin{proof}
\pf\ Immediate from Proposition \ref{prop:sets:retraction:id}. \qed
\end{proof}

\begin{prop}
\label{prop:sets:isomorphism:inverse}
If $i : A \cong B$ is an isomorphism then $\inv{i} : B \cong A$ is an
isomorphism and $\inv{(\inv{i})} = i$.
\end{prop}

\begin{proof}
\pf\ Immediate from the definition of isomorphism. \qed
\end{proof}

\begin{prop}
  \label{prop:sets:isomorphism:comp}
If $i : A \cong B$ and $j : B \cong C$ then $j \circ i : A \cong C$ and
$\inv{(j \circ i)} = \inv{i} \circ \inv{j}$.
\end{prop}

\begin{proof}
\pf\ Immediate from Proposition \ref{prop:sets:retraction:comp}. \qed
\end{proof}


\subsection{Parts of a Set}

\begin{df}[Part]
  A \emph{part} $S$ of a set $A$ consists of:
  \begin{itemize}
   \item a set $\dom S$;
   \item an injective function $i : S \hookrightarrow A$
  \end{itemize}
\end{df}

\begin{df}
  Two parts $i : S \hookrightarrow A$, $j : T \hookrightarrow A$ are
  \emph{equivalent}, $i \equiv_A j$, iff there exists an isomorphism $\phi :
S \cong T$ such that $i = j \circ \phi$.
\end{df}

\begin{prop}
 Any part of a set is equivalent to itself.
\end{prop}

\begin{proof}
 \pf
 For any part $i : X \hookrightarrow A$ of $A$ we have $i = i \circ \id{X}$ by the Right Unit Law. \qed
\end{proof}

\begin{prop}
 If $i \equiv_A j$ then $j \equiv_A i$.
\end{prop}

\begin{proof}
 \pf
 \step{<1>1}{\pflet{$i : S \hookrightarrow A$ and $j : T \hookrightarrow A$}}
 \step{<1>2}{\assume{$i \equiv_A j$}}
 \step{<1>3}{\pick\ an isomorphism $\phi : S \cong T$ such that $i = j \circ
   \phi$}
 \begin{proof}
   \pf\ From \stepref{<1>2}
 \end{proof}
 \step{<1>4}{$\inv{\phi} : T \cong S$}
 \begin{proof}
   \pf\ By Proposition \ref{prop:sets:isomorphism:inverse}.
 \end{proof}
 \step{<1>5}{$j = i \circ \inv{\phi}$}
 \begin{proof}
   \pf
   \begin{align*}
     j & = j \circ \id{T} & (\text{Right Unit Law})\\
     & = j \circ \phi \circ \inv{\phi} & (\text{\stepref{<1>3}})\\
     & = i \circ \inv{\phi} & (\text{\stepref{<1>3}})
   \end{align*}
 \end{proof}
 \qed
\end{proof}

\begin{prop}
  If $i \equiv_A j$ and $j \equiv_A k$ then $i \equiv_A k$.
\end{prop}

\begin{proof}
  \pf
  \step{<1>1}{\pflet{$i : R \hookrightarrow A$, $j : S \hookrightarrow A$ and $k : T \rightarrow A$}}
  \step{<1>2}{\pick\ isomorphisms $\phi : R \cong S$ and $\psi : S \cong T$ such that $i = j \circ \phi$ and $j = k \circ \psi$}
  \step{<1>3}{$\psi \circ \phi : R \cong T$}
  \begin{proof}
    \pf\ By Proposition \ref{prop:sets:isomorphism:comp}.
  \end{proof}
  \step{<1>4}{$i = k \circ \psi \circ \phi$}
\qed
\end{proof}

\begin{df}
  Given a set $A$, we write $A$ for the part $\id{A} : A \hookrightarrow A$.
\end{df}

(This is a part by Proposition \ref{prop:sets:identity:injective}.)

\begin{df}[Inclusion]
 Let $i : U \hookrightarrow A$ and $j : V \hookrightarrow A$ be parts of $A$.
 Then $i$ is \emph{included} in $j$, $i \subseteq_A j$, iff there exists a
 function $\phi : U \rightarrow V$ such that $i = j \circ \phi$.
\end{df}

\begin{prop}
  If $i \equiv_A i'$ and $j \equiv_A j'$ and $i \subseteq_A j$ then $i' \subseteq_A j'$.
\end{prop}

\begin{proof}
  \pf
  \step{<1>1}{\pflet{$i : S \hookrightarrow A$, $i' : S' \hookrightarrow A$, $j : T \hookrightarrow A$, $j' : T' \hookrightarrow A$}}
  \step{<1>2}{\pick\ $\phi : S \cong S'$, $\psi : T \cong T'$ and $\chi : S \rightarrow T$ such that $i = i' \circ \phi$, $j = j' \circ \psi$ and $i = j \circ \chi$}
  \step{<1>3}{$\psi \circ \chi \circ \inv{\phi} : S' \rightarrow T'$}
  \step{<1>4}{$i' = j' \circ \psi \circ \chi \circ \inv{\phi}$}
  \qed
\end{proof}

\begin{prop}
  For any part $i$ of $A$ we have $i \subseteq_A i$.
\end{prop}

\begin{proof}
  \pf
  \step{<1>1}{\pflet{$i : S \hookrightarrow A$}}
  \step{<1>2}{$\id{S} : S \rightarrow S$}
  \step{<1>3}{$i = i \circ \id{S}$}
  \qed
\end{proof}

\begin{prop}
  If $i \subseteq_A j$ and $j \subseteq_A k$ then $i \subseteq_A k$.
\end{prop}

\begin{proof}
  \pf
  \step{<1>1}{\pflet{$i : R \hookrightarrow A$, $j : S \hookrightarrow A$ and $k : T \hookrightarrow A$}}
  \step{<1>2}{\pick\ $\phi : R \rightarrow S$ and $\psi : S \rightarrow T$ such that $i = j \circ \phi$ and $j = k \circ \psi$}
  \step{<1>3}{$\psi \circ \phi : R \rightarrow T$}
  \step{<1>4}{$i = k \circ \psi \circ \phi$}
  \qed
\end{proof}

\begin{prop}
  If $i \subseteq_A j$ and $j \subseteq_A i$ then $i \equiv_A j$.
\end{prop}

\begin{proof}
  \pf
  \step{<1>1}{\pflet{$i : R \hookrightarrow A$, $j : S \hookrightarrow A$}}
  \step{<1>2}{\pick\ $\phi : R \rightarrow S$ and $\inv{\phi} : S \rightarrow R$ such that $i = j \circ \phi$ and $j = i \circ \inv{\phi}
  $}
  \step{<1>3}{$\phi \circ \inv{\phi} = \id{S}$}
  \begin{proof}
    \step{<2>1}{$j \circ \phi \circ \inv{\phi} = j$}
    \qedstep
    \begin{proof}
      \pf\ The result follows because $j$ is injective.
    \end{proof}
  \end{proof}
  \step{<1>4}{$\inv{\phi} \circ \phi = \id{T}$}
  \begin{proof}
    \pf\ Similar.
  \end{proof}
  \qed
\end{proof}

\begin{prop}
  For any part $i$ of $A$ we have $i \subseteq_A A$.
\end{prop}

\begin{proof}
  \pf\ For any part $i$ of $A$, we have $i = \id{A} \circ i$ by the Left Unit Law. \qed
\end{proof}

\subsection{The Empty Set}

\begin{ax}[Empty Set]
  There exists a set $\emptyset$, the \emph{empty set}, such that, for every
  set $X$, there exists a unique function $\magic_X : \emptyset \rightarrow
  X$.
\end{ax}

\begin{prop}[Uniqueness of Empty Set]
  \label{prop:sets:empty:unique}
  Let $E$ be any set. Then $E$ is empty if and only if there exists an isomorphism $E \cong \emptyset$, in which case the isomorphism is unique.
\end{prop}

\begin{proof}
  \pf
  \step{<1>1}{If $E$ is empty then $E \cong \emptyset$}
  \begin{proof}
    \step{<2>1}{\assume{$E$ is empty}}
    \step{<2>2}{\pflet{$\phi$ be the unique function $E \rightarrow \emptyset$}}
    \step{<2>3}{$\magic_E \circ \phi = \id{E}$}
    \begin{proof}
      \pf\ There is only one function $E \rightarrow E$.
    \end{proof}
    \step{<2>4}{$\phi \circ \magic_E = \id{\emptyset}$}
    \begin{proof}
      \pf\ There is only one function $\emptyset \rightarrow \emptyset$.
    \end{proof}
  \end{proof}
  \step{<1>2}{If $E \cong \emptyset$ then $E$ is empty}
  \begin{proof}
    \step{<2>1}{\pflet{$\phi : E \cong \emptyset$}}
    \step{<2>2}{\pflet{$X$ be a set} \prove{There is a unique function $E \rightarrow X$}}
    \step{<2>3}{$\magic_X \circ \phi : E \rightarrow X$}
    \step{<2>4}{If $f : E \rightarrow X$ then $f = \magic_X \circ \phi$}
    \begin{proof}
      \step{<3>1}{\pflet{$f : E \rightarrow X$}}
      \step{<3>2}{$f \circ \inv{\phi} : \emptyset \rightarrow X$}
      \step{<3>3}{$f \circ \inv{\phi} = \magic_X$}
      \begin{proof}
        \pf\ Uniqueness of $\magic_X$.
      \end{proof}
      \qedstep
    \end{proof}
  \end{proof}
  \step{<1>3}{There is at most one isomorphism $E \cong \emptyset$}
  \begin{proof}
    \pf\ This holds because there is at most one function $E \rightarrow \emptyset$.
  \end{proof}
  \qed
\end{proof}

\begin{prop}
  \[ \magic_\emptyset = \id{\emptyset} \]
\end{prop}

\begin{proof}
  \pf\ By the uniqueness of $\magic_\emptyset$. \qed
\end{proof}

\subsection{The Terminal Set}

\begin{ax}[Terminal Set]
  There exists a set $1$, the \emph{terminal set}, such that, for every set
  $X$, there exists a unique function $!_X : X \rightarrow 1$.
\end{ax}

\begin{prop}[Uniqueness of Terminal Set]
  Let $T$ be any set. Then $T$ is terminal if and only if there exists an isomorphism $T \cong 1$, in which case the isomorphism is unique.
\end{prop}

\begin{proof}
  \pf\ Dual to Proposition \ref{prop:sets:empty:unique}.
\end{proof}

\begin{prop}
  \[ !_1 = \id{1} \]
\end{prop}

\begin{proof}
  \pf\ From the uniqueness of $!_1$. \qed
\end{proof}

\subsection{Elements}

\begin{df}[Element]
  An \emph{element} of a set $A$ is a function $1 \rightarrow A$. We write $a
  \in A$ for $a : 1 \rightarrow A$. We write $f(a)$ for $f \circ a$ when $f :
  A \rightarrow B$ and $a \in A$.
\end{df}

\paragraph{The Axiom of Extensionality}

\begin{ax}[Extensionality]
 Let $A$ and $B$ be sets and $f, g : A \rightarrow B$ be functions. If, for
all $a \in A$, we have $f(a) = g(a) \in B$, then $f = g$.
\end{ax}

\begin{prop}
  \label{prop:sets:injective:elements}
  Let $f : A \rightarrow B$. Then $f$ is injective if and only if, for all $x, y \in A$, if $f(x) = f(y) \in B$ then $x = y \in A$.
\end{prop}

\begin{proof}
  \pf
  \step{<1>1}{If $f$ is injective and $f(x) = f(y) \in B$ then $x = y \in A$}
  \begin{proof}
    \pf\ Immediate from the definition of injective.
  \end{proof}
  \step{<1>2}{If, for all $x, y \in A$, if $f(x) = f(y) \in B$ then $x = y \in A$}
  \begin{proof}
    \step{<2>1}{\assume{For all $x, y \in A$, if $f(x) = f(y)$, then $x = y$}}
    \step{<2>2}{\pflet{$X$ be any set and $g,h : X \rightarrow A$ with $f \circ g = f \circ h$} \prove{$g = h$}}
    \step{<2>3}{\pflet{$x \in X$} \prove{$g(x) = h(x)$}}
    \step{<2>4}{$f(g(x)) = f(h(x))$}
    \begin{proof}
      \pf\ From \stepref{<2>2}.
    \end{proof}
    \step{<2>5}{$g(x) = h(x)$}
    \begin{proof}
      \pf\ By \stepref{<2>1}
    \end{proof}
  \end{proof}
  \qed
\end{proof}

\begin{prop}
  Any element $e \in X$ is a section of the unique function $!_X : X \rightarrow 1$.
\end{prop}

\begin{proof}
  \pf
  $!_X \circ e = \id{1}$ because there is only one function $1 \rightarrow 1$.
  \qed
\end{proof}

\begin{ax}[Non-degeneracy]
 The empty set $\emptyset$ has no elements.
\end{ax}

\begin{prop}
  For any set $X$, the function $\magic_X : \emptyset \rightarrow X$ is injective.
\end{prop}

\begin{proof}
  \pf\ From Proposition \ref{prop:sets:injective:elements}. \qed
\end{proof}

\begin{df}[Empty Part]
  For any set $X$, the \emph{empty part} of $X$ is $\emptyset = \magic_X : \emptyset \hookrightarrow X$.
\end{df}

  \begin{df}[Constant Function]
  A function $f : A \rightarrow B$ is \emph{constant} iff there exists $b \in
  B$ such that $f = b \circ !_A$.
\end{df}

\begin{df}[Membership]
 Let $i : U \hookrightarrow A$ be a part of $A$ and $a \in A$. Then $a$ is a
 \emph{member} of $i$, $a \in_A i$, iff there exists $\overline{a} \in U$ such that $i(\overline{a}) = a$.
\end{df}

\begin{prop}
 Let $A$ be a set. Let $i$, $j$ be parts of $A$ and $a \in A$. If $a \in_A i$
 and $i \subseteq_A j$ then $a \in_A j$.
\end{prop}

\begin{proof}
 \pf
 \step{<1>1}{\pick\ $\overline{a} \in \dom i$ such that $a =
i(\overline{a})$.}
 \step{<1>2}{\pick\ $\phi : \dom i \rightarrow \dom j$ such that $i = j \circ
   \phi$}
 \step{<1>3}{$a = j(\phi(\overline{a}))$}
 \qed
\end{proof}

\subsection{Products}

\begin{ax}[Products]
  For any sets $A$ and $B$, there exists a set $A \times B$, the
  \emph{product} of $A$ and $B$, and functions $\pi_1 : A \times B
  \rightarrow A$, $\pi_2 : A \times B \rightarrow B$, the \emph{projections},
  such that, for any set $C$ and functions $f : C \rightarrow A$, $g : C
  \rightarrow B$, there exists a unique function $\langle f, g \rangle : C
  \rightarrow A \times B$ such that
  \[ \pi_1 \circ \langle f, g \rangle = f; \qquad \pi_2 \circ \langle f,g
  \rangle
  = g \enspace . \]
\end{ax}

\begin{df}
  Given functions $f : A \rightarrow B$ and $g : C \rightarrow D$, define $f
  \times g : A \times C \rightarrow B \times D$ by
  \[ f \times g = \langle f \circ \pi_1, g \circ \pi_2 \rangle \]
\end{df}

\subsection{Coproducts}

\begin{ax}[Coproducts]
  For any sets $A$ and $B$, there exists a set $A \uplus B$, the
  \emph{coproduct} or \emph{sum} of $A$ and $B$, and functions $\kappa_1 : A
\rightarrow A
  \uplus B$, $\kappa_2 : B \rightarrow A \uplus B$, the \emph{injections},
  such that, for any set $C$ and functions $f : A \rightarrow C$, $g : B
  \rightarrow C$, there exists a unique function $[f, g] : A \uplus B
  \rightarrow
  C$ such that
  \[ [f,g] \circ \kappa_1 = f; \qquad [f,g] \circ \kappa_2 = g \enspace . \]
\end{ax}

  \begin{df}[Complement]
 Let $i : I \hookrightarrow J$ and $i' : I' \hookrightarrow J$ be parts of
 $J$. Then $i'$ is the \emph{complement} of $i$ iff $J$ is the sum of $I$ and
 $I'$ with injections $i$ and $i'$.
\end{df}


\subsection{Equalizers}

\begin{ax}[Equalizers]
  For any sets $A$ and $B$ and functions $f, g : A \rightarrow B$, there
  exists a set $E$ and function $e : E \rightarrow A$, the \emph{equalizer}
  of
  $A$ and $B$, such that:
  \begin{itemize}
    \item $f \circ e = g \circ e : E \rightarrow B$;
    \item For any set $C$ and function $h : C \rightarrow A$ such that $f
    \circ h
    = g \circ h$, there exists a unique function $\overline{h} : C
    \rightarrow
    E$
    such that $h = e \circ \overline{h}$.
  \end{itemize}
\end{ax}

\begin{prop}
  \label{prop:set_theory:equalizer:injective}
 All equalizers are injective.
\end{prop}

\begin{proof}
 \pf
 \step{<1>1}{\pflet{$e : E \rightarrow A$ be the equalizer of $f, g : A
     \rightarrow B$}}
 \step{<1>2}{\pflet{$x, y : X \rightarrow E$ with $e \circ x = e \circ y$}}
 \step{<1>3}{$f \circ e \circ x = g \circ e \circ x$}
 \begin{proof}
   \pf\ $f \circ e = g \circ e$ by \stepref{<1>1}1.
 \end{proof}
 \step{<1>4}{$x = y$}
 \begin{proof}
   \pf\ $x$ and $y$ are both the unique $z : X \rightarrow E$ such that $e
   \circ z = e \circ x$.
 \end{proof}
 \qed
\end{proof}

\subsection{Coequalizers}

\begin{ax}[Coequalizers]
  For any sets $A$ and $B$ and functions $f, g : A \rightarrow B$, there
  exists a set $C$ and function $c : B \rightarrow C$, the \emph{coequalizer}
  of $f$ and $g$, such that:
  \begin{itemize}
    \item $c \circ f = c \circ g : A \rightarrow C$
    \item For any set $X$ and function $h : B \rightarrow X$ such that $h
    \circ f
    = h \circ g$, there exists a unique function $\overline{h} : C
    \rightarrow
    X$
    such that $\overline{h} \circ c = h$.
  \end{itemize}
\end{ax}

\subsection{Pullbacks}

  \begin{df}[Pullback]
  The diagram below is a \emph{pullback diagram} iff:
  \begin{itemize}
    \item $f \circ p = g \circ q$
    \item for every set $X$ and
    functions $x : X \rightarrow B$ and $y : X \rightarrow C$ such that $f
    \circ x
    = g \circ y$, there exists a unique function $\langle x, y \rangle : X
    \rightarrow A$ such that $p \circ \langle x,y \rangle = x$ and $q \circ
    \langle
    x, y \rangle = y$.
  \end{itemize}
\[      \xymatrix{ A \ar[r]^p \ar[d]_q & B \ar[d]^f \\
      C \ar[r]_g & D } \]
\end{df}

   \begin{prop}
     \label{prop:sets:axioms:pullback}
 Let $f : A \rightarrow C$ and $g : B \rightarrow C$. Then $f$ and $g$ have a
 pullback.
\end{prop}

   \[ \xymatrix{
  E \ar[dr]^e \\
  & A \times B \ar[r]^{\pi_1} \ar[d]_{\pi_2} & A \ar[d]^f \\
  & B \ar[r]_g & C}
 \]

\begin{proof}
 \pf
 \step{<1>1}{Construct the product $\pi_1 : A \times B \rightarrow A$, $\pi_2
:
   A \times B \rightarrow B$.}
 \step{<1>2}{Construct the equalizer $e : E \rightarrow A$ of $f \circ \pi_1$
   and $g \circ \pi_2$.
   \prove{$\pi_1 \circ e$ and $\pi_2 \circ e$ form a pullback of $f$ and $g$}}
 \step{<1>3}{$f \circ \pi_1 \circ e = g \circ \pi_2 \circ e$}
 \step{<1>4}{\pflet{$X$ be a set and $x : X \rightarrow A$, $y : X
\rightarrow
     B$ satisfy $f \circ x = g \circ y$}}
 \step{<1>5}{$f \circ \pi_1 \circ \langle x, y \rangle = g \circ \pi_2 \circ
   \langle x, y \rangle$}
 \step{<1>6}{\pflet{$m : X \rightarrow E$ be the function such that $e \circ
     m = \langle x, y \rangle$}}
 \step{<1>7}{$\pi_1 \circ e \circ m = x$ and $\pi_2
   \circ e \circ m = y$}
 \step{<1>8}{$m$ is unique.}
 \begin{proof}
   \pf
   \step{<2>1}{\pflet{$n : X \rightarrow E$ be such that $\pi_1 \circ e \circ n
       = x$ and $\pi_2 \circ e \circ n = y$}}
   \step{<2>2}{$e \circ n = \langle x, y \rangle$}
   \step{<2>3}{$n = m$}
   \begin{proof}
     \pf\ By \stepref{<1>6}
   \end{proof}
 \end{proof}
 \qed
\end{proof}

\begin{prop}
 Pullbbacks are unique up to isomorphism.

 That is, let $P$ be a pullback of $f : A \rightarrow C$ and $g : B
\rightarrow C$ with projections $p : P \rightarrow A$ and $q : P \rightarrow
B$. Let $Q$ be a set and $p' : Q \rightarrow A$, $q' : Q \rightarrow B$. Then
$Q$ is a pullback of $f$ and $g$ with projections $p'$ and $q'$ if and only if
there exists a bijection $\phi : Q \cong P$ such that $p \circ \phi = p'$ and
$q \circ \phi = q'$, in which case $\phi$ is unique.

 \[ \xymatrix{
   Q \ar@/^/[rrd]^{p'} \ar@/_/[ddr]_{q'} \ar[dr]^{\phi} \\
  & P \ar[r]_p \ar[d]^q & A \ar[d]^{f} \\
  & B \ar[r]_g & C
}
 \]

\end{prop}

\begin{proof}
 \pf
 \step{<1>1}{If $Q$ is a pullback then there exists a bijection $\phi : Q
\cong
   P$ such that $p \circ \phi = p'$ and $q \circ \phi = q'$}
 \begin{proof}
   \step{<2>1}{\assume{$Q$ is a pullback with projections $p'$ and $q'$}}
   \step{<2>2}{\pflet{$\phi : Q \rightarrow P$ be the unique function such that
       $p \circ \phi = p'$ and $q \circ \phi = q'$}}
   \begin{proof}
     \pf\ Such a $\phi$ exists because $f \circ p' = g \circ q'$.
   \end{proof}
   \step{<2>3}{\pflet{$\inv{\phi} : P \rightarrow Q$ be the unique function such
       that $p' \circ \inv{\phi} = p$ and $q' \circ \inv{\phi} = q$}}
   \begin{proof}
     \pf\ Such a function exists because $f \circ p = g \circ q$.
   \end{proof}
   \step{<2>4}{$\phi \circ \inv{\phi} = \id{P}$}
   \begin{proof}
     \pf\ Each is the unique function $x$ such that $p \circ x = p$ and $q
     \circ x = q$.
   \end{proof}
   \step{<2>5}{$\inv{\phi} \circ \phi = \id{Q}$}
   \begin{proof}
     \pf\ Similar.
   \end{proof}
 \end{proof}
 \step{<1>2}{If $\phi : Q \cong P$ is a bijection then $Q$ is a pullback with
   projections $p \circ \phi$ and $q \circ \phi$}
 \begin{proof}
   \step{<2>1}{$f \circ p \circ \phi = g \circ q \circ \phi$}
   \begin{proof}
     \pf\ This holds because $f \circ p = g \circ q$
   \end{proof}
   \step{<2>2}{For any set $X$ and functions $x : X \rightarrow A$, $y : X
     \rightarrow B$ such that $f \circ x = g \circ y$, there exists a unique
     function $m : X \rightarrow Q$ such that $p \circ \phi \circ m = x$ and
$q \circ
     \phi \circ m = y$}
   \begin{proof}
     \pf\
     \begin{align*}
       & p \circ \phi \circ m = x \text{ and } q \circ \phi \circ m = y \\
      \Leftrightarrow & \phi \circ m = \langle x, y \rangle \\
      \Leftrightarrow & m = \inv{\phi} \circ \langle x, y \rangle
     \end{align*}
   \end{proof}
 \end{proof}
 \step{<1>3}{If $\phi, \phi' : P \cong Q$ are bijections such that $p \circ \phi
   = p \circ \phi'$ and $q \circ \phi = q \circ \phi'$}
 \begin{proof}
   \pf\ This follows from the definition of pullback.
 \end{proof}
 \qed
\end{proof}

\begin{prop}
  \label{prop:sets:pullback:injective}
 The pullback of an injective function is injective.

 That is, if the diagram below is a pullback diagram and $f$ is injective
then $q$ is injective.
\[      \xymatrix{ A \ar[r]^p \ar@{>->}[d]_q & B \ar@{>->}[d]^f \\
     C \ar[r]_g & D } \]
\end{prop}

\begin{proof}
 \pf
 \step{<1>1}{\pflet{$X$ be a set and $x, y : X \rightarrow A$ with $q \circ x =
     q \circ y$}}
 \step{<1>2}{$f \circ p \circ x = g \circ q \circ x$}
 \step{<1>3}{\pflet\ $z : X \rightarrow A$ be the function such that $p \circ z
   =      p \circ x$ and $q \circ z = q \circ x$}
 \step{<1>4}{$z = x$}
 \step{<1>5}{$z = y$}
 \begin{proof}
   \step{<2>1}{$q \circ x = q \circ y$}
   \begin{proof}
     \pf\ By \stepref{<1>1}.
   \end{proof}
   \step{<2>2}{$f \circ p \circ x = f \circ p \circ y$}
   \begin{proof}
     \pf
     \begin{align*}
       f \circ p \circ x & = g \circ q \circ x & (\text{\stepref{<1>2}}) \\
       & = g \circ q \circ y & (\text{\stepref{<1>1}}) \\
       & = f \circ p \circ y & (\text{the diagram is a pullback})
     \end{align*}
   \end{proof}
   \step{<2>3}{$p \circ x = p \circ y$}
   \begin{proof}
     \pf\ $f$ is injective.
   \end{proof}
 \end{proof}
 \qed
\end{proof}

\subsection{Function Sets}

\begin{ax}[Function Sets]
  For any sets $A$ and $B$, there exists a set $A^B$ and a function $\epsilon
  : A^B \times B \rightarrow A$, the \emph{evaluation} function, such that,
  for any set $C$ and function $f : C \times B \rightarrow A$, there exists a
  unique function $\lambda f : C \rightarrow A^B$ such that
  \[ \epsilon \circ (\lambda f \times \id{B}) = f \enspace . \]
\end{ax}

\subsection{The Subset Classifier}

\begin{df}
  The set $2$ is $1 + 1$. We write $\top$ (\emph{truth}) for $\kappa_1 : 1
  \rightarrow 2$, and $\bot$ (\emph{falsehood}) for $\kappa_2 : 1 \rightarrow
  2$.
\end{df}

\begin{ax}[Subset Classifier]
  For every injective function $m : A \rightarrowtail B$, there exists a
  unique function $\chi_m : B \rightarrow 2$, the \emph{characteristic
    function} of $m$, such that the following diagram is a pullback diagram:

  \[
  \xymatrix{
    A \ar[d]_m \ar[r]^{!} & 1 \ar[d]^{\top} \\
    B \ar[r]_{\chi_m} & 2
  }
  \]
\end{ax}

\begin{prop}
 Every function $\phi : A \rightarrow 2$ is the characteristic function of a
 part of $A$.
\end{prop}

\begin{proof}
 \pf
 \step{<1>1}{Construct a pullback
   \[ \xymatrix{
     I \ar[r] \ar[d]^{q} & 1 \ar[d]^{\top} \\
     A \ar[r]_{\phi} & 2 }
   \]}
 \begin{proof}
   \pf\ By Proposition \ref{prop:sets:axioms:pullback}.
 \end{proof}
 \step{<1>2}{$q$ is injective}
 \begin{proof}
   \pf\ By Proposition \ref{prop:sets:pullback:injective}.
 \end{proof}
 \qed
\end{proof}

\begin{ax}[Boolean]
  For any $p \in 2$ we have $p = \top$ or $p = \bot$.
\end{ax}

 \begin{prop}
Let $i : U \hookrightarrow A$ and $j : V \hookrightarrow A$ be parts of $A$.
Then the following are equivalent:
\begin{enumerate}
\item $i \subseteq_A j$ and $j \subseteq_A i$
\item There exist $h : U \rightarrow V$ and $k : V \rightarrow U$ such that $i
= j \circ h$, $j = i \circ k$, $k \circ h = \id{U}$ and $h \circ k = \id{V}$.
\item The characteristic function of $i$ is the characteristic function of $j$.
\end{enumerate}
\end{prop}

\begin{proof}
\pf
\step{<1>1}{$1 \Rightarrow 2$}
\begin{proof}
  \step{<2>1}{\assume{$i \subseteq_A j$ and $j \subseteq_A i$}}
  \step{<2>2}{\pflet{$h : U \rightarrow V$ be such that $i = j \circ h$}}
  \step{<2>3}{\pflet{$k : V \rightarrow U$ be such that $j = i \circ k$}}
  \step{<2>4}{$k \circ h = \id{U}$}
  \begin{proof}
    \step{<3>1}{$i \circ k \circ h = i$}
    \begin{proof}
      \pf\ From \stepref{<2>2} and \stepref{<2>3}.
    \end{proof}
    \qedstep
    \begin{proof}
      \pf\ Since $i$ is injective.
    \end{proof}
  \end{proof}
  \step{<2>5}{$h \circ k = \id{V}$}
  \begin{proof}
    \pf\ Similar.
  \end{proof}
\end{proof}
\step{<1>2}{$2 \Rightarrow 1$}
\begin{proof}
  \pf\ Trivial.
\end{proof}
\step{<1>3}{$2 \Rightarrow 3$}
\begin{proof}
  \step{<2>1}{\assume{2}}
  \step{<2>2}{\pflet{$\phi : A \rightarrow 2$ be the characteristic function
of
      $i$} \prove{$\phi$ is the characteristic function of $j$}
    \[ \xymatrix{
      X \ar[rrrdd] \ar[dddrr] \ar[dr] \\
      & V \ar[rrd] \ar[ddr] \ar[dr] \\
      & & U \ar[r] \ar[d] & 1 \ar[d]^{\top} \\
      & & A \ar[r]_{\phi} & 2
    }
    \]
  }
  \begin{proof}
    \pf\ By the Subset Classifier Axiom.
  \end{proof}
  \step{<2>3}{\pflet{$X$ be a set and $x : X \rightarrow 1$, $y : X
\rightarrow
      A$ satisfy $\phi \circ y = \top \circ x$}}
  \step{<2>4}{\pflet{$\langle x,y \rangle : X \rightarrow U$ be the unique
function such that $! \circ \langle x,y \rangle = x$ and $i \circ \langle x,y
\rangle = y$}}
\begin{proof}
\pf\ By \stepref{<2>2}.
\end{proof}
\step{<2>5}{$h \circ \langle x,y \rangle$ is the unique function $X \rightarrow
V$
such that $! \circ h \circ \langle x,y \rangle = x$ and $j \circ h \circ
\langle x,y \rangle = y$}
\begin{proof}
\step{<3>1}{$! \circ h \circ \langle x,y \rangle = x$}
\begin{proof}
  \pf\ Since 1 is terminal.
\end{proof}
\step{<3>2}{$j \circ h \circ \langle x,y \rangle = y$}
\begin{proof}
  \pf\ From \stepref{<2>1} and \stepref{<2>4}.
\end{proof}
\step{<3>3}{If $! \circ f = x$ and $j \circ f = y$ then $f = h  \circ \langle
x,y
  \rangle$}
\begin{proof}
  \step{<4>1}{\pflet{$f : X \rightarrow V$ satisfy $! \circ f = x$ and $j
\circ
      f = y$}}
  \step{<4>2}{$! \circ k \circ f = x$}
  \begin{proof}
    \pf\ As 1 is terminal.
  \end{proof}
  \step{<4>3}{$i \circ k \circ f = y$}
  \begin{proof}
    \pf\ From \stepref{<2>1} and \stepref{<4>1}.
  \end{proof}
  \step{<4>4}{$k \circ f = \langle x, y \rangle$}
  \begin{proof}
    \pf\ From \stepref{<2>4}, \stepref{<4>2} and \stepref{<4>3}.
  \end{proof}
  \step{<4>5}{$f = h \circ \langle x,y \rangle$}
  \begin{proof}
    \pf\ From \stepref{<2>1} and \stepref{<4>4}.
  \end{proof}
\end{proof}
\end{proof}
%TODO Extract lemma
\end{proof}
\step{<1>4}{$3 \Rightarrow 2$
  \[ \xymatrix{
    U \ar[dr] \ar[rrd] \ar[ddr]_{i} \\
    &  V \ar[r] \ar[d]_{j} \ar[ul] & 1 \ar[d]^{\top} \\
   &  A \ar[r]_{\phi} & 2
  } \]
}
\begin{proof}
  \step{<2>1}{\assume{3}}
  \step{<2>2}{\pflet{$\phi$ be the characteristic function of $i$ and $j$}}
  \step{<2>3}{\pflet{$h : U \rightarrow V$ be the unique function such that
$!
      \circ h = !$ and $j \circ h = i$}}
  \begin{proof}
    \step{<3>1}{$\top \circ ! = \phi \circ i$}
    \begin{proof}
      \pf\ This holds because $\phi$ is the characteristic function of $i$.
    \end{proof}
    \qedstep
    \begin{proof}
      \pf\ Since $\phi$ is the characteristic function of $j$.
    \end{proof}
  \end{proof}
  \step{<2>4}{\pflet{$k : V \rightarrow U$ be the unique function such that
$!
      \circ k = !$ and $i \circ k = j$}}
  \begin{proof}
    \pf\ Similar.
  \end{proof}
  \step{<2>5}{$k \circ h = \id{U}$}
  \begin{proof}
    \pf\ Each is the unique function $f$ such that $! \circ f = !$ and $i
\circ f = i$
\end{proof}
  \step{<2>6}{$h \circ k = \id{V}$}
  \begin{proof}
    \pf\ Each is the unique function $f$ such that $! \circ f = !$ and $j
\circ f = j$
\end{proof}
\end{proof}
\qed
\end{proof}

\section{The Basics}


\begin{lm}
  \label{lm:set_theory:union_of_subsets}
  Let $X$ be a set, $\mathcal{B} \subseteq \mathcal{P} X$ and $U \subseteq
  X$. Then the following are equivalent:
  \begin{enumerate}
    \item For all $x \in U$ there exists $B \in \mathcal{B}$ such that $x \in
    B \subseteq U$.
    \item There exists $\mathcal{B}_0 \subseteq \mathcal{B}$ such that $U =
    \bigcup \mathcal{B}_0$.
  \end{enumerate}
\end{lm}

\begin{proof}
  \pf
  \step{<1>1}{$1 \Rightarrow 2$}
  \begin{proof}
    \pf\ If 1 is true then $U = \bigcup \{ B \in \mathcal{B} : B \subseteq U
    \}$.
  \end{proof}
  \step{<1>2}{$2 \Rightarrow 1$}
  \begin{proof}
    \pf\ Trivial.
  \end{proof}
  \qed
\end{proof}

\begin{df}[Fixed Point]
  Let $X$ be a set, $f : X \rightarrow X$, and $x \in X$. Then $x$ is a
  \emph{fixed point} of $f$ iff $f(x) = x$.
\end{df}

\begin{df}[Saturated]
  Let $X$, $Y$ be sets and $p : X \twoheadrightarrow Y$ be a surjective
  function. Let $C \subseteq X$. Then $C$ is \emph{saturated} with respect to
  $p$ iff, for all $x, x' \in X$, if $x \in C$ and $p(x) = p(x')$ then $x'
  \in
  C$.
\end{df}

\begin{df}[Cover]
  Let $A$ be a set and $\mathcal{C} \subseteq \mathcal{P}
  A$. Then $\mathcal{C}$ \emph{covers} $A$ iff $\bigcup \mathcal{C} = A$.
\end{df}

\begin{df}[Finite Intersection Property]
  Let $X$ be a set and $\mathcal{C} \subseteq \mathcal{P} X$. Then
  $\mathcal{C}$ has the \emph{finite intersection property} if and only if
  every finite nonempty subset of $\mathcal{C}$ has nonempty intersection.
\end{df}

\begin{lm}[AC]
  \label{lm:sets:finite_intersection_property:maximal}
  Let $X$ be a set and $\mathcal{A} \subseteq \mathcal{P} X$ have the finite intersection property. Then there exists
  a maximal $\mathcal{D} \subseteq \mathcal{P} X$ that has the finite intersection property and includes $\mathcal{A}$.
\end{lm}

\begin{proof}
  \pf\ A straightforward application of Zorn's lemma, since the union of a chain of sets that has the finite intersection property has the finite intersection property. \qed
\end{proof}

\begin{lm}
  \label{lm:sets:finite_intersection_property:finite_intersection}
  Let $X$ be a set and $\mathcal{D} \subseteq \mathcal{P} X$ be maximal with respect to the finite intersection property. Then any finite intersection of elements of $\mathcal{D}$ is an element of $\mathcal{D}$.
\end{lm}

\begin{proof}
  \pf
  \step{<1>1}{\pflet{$A$ be a finite intersection of elements of $\mathcal{D}$}}
  \step{<1>2}{$\mathcal{D} \cup \{ A \}$ has the finite intersection property.}
  \step{<1>3}{$\mathcal{D} \cup \{ A \} = \mathcal{D}$}
  \qed
\end{proof}

\begin{lm}
  \label{lm:sets:finite_intersection_property:intersect_all}
  Let $X$ be a set and $\mathcal{D} \subseteq \mathcal{P} X$ be maximal with respect to the finite intersection property. If $A \subseteq X$ intersects every element of $\mathcal{D}$ then $A \in \mathcal{D}$.
\end{lm}

\begin{proof}
  \pf\ This holds because $\mathcal{D} \cup \{ A \}$ satisfies the finite intersection property. \qed
\end{proof}

\begin{df}[Graph]
  Let $f : A \rightarrow B$. The \emph{graph} of $f$ is the set $\{ (x, f(x))
  : x \in A \} \subseteq A \times B$.
\end{df}

 \begin{df}[Point-Finite]
 Let $X$ be a set and $\{ A_\alpha \}_{\alpha \in J}$ be a family of subsets
 of $X$. Then $\{ A_\alpha \}_{\alpha \in J}$ is \emph{point-finite} iff, for
all $x \in X$, there are only finitely many $\alpha \in J$ such that $x \in
A_\alpha$.
\end{df}

\begin{df}[Countable Intersection Property]
  A family of parts of a set $X$ has the \emph{countable intersection property} iff every countable subfamily has nonempty intersection.
\end{df}

\section{Refinements}

\begin{df}[Refinement]
  Let $X$ be a set and $\mathcal{A}, \mathcal{B} \subseteq \mathcal{P} X$. Then $\mathcal{B}$ is a \emph{refinement} of $\mathcal{A}$ iff, for all $B \in \mathcal{B}$, there exists $A \in \mathcal{A}$ such that $B \subseteq A$.
\end{df}

\section{Order Theory}

  \begin{df}[Cofinal]
  Let $J$ be a poset and $K \subseteq J$. Then $K$ is \emph{cofinal} iff, for
all $x \in J$, there exists $y \in K$ such that $x \leq y$.
\end{df}

  \begin{df}[Directed Set]
  A \emph{directed set} is a poset $J$ such that, for all $x, y \in J$, there
exists $z \in J$ such that $x \leq z$ and $y \leq z$.
\end{df}

\begin{df}[Linear Order]
  Let $X$ be a set. A \emph{linear order} on $X$ is a relation $\leq
  \subseteq X^2$ such that:
  \begin{itemize}
    \item For all $x \in X$, $x \leq x$
    \item For all $x, y, z \in X$, if $x \leq y$ and $y \leq z$ then $x \leq
    z$
    \item For all $x, y \in X$, if $x \leq y$ and $y \leq x$ then $x = y$
    \item For all $x, y \in X$, we have $x \leq y$ or $y \leq x$
  \end{itemize}
  We write $x < y$ iff $x \leq y$ and $x \neq y$.

  A \emph{linearly ordered set} consists of a set and a linear order on the
  set.
\end{df}

\begin{df}[Convex]
  Let $L$ be a linearly ordered set and $A \subseteq L$. Then $A$ is
  \emph{convex} iff, for all $x, y \in A$ and $z \in L$, if $x < z < y$ then
  $z \in A$.
\end{df}

\begin{df}[Least Upper Bound Property]
  A linearly ordered set $L$ has the \emph{least upper bound property} iff
  every subset of $L$ bounded above has a least upper bound.
\end{df}

\begin{df}[Linear Continuum]
  A \emph{linear continuum} is a linearly ordered set $L$ such that:
  \begin{itemize}
    \item $L$ has the least upper bound property.
    \item For all $x, y \in L$ with $x < y$, there exists $z \in L$ such that
    $x < z < y$.
  \end{itemize}
\end{df}

\begin{prop}
  If $L$ is a linear continuum then every convex subset of $L$ is a linear
  continuum.
\end{prop}

\begin{proof}
  \pf
  \step{<1>1}{\pflet{$L$ be a linear continuum and $C \subseteq L$ be convex}}
  \step{<1>2}{$C$ satisfies the least upper bound property.}
  \begin{proof}
    \step{<2>1}{\pflet{$S \subseteq C$ be nonempty and bounded above by $u$
        in
        $C$.}}
    \step{<2>2}{\pflet{$s$ be the supremum of $S$ in $L$}}
    \step{<2>3}{\pick\ $x \in S$}
    \step{<2>4}{$x \leq s \leq u$}
    \step{<2>5}{$s \in C$}
    \begin{proof}
      \pf\ $C$ is convex.
    \end{proof}
    \step{<2>6}{$s$ is the supremum of $S$ in $C$}
  \end{proof}
  \step{<1>3}{$C$ is dense.}
  \begin{proof}
    \pf
    \step{<2>1}{\pflet{$x, y \in C$ satisfy $x < y$}}
    \step{<2>2}{\pick\ $z \in L$ such that $x < z < y$}
    \step{<2>3}{$z \in C$}
    \begin{proof}
      \pf\ $C$ is convex.
    \end{proof}
  \end{proof}
  \qed
\end{proof}

\begin{lm}
  \label{lm:order:half_open}
  For any real numbers $a$, $b$ with $a < b$ we have $[a, b) \cong [0, 1)$.
\end{lm}

\begin{proof}
  \pf\ The map $\phi : [a, b) \cong [0, 1)$ where $\phi(x) = (x - a) / (b -
  a)$ is an order isomorphism. \qed
\end{proof}

\begin{prop}
  \label{prop:order:zero_one_twice}
  Let $X$ be a linearly ordered set. Let $a, b, c \in X$ with $a < c < b$.
  Then $[a, b) \cong [0, 1)$ if and only if $[a, c) \cong [c, b) \cong [0,
  1)$.
\end{prop}

\begin{proof}
  \pf
  \step{<1>1}{If $[a, b) \cong [0, 1)$ then $[a, c) \cong [c, b) \cong [0,
    1)$.}
  \begin{proof}
    \step{<2>1}{\assume{$\phi : [a, b) \cong [0, 1)$ is an order
        isomorphism.}}
    \step{<2>2}{$[a, c) \cong [0, 1)$}
    \begin{proof}
      \pf
      \begin{align*}
        [a ,c) & \cong [0, \phi(c)) & (\text{under } \phi) \\
        & \cong [0, 1) & (\text{Lemma \ref{lm:order:half_open} })
      \end{align*}
    \end{proof}
    \step{<2>3}{$[c, b) \cong [0, 1)$}
    \begin{proof}
      \pf\ Similar.
    \end{proof}
  \end{proof}
  \step{<1>2}{If $[a, c) \cong [c, b) \cong [0, 1)$ then $[a, b) \cong [0,
    1)$.}
  \begin{proof}
    \step{<2>1}{\assume{$[a, c) \cong [c, b) \cong [0, 1)$}}
    \step{<2>2}{\pflet{$\phi : [a, c) \cong [0, 1/2)$ and $\psi : [c, b)
        \cong
        [1/2, 1)$}}
    \step{<2>3}{\pflet{$\chi : [a, b) \rightarrow [0, 1)$ be given by
        $\chi(x)
        =
        \begin{cases} \phi(x) & \text{if } x < c \\ \psi(x) & \text{if } x
          \geq c \end{cases}$}}
    \step{<2>4}{$\chi : [a, b) \cong [0, 1)$}
    \begin{proof}
      \pf\ Easy to check. % TODO Extract lemma
    \end{proof}
  \end{proof}
  \qed
\end{proof}


\begin{prop}[CC]
  \label{prop:order:zero_one_countable}
  Let $X$ be a linearly ordered set. Let $\{ x_n \}_{n \geq 0}$ be an
  increasing sequence of points of $X$. Suppose $b$ is the supremum of $\{
  x_n : n \geq 0 \}$. Then $[x_0, b) \cong [0, 1)$ if and only if $[x_i,
  x_{i+1}) \cong [0,1)$ for all $i$.
\end{prop}

\begin{proof}
  \pf
  \step{<1>1}{If $[x_0, b) \cong [0, 1)$ then for all $i$ $[x_i, x_{i+1})
    \cong
    [0, 1)$.}
  \begin{proof}
    \pf\ If $\phi : [x_0, b) \cong [0, 1)$ then $[x_i, x_{i+1}) \cong
    [\phi(x_i), \phi(x_{i+1})) \cong [0, 1)$ by Lemma
    \ref{lm:order:half_open}.
  \end{proof}
  \step{<1>2}{If for all $i$ $[x_i, x_{i+1}) \cong [0, 1)$ then $[x_0, b)
    \cong
    [0, 1)$.}
  \begin{proof}
    \pf
    \step{<2>1}{\pflet{$\phi_i : [x_i, x_{i+1}) \cong [0, 1)$ for all $i$}}
    \step{<2>2}{Define $\phi : [x_0, b) \cong [0, 1)$ by: $\phi(y) =
      \phi_i(y) \qquad (x_0 \leq y < b)$ where $i$ is least such that $y <
      i_{i+1}$}
    \begin{proof}
      \pf\ There exists such an $i$ because $y$ is not an upper bound for
      $\{x_n : n \geq 0 \}$.
    \end{proof}
    \step{<2>3}{$\phi$ is an order isomorphism.}
    \begin{proof}
      \pf\ Easy to check.
    \end{proof}
  \end{proof}
  \qed
\end{proof}

\begin{prop}[CC]
  \label{prop:order:long_line_zero_one}
  For all $0 < \alpha < \Omega$, the interval $[(0, 0), (\alpha, 0))$ in
  $S_\Omega \times [0, 1)$ is order isomorphic to $[0, 1)$ in $\mathbb{R}$.
\end{prop}

\begin{proof}
  \pf
  \step{<1>1}{If $[(0, 0), (\alpha, 0)) \cong [0, 1)$ then $[(0, 0), (\alpha
    +
    1,
    0)) \cong [0, 1)$}
  \begin{proof}
    \pf\ By Proposition \ref{prop:order:zero_one_twice}.
  \end{proof}
  \step{<1>2}{Let $\lambda$ be a limit ordinal, $0 < \lambda < \Omega$. If,
    for
    all $\alpha$ with $0 < \alpha < \lambda$, we have $[(0, 0), (\alpha, 0))
    \cong [0, 1)$, then $[(0, 0), (\lambda, 0)) \cong [0, 1)$.}
  \begin{proof}
    \pf\ By Propositoin \ref{prop:order:zero_one_countable}.
  \end{proof}
  \qedstep
  \begin{proof}
    \pf\ By transfinite induction.
  \end{proof}
  \qed
\end{proof}
