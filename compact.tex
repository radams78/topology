\chapter{Compact Spaces}

\section{Countable Compactness}

\begin{df}[Countably Compact]
  A topological space is \emph{countably compact} iff every countable open
  covering has a finite subcovering.
\end{df}

\section{Limit Point Compactness}

\begin{df}[Limit Point Compact]
  A space is \emph{limit point compact} iff every infinite set has a
  limit point.
\end{df}

 \begin{prop}[CC]
 $S_\Omega \times \overline{S_\Omega}$ is limit point compact.
\end{prop}

\begin{proof}
\pf
\step{<1>1}{\pflet{$A \subseteq S_\Omega \times \overline{S_\Omega}$ be
    infinite}}
\step{<1>2}{\case{$\pi_1(A)$ is finite.}}
\begin{proof}
  \step{<2>1}{\pick\ $x$ such that there are infinitely many $y$ such that
$(x,
    y) \in A$}
  \step{<2>2}{\pick\ a limit point $l$ of $\{ y : (x,y) \in A \}$}
  \step{<2>3}{$(x, l)$ is a limit point of $A$}
\end{proof}
\step{<1>3}{\case{$\pi_1(A)$ is infinite.}}
\begin{proof}
  \step{<2>1}{\pick\ a limit point $l$ of $\pi_1(A)$.}
  \step{<2>2}{$l$ is a limit ordinal}
  \step{<2>3}{\pick\ a countable sequence $x_n$ with limit $l$}
  \step{<2>4}{For $n \geq 1$, \pick\ $a_n > x_n$ and $y_n$ such that $(a_n,
y_n)
    \in A$}
  \step{<2>5}{\case{$\{ y_n : n \geq 1 \}$ is finite}}
  \begin{proof}
    \step{<3>1}{\pick\ $y$ such that $y = y_n$ for infinitely many $n$}
    \step{<3>2}{$(l, y)$ is a limit point for $A$}
  \end{proof}
<\step{<2>6}{\case{$\{ y_n : n \geq 1 \}$ is infinite}}
  \begin{proof}
    \step{<3>1}{\pick\ a limit point $m$ for $\{ y_n : n \geq 1 \}$}
    \step{<3>2}{$(l, m)$ is a limit point for $A$}
  \end{proof}
\end{proof}
\qed
\end{proof}

\begin{prop}
The Sorgenfrey plane is not limit point compact.
\end{prop}

\begin{proof}
 \pf\ $\mathbb{Z}^2$ has no limit point. \qed
\end{proof}

\begin{prop}
 The space $\mathbb{R}^\omega$ under the box topology is not limit point
compact.
\end{prop}

\begin{proof}
\pf The set of all constant sequences of integers is an infinite set with no
limit point. \qed
\end{proof}

\begin{prop}
 Not every open subspace of a limit point compact space is limit point compact.
\end{prop}

\begin{proof}
 \pf\ The space $[0,1]$ is limit point compact but $(0,1)$ is not. \qed
\end{proof}

\begin{prop}
 The product of two limit point compact spaces is not necessarily limit point compact.
\end{prop}

\begin{proof}
 \pf\ See Steen and Seebach \emph{Countexamples in Topology} Example 112. \qed
\end{proof}

\begin{prop}
 The continuous image of a limit point comapct space is not necessarily limit point comapct.
\end{prop}

\begin{proof}
 \pf\ Let $Y$ be a two-point set under the indiscrete topology. Then $\mathbb{N}
 \times Y$ is limit point compact, but $\mathbb{N}$ is not. \qed
\end{proof}

\section{Lindel\"{o}f Spaces}

  \begin{df}[Lindel\"{o}f Space]
  A topological space $X$ is \emph{Lindel\"{o}f} iff every open covering has
  a countable subcovering.
\end{df}

  \begin{thm}[CC]
    \label{thm:topology:lindelof:second_countable}
  Every second countable space is Lindel\"{o}f.
\end{thm}

\begin{proof}
 \pf
 \step{<1>1}{\pflet{$X$ be a second countable space}}
 \step{<1>2}{\pick\ a countable basis $\mathcal{B}$ for $X$.}
 \step{<1>3}{\pflet{$\mathcal{A}$ be an open cover of $X$}}
 \step{<1>4}{For every $B \in \mathcal{B}$ such that there exists $U \in
   \mathcal{A}$ such that $B \subseteq U$, \pick\ $U_B \in \mathcal{A}$ such
   that $B \subseteq U_B$}
 \step{<1>5}{$\{ U_B : B \in \mathcal{B}, \exists U \in \mathcal{A}. B
\subseteq
   U \}$ covers $X$.}
 \begin{proof}
   \step{<2>1}{\pflet{$x \in X$}}
   \step{<2>2}{\pick\ $U \in \mathcal{A}$ such that $x \in U$}
   \step{<2>3}{\pick\ $B \in \mathcal{B}$ such that $x \in B \subseteq U$}
   \step{<2>4}{$x \in U_B$}
 \end{proof}
 \qed
\end{proof}

\begin{cor}
  The space $\mathbb{R}^\omega$ is Lindel\"{o}f.
\end{cor}

\begin{cor}
  The space $\mathbb{R}_K$ is Lindel\"{o}f.
\end{cor}

  \begin{prop}
  The space $S_\Omega$ is not Lindel\"{o}f.
\end{prop}

\begin{proof}
  \pf $\{ (- \infty, \alpha) : \alpha \in S_\Omega \}$ is an open cover that
has no countable subcover. \qed
\end{proof}

  \begin{prop}[CC]
  The space $\overline{S_\Omega}$ is Lindel\"{o}f.
\end{prop}

\begin{proof}
 \pf
 \step{<1>1}{\pflet{$\mathcal{A}$ be an open cover of $\overline{S_\Omega}$}}
 \step{<1>2}{\pick\ $U \in \mathcal{A}$ such that $\Omega \in U$}
 \step{<1>3}{\pick\ $\alpha < \Omega$ such that $(\alpha, \Omega] \subseteq U$}
 \step{<1>4}{For $\beta \leq \alpha$, \pick\ $U_\beta \in \mathcal{A}$ such that
   $\beta \in U_\beta$}
 \step{<1>5}{$\{ U \} \cup \{ U_\beta : \beta \leq \alpha \}$ is a countable
   subcover of $\mathcal{A}$.}
 \qed
\end{proof}

\begin{prop}[CC]
  The continuous image of a Lindel\"{o}f space is Lindel\"{o}f.
\end{prop}

\begin{proof}
 \pf
 \step{<1>1}{\pflet{$X$ be a Lindel\"{o}f space, $Y$ a space and $f : X
     \rightarrow Y$ continuous.}}
 \step{<1>2}{\pflet{$\mathcal{A}$ be an open covering of $Y$}}
 \step{<1>3}{$\{ \inv{f}(V) : V \in \mathcal{A} \}$ is an open covering of
$X$}
 \step{<1>4}{\pick\ a countable subcovering $\{ \inv{f}(V_1), \inv{f}(V_2),
   \ldots \}$ of $\{ \inv{f}(V) : V \in \mathcal{A} \}$}
 \step{<1>5}{$\{ V_1, V_2, \ldots \}$ is a countable subcovering of
   $\mathcal{A}$}
 \qed
\end{proof}

\begin{prop}
 The Sorgenfrey plane is not Lindel\"{o}f.
\end{prop}

\begin{proof}
\pf
\step{<1>1}{\pflet{$L = \{ (x, -x) : x \in \mathbb{R} \}$}}
\step{<1>2}{$L$ is closed in $\mathbb{R}_l^2$}
\begin{proof}
  \step{<2>1}{\pflet{$(x, y) \notin L$, so $y \neq -x$} \prove{There exists a
      neighbourhood $U$ of $(x,y)$ that does not intersect $L$}}
  \step{<2>2}{\case{$y > -x$}}
  \begin{proof}
    \pf\ In this case, take $U = [x,+\infty) \times [y, + \infty)$
  \end{proof}
  \step{<2>3}{\case{$y < -x$}}
  \begin{proof}
    \pf\ In this case, take $U = [x,(x-y)/2) \times [y,(y-x)/2)$.
  \end{proof}
\end{proof}
\step{<1>3}{\pflet{$\mathcal{U} = \{ \mathbb{R}_l^2 \setminus L \} \cup \{
[a,b)
    \times [-a,d) : a,b,d \in \mathbb{R} \}$}}
\step{<1>4}{$\mathcal{U}$ is an open covering of $\mathbb{R}_l^2$}
\step{<1>5}{No countable subset of $\mathcal{U}$ covers $\mathbb{R}_l^2$}
\begin{proof}
  \pf\ Every set $[a,b) \times [-a,d)$ intersects $L$ in exactly one point,
namely $(a, -a)$.
\end{proof}
\qed
\end{proof}

\begin{cor}
The Sorgenfrey plane is not second countable.
\end{cor}

\begin{cor}
The product of two Lindel\"{o}f spaces is not necessarily Lindel\"{o}f.
\end{cor}

\begin{prop}
The space $\mathbb{R}^\omega$ under the box topology is not Lindel\"{o}f.
\end{prop}

\begin{proof}
\pf\ The set $\{ \prod_{n=0}^\infty (a_n, a_n + 1) : \forall n. a_n \in \mathbb{Z} \}$ covers the space but has no countable subcover. \qed
\end{proof}

\begin{prop}
Not every open subspace of a Lindel\"{o}f space is Lindel\"{o}f.
\end{prop}

\begin{proof}
\pf\ The ordered square is Lindel\"{o}f but the subspace $[0,1]
\times (0,1)$ is not. \qed
\end{proof}

\section{Paracompactness}

\begin{df}[Paracompact]
A topological space $X$ is \emph{paracompact} iff every open covering of $X$ has a locally finite open refinement that covers $X$.
\end{df}

\begin{thm}
\label{thm:topology:paracompact:Hausdorff_normal}
Every paracompact Hausdorff space is normal.
\end{thm}

\begin{proof}
\pf
\step{<1>1}{\pflet{$X$ be a paracompact Hausdorff space.}}
\step{<1>2}{$X$ is regular.}
\begin{proof}
  \step{<2>1}{\pflet{$A$ be a closed set.}}
  \step{<2>2}{\pflet{$a \notin A$}}
  \step{<2>3}{For all $x \in A$, there exists an open set $U$ such that $x \in U$ and $a \notin \overline{U}$}
  \begin{proof}
    \step{<3>1}{\pflet{$x \in A$}}
    \step{<3>2}{$x \neq a$}
    \begin{proof}
      \pf\ \stepref{<2>2}, \stepref{<3>1}
    \end{proof}
    \step{<3>3}{\pick\ disjoint open neighbourhoods $U$ of $x$ and $V$ of $a$}
    \begin{proof}
      \pf\ \stepref{<1>1}, \stepref{<3>2}
    \end{proof}
    \step{<3>4}{$a \notin \overline{U}$}
    \begin{proof}
      \pf\ Theorem \ref{thm:topology:closure:neighbourhoods}, \stepref{<3>3}.
    \end{proof}
  \end{proof}
  \step{<2>4}{\pick\ a locally finite open refinement $\mathcal{C}$ of $\{ U \text{ open in } X : a \notin \overline{U} \} \cup \{ X \setminus A \}$ that covers $X$}
  \begin{proof}
    \pf\ By \stepref{<2>3}, $\{ U \text{ open in } X : a \notin \overline{U} \} \cup \{ X \setminus A \}$ is an open covering of $X$.
  \end{proof}
  \step{<2>5}{\pflet{$\mathcal{D} = \{ U \in \mathcal{C} : U \cap A \neq \emptyset \}$}}
  \step{<2>6}{$\mathcal{D}$ covers $A$}
  \begin{proof}
    \pf\ From \stepref{<2>4} and \stepref{<2>5}.
  \end{proof}
  \step{<2>7}{For all $U \in \mathcal{D}$ we have $a \notin \overline{U}$}
  \begin{proof}
    \pf\ From \stepref{<2>4} and \stepref{<2>5}.
  \end{proof}
  \step{<2>8}{\pflet{$V = \bigcup \mathcal{D}$}}
  \step{<2>9}{$V$ is open}
  \begin{proof}
    \step{<3>1}{Every member of $\mathcal{D}$ is open.}
    \begin{proof}
      \pf\ From \stepref{<2>4} and \stepref{<2>5}.
    \end{proof}
    \qedstep
    \begin{proof}
      \pf\ By \stepref{<2>8}.
    \end{proof}
  \end{proof}
  \step{<2>10}{$A \subseteq V$}
  \begin{proof}
    \pf\ From \stepref{<2>6} and \stepref{<2>7}.
  \end{proof}
  \step{<2>11}{$a \notin \overline{V}$}
  \begin{proof}
    \step{<3>1}{$\mathcal{D}$ is locally finite.}
    \begin{proof}
      \pf\ Lemma \ref{lm:topology:locally_finite:subfamily}, \stepref{<2>4}, \stepref{<2>5}.
    \end{proof}
    \step{<3>2}{$\overline{V} = \bigcup_{U \in \mathcal{D}} \overline{U}$}
    \begin{proof}
      \pf\ By Lemma \ref{lm:topology:closure:locally_finite_union}, \stepref{<2>8} and \stepref{<3>1}.
    \end{proof}
    \qedstep
    \begin{proof}
      \pf\ By \stepref{<2>7}.
    \end{proof}
  \end{proof}
  \qedstep
  \begin{proof}
    \pf\ Proposition \ref{prop:topology:regular:closure}.
  \end{proof}
\end{proof}
\step{<1>3}{$X$ is normal.}
\begin{proof}
  \step{<2>1}{\pflet{$A$, $B$ be disjoint closed sets.}}
  \step{<2>2}{For all $x \in A$, there exists an open set $U$ such that $x \in U$ and $B$ is disjoint from $\overline{U}$}
  \begin{proof}
    \step{<3>1}{\pflet{$x \in A$}}
    \step{<3>2}{$x \notin B$}
    \begin{proof}
      \pf\ \stepref{<2>2}, \stepref{<3>1}
    \end{proof}
    \step{<3>3}{\pick\ disjoint open neighbourhoods $U$ of $x$ and $V$ of $B$}
    \begin{proof}
      \pf\ \stepref{<1>2}, \stepref{<3>2}
    \end{proof}
    \step{<3>4}{$B$ is disjoint from $\overline{U}$}
    \begin{proof}
      \pf\ $B \subseteq V \subseteq X \setminus \overline{U}$
    \end{proof}
  \end{proof}
  \step{<2>3}{\pick\ a locally finite open refinement $\mathcal{C}$ of $\{ U \text{ open in } X : B \cap \overline{U} = \emptyset \} \cup \{ X \setminus A \}$ that covers $X$}
  \begin{proof}
    \pf\ By \stepref{<2>2}, $\{ U \text{ open in } X : B \cap \overline{U} = \emptyset \} \cup \{ X \setminus A \}$ is an open covering of $X$.
  \end{proof}
  \step{<2>4}{\pflet{$\mathcal{D} = \{ U \in \mathcal{C} : U \cap A \neq \emptyset \}$}}
  \step{<2>5}{$\mathcal{D}$ covers $A$}
  \begin{proof}
    \pf\ From \stepref{<2>3} and \stepref{<2>4}.
  \end{proof}
  \step{<2>6}{For all $U \in \mathcal{D}$ we have $B \cap \overline{U} = \emptyset$}
  \begin{proof}
    \pf\ From \stepref{<2>3} and \stepref{<2>4}.
  \end{proof}
  \step{<2>7}{\pflet{$V = \bigcup \mathcal{D}$}}
  \step{<2>8}{$V$ is open}
  \begin{proof}
    \step{<3>1}{Every member of $\mathcal{D}$ is open.}
    \begin{proof}
      \pf\ From \stepref{<2>3} and \stepref{<2>4}.
    \end{proof}
    \qedstep
    \begin{proof}
      \pf\ By \stepref{<2>7}.
    \end{proof}
  \end{proof}
  \step{<2>9}{$A \subseteq V$}
  \begin{proof}
    \pf\ From \stepref{<2>5} and \stepref{<2>6}.
  \end{proof}
  \step{<2>10}{$B \cap \overline{V} = \emptyset$}
  \begin{proof}
    \step{<3>1}{$\mathcal{D}$ is locally finite.}
    \begin{proof}
      \pf\ Lemma \ref{lm:topology:locally_finite:subfamily}, \stepref{<2>3}, \stepref{<2>4}.
    \end{proof}
    \step{<3>2}{$\overline{V} = \bigcup_{U \in \mathcal{D}} \overline{U}$}
    \begin{proof}
      \pf\ By Lemma \ref{lm:topology:closure:locally_finite_union}, \stepref{<2>7} and \stepref{<3>1}.
    \end{proof}
    \qedstep
    \begin{proof}
      \pf\ By \stepref{<2>6}.
    \end{proof}
  \end{proof}
  \qedstep
  \begin{proof}
    \pf\ $V$ and $X \setminus \overline{V}$ are disjoint open neighbourhoods of $A$ and $B$ respectively.
  \end{proof}
\end{proof}
\qed
\end{proof}

\begin{thm}
  \label{thm:topology:paracompact:closed_subspace}
Every closed subspace of a paracompact space is paracompact.
\end{thm}

\begin{proof}
\pf
\step{<1>1}{\pflet{$X$ be a paracompact space.}}
\step{<1>2}{\pflet{$Y$ be closed in $X$.}}
\step{<1>3}{\pflet{$\mathcal{A}$ be an open covering of $Y$.}}
\step{<1>4}{$\{ U \text{ open in } X : U \cap Y \in \mathcal{A} \} \cup \{ X \setminus Y \}$ is an open covering of $X$.}
\step{<1>5}{\pick\ a locally finite open refinement $\mathcal{B}$ that covers $X$.}
\step{<1>6}{$\{ U \cap Y : U \in \mathcal{B} \}$ is a locally finite open refinement of $\mathcal{A}$ that covers $Y$.}
\begin{proof}
  \step{<2>1}{\pflet{$\mathcal{C} = \{ U \cap Y : U \in \mathcal{B} \}$}}
  \step{<2>2}{$\mathcal{C}$ is locally finite.}
  \begin{proof}
    \pf\ Proposition \ref{prop:topology:locally_finite:subset},\stepref{<1>5}, \stepref{<2>1}.
  \end{proof}
  \step{<2>3}{$\mathcal{C}$ refines $\mathcal{A}$}
\end{proof}
\qed
\end{proof}

\begin{lm}[E. Michael (AC)]
\label{lm:topology:paracompact:michaels_lemma}
Let $X$ be a regular space. Then the following are equivalent.
\begin{enumerate}
  \item
  \label{one}
  Every open covering of $X$ has a countably locally finite open refinement that covers $X$.
  \item
  \label{two}
  Every open covering of $X$ has a locally finite refinement that covers $X$.
  \item
  \label{three}
  Every open covering of $X$ has a locally finite closed refinement that covers $X$.
  \item
  \label{four}
  $X$ is paracompact.
\end{enumerate}
\end{lm}

\begin{proof}
\pf
\step{<1>1}{\pflet{$X$ be a regular space.}}
\step{<1>2}{$\ref{one} \Rightarrow \ref{two}$}
\begin{proof}
  \step{<2>1}{\assume{\ref{one}}}
  \step{<2>2}{\pflet{$\mathcal{A}$ be an open covering of $X$.}}
  \step{<2>3}{\pick\ a countably locally finite open refinement $\mathcal{B}$ of $\mathcal{A}$ that covers $X$.}
  \begin{proof}
    \pf\ \stepref{<2>1}, \stepref{<2>2}
  \end{proof}
  \step{<2>4}{\pick\ locally finite sets $\mathcal{B}_n$ for $n \in \mathbb{N}$ such that $\mathcal{B} = \bigcup_{n=0}^\infty \mathcal{B}_n$}
  \begin{proof}
    \pf\ From \stepref{<2>3}
  \end{proof}
  \step{<2>5}{For $n \in \mathbb{N}$, \pflet{$V_n = \bigcup \mathcal{B}_n$}}
  \step{<2>6}{For $n \in \mathbb{N}$ and $U \in \mathcal{B}_n$, \pflet{$S_n(U) = U \setminus \bigcup_{i<n} V_i$}}
  \step{<2>7}{For $n \in \mathbb{N}$, \pflet{$\mathcal{C}_n = \{ S_n(U) : U \in \mathcal{B}_n \}$}}
  \step{<2>8}{For $n \in \mathbb{N}$, we have $\mathcal{C}_n$ refines $\mathcal{B}_n$}
  \begin{proof}
    \pf\ This holds because $S_n(U) \subseteq U$.
  \end{proof}
  \step{<2>9}{\pflet{$\mathcal{C} = \bigcup_n \mathcal{C}_n$}}
  \step{<2>10}{$\mathcal{C}$ is locally finite}
  \begin{proof}
    \step{<3>1}{\pflet{$x \in X$}}
    \step{<3>2}{\pflet{$N$ be least such that there exists $U \in \mathcal{B}_N$ such that $x \in U$}}
    \begin{proof}
      \pf\ By \stepref{<2>3} and \stepref{<2>4}
    \end{proof}
    \step{<3>3}{\pick\ $U \in \mathcal{B}_N$ such that $x \in U$}
    \step{<3>4}{For $1 \leq i \leq N$, \pick\ a neighbourhood $W_i$ of $x$ that intersects only finitely many elements of $\mathcal{B}_i$}
    \begin{proof}
      \pf\ By \stepref{<2>4}
    \end{proof}
    \step{<3>5}{For $1 \leq i \leq N$, $W_i$ intersects only finitely many elements of $\mathcal{C}_i$}
    \begin{proof}
      \pf\ If $W_i$ intersects $S_i(U)$ then $W_i$ intersects $U$.
    \end{proof}
    \step{<3>6}{\pflet{$W = U \cap W_1 \cap \cdots \cap W_N$}}
    \step{<3>7}{$W$ intersects only finitely many elements of $\mathcal{C}$}
    \begin{proof}
      \step{<4>1}{For $i \leq N$, $W$ intersects only finitely many elements of $\mathcal{C}_i$}
      \begin{proof}
        \pf\ From \stepref{<3>5} and \stepref{<3>6}.
      \end{proof}
      \step{<4>2}{For $i > N$, $W$ intersects no elements of $\mathcal{C}_i$.}
      \begin{proof}
        \pf\ This holds because $W \subseteq U \subseteq V_N$.
      \end{proof}
    \end{proof}
  \end{proof}
  \step{<2>11}{$\mathcal{C}$ refines $\mathcal{A}$}
  \begin{proof}
    \pf\ From \stepref{<2>3} and \stepref{<2>8}
  \end{proof}
  \step{<2>12}{$\mathcal{C}$ covers $X$}
  \begin{proof}
    \step{<3>1}{\pflet{$x \in X$}}
    \step{<3>2}{\pflet{$N$ be least such that there exists $U \in \mathcal{B}_N$ such that $x \in U$}}
    \step{<3>3}{\pick\ $U \in \mathcal{B}_N$ such that $x \in U$}
    \step{<3>4}{$x \in S_N(U)$}
  \end{proof}
\end{proof}
\step{<1>3}{$\ref{two} \Rightarrow \ref{three}$}
\begin{proof}
  \step{<2>1}{\assume{\ref{two}}}
  \step{<2>2}{\pflet{$\mathcal{A}$ be an open covering of $X$.}}
  \step{<2>3}{\pflet{$\mathcal{B} = \{ U \text{ open in } X : \exists V \in \mathcal{A}. \overline{U} \subseteq V \}$}}
  \step{<2>4}{$\mathcal{B}$ covers $X$}
  \begin{proof}
    \step{<3>1}{\pflet{$x \in X$}}
    \step{<3>2}{\pick\ $V \in \mathcal{A}$ such that $x \in V$}
    \begin{proof}
      \pf\ From \stepref{<2>2}
    \end{proof}
    \step{<3>3}{\pick\ $U$ an open neighbourhood of $x$ such that $\overline{U} \subseteq V$}
    \begin{proof}
      \pf\ From Proposition \ref{prop:topology:regular:closure}, \stepref{<1>1}, \stepref{<3>1}, \stepref{<3>3}.
    \end{proof}
    \step{<3>4}{$U \in \mathcal{B}$}
    \begin{proof}
      \pf\ \stepref{<2>3}, \stepref{<3>2}, \stepref{<3>3}.
    \end{proof}
  \end{proof}
  \step{<2>5}{\pick\ a locally finite refinement $\mathcal{C}$ of $\mathcal{B}$ that covers $X$.}
  \begin{proof}
    \pf\ \stepref{<2>1}, \stepref{<2>3}, \stepref{<2>4}.
  \end{proof}
  \step{<2>6}{\pflet{$\mathcal{D} = \{ \overline{C} : C \in \mathcal{C} \}$}}
  \step{<2>7}{$\mathcal{D}$ is a locally finite closed refinement of $\mathcal{A}$ that covers $X$.}
  \begin{proof}
    \step{<3>1}{$\mathcal{D}$ is locally finite.}
    \begin{proof}
      \pf\ Lemma \ref{lm:topology:closure:locally_finite}, \stepref{<2>5}, \stepref{<2>6}.
    \end{proof}
    \step{<3>2}{Every member of $\mathcal{D}$ is closed.}
    \begin{proof}
      \pf\ Proposition \ref{prop:topology:closure:closed}, \stepref{<2>6}.
    \end{proof}
    \step{<3>3}{$\mathcal{D}$ refines $\mathcal{A}$.}
    \begin{proof}
      \step{<4>1}{\pflet{$D \in \mathcal{D}$}}
      \step{<4>2}{\pick\ $C \in \mathcal{C}$ such that $D = \overline{C}$}
      \begin{proof}
        \pf\ \stepref{<2>6}, \stepref{<4>1}
      \end{proof}
      \step{<4>3}{\pick\ $U \in \mathcal{B}$ such that $C \subseteq U$}
      \begin{proof}
        \pf \stepref{<2>5}, \stepref{<4>2}
      \end{proof}
      \step{<4>4}{\pick\ $V \in \mathcal{A}$ such that $\overline{U} \subseteq V$}
      \begin{proof}
        \pf\ \stepref{<2>3}, \stepref{<4>3}
      \end{proof}
      \step{<4>5}{$D \subseteq V$}
      \begin{proof}
        \pf
        \begin{align*}
          D & = \overline{C} & (\text{\stepref{<4>2}}) \\
          & \subseteq \overline{U} & (\text{\stepref{<4>3}, Proposition \ref{prop:topology:closure:monotone}})\\
          & \subseteq V & (\text{\stepref{<4>4}})
        \end{align*}
      \end{proof}
    \end{proof}
    \step{<3>4}{$\mathcal{D}$ covers $X$.}
    \begin{proof}
      \step{<4>1}{\pflet{$x \in X$}}
      \step{<4>2}{\pick\ $C \in \mathcal{C}$ such that $x \in C$}
      \begin{proof}
        \pf\ \stepref{<2>5}, \stepref{<4>1}
      \end{proof}
      \step{<4>3}{$x \in \overline{C} \in \mathcal{D}$}
      \begin{proof}
        \step{<5>1}{$x \in \overline{C}$}
        \begin{proof}
          \pf\ Proposition \ref{prop:topology:closure:A_sub_Abar}, \stepref{<4>2}.
        \end{proof}
        \step{<5>2}{$\overline{C} \in \mathcal{D}$}
        \begin{proof}
          \pf\ \stepref{<2>6}, \stepref{<4>2}.
        \end{proof}
      \end{proof}
    \end{proof}
  \end{proof}
\end{proof}
\step{<1>4}{$\ref{three} \Rightarrow \ref{four}$}
\begin{proof}
  \step{<2>1}{\assume{\ref{three}}}
  \step{<2>2}{\pflet{$\mathcal{A}$ be an open covering of $X$}}
  \step{<2>3}{\pick\ a locally finite refinement $\mathcal{B}$ of $\mathcal{A}$ that covers $X$.}
  \begin{proof}
    \pf\ \stepref{<2>1}, \stepref{<2>2}
  \end{proof}
  \step{<2>4}{$\{ U \text{ open in } X : U \text{ intersects only finitely many elements of } \mathcal{B} \}$ is an open covering of $X$.}
  \begin{proof}
    \pf\ From \stepref{<2>3}
  \end{proof}
  \step{<2>5}{\pick\ a locally finite closed refinement $\mathcal{C}$ of $\{ U \text{ open in } X : U \text{ intersects only finitely many elements of } \mathcal{B} \}$ that covers $X$.}
  \begin{proof}
    \pf\ \stepref{<2>1}, \stepref{<2>4}.
  \end{proof}
  \step{<2>6}{Every element of $\mathcal{C}$ intersects only finitely many elements of $\mathcal{B}$}
  \begin{proof}
    \step{<3>1}{\pflet{$C \in \mathcal{C}$}}
    \step{<3>2}{There exists $U$ open in $X$ such that $U$ intersects only finitely many elements of $\mathcal{B}$ and $C \subseteq U$}
    \begin{proof}
      \pf\ \stepref{<2>5}, \stepref{<3>1}
    \end{proof}
    \step{<3>3}{$C$ intersects only finitely many elements of $\mathcal{B}$}
    \begin{proof}
      \pf\ From \stepref{<3>2}
    \end{proof}
  \end{proof}
  \step{<2>7}{For $B \in \mathcal{B}$, \pflet{$C(B) = \{ C \in \mathcal{C} : C \subseteq X \setminus B \}$}}
  \step{<2>8}{For $B \in \mathcal{B}$, \pflet{$E(B) = X \setminus \bigcup C(B)$}}
  \step{<2>9}{The union of any subset of $\mathcal{C}$ is closed.}
  \begin{proof}
    \pf\ Lemma \ref{lm:topology:closure:locally_finite_union}, \stepref{<2>5}.
  \end{proof}
  \step{<2>10}{For all $B \in \mathcal{B}$, we have $E(B)$ is open.}
  \begin{proof}
    \pf\ \stepref{<2>7}, \stepref{<2>8}, \stepref{<2>9}.
  \end{proof}
  \step{<2>11}{For all $B \in \mathcal{B}$, we have $B \subseteq E(B)$.}
  \begin{proof}
    \pf\ \stepref{<2>7}, \stepref{<2>8}.
  \end{proof}
  \step{<2>12}{For $B \in \mathcal{B}$, \pick\ $F(B) \in \mathcal{A}$ such that $B \subseteq F(B)$.}
  \begin{proof}
    \pf\ \stepref{<2>3}
  \end{proof}
  \step{<2>13}{\pflet{$\mathcal{D} = \{ E(B) \cap F(B) : B \in \mathcal{B} \}$}}
  \step{<2>14}{$\mathcal{D}$ refines $\mathcal{A}$.}
  \begin{proof}
    \pf\ \stepref{<2>12}, \stepref{<2>13}
  \end{proof}
  \step{<2>15}{$\mathcal{D}$ covers $X$.}
  \begin{proof}
    \step{<3>1}{\pflet{$x \in X$}}
    \step{<3>2}{\pick\ $B \in \mathcal{B}$ such that $x \in B$}
    \begin{proof}
      \pf\ \stepref{<2>3}, \stepref{<3>1}.
    \end{proof}
    \step{<3>3}{$x \in E(B) \cap F(B) \in \mathcal{D}$}
    \begin{proof}
      \pf\ \stepref{<2>11}, \stepref{<2>12}, \stepref{<2>13}, \stepref{<3>2}.
    \end{proof}
  \end{proof}
  \step{<2>16}{$\mathcal{D}$ is locally finite.}
  \begin{proof}
    \step{<3>1}{\pflet{$x \in X$}}
    \step{<3>2}{\pick\ an open neighbourhood $W$ of $x$ that intersects only finitely many elements of $\mathcal{C}$, say $C_1$, \ldots, $C_k$.
    \prove{$W$ intersects only finitely many elements of $\mathcal{D}$.}}
    \begin{proof}
      \pf\ \stepref{<2>5}, \stepref{<3>1}
    \end{proof}
    \step{<3>3}{$W$ is covered by $C_1$, \ldots, $C_k$.}
    \begin{proof}
      \pf\ \stepref{<2>5}, \stepref{<3>2}.
    \end{proof}
    \step{<3>4}{Every element of $\mathcal{C}$ intersects only finitely many elements of $\mathcal{D}$.}
    \begin{proof}
      \step{<4>1}{\pflet{$C \in \mathcal{C}$}}
      \step{<4>2}{If $C$ intersects $E(B) \cap F(B)$ for $B \in \mathcal{B}$ then $C$ intersects $B$}
      \begin{proof}
        \step{<5>1}{\pflet{$x \in C \cap E(B) \cap F(B)$}}
        \step{<5>2}{$C \notin C(B)$}
        \begin{proof}
          \pf\ \stepref{<2>8}, \stepref{<5>1}
        \end{proof}
        \step{<5>3}{$C$ intersects $B$}
        \begin{proof}
          \pf\ \stepref{<2>7}, \stepref{<5>2}
        \end{proof}
      \end{proof}
      \step{<4>3}{$C$ intersects only finitely many elements of $\mathcal{B}$}
      \begin{proof}
        \pf\ \stepref{<2>6}, \stepref{<4>1}
      \end{proof}
      \qedstep
      \begin{proof}
        \pf\ Using \stepref{<2>13}.
      \end{proof}
    \end{proof}
  \end{proof}
  \step{<2>17}{Every element of $\mathcal{D}$ is open.}
  \begin{proof}
    \step{<3>1}{\pflet{$B \in \mathcal{B}$.}}
    \step{<3>2}{$E(B)$ is open.}
    \begin{proof}
      \pf\ \stepref{<2>10}, \stepref{<3>1}.
    \end{proof}
    \step{<3>3}{$F(B)$ is open.}
    \begin{proof}
      \pf\ \stepref{<2>2}, \stepref{<2>12}
    \end{proof}
    \qedstep
    \begin{proof}
      \pf\ Using \stepref{<2>13}.
    \end{proof}
  \end{proof}
\end{proof}
\step{<1>5}{$\ref{four} \Rightarrow \ref{one}$}
\begin{proof}
  \pf\ Trivial.
\end{proof}
\qed
\end{proof}

\begin{cor}
Every regular Lindel\"{o}f space is paracompact.
\end{cor}

\begin{lm}[Shrinking Lemma (AC)]
Let $X$ be a paracompact Hausdorff space. Let $\{ U_\alpha \}_{\alpha in J}$ be a family of open sets that covers $X$. Then there exists a locally finite family $\{ V_\alpha \}_{\alpha \in J}$ of open sets that covers $X$ such that, for all $\alpha \in J$, we have $\overline{V_\alpha} \subseteq U_\alpha$.
\end{lm}

\begin{proof}
\pf
\step{<1>1}{\pflet{$X$ be a paracompact Hausdorff space.}}
\step{<1>2}{\pflet{$\{ U_\alpha \}_{\alpha \in J}$ be a family of open sets that covers $X$.}}
\step{<1>3}{\pflet{$\mathcal{A} = \{ V \text{ open in } X : \exists \alpha \in J. \overline{V} \subseteq U_\alpha \}$.}}
\step{<1>4}{$\mathcal{A}$ covers $X$.}
\begin{proof}
  \step{<2>1}{\pflet{$x \in X$.}}
  \step{<2>2}{\pick\ $\alpha \in J$ such that $x \in U_\alpha$.}
  \begin{proof}
    \pf\ \stepref{<1>2}
  \end{proof}
  \step{<2>3}{\pick\ $V$ open such that $x \in V$ and $\overline{V} \subseteq U_\alpha$}
  \begin{proof}
    \pf\ Theorem \ref{thm:topology:paracompact:Hausdorff_normal}, \stepref{<2>2}.
  \end{proof}
  \step{<2>4}{$x \in V \in \mathcal{A}$}
  \begin{proof}
    \pf\ \stepref{<1>3}, \stepref{<2>3}
  \end{proof}
\end{proof}
\step{<1>5}{\pick\ a locally finite open refinment $\mathcal{B}$ of $\mathcal{A}$ that covers $X$.}
\begin{proof}
  \pf\ \stepref{<1>1}, \stepref{<1>3}, \stepref{<1>4}
\end{proof}
\step{<1>6}{For $B \in \mathcal{B}$ \pick\ $f(B) \in J$ such that $\overline{B} \subseteq U_{f(B)}$}
\begin{proof}
  \step{<2>1}{\pflet{$B \in \mathcal{B}$}}
  \step{<2>2}{\pick\ $V \in \mathcal{A}$ such that $B \subseteq V$}
  \begin{proof}
    \pf\ \stepref{<1>5}, \stepref{<2>1}
  \end{proof}
  \step{<2>3}{\pick\ $\alpha \in J$ such that $\overline{V} \subseteq U_\alpha$.}
  \begin{proof}
    \pf\ \stepref{<1>3}, \stepref{<2>2}
  \end{proof}
  \step{<2>4}{$\overline{B} \subseteq U_\alpha$}
  \begin{proof}
    \pf
    \begin{align*}
      \overline{B} & \subseteq \overline{V} & (\text{Proposition \ref{prop:topology:closure:monotone}, \stepref{<2>2}})\\
      & \subseteq U_\alpha & (\text{\stepref{<2>3}})
    \end{align*}
  \end{proof}
\end{proof}
\step{<1>7}{For $\alpha \in J$ \pflet{$V_\alpha = \bigcup_{f(B) = \alpha} B$}}
\step{<1>8}{For all $\alpha \in J$ we have $\overline{V_\alpha} \subseteq U_\alpha$}
\begin{proof}
  \step{<2>1}{\pflet{$\alpha \in J$}}
  \step{<2>2}{$\overline{V_\alpha} \subseteq U_\alpha$}
  \begin{proof}
    \pf
    \begin{align*}
      \overline{V_\alpha} & = \overline{\bigcup_{f(B) = \alpha} B} & (\text{\stepref{<1>7}}) \\
      & = \bigcup_{f(B) = \alpha} \overline{B} & (\text{Lemma \ref{lm:topology:closure:locally_finite_union}, Lemma \ref{lm:topology:locally_finite:subfamily}, \stepref{<1>5}})\\
      & \subseteq \bigcup_{f(B) = \alpha} U_{f(B)} & (\text{\stepref{<1>6}})\\
      & = U_\alpha
    \end{align*}
  \end{proof}
\end{proof}
\step{<1>9}{$\{ V_\alpha \}_{\alpha \in J}$ is locally finite.}
\begin{proof}
  \step{<2>1}{\pflet{$x \in X$}}
  \step{<2>2}{\pick\ an open neighbourhood $W$ of $x$ that intersects only finitely many elements of $\mathcal{B}$, say $B_1$, \ldots, $B_n$}
  \begin{proof}
    \pf\ \stepref{<1>5}, \stepref{<2>1}
  \end{proof}
  \step{<2>3}{For all $\alpha \in J$, if $W$ intersects $V_\alpha$ then $\alpha$ is one of $f(B_1)$, \ldots, $f(B_n)$.}
  \begin{proof}
    \step{<3>1}{\pflet{$\alpha \in J$}}
    \step{<3>2}{\assume{$W$ intersects $V_\alpha$}}
    \step{<3>3}{\pick\ $y \in W \cap V_\alpha$}
    \begin{proof}
      \pf\ \stepref{<3>2}
    \end{proof}
    \step{<3>4}{\pick\ $B$ such that $f(B) = \alpha$ and $y \in B$}
    \begin{proof}
      \pf\ \stepref{<1>7}, \stepref{<3>3}
    \end{proof}
    \step{<3>5}{$B$ is one of $B_1$, \ldots, $B_n$}
    \begin{proof}
      \pf\ \stepref{<2>2}, \stepref{<3>3}, \stepref{<3>4}
    \end{proof}
  \end{proof}
  \step{<2>4}{$W$ intersects only finitely many $V_\alpha$}
  \begin{proof}
    \pf\ \stepref{<2>3}
  \end{proof}
\end{proof}
\qed
\end{proof}

\begin{thm}
  Let $X$ be a paracompact Hausdorff space. Let $\mathcal{C} \subseteq \mathcal{P} X$ be locally finite. For $C \in \mathcal{C}$ let $\epsilon_C > 0$. Then there exists a continuous function $f : X \rightarrow \mathbb{R}$ such that $f(x) > 0$ for all $x \in X$, and $f(x) \leq \epsilon_C$ for all $C \in \mathcal{C}$ and $x \in C$.
\end{thm}

\begin{proof}
  \pf
  \step{<1>1}{\pflet{$\mathcal{A} = \{ U \text{ open in } X : U \text{ intersects at most finitely many elements of } \mathcal{C} \}$}}
  \step{<1>2}{$\mathcal{A}$ covers $X$.}
  \begin{proof}
    \pf\ Holds since $\mathcal{C}$ is locally finite.
  \end{proof}
  \step{<1>3}{\pick\ a partition of unity $\{ \phi_U \}_{U \in \mathcal{A}}$ dominated by $\{ U \}_{U \in \mathcal{A}}$.}
  \begin{proof}
    \pf\ Theorem \ref{thm:metric:paracompact_Hausdorff:partition_of_unity}, \stepref{<1>1}, \stepref{<1>2}.
  \end{proof}
  \step{<1>4}{For $U \in \mathcal{A}$, \pflet{
  $$ \delta_U = \begin{cases}
  \min \{ \epsilon_C : C \in \mathcal{C}, C \cap \supp\ \phi_U \neq \emptyset \} & \text{if there exists at least one such } C \\
  1 & \text{if not}
\end{cases} $$
  }}
  \step{<1>5}{\pflet{$f(x) = \sum_{U \in \mathcal{A}} \delta_U \phi_U(x)$}}
  \begin{proof}
    %TODO Extract lemma
    \step{<2>1}{For $x \in X$ we have $\phi_U(x) = 0$ for all but finitely many $U$}
    \begin{proof}
      \step{<3>1}{\pflet{$x \in X$}}
      \step{<3>2}{\pick\ an open neighbourhood $W$ of $x$ that intersects $\supp \phi_U$ for only finitely many $U$, say $U_1$, \ldots, $U_n$}
      \begin{proof}
        \pf\ \stepref{<1>3}, \stepref{<3>1}
      \end{proof}
      \step{<3>3}{For all $U \in \mathcal{A}$, if $\phi_U(x) \neq 0$ then $U$ is one of $U_1$, \ldots, $U_n$}
      \begin{proof}
        \step{<4>1}{\pflet{$U \in \mathcal{A}$}}
        \step{<4>2}{\assume{$\phi_U(x) \neq 0$}}
        \step{<4>3}{$x \in \supp \phi_U$}
        \begin{proof}
          \pf\ Proposition \ref{prop:topology:closure:A_sub_Abar}, \stepref{<4>2}.
        \end{proof}
        \step{<4>4}{$U$ is one of $U_1$, \ldots, $U_n$}
        \begin{proof}
          \pf\ \stepref{<3>2}, \stepref{<4>3}
        \end{proof}
      \end{proof}
    \end{proof}
  \end{proof}
  \step{<1>6}{$f(x) > 0$ for all $x \in X$.}
  \begin{proof}
    \step{<2>1}{\pflet{$x \in X$}}
    \step{<2>2}{\pick\ $U \in \mathcal{A}$ such that $\phi_U(x) > 0$}
    \begin{proof}
      \pf\ Such a $U$ exists since $\sum_{U \in \mathcal{A}} \phi_U(x) = 1$ by \stepref{<1>3}.
    \end{proof}
    \step{<2>3}{$\delta_U > 0$}
    \begin{proof}
      \pf\ \stepref{<1>4}
    \end{proof}
    \qedstep
    \begin{proof}
      \pf\ \stepref{<1>5}
    \end{proof}
  \end{proof}
  \step{<1>7}{For $C \in \mathcal{C}$ and $x \in C$ we have $f(x) \leq \epsilon_C$.}
  \begin{proof}
    \step{<2>1}{\pflet{$C \in \mathcal{C}$}}
    \step{<2>2}{\pflet{$x \in C$}}
    \step{<2>3}{For all $U \in \mathcal{A}$ we have $\delta_U \phi_U(x) \leq \epsilon_C \phi_U(x)$}
    \begin{proof}
      \step{<3>1}{\pflet{$U \in \mathcal{A}$} \prove{$\delta_U \phi_U(x) \leq \epsilon_C \phi_U(x)$}}
      \step{<3>2}{\case{$x \in \supp \phi_U$}}
      \begin{proof}
        \pf\ In this case, $\delta_U \leq \epsilon_C$ by \stepref{<1>4}, \stepref{<2>2}.
      \end{proof}
      \step{<3>3}{\case{$x \notin \supp \phi_U$}}
      \begin{proof}
        \pf\ In this case we have $\phi_U(x) = 0$ by Proposition \ref{prop:topology:closure:A_sub_Abar}.
      \end{proof}
    \end{proof}
    \step{<2>4}{$f(x) \leq \epsilon_C$}
    \begin{proof}
      \pf
      \begin{align*}
        f(x) & = \sum_{U \in \mathcal{A}} \delta_U \phi_U(x) & (\text{\stepref{<1>5}})\\
        & \leq \sum_{U \in \mathcal{A}} \epsilon_C \phi_U(x) & (\text{\stepref{<2>3}})\\
        & = \epsilon_C \sum_{U \in \mathcal{A}} \phi_U(x) \\
        & = \epsilon_C & (\text{\stepref{<1>3}})
      \end{align*}
    \end{proof}
  \end{proof}
  \qed
\end{proof}

\begin{lm}[Expansion Lemma]
  Let $\{ B_\alpha \}_{\alpha \in J}$ be a locally finite family of subsets of the paracompact Hausdorff space $X$. Then there exists a locally finite family $\{ U_\alpha \}_{\alpha \in J}$ of open sets such that $B_\alpha \subseteq U_\alpha$ for all $\alpha \in J$.
\end{lm}

\begin{proof}
  \pf
  \step{<1>1}{\pflet{$X$ be a paracompact Hausdorff space.}}
  \step{<1>2}{\pflet{$\{ B_\alpha \}_{\alpha \in J}$ be locally finite}}
  \step{<1>3}{\pflet{$\mathcal{A} = \{ U \text{ open in } X : U \text{ intersects } B_\alpha \text{ for only finitely many } \alpha \}$}}
  \step{<1>4}{\pick\ a locally finite open refinement $\mathcal{B}$ of $\mathcal{A}$ that covers $X$.}
  \begin{proof}
    \step{<2>1}{Every element of $\mathcal{A}$ is open.}
    \begin{proof}
      \pf\ From \stepref{<1>3}.
    \end{proof}
    \step{<2>2}{$\mathcal{A}$ covers $X$}
    \begin{proof}
      \pf\ From \stepref{<1>2}, \stepref{<1>3}.
    \end{proof}
    \qedstep
    \begin{proof}
      \pf\ From \stepref{<1>1}.
    \end{proof}
  \end{proof}
  \step{<1>5}{For $\alpha \in J$, \pflet{$U_\alpha = \bigcup \{ V \in \mathcal{B} : V \cap B_\alpha \neq \emptyset \}$}}
  \step{<1>6}{$\{ U_\alpha \}_{\alpha \in J}$ is locally finite.}
  \begin{proof}
    \step{<2>1}{Every element of $\mathcal{B}$ intersects $B_\alpha$ for only finitely many $\alpha$.}
    \begin{proof}
      \step{<3>1}{\pflet{$V \in \mathcal{B}$}}
      \step{<3>2}{\pick\ $U \in \mathcal{A}$ such that $U \subseteq V$}
      \begin{proof}
        \pf\ \stepref{<1>4}, \stepref{<3>1}
      \end{proof}
      \step{<3>3}{$U$ intersects $B_\alpha$ for only finitely many $\alpha$}
      \begin{proof}
        \pf\ \stepref{<1>3}, \stepref{<3>2}
      \end{proof}
      \step{<3>4}{$V$ intersects $B_\alpha$ for only finitely many $\alpha$}
      \begin{proof}
        \pf\ \stepref{<3>2}, \stepref{<3>3}
      \end{proof}
    \end{proof}
    \step{<2>2}{\pflet{$x \in X$}}
    \step{<2>3}{\pick\ an open neighbourhood $W$ of $x$ that intersects only finitely many elements of $\mathcal{B}$, say $V_1$, \ldots, $V_n$.}
    \begin{proof}
      \pf\ \stepref{<1>4}, \stepref{<2>2}
    \end{proof}
    \step{<2>4}{For $1 \leq i \leq n$, \pflet{$\alpha_{i1}$, \ldots, $\alpha_{ir_i}$ be the finitely many values of $\alpha$ such that $V_i$ intersects $B_\alpha$}
    \prove{If $W$ intersects $B_\alpha$ then $\alpha = \alpha_{ij}$ for some $i$, $j$}}
    \begin{proof}
      \pf\ \stepref{<2>1}, \stepref{<2>3}.
    \end{proof}
    \step{<2>5}{\pflet{$y \in W \cap B_\alpha$}}
    \step{<2>6}{\pick\ $V \in \mathcal{B}$ such that $y \in V$}
    \begin{proof}
      \pf\ \stepref{<1>4}
    \end{proof}
    \step{<2>7}{\pflet{$V = V_i$}}
    \begin{proof}
      \pf\ \stepref{<2>3}, \stepref{<2>5}, \stepref{<2>6}
    \end{proof}
    \step{<2>8}{$V_i$ intersects $B_\alpha$}
    \begin{proof}
      \pf\ \stepref{<2>5}, \stepref{<2>6}, \stepref{<2>7}
    \end{proof}
    \step{<2>9}{$\alpha = \alpha_{ij}$ for some $j$.}
    \begin{proof}
      \pf\ \stepref{<2>4}, \stepref{<2>8}
    \end{proof}
  \end{proof}
  \step{<1>7}{For all $\alpha \in J$, we have $U_\alpha$ is open.}
  \begin{proof}
    \pf\ \stepref{<1>5}
  \end{proof}
  \step{<1>8}{For all $\alpha \in J$, we have $B_\alpha \subseteq U_\alpha$.}
  \begin{proof}
    \step{<2>1}{\pflet{$\alpha \in J$}}
    \step{<2>2}{\pflet{$x \in B_\alpha$}}
    \step{<2>3}{\pick\ $V \in \mathcal{B}$ such that $x \in V$}
    \begin{proof}
      \pf\ \stepref{<1>4}
    \end{proof}
    \step{<2>4}{$V \cap B_\alpha \neq \emptyset$}
    \begin{proof}
      \pf\ \stepref{<2>2}, \stepref{<2>3}
    \end{proof}
    \step{<2>5}{$x \in U_\alpha$}
    \begin{proof}
      \pf\ \stepref{<1>5}, \stepref{<2>3}, \stepref{<2>4}
    \end{proof}
  \end{proof}
  \qed
\end{proof}

\section{Compactness}

\begin{df}[Compact]
  A topological space is \emph{compact} iff every open cover has a finite
  subcover.
\end{df}

 \begin{prop}
   \label{prop:topology:compact:S_omega}
$S_\Omega$ is not compact.
\end{prop}

\begin{proof}
 \pf\ The open covering $\{ (- \infty, \alpha) : \alpha \in S_\Omega \}$ has
 no finite subcovering. \qed
\end{proof}

\begin{prop}
 $\mathbb{R}_l$ is not compact.
\end{prop}

\begin{proof}
 \pf\ $\{ [n, n+1) : n \in \mathbb{Z} \}$ has no finite subcover. \qed
\end{proof}

\begin{prop}
 The space $\mathbb{R}^\omega$ under the box topology is not compact.
\end{prop}

\begin{proof}
 \pf\ The set $\{ \prod_{n=0}^\infty (a_n, a_n+1) : n \in \mathbb{Z}
\}$ is a cover that has no finite subcover. \qed
\end{proof}

\begin{prop}
  \label{prop:topology:compact:subspace}
  Let $Y$ be a subspace of $X$. Then $Y$ is compact if and only if every
  covering of $Y$ by sets open in $X$ contains a finite subcollection
  covering
  $Y$.
\end{prop}

\begin{proof}
  \pf
  \step{<1>1}{If $Y$ is compact then every covering of $Y$ by sets open in
    $X$
    contains a finite subcollection covering $Y$.}
  \begin{proof}
    \step{<2>1}{\assume{$Y$ is compact.}}
    \step{<2>2}{\pflet{$\mathcal{A}$ be a covering of $Y$ by sets open in
        $X$.}}
    \step{<2>3}{$\{ U \cap Y : U \in \mathcal{A} \}$ is an open covering of
      $Y$.}
    \step{<2>4}{\pick\ a finite subcovering $V_1$, \ldots, $V_n$ of $\{ U
      \cap
      Y
      : U \in \mathcal{A} \}$}
    \step{<2>5}{For $1 \leq i \leq n$, \pick\ $U_i \in \mathcal{A}$ such that
      $V_i = U_i \cap Y$.}
    \step{<2>6}{$\{ U_1, \ldots, U_n \}$ is a finite subset of $\mathcal{A}$
      that
      covers $Y$.}
  \end{proof}
  \step{<1>2}{If every covering of $Y$ by sets open in $X$ contains a finite
    subcollection covering $Y$ then $Y$ is compact.}
  \begin{proof}
    \step{<2>1}{\assume{Every covering of $Y$ by sets open in $X$ contains a
        finite subcollection covering $Y$.}}
    \step{<2>2}{\pflet{$\mathcal{A}$ be an open covering of $Y$}}
    \step{<2>3}{\pflet{$\mathcal{B} = \{ U \text{ open in } X : U \cap Y \in
        \mathcal{A} \}$}}
    \step{<2>4}{$\mathcal{B}$ covers $Y$}
    \step{<2>5}{\pick\ a finite subcollection $\{ U_1, \ldots, U_n \}
      \subseteq
      \mathcal{B}$ that covers $Y$}
    \step{<2>6}{$\{ U_1 \cap Y, \ldots, U_n \cap Y \}$ is a finite subcover
      of
      $\mathcal{A}$.}
  \end{proof}
  \qed
\end{proof}

\begin{prop}
  \label{prop:topology:compact:closed_is_compact}
  Every closed subspace of a compact space is compact.
\end{prop}

\begin{proof}
  \pf
  \step{<1>1}{\pflet{$X$ be a compact space and $Y \subseteq X$ be closed.}}
  \step{<1>2}{\pflet{$\mathcal{A}$ be a covering of $Y$ by spaces open in
      $X$}}
  \step{<1>3}{$\mathcal{A} \cup \{ X \setminus Y \}$ is an open covering of
    $X$.}
  \step{<1>4}{\pick\ a finite subcovering $\{ U_1, \ldots, U_n \}$ or $\{
    U_1,
    \ldots, U_n, X \setminus Y \}$}
  \step{<1>5}{$\{ U_1, \ldots, U_n \}$ is a finite subset of $\mathcal{A}$
    that
    covers $Y$.}
  \qedstep
  \begin{proof}
    \pf\ Proposition \ref{prop:topology:compact:subspace}.
  \end{proof}
  \qed
\end{proof}

\begin{cor}
Not every compact Hausdorff space is connected.
\end{cor}

\begin{proof}
\pf\ The space $[0,1] \cup [2,3]$ is compact Hausdorff and disconnected. \qed
\end{proof}

\begin{cor}
Not every compact Hausdorff space is path connected.
\end{cor}

\begin{cor}
Not every compact Hausdorff space is locally connected.
\end{cor}

\begin{proof}
The space $[0,1] \cap \mathbb{Q}$ is not locally connected.
\end{proof}

\begin{cor}
Not every compact Hausdorff space is locally path connected.
\end{cor}

\begin{prop}
Not every open subspace of a compact space is compact.
\end{prop}

\begin{proof}
\pf\ The space $[0,1]$ is compact but $(0,1)$ is not. \qed
\end{proof}

\begin{lm}
  \label{lm:topology:compact:regular}
  If $Y$ is a compact subspace of the Hausdorff space $X$ and $a \notin Y$,
  then there exist disjoint open sets $U$ and $V$ of $X$ containing $a$ and
  $Y$,
  respectively.
\end{lm}

\begin{proof}
  \pf
  \step{<1>1}{For $y \in Y$, there exist disjoint open sets $U$ and $V$ such
    that
    $a \in U$ and $y \in V$.}
  \step{<1>2}{$\{ V \text{ open in } X : \exists U \text{ open and disjoint
      from
    } V. a \in U \}$ is a covering of $Y$ by open sets in $X$.}
  \step{<1>3}{\pick\ a finite subset $\{ V_1, \ldots, V_n \}$ that covers
    $Y$.}
  \step{<1>4}{For $1 \leq i \leq n$, \pick $U_i$ disjoint from $V_i$ such
    that
    $a
    \in U_i$}
  \step{<1>5}{\pflet{$U = U_1 \cap \cdots \cap U_n$ and $V = V_1 \cup \cdots
      \cup
      V_n$}}
  \qed
\end{proof}

\begin{prop}
  \label{prop:topology:compact:compact_is_closed}
  Every compact subspace of a Hausdorff space is closed.
\end{prop}

\begin{proof}
  \pf
  \step{<1>1}{\pflet{$X$ be a Hausdorff space and $Y \subseteq X$ be
      compact.}}
  \step{<1>2}{Every point $a \notin Y$ has an open neighbourhood disjoint
    from
    $Y$.}
  \begin{proof}
    \pf\ By Lemma \ref{lm:topology:compact:regular}.
  \end{proof}
  \qedstep
  \begin{proof}
    \pf\ By Proposition \ref{prop:topology:neighbourhood:open}.
  \end{proof}
\end{proof}


\begin{prop}
  \label{prop:topology:compact:image}
  The image of a compact space under a continuous map is compact.
\end{prop}

\begin{proof}
  \pf
  \step{<1>1}{\pflet{$f : X \rightarrow Y$ be continuous where $X$ is
      compact.}}
  \step{<1>2}{\pflet{$\mathcal{A}$ be a covering of $f(X)$ by open sets in
      $Y$.}}
  \step{<1>3}{$\{ f^{-1}(U) : U \in \mathcal{A} \}$ is an open covering of
    $X$.}
  \step{<1>4}{\pick\ a finite subcovering $\{ f^{-1}(U_1), \ldots,
    f^{-1}(U_n)
    \}$}
  \step{<1>5}{$\{ U_1, \ldots, U_n \}$ is a finite subset of $\mathcal{A}$
    that
    covers $f(X)$.}
  \qedstep
  \begin{proof}
    \pf\ By Proposition \ref{prop:topology:compact:subspace}.
  \end{proof}
  \qed
\end{proof}

\begin{cor}
  Let $\{ X_\alpha \}_{\alpha \in J}$ be a family of topological spaces. If
  $\prod_{\alpha \in J} X_\alpha$ is compact then each $X_\alpha$ is compact.
\end{cor}

\begin{cor}
  $S_\Omega \times \overline{S_\Omega}$ is compact.
\end{cor}

\begin{cor}
 The Sorgenfrey plane is not compact.
\end{cor}

\begin{cor}
  For any nonempty set $I$,
  the sapce $\mathbb{R}^I$ is not compact.
\end{cor}

\begin{cor}
  Let $\mathcal{T}$ and $\mathcal{T}'$ be topologies on the same
set $X$. If $\mathcal{T} \subseteq \mathcal{T}'$ and
$\mathcal{T}'$ is compact then $\mathcal{T}$ is compact.
\end{cor}

      \begin{cor}
        The space $\mathbb{R}_K$ is not compact.
      \end{cor}

\begin{thm}
  \label{thm:topology:compact:homeomorphism}
  Let $f : X \rightarrow Y$ be a bijective continuous function. If $X$ is
  compact and $Y$ is Hausdorff then $f$ is a homeomorphism.
\end{thm}

\begin{proof}
  \pf
  \step{<1>1}{\pflet{$C$ be closed in $X$}}
  \step{<1>2}{$C$ is compact}
  \begin{proof}
    \pf\ Proposition \ref{prop:topology:compact:closed_is_compact}.
  \end{proof}
  \step{<1>3}{$f(C)$ is compact}
  \begin{proof}
    \pf\ Proposition \ref{prop:topology:compact:image}
  \end{proof}
  \step{<1>4}{$f(C)$ is closed}
  \begin{proof}
    \pf\ Proposition \ref{prop:topology:compact:compact_is_closed}.
  \end{proof}
  \qedstep
  \begin{proof}
    \pf\ By Theorem
    \ref{thm:topology:continuous:characterisation}
    we have that      $f^{-1}$ is continuous.
  \end{proof}
  \qed
\end{proof}

\begin{cor}
  \label{cor:topology:compact_hausdorff:finer_coarser}
 Let $\mathcal{T}$ and $\mathcal{T}'$ be topologies on the same set $X$. If $\mathcal{T} \subseteq \mathcal{T}'$, $\mathcal{T}$ is Hausdorff and $\mathcal{T}'$ is compact then $\mathcal{T} = \mathcal{T}'$.
\end{cor}

\begin{cor}
  \label{cor:topology:compact_hausdorff:01K}
  The space $[0,1]$ is not compact as a subspace of $\mathbb{R}_K$.
\end{cor}

\begin{thm}[Tube Lemma]
  Let $A$ and $B$ be subspaces of $X$ and $Y$, respectively; let $N$ be an
  open set in $X \times Y$ including $A \times B$. If $A$ and $B$ are
  compact,
  then there exist open sets $U$ and $V$ in $X$ and $Y$, respectively, such
  that
  \[ A \times B \subseteq U \times V \subseteq N \enspace . \]
\end{thm}

\begin{proof}
  \pf
  \step{<1>1}{For all $a \in A$, there exist open sets $U$ and $V$ in $X$ and
    $Y$, respectively, such that
    \[ \{ a \} \times B \subseteq U \times V \subseteq N \enspace . \]}
  \begin{proof}
    \step{<2>1}{\pflet{$a \in A$}}
    \step{<2>2}{For all $b \in B$, there exist open sets $U$ and $V$ in $X$
      and
      $Y$, respectively, such that $(a, b) \in U \times V \subseteq N$.}
    \step{<2>3}{$\{ V \text{ open in } Y : \exists U \text{ open in } X. a
      \in
      U,
      U \times V \subseteq N \}$ covers $B$}
    \step{<2>4}{\pick\ a finite subset $\{ V_1, \ldots, V_n \}$ that covers
      $B$.}
    \step{<2>5}{For $1 \leq i \leq n$, \pick\ $U_i$ open in $X$ such that $a
      \in
      U_i$ and        $U_i \times V_i \subseteq N$}
    \step{<2>6}{\pflet{$U = U_1 \cap \cdots \cap U_n$ and $V = V_1 \cup
        \cdots
        \cup V_n$}}
  \end{proof}
  \step{<1>2}{$\{ U \text{ open in } X : \exists V \text{ open in } Y. B
    \subseteq V \text{ and } U \times V \subseteq N \}$ covers
    $A$.}
  \step{<1>3}{\pick\ a finite subset $\{ U_1, \ldots, U_n \}$ that covers
    $A$.}
  \step{<1>4}{For $1 \leq i \leq n$, \pick\ $V_i$ open in $B$ such that $B
    \subseteq V_i$ and $U_i \times V_i \subseteq N$.}
  \step{<1>5}{\pflet{$U = U_1 \cup \cdots \cup U_n$ and $V = V_1 \cap \cdots
      \cap
      V_n$}}
  \step{<1>6}{$A \times B \subseteq U \times V \subseteq N$}
  \qed
\end{proof}

\begin{lm}
  \label{lm:topology:compact:tube_lemma2}
  Let $\mathcal{A}$ be a set of basis elements for $X \times Y$ such that no finite subset of $\mathcal{A}$ covers $X \times Y$.
  If $X$ is compact, then there exists a point $x \in X$ such that no finite subset of $\mathcal{A}$ covers $\{x\} \times Y$.
\end{lm}

\begin{proof}
  \pf
  \step{<1>1}{\assume{$X$ is compact.}}
  \step{<1>2}{\assume{For all $x \in X$, there is a finite subset of $\mathcal{A}$ that covers $\{x\} \times Y$}
  \prove{A finite subset of $\mathcal{A}$ covers $X \times Y$}}
  \step{<1>3}{$\{ U \text{ open in } X : \exists U_1 \times V_1, \ldots, U_r \times V_r \in \mathcal{A}. U = U_1 \cap \cdots \cap U_r, Y = V_1 \cup \cdots \cup V_r \}$ covers $X$.}
  \step{<1>4}{\pick\ a finite subcover $\{ U_1, \ldots, U_n \}$}
  \step{<1>5}{For $1 \leq i \leq n$, \pick $U_{i1} \times V_{i1}, \ldots, U_{ir_i} \times V_{ir_i} \in \mathcal{A}$ such that $U_i = U_{i1} \cap \cdots \cap U_{ir_i}$ and $Y = V_{i1} \cup \cdots \cup V_{ir_i}$}
  \step{<1>6}{$\{ U_{ij} : 1 \leq i \leq n, 1 \leq j \leq r_i \}$ covers $X \times Y$}
  \qed
\end{proof}

\begin{prop}
  \label{prop:topology:compact:product}
  The product of two compact spaces is compact.
\end{prop}

\begin{proof}
  \pf
  \step{<1>1}{\pflet{$X$ and $Y$ be compact spaces.}}
  \step{<1>2}{\pflet{$\mathcal{A}$ be an open covering of $X \times Y$}}
  \step{<1>3}{For all $x \in X$, there exists a neighbourhood $W$ of $x$ such
    that $W \times Y$ is      covered by finitely many elements of
    $\mathcal{A}$.}
  \begin{proof}
    \step{<2>1}{\pflet{$x \in X$}}
    \step{<2>2}{$\{x\} \times Y$ is compact.}
    \begin{proof}
      \pf\ It is homeomorphic to $Y$.
    \end{proof}
    \step{<2>3}{\pick\ a finite subset $\{ U_1, \ldots, U_m \}$ of
      $\mathcal{A}$
      that covers $\{x\} \times Y$}
    \begin{proof}
      \pf\ By Proposition \ref{prop:topology:compact:subspace}.
    \end{proof}
    \step{<2>4}{There exists a neighbourhood $W$ of $x$ such that $W \times Y
      \subseteq U_1 \cup \cdots \cup U_m$}
    \begin{proof}
      \pf\ By the Tube Lemma.
    \end{proof}
  \end{proof}
  \step{<1>4}{$\{ W \text{ open in } X : W \times Y \text{ is covered by
      finitely
      many        elements of } \mathcal{A} \}$ is an open covering of $X$.}
  \step{<1>5}{\pick\ a finite subcovering $\{ W_1, \ldots, W_n \}$}
  \step{<1>6}{For $1 \leq i \leq n$, \pick\ a finite subset $\{ U_{i1},
    \ldots,
    U_{ir_i} \}$ of $\mathcal{A}$ that covers $W_i \times Y$}
  \step{<1>7}{$\{ U_{11}, \ldots, U_{nr_n} \}$ is a finite subcovering of
    $\mathcal{A}$.}
  \qed
\end{proof}

\begin{prop}
  \label{prop:topology:compact:finite_intersection}
  A topological space is compact if and only if every nonempty set of closed
  sets that has the finite intersection property has nonempty intersection.
\end{prop}

\begin{proof}
  \pf\ Immediate from definitions. \qed
\end{proof}

\begin{lm}
  \label{lm:topology:compact:projection_closed}
  If $Y$ is compact then $\pi_1 : X \times Y \rightarrow X$ is a closed map.
\end{lm}

\begin{proof}
  \pf
  \step{<1>1}{\pflet{$C \subseteq X \times Y$ be closed}}
  \step{<1>2}{\pflet{$x \in X \setminus \pi_1(C)$}}
  \step{<1>3}{For all $y \in Y$, we have $(x, y) \notin C$}
  \step{<1>4}{For all $y \in Y$, there exist open neighbourhoods $U$ of $x$
    and
    $V$ of $y$ such that $U \times V \subseteq (X \times Y) \setminus C$}
  \step{<1>5}{$\{ V \text{ open in } Y : \exists U \text{ an open
      neighbourhood
      of } x \text{ such that } U \times V \subseteq (X \times Y) \setminus C
    \}$ is an open covering of $Y$.}
  \step{<1>6}{\pick\ a finite subcovering $\{ V_1, \ldots, V_n \}$}
  \step{<1>7}{For $1 \leq i \leq n$, \pick\ an open neighbourhood $U_i$ of
    $x$
    such that $U_i \times V_i \subseteq (X \times Y) \setminus C$}
  \step{<1>8}{$x \in U_1 \cap \cdots \cap U_n \subseteq X \setminus \pi_1(C)$}
  \qed
\end{proof}

\begin{thm}
  Let $X$ be a compact space.
  Let $f_n : X \rightarrow \mathbb{R}$ be a sequence of continuous functions
  such that, for all $x \in X$, $f_n(x) \rightarrow f(x)$ as $n \rightarrow
  \infty$. If $f$ is continuous, and if the sequence $(f_n)_n$ is monotone
  increasing, and if $X$ is compact, then the convergence is uniform.
\end{thm}

\begin{proof}
  \pf
  \step{<1>1}{\pflet{$\epsilon > 0$} \prove{There exists $N$ such that, for
      all
      $n \geq N$, we have $|f_n(x) - f(x)| < \epsilon$}}
  \step{<1>2}{For $n \in \mathbb{Z}^+$, \pflet{$U_n = \{ x \in X : f(x) -
      f_n(x)
      <        \epsilon \}$}}
  \step{<1>3}{Each $U_n$ is open}
  \begin{proof}
    \pf\ Let $g(x) = f(x) - f_n(x)$. Then $g$ is continuous and $U_n =
    g^{-1}((- \infty, \epsilon))$.
  \end{proof}
  \step{<1>4}{$\{ U_n : n \geq 1 \}$ is an open covering of $X$}
  \begin{proof}
    \step{<2>1}{\pflet{$x \in X$}}
    \step{<2>2}{\pick\ $N$ such that, for all $n \geq N$, $|f(x) - f_n(x)| <
      \epsilon$}
    \begin{proof}
      \pf\ $f_n(x) \rightarrow f(x)$ as $n \rightarrow \infty$
    \end{proof}
    \step{<2>3}{$f(x) - f_N(x) < \epsilon$}
    \begin{proof}
      \pf\ This holds since the sequece $(f_n)_n$ is monotone.
    \end{proof}
  \end{proof}
  \step{<1>5}{\pick\ a finite subcovering $\{ U_{n_1}, \ldots, U_{n_k} \}$}
  \step{<1>6}{\pflet{$N = \max(n_1, \ldots, n_k)$}}
  \step{<1>7}{For all $n \geq N$ we have $|f_n(x) - f(x)| < \epsilon$}
  \qed
\end{proof}

\begin{lm}
  \label{lm:topology:compact:normal}
  Every compact Hausdorff space is normal.
\end{lm}

\begin{proof}
  \pf
  From Thearem \ref{thm:topology:paracompact:Hausdorff_normal}
\end{proof}

\begin{cor}
 The ordered square is normal.
\end{cor}

\begin{thm}
  \label{thm:topology:compact:closed_interval}
  Let $X$ be a complete linearly ordered set under the order topology. Then
  every closed interval in $X$ is compact.
\end{thm}

\begin{proof}
  \pf
  \step{<1>1}{\pflet{$X$ be a complete linearly ordered set in the order
      topology}}
  \step{<1>2}{\pflet{$a, b \in X$, $a < b$} \prove{$[a, b]$ is compact}}
  \step{<1>3}{\pflet{$\mathcal{A}$ be a set of open sets that covers $[a,b]$}}
  \step{<1>4}{For all $x \in [a, b)$, there exists $y \in (x, b]$ such that
    $[x,
    y]$ is covered by at most two points of $\mathcal{A}$}
  \begin{proof}
    \step{<2>1}{\pflet{$x \in [a,b]$}}
    \step{<2>2}{\pick\ $U \in \mathcal{A}$ such that $x \in U$}
    \begin{proof}
      \pf\ By \stepref{<1>3} and \stepref{<2>1}
    \end{proof}
    \step{<2>3}{\pick\ $y \in (x, b]$ such that $[x, y) \subseteq U$}
    \begin{proof}
      \pf\ By Lemma \ref{lm:topology:order:open}.
    \end{proof}
    \step{<2>4}{\pick\ $V \in \mathcal{A}$ such that $y \in V$}
    \begin{proof}
      \pf\ By \stepref{<1>3} and \stepref{<2>3}.
    \end{proof}
    \step{<2>5}{$[x, y]$ is covered by $\{ U, V \}$}
    \begin{proof}
      \pf\ By \stepref{<2>3} and \stepref{<2>4}.
    \end{proof}
  \end{proof}
  \step{<1>5}{\pflet{$C = \{ y \in (a, b] : [a,y] \text{ is covered by a
        finite
        subset of } \mathcal{A} \}$}}
  \step{<1>6}{$C$ is nonempty}
  \begin{proof}
    \pf\ By \stepref{<1>4}.
  \end{proof}
  \step{<1>7}{\pflet{$c = \sup C$}}
  \begin{proof}
    \pf\ By \stepref{<1>1}.
  \end{proof}
  \step{<1>8}{$c \in C$}
  \begin{proof}
    \step{<2>1}{\pick\ $U \in \mathcal{A}$ such that $c \in U$}
    \step{<2>2}{\pick\ $y \in [a, c)$ such that $(y, c] \subseteq U$}
    \begin{proof}
      \pf\ By Lemma \ref{lm:topology:order:open}
    \end{proof}
    \step{<2>3}{\pick\ $z$ such that $y < z$ and $z \in C$}
    \begin{proof}
      \pf\ This exists because $y$ is not an upper bound for $C$.
    \end{proof}
    \step{<2>4}{\pick\ a finite $\mathcal{A}_0 \subseteq \mathcal{A}$ such
      that
      $[a, z]$ is covered by $\mathcal{A}_0$}
    \step{<2>5}{$[a, c]$ is covered by $\mathcal{A}_0 \cup \{ U \}$}
  \end{proof}
  \step{<1>9}{$c = b$}
  \begin{proof}
    \step{<2>1}{\assume{for a contradiction $c < b$}}
    \step{<2>2}{\pick\ $y \in (c, b]$ such that $[c, y]$ is covered by at
      most
      two elements of $\mathcal{A}$.}
    \begin{proof}
      \pf\ By \stepref{<1>4}
    \end{proof}
    \step{<2>3}{$y > c$ and $y \in C$}
    \qedstep
    \begin{proof}
      \pf\ This contradicts \stepref{<1>7}.
    \end{proof}
  \end{proof}
  \qedstep
\end{proof}

\begin{cor}
  \label{cor:topology:compact:real_closed_interval}
  Every closed interval in $\mathbb{R}$ is compact.
\end{cor}

 \begin{cor}[CC]
   \label{cor:topology:limit_point_compact:S_omega}
$S_\Omega$ is limit point compact.
\end{cor}

\begin{proof}
\pf
\step{<1>1}{\pflet{$A$ be an infinite subset of $S_\Omega$}}
\step{<1>2}{\pick\ a countably infinite subset $B \subseteq A$}
\step{<1>3}{\pflet{$b = \sup B$}}
\step{<1>4}{$B \subseteq [0, b]$}
\step{<1>5}{$[0, b]$ is compact}
\begin{proof}
  \pf\ By the theorem.
\end{proof}
\step{<1>6}{$B$ has a limit point in $[0,b]$}
\step{<1>7}{$A$ has a limit point in $[0,b]$}
\qed
\end{proof}

\begin{cor}
The ordered square is compact.
\end{cor}

\begin{cor}
The ordered square is limit point compact.
\end{cor}

\begin{cor}
Not every subspace of a compact space is compact.
\end{cor}

\begin{proof}
\pf\ $[0,1]$ is compact but $(0,1)$ is not. \qed
\end{proof}

\begin{thm}[Extreme Value Theorem]
  Let $f : X \rightarrow Y$ be continuous where $Y$ is a linearly ordered set
  in the order topology. If $X$ is compact, then there exist $c, d \in X$
  such
  that, for all $x \in X$, we have $f(c) \leq f(x) \leq f(d)$.
\end{thm}

\begin{proof}
  \pf
  \step{<1>1}{$f(X)$ is compact.}
  \begin{proof}
    \pf\ By Proposition \ref{prop:topology:compact:image}.
  \end{proof}
  \step{<1>2}{$f(X)$ has a greatest element.}
  \begin{proof}
    \step{<2>1}{\assume{for a contradiction $f(X)$ has no greatest element.}}
    \step{<2>2}{$\{ (- \infty, f(x)) : x \in X \}$ is a set of open sets
      that covers $f(X)$.}
    \step{<2>3}{\pick\ a finite subset $\{ (- \infty, f(x_1)), \ldots, (-
      \infty,
      f(x_n)) \}$ that covers $f(X)$.}
    \begin{proof}
      \pf\ By Proposition \ref{prop:topology:compact:subspace}
    \end{proof}
    \step{<2>4}{\pflet{$f(x_N)$ be largest out of $f(x_1)$, \ldots, $f(x_n)$}}
    \step{<2>5}{$f(x_N) < f(x_N)$}
    \qedstep
    \begin{proof}
      \pf\ This is a contradiction.
    \end{proof}
  \end{proof}
  \step{<1>3}{$f(X)$ has a least element.}
  \begin{proof}
    \pf\ Similar.
  \end{proof}
  \qed
\end{proof}

\begin{thm}[DC]
  A nonempty compact Hausdorff space with no isolated points is uncountable.
\end{thm}

\begin{proof}
  \pf
  \step{<1>1}{\pflet{$X$ be a nonempty compact Hausdorff space with no
      isolated
      points.}}
  \step{<1>2}{For every nonempty open $U \subseteq X$ and point $x \in X$,
    there
    exists a nonempty open $V \subseteq U$ such that $x \notin \overline{V}$}
  \begin{proof}
    \step{<2>1}{\pflet{$U \subseteq X$ be nonempty and open and $x \in X$}}
    \step{<2>2}{\pick\ $y \in U$ such that $y \neq x$}
    \begin{proof}
      \pf\ This is possible because $U \neq \{ x \}$ since $x$ is not an
      isolated point.
    \end{proof}
    \step{<2>3}{\pick\ disjoint open neighbourhoods $W_1$ and $W_2$ of $x$
      and
      $y$}
    \begin{proof}
      \pf\ Since $X$ is Hausdorff
    \end{proof}
    \step{<2>4}{\pflet{$V = U \cap W_2$}}
    \step{<2>5}{$x \notin \overline{V}$}
    \begin{proof}
      \pf\ We have $\overline{V} \subseteq \overline{W_2} \subseteq X
      \setminus W_1$.
    \end{proof}
  \end{proof}
  \step{<1>3}{\pflet{$f : \mathbb{Z}^+ \rightarrow X$} \prove{$f$ is not
      surjective}}
  \step{<1>4}{\pick\ a sequence of open sets $V_1 \supseteq V_2 \supseteq
    \cdots$
    such that $f(n) \notin \overline{V_n}$}
  \begin{proof}
    \pf\ By \stepref{<1>2} and Dependent Choice.
  \end{proof}
  \step{<1>5}{\pick\ a point $b \in \bigcap_{i=1}^\infty \overline{V_i}$}
  \begin{proof}
    \pf\ By Proposition \ref{prop:topology:compact:finite_intersection}.
  \end{proof}
  \step{<1>6}{$b \neq f(n)$ for all $n$}
  \begin{proof}
    \pf\ For each $n$ we have $b \in \overline{V_n}$ (\stepref{<1>5}) and
    $f(n) \notin      \overline{V_n}$ (\stepref{<1>4}).
  \end{proof}
  \qed
\end{proof}

\begin{cor}
  Every closed interval in $\mathbb{R}$ is uncountable.
\end{cor}

\begin{thm}
  \label{thm:topology:compact:limit_point_compact}
  Every compact space is limit point compact.
\end{thm}

\begin{proof}
  \pf
  \step{<1>1}{\pflet{$X$ be a compact space.}}
  \step{<1>2}{\pflet{$A \subseteq X$ be a set with no limit points.}
    \prove{$A$
      is finite.}}
  \step{<1>3}{$A$ is closed.}
  \begin{proof}
    \pf\ By Corollary \ref{cor:topology:limit_point:closed}.
  \end{proof}
  \step{<1>4}{$A$ is compact.}
  \begin{proof}
    \pf\ By Proposition \ref{prop:topology:compact:closed_is_compact}.
  \end{proof}
  \step{<1>5}{$\{ U \text{ open in } X : U \cap A \text{ is a singleton} \}$
    covers $A$}
  \begin{proof}
    \step{<2>1}{\pflet{$a \in A$}}
    \step{<2>2}{\pick\ an open neighbourhood $U$ of $a$ such that $U$ does
      not
      intersect $A$ at a point other than $a$}
    \begin{proof}
      \pf\ One must exist because $a$ is not a limit point of $A$
      (\stepref{<1>2}).
    \end{proof}
    \step{<2>3}{$U \cap A = \{ a \}$}
  \end{proof}
  \step{<1>6}{\pick\ a finite subcover $\{ U_1, \ldots, U_n \}$}
  \begin{proof}
    \pf\ By \stepref{<1>4} using Proposition
    \ref{prop:topology:compact:subspace}.
  \end{proof}
  \step{<1>7}{For $1 \leq i \leq n$, \pflet{$U_i \cap A = \{ a_i \}$}}
  \step{<1>8}{$A = \{ a_1, \ldots, a_n \}$}
  \qed
\end{proof}

\begin{prop}
  \label{prop:topology:compact:union}
  Let $X$ be a space and $C, D \subseteq X$ be compact. Then $C \cup D$ is
  compact.
\end{prop}

\begin{proof}
  \pf
  \step{<1>1}{\pflet{$\mathcal{A}$ be a set of open sets that covers $C \cup
      D$}}
  \step{<1>2}{\pick\ a finite subset $\mathcal{A}_1$ that covers $C$ and a
    finite
    subset $\mathcal{A}_2$ that covers $D$.}
  \step{<1>3}{$\mathcal{A}_1 \cup \mathcal{A}_2$ is a finite subset of
    $\mathcal{A}$ that covers $C \cup D$.}
  \qedstep
\end{proof}

\begin{prop}
  Not every compact Hausdorff space is first countable.
\end{prop}

\begin{proof}
  \pf\ The space $\overline{S_\Omega}$ is compact Hausdorff but not first countable. \qed
\end{proof}

\begin{cor}
Not every compact Hausdorff space is second countable.
\end{cor}

\begin{thm}[Tychonoff (AC)]
The product of a family of compact spaces is compact.
\end{thm}

\begin{proof}
\pf
\step{<1>1}{\pflet{$\{X_\alpha\}_{\alpha \in J}$ be a family of compact spaces.} \pflet{$X = \prod_{\alpha \in J} X_\alpha$}}
\step{<1>2}{\pflet{$\mathcal{A} \subseteq \mathcal{P} X$ satisfy the finite intersection property.}
\prove{$\bigcap_{A \in \mathcal{A}} \overline{A}$ is nonempty.}}
\step{<1>3}{\pick\ a set $\mathcal{D} \subseteq \mathcal{P} X$ that includes $\mathcal{A}$ and is maximal with respect to the finite intersection property.}
\begin{proof}
  \pf\ By Lemma \ref{lm:sets:finite_intersection_property:maximal}.
\end{proof}
\step{<1>4}{For $\alpha \in J$, \pick\ $x_\alpha \in \bigcap_{D \in \mathcal{D}} \overline{\pi_\alpha(D)}$}
\begin{proof}
  \step{<2>1}{\pflet{$\alpha \in J$}}
  \step{<2>2}{$\{ \overline{\pi_\alpha(D)} : D \in \mathcal{D}\}$ satisfies the finite intersection property.}
  \qedstep
  \begin{proof}
    \pf\ By Proposition \ref{prop:topology:compact:finite_intersection}
  \end{proof}
\end{proof}
\step{<1>5}{\pflet{$x = (x_\alpha)_{\alpha \in J}$}}
\step{<1>6}{For all $D \in \mathcal{D}$ we have $(x_\alpha)_{\alpha \in J} \in \overline{D}$}
\begin{proof}
  \pf
  \step{<2>1}{Every subbasis element containing $x$ intersects every member of $\mathcal{D}$}
  \begin{proof}
    \step{<3>1}{\pflet{$\inv{\pi_\alpha(U)}$ be a subbasis element containing $x$ where $U$ is open in $X_
    \alpha$}}
    \step{<3>2}{\pflet{$D \in \mathcal{D}$}}
    \step{<3>3}{$U$ intersects $\pi_\alpha(D)$}
  \end{proof}
  \step{<2>2}{Every subbasis element containing $x$ is a member of $\mathcal{D}$}
  \begin{proof}
    \pf\ By Lemma \ref{lm:sets:finite_intersection_property:intersect_all}
  \end{proof}
  \step{<2>3}{Every basis element containing $x$ is a member of $\mathcal{D}$}
  \begin{proof}
    \pf\ By Lemma \ref{lm:sets:finite_intersection_property:finite_intersection}
  \end{proof}
  \step{<2>4}{Every basis element containing $x$ intersects every member of $\mathcal{D}$}
  \begin{proof}
    \pf\ This follows because $\mathcal{D}$ satisfies the finite intersection property.
  \end{proof}
\end{proof}
\qedstep
\begin{proof}
  \pf\ By Proposition \ref{prop:topology:compact:finite_intersection}
\end{proof}
\qed
\end{proof}

\begin{proof}
\pf
\step{<1>1}{\pflet{$\{ X_\alpha \}_{\alpha \in J}$ be a family of compact spaces and $X = \prod_{\alpha \in J} X_\alpha$.}}
\step{<1>2}{\pick\ a well-ordering $<$ of $J$ such that $J$ has a greatest elemeent $\top$}
\step{<1>3}{For all $\alpha \in J$ and every family of points $p = \{ p_i \in X_i \}_{i \leq \alpha}$,
\pflet{$Y_\alpha(p) = \{ x \in X : \forall i \leq \alpha. x_i = p_i \}$}}
\step{<1>4}{For all $\beta \in J$ and every family of points $p = \{ p_i \in X_i \}_{i < \beta}$,
\pflet{$Z_\beta(p) = \bigcap_{\alpha < \beta} Y_\alpha = \{ x \in X : \forall i < \beta. x_i = p_i \}$}}
\step{<1>5}{Given $\beta \in J$, a family of points $\{ p_i \in X_i \}_{i < \beta}$, and a finite set $\mathcal{A}$ of basis elements
that covers $Z_\beta(p)$, there exists $\alpha < \beta$ such that $\mathcal{A}$ covers $Y_\alpha(p)$}
\begin{proof}
  \step{<2>1}{\assume(w.l.o.g. $\beta$ has no immediate predecessor)}
  \step{<2>2}{For $A \in \mathcal{A}$, \pflet{$J_A = \{ i < \beta : \pi_i(A) \neq X_i \}$}}
  \step{<2>3}{\pflet{$\alpha$ be the largest element of $\bigcup_{A \in \mathcal{A}} J_A$}}
  \begin{proof}
    \pf\ The set has a greatest element because each $J_A$ is finite and $\mathcal{A}$ is finite.
  \end{proof}
  \step{<2>4}{$\mathcal{A}$ covers $Y_\alpha(p)$}
  \begin{proof}
    \step{<3>1}{\pflet{$x \in Y_\alpha(p)$}}
    \step{<3>2}{\pflet{$y \in Z_\beta(p)$ be the point with
      $$ y_i = \begin{cases}
      p_i & \text{if } i < \beta \\
      x_i & \text{if } i \geq \beta
      \end{cases} $$}}
    \step{<3>3}{\pick\ $A \in \mathcal{A}$ such that $y \in A$}
    \step{<3>4}{$x \in A$}
    \begin{proof}
      \step{<4>1}{For $i \leq \alpha$ we have $x_i \in \pi_i(A)$}
      \begin{proof}
        \step{<5>1}{$x_i = p_i$}
        \begin{proof}
          \pf\ From \stepref{<3>1} and \stepref{<1>3}.
        \end{proof}
        \step{<5>2}{$y_i = p_i$}
        \begin{proof}
          \pf\ From \stepref{<3>2}
        \end{proof}
        \step{<5>3}{$y_i \in \pi_i(A)$}
        \begin{proof}
          \pf\ From \stepref{<3>3}.
        \end{proof}
      \end{proof}
      \step{<4>2}{For $\alpha < i < \beta$ we have $x_i \in \pi_i(A)$}
      \begin{proof}
        \step{<5>1}{$i \notin J_A$}
        \begin{proof}
          \pf\ From \stepref{<2>3}
        \end{proof}
        \step{<5>2}{$\pi_i(A) = X_i$}
        \begin{proof}
          \pf\ From \stepref{<2>2}
        \end{proof}
      \end{proof}
      \step{<4>3}{For $i \geq \beta$ we have $x_i \in \pi_i(A)$}
      \begin{proof}
        \step{<5>1}{$x_i = y_i$}
        \begin{proof}
          \pf\ By \stepref{<3>2}
        \end{proof}
        \step{<5>2}{$y_i \in \pi_i(A)$}
        \begin{proof}
          \pf\ By \stepref{<3>3}
        \end{proof}
      \end{proof}
    \end{proof}
  \end{proof}
\end{proof}
\step{<1>6}{\assume{for a contradiction $\mathcal{A}$ is a set of basis elements such that no finite subset covers $X$}}
\step{<1>7}{For all $\alpha \in J$ there exists a family of points $\{p_i \in X_i \}_{i \leq \alpha}$ such that no finite subset of $\mathcal{A}$ covers $Y_\alpha(p)$}
\begin{proof}
  \step{<2>1}{\assume{as induction hypothesis $\beta \in J$ and $p_i$ has been chosen for all $i < \beta$ such that, for all $\alpha < \beta$, no finite subset of $\mathcal{A}$ covers $Y_\alpha(p)$}}
  \step{<2>2}{No finite subset of $\mathcal{A}$ covers $Z_\beta(p)$}
  \begin{proof}
    \pf\ By \stepref{<1>5}
  \end{proof}
  \step{<2>3}{\pick\ $p_\beta \in X_\beta$ such that no finite subset of $\mathcal{A}$ covers $Z_\beta(p) \times \{ p_\beta \} = Y_\beta(p)$}
  \begin{proof}
    \pf\ By Lemma \ref{lm:topology:compact:tube_lemma2}.
  \end{proof}
\end{proof}
\qedstep
\begin{proof}
  \pf\ This is a contradiction since $Y_\top(p) = \{ p \}$ and so must be covered by a single element of $\mathcal{A}$.
\end{proof}
\qed
\end{proof}

\begin{thm}
In a compact Hausdorff space, the components and the quasicomponents coincide.
\end{thm}

\begin{proof}
\pf
\step{<1>1}{\pflet{$X$ be a compact Hausdorff space and $x, y \in X$ lie in the same quasicomponent.} \prove{$x$ and $y$ are in the same component.}}
\step{<1>2}{\pflet{$\mathcal{A}$ be the set of all closed subspaces $A$ of $X$ such that $x$ and $y$ lie in the same quasicomponent of $A$.}}
\step{<1>3}{Every chain in $\mathcal{A}$ has a lower bound.}
\begin{proof}
  \step{<2>1}{\pflet{$\mathcal{B} \subseteq \mathcal{A}$ be a chain} \prove{$Y = \bigcap \mathcal{B} \in \mathcal{A}$}}
  \step{<2>2}{\assume{for a contradiction $Y = C \cup D$ were $C$ and $D$ are disjoint and open in $Y$, $x \in C$ and $y \in D$}}
  \step{<2>3}{\pick\ disjoint open sets $U$ and $V$ in $X$ such that $C \subseteq U$ and $D \subseteq V$}
  \begin{proof}
    \pf\ By Lemma \ref{lm:topology:compact:normal}.
  \end{proof}
  \step{<2>4}{$\{ B \setminus (U \cup V) : B \in \mathcal{B} \}$ satisfies the finite intersection property.}
  \begin{proof}
    \step{<3>1}{\pflet{$B_1, \ldots, B_n \in \mathcal{B}$}}
    \step{<3>2}{$B_1 \cap \cdots \cap B_n \in \mathcal{B}$}
    \begin{proof}
      \pf\ By \stepref{<2>1}.
    \end{proof}
    \step{<3>3}{$B_1 \cap \cdots \cap B_n \setminus (U \cap V)$ is nonempty}
    \begin{proof}
      \pf\ $B_1 \cap \cdots \cap B_n \cap U$ and $B_1 \cap \cdots \cap B_n \cap V$ cannot be disjoint, because $x$ and $y$ are in the same quasicomponent of $B_1 \cap \cdots \cap B_n$.
    \end{proof}
  \end{proof}
  \step{<2>5}{$Y \setminus (U \cup V)$ is nonempty.}
  \begin{proof}
    \pf\ By Proposition \ref{prop:topology:compact:finite_intersection}.
  \end{proof}
  \qedstep
  \begin{proof}
    \pf\ This is a contradiction since $Y \setminus (U \cup V) = Y \setminus (C \cup D)$.
  \end{proof}
\end{proof}
\step{<1>4}{\pick\ a minimal element $D \in \mathcal{A}$}
\begin{proof}
  \pf\ One exists by Zorn's Lemma.
\end{proof}
\step{<1>5}{$D$ is connected.}
\begin{proof}
  \step{<2>1}{\assume[for a contradiction $D = U \uplus V$ is a separation of $D$]}
  \step{<2>2}{\case{$x, y \in U$}}
  \begin{proof}
    \pf\ In this case we have $U \in \mathcal{A}$ contradicting the minimality of $D$.
  \end{proof}
  \step{<2>3}{\case{$x \in U, y \in V$}}
  \begin{proof}
    \pf\ This is a contradiction because $x$ and $y$ are in the same quasicomponent of $D$.
  \end{proof}
  \step{<2>4}{\case{$x \in V, y \in U$}}
  \begin{proof}
    \pf\ Similar to \stepref{<2>3}.
  \end{proof}
  \step{<2>5}{\case{$x, y \in V$}}
  \begin{proof}
    \pf\ Similar to \stepref{<2>2}.
  \end{proof}
\end{proof}
\qed
\end{proof}

\section{Perfect Maps}

  \begin{prop}
    \label{prop:topology:perfect:neighbourhood}
 Let $p : X \twoheadrightarrow Y$ be a closed continuous surjective map. For
 all $y \in Y$ and $U$ an open neighbourhood of $\inv{p}(y)$, there exists an
 open neighbourhood $W$ of $y$ such that $\inv{p}(W) \subseteq U$.
\end{prop}

\begin{proof}
 \pf\ Take $W = Y \setminus p(X \setminus U)$. \qed
\end{proof}

    \begin{prop}[AC]
 Let $p : X \twoheadrightarrow Y$ be a closed continuous surjective map. If
$X$ is normal then $Y$ is normal.
\end{prop}

\begin{proof}
 \pf
 \step{<1>1}{\pflet{$A, B \subseteq Y$ be closed}}
 \step{<1>2}{$\inv{p}(A)$, $\inv{p}(B)$ are closed in $X$.}
 \step{<1>3}{\pick\ disjoint open sets $U$, $V$ of $\inv{p}(A)$, $\inv{p}(B)$
   respectively.}
 \step{<1>4}{For all $a \in A$, \pick\ an open neighbourhood $W_a$ of $a$
such
   that $\inv{p}(W_a) \subseteq U$}
 \begin{proof}
   \pf\ By Proposition \ref{prop:topology:perfect:neighbourhood}.
 \end{proof}
 \step{<1>5}{For all $b \in B$, \pick\ an open neighbourhood $W'_b$ of $b$
such
   that $\inv{p}(W'_b) \subseteq V$}
 \begin{proof}
   \pf\ By Proposition \ref{prop:topology:perfect:neighbourhood}.
 \end{proof}
 \step{<1>6}{\pflet{$W = \bigcup_{a \in A} W_a$ and $W' = \bigcup_{b \in B}
     W'_b$}}
 \step{<1>7}{$W \cap W' = \emptyset$}
 \begin{proof}
   \pf\ This holds because $\inv{p}(W) \subseteq U$, $\inv{p}(W') \subseteq
V$, and $p$ is surjective.
 \end{proof}
 \qed
\end{proof}

\begin{df}[Perfect Map]
  Let $X$ and $Y$ be topological spaces and $p : X \rightarrow Y$. Then $p$
  is
  \emph{perfect} iff $p$ is closed, continuous, surjective, and $p^{-1}(y)$
  is
  compact for all $y \in Y$.
\end{df}

  \begin{prop}
    \label{prop:topology:perfect:Hausdorff}
 Let $p : X \rightarrow Y$ be a perfect map. If $X$ is Hausdorff then so is
$Y$.
\end{prop}

\begin{proof}
 \pf
 \step{<1>1}{\pflet{$a, b \in Y$ with $a \neq b$}}
 \step{<1>2}{\pick\ disjoint open neighbourhoods $U$ and $V$ of
$\inv{\pi}(a)$
   and $\inv{\pi}(b)$, respectively.}
 \begin{proof}
   \pf\ By Lemma \ref{lm:topology:compact:normal}.
 \end{proof}
 \step{<1>3}{\pick\ open neighbourhoods $W$ and $W'$ of $a$ and $b$ such that
   $\inv{\pi}(W) \subseteq U$ and $\inv{\pi}(W') \subseteq V$}
 \begin{proof}
   \pf\ By Proposition \ref{prop:topology:perfect:neighbourhood}.
 \end{proof}
 \step{<1>4}{$W$ and $W'$ are disjoint.}
 \qed
\end{proof}

  \begin{prop}
 Let $p : X \twoheadrightarrow Y$ be perfect. If $X$ is regular then so is
$Y$.
\end{prop}

\begin{proof}
 \pf
 \step{<1>1}{$Y$ is $T_1$}
 \begin{proof}
   \pf\ By Proposition \ref{prop:topology:perfect:Hausdorff}.
 \end{proof}
 \step{<1>2}{\pflet{$C \subseteq Y$ be closed and $a \in Y \setminus C$}}
 \step{<1>3}{$\inv{p}(C)$ is closed and $\inv{p}(a)$ is disjoint from
   $\inv{p}(C)$.}
 \step{<1>4}{\pick\ disjoint open neighbourhoods $U$, $V$ of $\inv{p}(C)$,
   $\inv{p}(a)$ respectively.}
 \begin{proof}
   \pf\ By Lemma \ref{lm:topology:compact:regular}.
 \end{proof}
 \step{<1>5}{\pick\ an open neighbourhood $W'$ of $a$ such that $\inv{p}(W')
   \subseteq V$}
 \begin{proof}
   \pf\ By Proposition \ref{prop:topology:perfect:neighbourhood}.
 \end{proof}
 \step{<1>6}{For $c \in C$, \pick\ an open neighbourhood $W_c$ such that
   $\inv{p}(W_c) \subseteq U$}
 \begin{proof}
   \pf\ By Proposition \ref{prop:topology:perfect:neighbourhood}.
 \end{proof}
 \step{<1>7}{$W = \bigcup_{c \in C} W_c$ is an open neighbourhood of $C$
   disjoint from $W'$}
 \qed
\end{proof}

  \begin{prop}[AC]
 Let $p : X \twoheadrightarrow Y$ be perfect. If $X$ is locally compact then
 so is $Y$.
\end{prop}

\begin{proof}
 \pf
 \step{<1>1}{\pflet{$b \in Y$}}
 \step{<1>2}{$\{ U \text{ open in } X : \exists C \subseteq X \text{ compact}.
U
   \subseteq C \}$ covers $\inv{p}(b)$}
 \step{<1>3}{\pick\ a finite subcover $\{ U_1, \ldots, U_n \}$}
 \step{<1>4}{For $1 \leq i \leq n$, \pick\ a compact $C_i \subseteq X$ such
that
   $U_i \subseteq C_i$}
 \step{<1>5}{For $1 \leq i \leq n$, \pick\ a neighbourhood $W_i$ of $b$ such
   that $\inv{p}(W_i) \subseteq U_i$}
 \begin{proof}
   \pf\ By Proposition \ref{prop:topology:perfect:neighbourhood}
 \end{proof}
 \step{<1>6}{$b \in W_1 \cup \cdots \cup W_n \subseteq p(C_1) \cup \cdots
\cup
   p(C_n)$}
 \step{<1>7}{$p(C_1) \cup \cdots \cup p(C_n)$ is compact.}
 \begin{proof}
   \step{<2>1}{Each $p(C_i)$ is compact.}
   \begin{proof}
     \pf\ By Proposition \ref{prop:topology:compact:image}.
   \end{proof}
   \qedstep
   \begin{proof}
      \pf\ By Proposition \ref{prop:topology:compact:union}.
    \end{proof}
 \end{proof}
 \qed
\end{proof}

\begin{prop}
  \label{prop:topology:regular:perfect_map}
  The image of a regular space under a perfect map is regular.
\end{prop}

\begin{proof}
  \pf
  \step{<1>1}{\pflet{$p : X \twoheadrightarrow Y$ be a perfect map where $X$ is regular.}}
  \step{<1>2}{\pflet{$A \subseteq Y$ be closed and $a \notin A$.}}
  \step{<1>3}{\pick\ disjoint open neighbourhoods $U$ and $V$ of $\inv{p}(A)$ and $\inv{p}(a)$ respectively.}
  \begin{proof}
    \pf\ Lemma \ref{lm:topology:compact:regular}
  \end{proof}
  \step{<1>4}{\pick\ neighbourhoods $U'$ of $A$ and $V'$ of $a$ such that $\inv{p}(U') \subseteq U$ and $\inv{p}(V') \subseteq V$.}
  \begin{proof}
    \pf\ Lemma \ref{lm:topology:closed_map:open}.
  \end{proof}
  \step{<1>5}{$U'$ and $V'$ are disjoint.}
  \qed
\end{proof}

\section{Sequential Compactness}

\begin{df}[Sequentially Compact]
  A space is \emph{sequentially compact} iff every sequence has a convergent
  subsequence.
\end{df}

 \begin{prop}
 $\overline{S_\Omega}$ is not sequentially compact.
\end{prop}

\begin{proof}
\pf\ $\Omega$ is a limit point of $S_\Omega$ but is not the limit of any
sequence of points in $S_\Omega$. \qed
\end{proof}

\section{Local Compactness}

\begin{df}[Local Compactness]
  Let $X$ be a topological space.

  For $x \in X$, the space $X$ is \emph{locally compact} at $x$ iff there
  exists a compact subspace $C \subseteq X$ that includes a neighbourhood of
  $x$.

  The space $X$ is \emph{locally compact} iff it is locally compact at every
  point.
\end{df}

 \begin{prop}
Every complete linearly ordered set is locally compact under the order
topology.
\end{prop}

\begin{proof}
\pf
\step{<1>1}{\pflet{$L$ be a complete linearly ordered set and $x \in L$}
\prove{There exists a compact subspace $C \subseteq L$ that includes a
neighbourhood $U$ of $x$}}
\step{<1>2}{\case{$x$ is least and greatest in $L$}}
\begin{proof}
\pf\ In this case, $L = \{ x \}$ is compact.
\end{proof}
\step{<1>3}{\case{$x$ is least in $L$ but not greatest}}
\begin{proof}
\step{<2>1}{\pick\ $a < x$}
\step{<2>2}{Take $C = [a,x]$ and $U = (a, x]$}
\end{proof}
\step{<1>4}{\case{$x$ is greatest in $L$ but not least}}
\begin{proof}
  \pf\ Similar.
\end{proof}
\step{<1>5}{\case{$x$ is neither least nor greatest}}
\begin{proof}
  \step{<2>1}{\pick\ $a < x$ and $b > x$}
  \step{<2>2}{Take $C = [a,b]$ and $U = (a, b)$}
\end{proof}
\qed
\end{proof}

\begin{cor}
For every ordinal $\alpha$, the space $S_\alpha$ is locally compact.
\end{cor}


\begin{thm}
  \label{thm:topology:locally_compact:closed_subspace}
  Every closed subspace of a locally compact Hausdorff space is locally
  compact.
\end{thm}

\begin{proof}
  \pf
  \step{<1>1}{\pflet{$X$ be locally compact Hausdorff and $C \subseteq X$ be
      closed.}}
  \step{<1>2}{\pflet{$x \in C$}}
  \step{<1>3}{\pick\ $D \subseteq X$ compact and $U \subseteq D$ open such
    that
    $x \in U$}
  \step{<1>4}{$D$ is closed.}
  \begin{proof}
    \pf\ Proposition \ref{prop:topology:compact:compact_is_closed}.
  \end{proof}
  \step{<1>5}{$C \cap D$ is closed}
  \begin{proof}
    \pf\ Propositon \ref{prop:topology:closed:intersection}.
  \end{proof}
  \step{<1>6}{$C \cap D$ is compact}
  \begin{proof}
    \pf\ Proposition \ref{prop:topology:compact:closed_is_compact}.
  \end{proof}
  \qedstep
  \begin{proof}
    \pf\ $C \cap D \subseteq C$ is compact and includes the open
    neighbourhood
    $U \cap C$ of $x$.
  \end{proof}
  \qed
\end{proof}

 \begin{prop}
 Let $\{X_\alpha\}_{\alpha \in J}$ be a family of nonempty topological
spaces. If    $\prod_{\alpha \in J} X_\alpha$ is locally compact, then each
$X_\alpha$ is locally compact.
\end{prop}

\begin{proof}
\pf
\step{<1>1}{\pflet{$\alpha \in J$ and $x_\alpha \in X_\alpha$}}
\step{<1>2}{\pick\ $x_\beta \in X_\beta$ for all $\beta \in J \setminus \{
  \alpha \}$}
\step{<1>3}{\pick\ a compact subspace $C \subseteq \prod_{\alpha \in J}
  X_\alpha$ that a neighbourhood $U$ of $x$ included in $C$}
\step{<1>4}{\pick\ a basic open set $\prod_{\alpha \in J} U_\alpha$ such that
$x
  \in \prod_{\alpha \in J} U_\alpha \subseteq U$}
\step{<1>5}{$x_\alpha \in U_\alpha \subseteq \pi_\alpha(C)$}
\step{<1>6}{$\pi_\alpha(C)$ is compact.}
\begin{proof}
  \pf\ By Proposition \ref{prop:topology:compact:image}.
\end{proof}
\qed
\end{proof}

\begin{prop}
 Let $\{ X_\alpha \}_{\alpha \in J}$ be a family of locally compact spaces
such that $X_\alpha$ is compact for all but finitely many values of $\alpha$.
Then $\prod_{\alpha \in J} X_\alpha$ is locally compact.
\end{prop}

\begin{proof}
\pf
\step{<1>1}{\assume{$X_\alpha$ is compact if $\alpha \neq \alpha_1, \ldots,
    \alpha_n$}}
\step{<1>2}{\pflet{$\vec{x} \in \prod_{\alpha \in J} X_\alpha$}}
\step{<1>3}{For $1 \leq i \leq n$, \pick\ $C_{\alpha_i} \subseteq
X_{\alpha_i}$
  compact     and $U_{\alpha_i}$ open such that $x_{\alpha_i} \in
  U_{\alpha_i} \subseteq C_{\alpha_i}$}
\step{<1>4}{For $\alpha \neq \alpha_1, \ldots, \alpha_n$, \pflet{$C_\alpha =
    U_\alpha = X_\alpha$}}
\step{<1>5}{$\vec{x} \in \prod_{\alpha \in J} U_\alpha \subseteq
\prod_{\alpha
    \in J} C_\alpha$}
\step{<1>6}{$\prod_{\alpha \in J} C_\alpha$ is compact}
\begin{proof}
  \pf\ By Tychonoff's Theorem.
\end{proof}
\qed
\end{proof}

\begin{prop}
 $\mathbb{R}_l$ is not locally compact.
\end{prop}

\begin{proof}
\pf\ $[0, +\infty)$ can be partitioned into infinitely many disjoint open
sets, which therefore do not have a finite subcover. \qed
\end{proof}

\begin{cor}
The Sorgenfrey plane is not locally compact.
\end{cor}

  \begin{prop}
 Let $\{X_\alpha\}_{\alpha \in J}$ be a family of nonempty topological
spaces. If    $\prod_{\alpha \in J} X_\alpha$ is locally compact, then all but
finitely many of the $X_\alpha$ are compact.
\end{prop}

\begin{proof}
\pf
\step{<1>1}{\pick\ a point $a = (a_\alpha)_{\alpha \in J} \in \prod_{\alpha
\in
    J}       X_\alpha$}
\step{<1>2}{\pick\ a compact $C \subseteq \prod_{\alpha \in J} X_\alpha$ that
  includes the       basic neighbourhood $\prod_{\alpha \in J} U_\alpha$ of
  $a$,  where $U_\alpha = X_\alpha$ for all $\alpha$ except $\alpha =
  \alpha_1,  \ldots, \alpha_n$}
\step{<1>3}{For $\alpha \neq \alpha_1, \ldots, \alpha_n$, we have $X_\alpha$
is
  compact.}
\begin{proof}
  \pf\ $X_\alpha$ is homeomorphic to a closed subspace of $C$.
\end{proof}
\qed
\end{proof}

\begin{cor}
For any infinite set $I$, the space $\mathbb{R}^I$ is not locally compact.
\end{cor}

\begin{prop}
 $[0,1]^\omega$ is not compact under the uniform topology.
\end{prop}

\begin{proof}
 \pf $\{ a_i : i \geq 0 \}$ is an infinite set with no limit point, where
$a_i$ is the point with $i$th component 1 and all other components 0. \qed
\end{proof}

\begin{cor}
 $\mathbb{R}^\omega$ under the uniform topology is not locally compact.
\end{cor}

\begin{proof}
\pf
\step{<1>1}{\assume{$\mathbb{R}^\omega$ is locally compact}}
\step{<1>2}{\pflet{$C$ be a compact subspace such that $B(\vec{0}, \epsilon)
    \subseteq C$}}
\step{<1>3}{$\overline{B(\vec{0}, \epsilon)}$ is compact.}
\qedstep
\begin{proof}
  \pf\ This contradicts the proposition.
\end{proof}
\qed
\end{proof}

\begin{prop}
 Not every subspace of a locally compact Hausdorff space is locally compact.
\end{prop}

\begin{proof}
 \pf\ $\mathbb{R}$ is locally compact Hausdoff, $\mathbb{Q}$ is not locally compact. \qed
\end{proof}

\begin{prop}
 The continuous image of a locally compact Hausdorff space is not necessarily locally compact.
\end{prop}

\begin{proof}
 \pf
 \step{<1>1}{\pflet{$\{ q_0, q_1, \ldots \}$ be an enumeration of $[0,1] \cap \mathbb{Q}$.}}
 \step{<1>2}{Define $f : (0, +\infty) \setminus \mathbb{Z} \rightarrow [0,1] \cap \mathbb{Q}$ by: $f(x) = q_n$ for $x \in (n, n+1)$}
 \step{<1>3}{$f$ is continuous.}
 \begin{proof}
   \pf\ The inverse image of any set is a union of open intervals.
 \end{proof}
 \qed
\end{proof}

\section{Compactifications}

\begin{df}[Compactification]
  Let $X$ and $Y$ be spaces. Then $Y$ is a \emph{compactification} of $X$ iff
  $Y$ is a compact Hausdorff space and $X$ is a subspace of $Y$ with
  $\overline{X} = Y$.

  Two compcactificatons $Y_1$, $Y_2$ of $X$ are \emph{equivalent} iff there exists a homeomorphism between $Y_1$ and $Y_2$ that is the identity on $X$.
\end{df}

\begin{lm}
  \label{lm:topology:compactification:factorization}
  Let $h : X \rightarrow Z$ be an imbedding. Then there exists a compactification $c : X \rightarrow Y$ of $X$, unique up to equivalence,
  and an imbedding $i : Y \rightarrow Z$ such that $h = i \circ c$.
\end{lm}

\begin{proof}
  \pf\ Simply take $Y$ to be the closure of $X$ in $Z$. \qed
\end{proof}

\begin{df}[One-Point Compactification]
  A \emph{one-point compactification} of $X$ is a compactification $Y$ of $X$
  such that $Y \setminus X$ is a singleton.
\end{df}

\begin{thm}
  \label{thm:topology:locally_compact:one_point_compactification}
  Let $X$ be a topological space. Then $X$ is locally compact Hausdorff if
  and
  only if there exists a space $Y$ such that:
  \begin{enumerate}
    \item $X$ is a subspace of $Y$
    \item The set $Y \setminus X$ is a singleton.
    \item $Y$ is a compact Hausdorff space.
  \end{enumerate}
  If $Y$ and $Y'$ are two spaces satisfying these conditions, then there
  exists a
  unique homeomorphism between $Y$ and $Y'$ that is the identity on $X$.
\end{thm}

\begin{proof}
  \pf
  \step{<1>1}{If $X$ is locally compact Hausdorff then there exists a space
    $Y$
    satisfying 1--3.}
  \begin{proof}
    \step{<2>1}{\pflet{$Y = X \cup \{ \infty \}$ under the topology
        $\mathcal{T} = \{ U \subseteq X : U \text{ is open in } X \} \cup \{
        Y
        \setminus C : C \subseteq X \text{ is compact} \}$.}}
    \begin{proof}
      \step{<3>1}{$Y \in \mathcal{T}$}
      \begin{proof}
        \pf\ This holds because $Y = Y \setminus \emptyset$.
      \end{proof}
      \step{<3>2}{For all $U, V \in \mathcal{T}$ we have $U \cap V \in
        \mathcal{T}$.}
      \begin{proof}
        \step{<4>1}{\pflet{$U, V \in \mathcal{T}$}}
        \step{<4>2}{\case{$U$, $V$ are open in $X$}}
        \begin{proof}
          \pf\ In this case, $U \cap V$ is open in $X$.
        \end{proof}
        \step{<4>3}{\case{$U$ is open in $X$, $V = Y \setminus C$ where $C
            \subseteq X$ is compact.}}
        \begin{proof}
          \step{<5>1}{$U \cap V = U \setminus C$}
          \step{<5>2}{$C$ is closed in $X$}
          \begin{proof}
            \pf\ Proposition \ref{prop:topology:compact:compact_is_closed}.
          \end{proof}
          \step{<5>3}{$U \cap V$ is open in $X$}
        \end{proof}
        \step{<4>4}{\case{$U = Y \setminus C$ where $C \subseteq X$ is
            compact,
            $V$ is open in $X$.}}
        \begin{proof}
          \pf\ Similar.
        \end{proof}
        \step{<4>5}{\case{$U = Y \setminus C$, $V = Y \setminus D$ where $C,
            D
            \subseteq X$ are compact.}}
        \begin{proof}
          \step{<5>1}{$U \cap V = Y \setminus (C \cup D)$}
          \step{<5>2}{$C$ and $D$ are closed in $X$}
          \begin{proof}
            \pf\ Proposition \ref{prop:topology:compact:compact_is_closed}.
          \end{proof}
          \step{<5>3}{$C \cup D$ is closed in $X$}
          \begin{proof}
            \pf\ Proposition \ref{prop:topology:closed:union}.
          \end{proof}
          \step{<5>4}{$C \cup D$ is compact.}
          \begin{proof}
            \pf\ By Proposition
            \ref{prop:topology:compact:union}. \qed
          \end{proof}
        \end{proof}
      \end{proof}
      \step{<3>3}{For all $\mathcal{A} \subseteq \mathcal{T}$ we have
        $\bigcup
        \mathcal{A} \in \mathcal{T}$.}
      \begin{proof}
        \step{<4>1}{\pflet{$\mathcal{A} \subseteq \mathcal{T}$}}
        \step{<4>2}{\case{Every element of $\mathcal{A}$ is an open set in
            $X$.}}
        \begin{proof}
          \pf\ In this case, $\bigcup \mathcal{A}$ is open in $X$.
        \end{proof}
        \step{<4>3}{\case{There exists $C$ compact in $X$ such that $Y
            \setminus
            C \in \mathcal{A}$}}
        \begin{proof}
          \step{<5>1}{$\bigcup \mathcal{A} = Y \setminus (\bigcap \{ D
            \subseteq X : D \text{ compact}, Y \setminus D \in \mathcal{A} \}
            \setminus \bigcup \{ U \text{ open in } X : U \in \mathcal{A}
            \})$}
          \begin{proof}
            \pf\ Set theory.
          \end{proof}
          \step{<5>2}{$\bigcap \{ D
            \subseteq X : D \text{ compact}, Y \setminus D \in \mathcal{A} \}
            \setminus \bigcup \{ U \text{ open in } X : U \in \mathcal{A} \}$
            is
            compact.}
          \begin{proof}
            \pf\ It is a closed subset of the compact set $C$.
          \end{proof}
        \end{proof}
      \end{proof}
    \end{proof}
    \step{<2>2}{$X$ is a subspace of $Y$}
    \begin{proof}
      \step{<3>1}{For every open set $U$ of $X$, there exists $V$ open in $Y$
        such that $U = V \cap X$}
      \begin{proof}
        \pf\ Take $V = U$.
      \end{proof}
      \step{<3>2}{For every open set $V$ in $Y$, we have $V \cap X$ is open
        in
        $X$.}
      \begin{proof}
        \step{<4>1}{\pflet{$V$ be open in $Y$}}
        \step{<4>2}{\case{$V$ is open in $X$}}
        \begin{proof}
          \pf\ In this case, $V \cap X = V$.
        \end{proof}
        \step{<4>3}{\case{$V = Y \setminus C$ where $C \subseteq X$ is
            compact.}}
        \begin{proof}
          \step{<5>1}{$C$ is closed in $X$.}
          \begin{proof}
            \pf\ By Proposition \ref{prop:topology:compact:compact_is_closed}.
          \end{proof}
          \step{<5>2}{$V \cap X = X \setminus C$}
        \end{proof}
      \end{proof}
    \end{proof}
    \step{<2>3}{$Y \setminus X = \{ \infty \}$}
    \step{<2>4}{$Y$ is compact.}
    \begin{proof}
      \step{<3>1}{\pflet{$\mathcal{A}$ be an open covering of $Y$}}
      \step{<3>2}{\pick\ $U \in \mathcal{A}$ such that $\infty \in U$}
      \step{<3>3}{\pick\ $C \subseteq X$ compact such that $U = Y \setminus
        C$.}
      \step{<3>4}{$\{ V \cap X : V \in \mathcal{A} \}$ is set of open sets
        that
        covers $C$}
      \step{<3>5}{\pick\ a finite subset $\{ V_1, \ldots, V_n \}$ such that
        $\{
        V_1 \cap X, \ldots, V_n \cap X \}$ covers $C$.}
      \step{<3>6}{$\{ U, V_1, \ldots, V_n \}$ is a finite subcover of $Y$.}
    \end{proof}
    \step{<2>5}{$Y$ is Hausdorff.}
    \begin{proof}
      \step{<3>1}{\pflet{$x, y \in Y$ with $x \neq y$} \prove{There exist
          disjoint open neighbourhoods $U$, $V$ of $x$ and $y$.}}
      \step{<3>2}{\case{$x, y \in X$}}
      \begin{proof}
        \pf\ In this case, we just use the fact that $X$ is Hausdorff.
      \end{proof}
      \step{<3>3}{\case{$x = \infty$, $y \in X$}}
      \begin{proof}
        \step{<4>1}{\pick\ $C \subseteq X$ compact such that $C$ includes an
          open
          neighbourhood $V$ of $y$}
        \step{<4>2}{\pflet{$U = Y \setminus C$}}
      \end{proof}
      \step{<3>4}{\case{$x \in X$, $y = \infty$}}
      \begin{proof}
        \pf\ Simlar.
      \end{proof}
    \end{proof}
  \end{proof}
  \step{<1>2}{If there exists a space $Y$ satisfying 1--3 then $X$ is locally
    compact Hausdorff.}
  \begin{proof}
    \step{<2>1}{\pflet{$Y$ be a space satisfying 1--3}}
    \step{<2>2}{\pflet{$\infty$ be the point in $Y \setminus X$}}
    \step{<2>3}{$X$ is locally compact}
    \begin{proof}
      \step{<3>1}{\pflet{$x \in X$}}
      \step{<3>2}{\pick\ disjoint open neighbourhoods $U$ of $x$ and $V$ of
        $\infty$}
      \step{<3>3}{$X \setminus V$ is compact and includes $U$}
      \begin{proof}
        \pf\ $X \setminus V = Y \setminus V$ is compact because it is a
        closed
        subset of $Y$ (Proposition
        \ref{prop:topology:compact:closed_is_compact}).
      \end{proof}
    \end{proof}
    \step{<2>4}{$X$ is Hausdorff.}
    \begin{proof}
      \pf\ By Corollary \ref{cor:topology:Hausdorff:subspace}.
    \end{proof}
  \end{proof}
  \step{<1>3}{If $Y$ and $Y'$ are two spaces satisfying 1--3 then there
    exists a
    unique homemorphism between $Y$ and $Y'$ that is the identity on $X$.}
  \begin{proof}
    \step{<2>1}{\pflet{$Y$ and $Y'$ be two spaces that satisfy 1--3.}}
    \step{<2>2}{\pflet{$Y \setminus X = \{ p \}$ and $Y' \setminus X = \{ q
        \}$}}
    \step{<2>3}{\pflet{$h : Y \rightarrow Y'$ be given by
        \begin{align*}
          h(x) & = x & (x \in X) \\
          h(p) & = q
        \end{align*}}}
    \step{<2>4}{$h$ is a homeomorphism}
    \begin{proof}
      \step{<3>1}{$h$ is bijective.}
      \step{<3>2}{$h$ is continuous.}
      \begin{proof}
        \step{<4>1}{\pflet{$V \subseteq Y'$ be open.} \prove{$\inv{h}(V)$ is
            open.}}
        \step{<4>2}{\case{$V \subseteq X$}}
        \begin{proof}
          \step{<5>1}{$\inv{h}(V) = V$}
          \step{<5>2}{$V$ is open in $X$}
          \begin{proof}
            \pf\ Condition 1 for $Y'$.
          \end{proof}
          \step{<5>3}{$V$ is open in $Y$}
          \begin{proof}
            \pf\ Condition 1 for $Y$.
          \end{proof}
        \end{proof}
        \step{<4>3}{\case{$q \in V$}}
        \begin{proof}
          \step{<5>1}{$Y' \setminus V$ is compact.}
          \begin{proof}
            \pf\ Proposition \ref{prop:topology:compact:closed_is_compact}.
          \end{proof}
          \step{<5>2}{$Y' \setminus V$ is closed in $Y$.}
          \begin{proof}
            \pf\ Proposition \ref{prop:topology:compact:compact_is_closed}.
          \end{proof}
          \step{<5>3}{$\inv{h}(V) = Y \setminus (Y' \setminus V)$}
        \end{proof}
      \end{proof}
      \step{<3>3}{$\inv{h}$ is continuous.}
      \begin{proof}
        \pf\ Similar.
      \end{proof}
    \end{proof}
    \step{<2>5}{If $h' : Y \rightarrow Y'$ is a homeomorphism such that $h'
      \restriction_X = \id{X}$ then $h' = h$}
  \end{proof}
  \qed
\end{proof}

\begin{thm}
  \label{thm:topology:locally_compact:neighbourhood}
  Let $X$ be a Hausdorff space. Then $X$ is locally compact if and only if,
  for all $x \in X$ and any neighbourhood $U$ of $x$, there exists an open
  neighbourhood $V$ of $x$ such that $\overline{V}$ is compact and
  $\overline{V}
  \subseteq U$.
\end{thm}

\begin{proof}
  \pf
  \step{<1>1}{If $X$ is locally compact then, for all $x \in X$ and any
    neighbourhood $U$ of $x$, there exists an open
    neighbourhood $V$ of $x$ such that $\overline{V}$ is compact and
    $\overline{V}
    \subseteq U$.}
  \begin{proof}
    \step{<2>1}{\assume{$X$ is locally compact.}}
    \step{<2>2}{\pflet{$x \in X$ and $U$ be a neighbourhood of $x$.}}
    \step{<2>3}{\pflet{$Y$ be the one-point compactification of $X$.}}
    \begin{proof}
      \pf\ By Theorem
      \ref{thm:topology:locally_compact:one_point_compactification}.
    \end{proof}
    \step{<2>4}{\pflet{$C = Y \setminus U$}}
    \step{<2>5}{$C$ is compact}
    \begin{proof}
      \pf\ By Proposition \ref{prop:topology:compact:closed_is_compact}.
    \end{proof}
    \step{<2>6}{\pick\ disjoint open sets $V$, $W$ containing $x$ and $C$}
    \begin{proof}
      \pf\ Lemma \ref{lm:topology:compact:regular}
    \end{proof}
    \step{<2>7}{$V$ is open in $X$}
    \begin{proof}
      \pf\ $V \subseteq X$ since $\infty \in W$.
    \end{proof}
    \step{<2>8}{The closure of $V$ in $X$ is compact}
    \begin{proof}
      \step{<3>1}{The closure of $V$ is $X$ is the same as the closure of $V$
        in
        $Y$.}
      \begin{proof}
        \pf\ The point $\infty$ cannot be a limit point of $V$ since $W$ is a
        neighbourhood disjoint from $V$.
      \end{proof}
      \step{<3>2}{The closure of $V$ in $Y$ is compact.}
      \begin{proof}
        \pf\ By Proposition \ref{prop:topology:compact:closed_is_compact}.
      \end{proof}
    \end{proof}
    \step{<2>9}{$\overline{V} \subseteq U$}
    \begin{proof}
      \pf
      \begin{align*}
        \overline{V} & \subseteq Y \setminus W \\
        & \subseteq Y \setminus C \\
        & = U
      \end{align*}
    \end{proof}
  \end{proof}
  \step{<1>2}{If, for all $x \in X$ and any neighbourhood $U$ of $x$, there
    exists an
    open
    neighbourhood $V$ of $x$ such that $\overline{V}$ is compact and
    $\overline{V}
    \subseteq U$, then $X$ is locally compact.}
  \begin{proof}
    \step{<2>1}{\assume{for all $x \in X$ and any neighbourhood $U$ of $x$,
        there
        exists an
        open
        neighbourhood $V$ of $x$ such that $\overline{V}$ is compact and
        $\overline{V}
        \subseteq U$}}
    \step{<2>2}{\pflet{$x \in X$} \prove{There exists $C \subseteq X$ compact
        such
        that $C$ includes a neighbourhood $U$ of $x$}}
    \step{<2>3}{\pick\ an open neighbourhood $V$ of $x$ such that
      $\overline{V}$ is
      compact and $\overline{V} \subseteq X$}
    \step{<2>4}{Take $C = \overline{V}$ and $U = V$}
  \end{proof}
  \qed
\end{proof}

\begin{cor}
  Every open subspace of a locally compact Hausdorff space is locally compact.
\end{cor}

\begin{cor}
  A space is locally compact Hausdorff if and only if it is an open subspace
  of a compact Hausdorff space.
\end{cor}

\begin{cor}
 Every locally compact Hausdorff space is completely regular.
\end{cor}

\begin{cor}
  The space $\mathbb{R}_K$ is not locally compact.
\end{cor}

  \begin{lm}[AC]
    \label{lm:topology:locally_compact:quotient}
 If $p : X \rightarrow Y$ is a quotient map and $Z$ is a locally compact
Hausdorff space, then the map
\[ \pi = p \times \id{Z} : X \times Z \rightarrow Y \times Z \]
is a quotient map.
\end{lm}

\begin{proof}
 \pf
 \step{<1>1}{$\pi$ is surjective.}
 \begin{proof}
   \pf\ This holds because $p$ is surjective.
 \end{proof}
 \step{<1>2}{$\pi$ is continuous.}
 \begin{proof}
   \pf\ By Theorem \ref{thm:topology:continuous:product}. % TODO Extract lemma
 \end{proof}
 \step{<1>3}{For $A \subseteq Y \times Z$, if $\inv{\pi}(A)$ is open in $X
   \times Z$ then $A$ is open in $Y \times Z$.}
 \begin{proof}
   \step{<2>1}{\pflet{$A \subseteq Y \times Z$}}
   \step{<2>2}{\assume{$\inv{\pi}(A)$ is open in $X \times Z$}}
   \step{<2>3}{\pflet{$(y, z) \in A$}}
   \step{<2>4}{\pick\ $x \in X$ such that $p(x) = y$}
   \begin{proof}
     \pf\ Since $p$ is surjective.
   \end{proof}
   \step{<2>5}{\pick\ open sets $U_1$, $V$ with $\overline{V}$ compact such
that
     $(x, y) \in U_1 \times V$ and $U_1 \times \overline{V} \subseteq
     \inv{\pi}(A)$}
   \begin{proof}
     \pf\ Using Theorem \ref{thm:topology:locally_compact:neighbourhood}
   \end{proof}
   \step{<2>6}{\pick\ a sequence of open sets $U_1$, $U_2$, \ldots in $X$
such
     that $\inv{p}(p(U_n)) \subseteq U_{n+1}$ and $U_n \times \overline{V}
     \subseteq \inv{\pi}(A)$ for all $n$}
   \begin{proof}
     \step{<3>1}{\pflet{$U$ be open with $U \times \overline{V} \subseteq
         \inv{\pi}(A)$} \prove{There exists $W$ open with $\inv{p}(p(U))
         \subseteq W$ and $W \times \overline{V} \subseteq \inv{\pi}(A)$}}
     \step{<3>2}{For all $x \in \inv{p}(p(U))$, \pick\ open sets $U_x$, $V_x$
       such that $x \in U_x$, $\overline{V} \subseteq V_x$ and $U_x \times
       V_x \subseteq \inv{\pi}(A)$}
     \begin{proof}
       \pf\ By the Tube Lemma.
     \end{proof}
     \step{<3>3}{\pflet{$W = \bigcup_{x \in \inv{p}(p(U))} U_x$}}
   \end{proof}
   \step{<2>7}{\pflet{$U = \bigcup_{n=1}^\infty U_n$}}
   \step{<2>8}{$U$ is saturated with respect to $p$}
   \begin{proof}
     \step{<3>1}{\pflet{$a \in U$, $b \in X$, $p(a) = p(b)$}}
     \step{<3>2}{\pick\ $n$ such that $a \in U_n$}
     \step{<3>3}{$b \in \inv{p}(p(U_n))$}
     \step{<3>4}{$b \in U_{n+1}$}
     \step{<3>5}{$b \in U$}
   \end{proof}
   \step{<2>9}{$p(U)$ is open in $Y$}
   \begin{proof}
     \pf\ By Lemma \ref{lm:topology:quotient:saturated}.
   \end{proof}
   \step{<2>10}{$(y, z) \in p(U) \times V \subseteq A$}
   \qedstep
   \begin{proof}
     \pf\ By Proposition \ref{prop:topology:neighbourhood:open}.
   \end{proof}
 \end{proof}
 \qed
\end{proof}

  \begin{thm}
 Let $p : A \rightarrow B$ and $q : C \rightarrow D$ be quotient maps. If $B$
and $C$ are locally compact Hausdorff spaces, then $p \times q : A \times C
\rightarrow B \times D$ is a quotient map.
\end{thm}

\begin{proof}
  \pf\ This holds by Lemma
\ref{lm:topology:locally_compact:quotient} and Proposition
\ref{prop:topology:quotient:composite} because $p \times q = (\id{B} \times q)
\circ (p \times
\id{C})$. \qed
\end{proof}

\begin{thm}
  Let $X$ be a completely regular space. Let $Y$ be a compactification of $X$ such that every bounded continuous map $X \rightarrow \mathbb{R}$
  extends uniquely to a continuous map $Y \rightarrow \mathbb{R}$. Then, for every compact Hausdorff space $C$, every continuous map $X \rightarrow C$ extends uniquely to a continuous map $Y \rightarrow C$.
\end{thm}

\begin{proof}
  \pf
  \step{<1>1}{\pflet{$C$ be a compact Hausdorff space and $f : X \rightarrow C$ a continuous function}}
  \step{<1>2}{\pick\ a set $J$ and an imedding $C \subseteq [0,1]^J$}
  \begin{proof}
    \step{<2>1}{$C$ is normal}
    \begin{proof}
      \pf\ By Lemma \ref{lm:topology:compact:normal}
    \end{proof}
    \qedstep
    \begin{proof}
      \pf\ By Theorem \ref{thm:topology:completely_regular:imdbeddable}.
    \end{proof}
  \end{proof}
  \step{<1>3}{For $\alpha \in J$, \pflet{$g_\alpha : Y \rightarrow \mathbb{R}$ be the unique continuous extension of $\pi_\alpha \circ f$}}
  \step{<1>4}{Define $g : Y \rightarrow \mathbb{R}^J$ by $g(y)_\alpha = g_\alpha(y)$}
  \step{<1>5}{$g$ is continuous}
  \begin{proof}
    \pf\ By Theorem \ref{thm:topology:continuous:product}.
  \end{proof}
  \step{<1>6}{$g$ extends $f$}
  \step{<1>7}{We have $g : Y \rightarrow C$}
  \begin{proof}
    \pf
    \begin{align*}
      g(Y) & = g(\overline{X}) \\
      & \subseteq \overline{g(X)} & (\text{Theorem \ref{thm:topology:continuous:characterisation}})\\
      & = \overline{f(X)} & (\text{\stepref{<1>6}})\\
      & \subseteq \overline{C} \\
      & = C & (\text{Proposition \ref{prop:topology:compact:compact_is_closed}})
    \end{align*}
  \end{proof}
  \step{<1>8}{$g$ is unique}
  \begin{proof}
    \step{<2>1}{\pflet{$h : Y \rightarrow C$ be a continuous extension of $f$}}
    \step{<2>2}{For all $\alpha \in J$, $\pi_\alpha \circ h$ extends $\pi_\alpha \circ f$}
    \step{<2>3}{For all $\alpha \in J$, $\pi_\alpha \circ h = g_\alpha$}
    \begin{proof}
      \pf\ By \stepref{<1>3}
    \end{proof}
    \step{<2>4}{$h = g$}
    \begin{proof}
      \pf\ By \stepref{<1>4}
    \end{proof}
  \end{proof}
  \qed
\end{proof}

\begin{cor}
Let $X$ be a completely regular space. Let $Y_1$ and $Y_2$ be compactifications of $X$ such that every bounded continuous map $X \rightarrow \mathbb{R}$
extends uniquely to a continuous map $Y_i \rightarrow \mathbb{R}$. Then $Y_1$ and $Y_2$ are equivalent.
\end{cor}

\begin{df}[Stone-\u{C}ech Compactification]
Let $X$ be a completely regular space. The \emph{Stone-\u{C}ech compactification} of $X$, $\beta(X)$, is the compactification of $X$ such that, for every compact Hausdorff space $C$,
every continuous function $X \rightarrow C$ extends uniquely to a continuous function $\beta(X) \rightarrow C$.
\end{df}

\section{Compactly Generated Spaces}

\begin{df}[Compactly Generated]
  A topological space $X$ is \emph{compactly generated} iff, for every $U \subseteq X$, if $U \cap C$ is open in $C$ for every compact $C \subseteq X$, then $U$ is open in $X$.
\end{df}

\begin{lm}
  Let $X$ be a topological space. Then $X$ is compactly generated iff, for every $A \subseteq X$, if $A \cap C$ is closed in $C$ for every compact $C \subseteq X$, then $A$ is closed in $X$.
\end{lm}

\begin{proof}
  \pf\ Easy. \qed
\end{proof}

\begin{prop}
  Every locally compact space is compactly generated.
\end{prop}

\begin{proof}
  \pf
  \step{<1>1}{\pflet{$X$ be locally compact.}}
  \step{<1>2}{\pflet{$U$ be a set such that $U \cap C$ is open in $C$ for every compact $C \subseteq X$} \prove{$U$ is open in $X$}}
  \step{<1>3}{\pflet{$x \in U$}}
  \step{<1>4}{\pick\ a neighbourhood $V$ of $x$ and a compact $C \subseteq X$ such that $V \subseteq C$}
  \step{<1>5}{$U \cap C$ is open in $C$}
  \step{<1>6}{\pick\ an open set $W$ in $X$ such that $U \cap C = W \cap C$}
  \step{<1>7}{$x \in V \cap W \subseteq U$}
  \qed
\end{proof}

\begin{prop}
  Every first countable space is compactly generated.
\end{prop}

\begin{proof}
  \pf
  \step{<1>1}{\pflet{$X$ be first countable.}}
  \step{<1>2}{\pflet{$A$ be a set such that $A \cap C$ is closed in $C$ for every compact $C \subseteq X$.} \prove{$A$ is closed in $X$.}}
  \step{<1>3}{\pflet{$x \in \overline{A}$.} \prove{$x \in A$.}}
  \step{<1>4}{\pick\ a sequence $(x_n)$ in $A$ that converges to $x$.}
  \begin{proof}
    \pf\ Theorem \ref{sequence_lemma}.
  \end{proof}
  \step{<1>5}{\pflet{$C = \{ x \} \cup \{ x_n : n \in \mathbb{Z}^+ \}$}}
  \step{<1>6}{$C$ is compact.}
  \step{<1>7}{$A \cap C$ is closed in $C$.}
  \step{<1>8}{$x \in A$}
  \qed
\end{proof}

\begin{prop}
  \label{compactly_generated:continuous}
  Let $X$ be a compactly generated space, $Y$ a topological space, and $f : X \rightarrow Y$. Then $f$ is continuous if and only if, for every compact $C \subseteq X$, the function $f \restriction_C : C \rightarrow Y$ is continuous.
\end{prop}

\begin{proof}
  \pf
  \step{<1>1}{If $f$ is continuous then, for every compact $C \subseteq X$, the function $f \restriction_C$ is continuous.}
  \begin{proof}
    \pf\ Proposition \ref{continuous:restriction}.
  \end{proof}
  \step{<1>2}{If, for every compact $C \subseteq X$, the function $f \restriction_C$ is continuous, then $f$ is continuous.}
  \begin{proof}
    \step{<2>1}{\pflet{$V \subseteq Y$ be open.}}
    \step{<2>2}{For every compact $C \subseteq X$ we have $\inv{f}(V) \cap C$ is open.}
    \step{<2>3}{$\inv{f}(V)$ is open.}
  \end{proof}
  \qed
\end{proof}
