\chapter{Separation Axioms}

\section{$T_1$ Spaces}

\begin{df}[$T_1$ Space]
  A \emph{$T_1$ space} is a topological space in which every one-point set is closed.
\end{df}

\begin{ex}$ $
  \begin{enumerate}
    \item
    The discrete topology is $T_1$.
    \item
    The indiscrete topology on $X$ is $T_1$ iff $X$ is a singleton.
    \item
    The finite complement topology is $T_1$.
    \item
    The countable complement topology is $T_1$.
    \item
    The lower limit topology on $\mathbb{R}$ is $T_1$.
    \item
    The $K$-topology on $\mathbb{R}$ is $T_1$.
    \item
    If $\mathcal{T}$ is finer than $\mathcal{T}'$ and $\mathcal{T}'$ is $T_1$ then $\mathcal{T}$ is $T_1$.
    \item
    The order topology on a linearly ordered set $X$ is $T_1$, because $X \setminus \{a\} = (-\infty, a) \cup (a, +\infty)$ is open.
    \item
    A subspace of a $T_1$ space is $T_1$.
    \item
    The ordered square is $T_1$.
  \end{enumerate}
\end{ex}

\begin{prop}
  A topological space is a $T_1$ space if and only if every finite set is closed.
\end{prop}

\begin{proof}
  \pf\ By Proposition \ref{prop:closed}, a finite union of closed sets is closed.
\end{proof}

\begin{prop}
  Let $X$ be a $T_1$ space, $A \subseteq X$ and $a \in X$. Then $a$ is a limit point of $A$ if and only if every neighbourhood of $a$ contains infinitely many points of $A$.
\end{prop}

\begin{proof}
  \pf
  \step{<1>1}{\pflet{$X$ be a $T_1$ space.}}
  \step{<1>2}{\pflet{$A \subseteq X$}}
  \step{<1>3}{\pflet{$a \in X$}}
  \step{<1>4}{If $a$ is a limit point of $A$ then every neighbourhood of $a$ contains infinitely many points of $A$.}
  \begin{proof}
    \step{<2>1}{\assume{$a$ is a limit point of $A$}}
    \step{<2>2}{\assume{for a contradiction $N$ is a neighbourhood of $a$ that contains only finitely many points of $A$, say $a_1$, \ldots, $a_n$ and also possibly $a$.}}
    \step{<2>3}{$N \setminus \{ a_1, \ldots, a_n \}$ is a neighbourhood of $a$ that does not intersect $A$ except possibly at $a$.}
    \begin{proof}
      \pf\ Proposition \ref{prop:neighbourhood:minus_closed}.
    \end{proof}
    \qedstep
    \begin{proof}
      \pf\ \stepref{<2>1} and \stepref{<2>3} are a contradiction.
    \end{proof}
  \end{proof}
  \step{<1>5}{If every neighbourhood of $a$ contains infinitely many points of $A$ then $a$ is a limit point of $A$.}
  \begin{proof}
    \pf\ Immediate from definitions.
  \end{proof}
  \qed
\end{proof}

\begin{prop}
  The product of a family of $T_1$ spaces is $T_1$.
\end{prop}

\begin{proof}
  \pf\ From Proposition \ref{prop:product:closed}. \qed
\end{proof}

\begin{prop}
  A topological space $X$ is $T_1$ if and only if, for any points $a,b \in X$ with $a \neq b$, there exists a neighbourhood $N$ of $a$ that does not contain $b$.
\end{prop}

\begin{proof}
  \pf\ Immediate from definitions. \qed
\end{proof}

\section{Hausdorff Spaces}

\begin{df}[Hausdorff Space]
  A \emph{Hausdorff space} is a topological space such that any two distinct points have two disjoint neighbourhoods.
\end{df}

\begin{ex}$ $
  \begin{enumerate}
    \item
    The discrete topology is Hausdorff because $\{a\}$ and $\{b\}$ are two disjoint neighbourhoods of $a$ and $b$ if $a \neq b$.
    \item
    The indiscrete topology on $X$ is Hausdorff iff $X$ is a singleton.
    \item
    The finite complement topology on $X$ is Hausdorff iff $X$ is finite.
    \item
    The countable complement topology on $X$ is Hausdorff iff $X$ is countable.
    \item
    The lower limit topology on $\mathbb{R}$ is Hausdorff.
    \item
    The $K$-topology on $\mathbb{R}$ is Hausdorff.
  \end{enumerate}
\end{ex}

\begin{prop}
  Every Hausdroff space is $T_1$.
\end{prop}

\begin{proof}
  \pf
  \step{<1>1}{\pflet{$X$ be a Hausdorff space.}}
  \step{<1>2}{\pflet{$a \in X$} \prove{$X \setminus \{a\}$ is open.}}
  \step{<1>3}{\pflet{$b \in X \setminus \{ a \}$}}
  \step{<1>4}{\pick\ disjoint neighbourhoods $M$ of $a$ and $N$ of $b$.}
  \begin{proof}
    \pf\ \stepref{<1>3}, \stepref{<1>3}
  \end{proof}
  \step{<1>5}{$N \subseteq X \setminus \{a\}$}
  \begin{proof}
    \pf\ From \stepref{<1>4}
  \end{proof}
  \step{<1>6}{$X \setminus \{a\}$ is a neighbourhood of $b$}
  \begin{proof}
    \pf\ Proposition \ref{prop:neighbourhood}, \stepref{<1>4}, \stepref{<1>5}
  \end{proof}
  \qedstep
  \begin{proof}
    \pf\ Proposition \ref{prop:neighbourhood}.
  \end{proof}
\end{proof}

\begin{prop}
  In a Hausdorff space, a sequence has at most one limit.
\end{prop}

\begin{proof}
  \pf
  \step{<1>1}{\pflet{$X$ be a Hausdorff space.}}
  \step{<1>2}{\assume{for a contradiction $x_n \rightarrow l$ as $n \rightarrow \infty$ and $x_n \rightarrow m$ as $n \rightarrow \infty$ and $l \neq m$}}
  \step{<1>3}{\pick\ disjoint neighbourhoods $L$ and $M$ of $l$ and $m$ respectively.}
  \begin{proof}
    \pf\ \stepref{<1>3}, \stepref{<1>2}
  \end{proof}
  \step{<1>4}{\pick\ $N$ such that, for all $n \geq N$, we have $x_n \in L$ and $x_n \in M$}
  \begin{proof}
    \pf\ \stepref{<1>2}, \stepref{<1>3}
  \end{proof}
  \qedstep
  \begin{proof}
    \pf\ \stepref{<1>4} contradicts the fact that $L$ and $M$ are disjoint (\stepref{<1>3}).
  \end{proof}
  \qed
\end{proof}

\begin{prop}
  Every linearly ordered set is Hausdorff under the order topology.
\end{prop}

\begin{proof}
  \pf
  \step{<1>1}{\pflet{$X$ be a linearly ordered set under the order topology.}}
  \step{<1>2}{\pflet{$a, b \in X$ with $a \neq b$} \prove{There exist disjoint neighbourhoods $L$ and $M$ of $a$ and $b$ respectively.}}
  \step{<1>3}{\assume{w.l.o.g.~$a < b$}}
  \step{<1>4}{\case{There exists $c$ such that $a < c < b$}}
  \begin{proof}
    \pf\ Take $L = (- \infty, c)$ and $M = (c, + \infty)$.
  \end{proof}
  \step{<1>5}{\case{There is no $c$ such that $a < c < b$}}
  \begin{proof}
    \pf\ Take $L = (- \infty, b)$ and $M = (a, + \infty)$.
  \end{proof}
  \qed
\end{proof}

\begin{prop}
  \label{prop:Hausdorff:product}
  The product of a family of Hausdorff spaces is Hausdorff.
\end{prop}

\begin{proof}
  \pf
  \step{<1>1}{\pflet{$\{ X_\alpha \}_{\alpha \in J}$ be a family of Hausdorff spaces.}}
  \step{<1>2}{\pflet{$(a_\alpha), (b_\alpha) \in \prod_{\alpha \in J} X_\alpha$ be distinct.}}
  \step{<1>3}{\pick\ $\alpha \in J$ such that $a_\alpha \neq b_\alpha$}
  \step{<1>4}{\pick\ disjoint neighbourhoods $L$ and $M$ of $a_\alpha$ and $b_\alpha$ in $X_\alpha$}
  \step{<1>5}{$\inv{\pi_\alpha}(L)$ and $\inv{\pi_\alpha}(M)$ are disjoint neighbourhoods of $(a_\alpha)$ and $(b_\alpha)$ respectively.}
  \qed
\end{proof}

\begin{cor}
  Let $\{ X_\alpha \}_{\alpha \in J}$ be a family of Hausdorff spaces. Then $\prod_{\alpha \in J} X_\alpha$ under the box topology is Hausdorff.
\end{cor}

\begin{proof}
  \pf\ The box topology is finer than the product topology. \qed
\end{proof}

\begin{prop}
  A subspace of a Hausdorff space is Hausdorff.
\end{prop}

\begin{proof}
  \pf
  \step{<1>1}{\pflet{$X$ be Hausdorff and $Y \subseteq X$}}
  \step{<1>2}{\pflet{$a, b \in Y$ with $a \neq b$}}
  \step{<1>3}{\pick\ disjoint neighbourhoods $M$ and $N$ of $a$ and $b$ in $X$}
  \step{<1>4}{$M \cap Y$ and $N \cap Y$ are disjoint neighbourhoods of $a$ and $b$ in $Y$}
  \begin{proof}
    \pf\ Proposition \ref{prop:subspace:neighbourhood}
  \end{proof}
  \qed
\end{proof}

\begin{cor}
  The ordered square is Hausdorff.
\end{cor}

\begin{prop}
  A topological space $X$ is Hausdorff iff the diagonal $\Delta = \{ (x,x) : x \in X \}$ is closed in $X \times X$.
\end{prop}

\begin{proof}
  \pf\ Both '$X$ is Hausdorff' and '$(X \times X) \setminus \Delta$ is open' are equivalent to the statement: for all $(a,b) \in X \times X$, if $(a,b) \notin \Delta$ then there exist open $U$, $V$ such that $a \in U$, $b \in V$ and $U \times V \subseteq (X \times X) \setminus \Delta$. \qed
\end{proof}

\section{Regular Spaces}

\begin{df}[Regular]
  A topological space $X$ is \emph{regular} iff it is $T_1$ and, for every closed $A \subseteq X$ and point $a \in X \setminus A$, there exist disjoint neighbourhoods $M$ and $N$ of $A$ and $a$ respectively.
\end{df}
