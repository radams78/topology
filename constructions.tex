\chapter{Constructions of Topological Spaces}

\section{The Order Topology}

\begin{df}[Order Topology]
  Let $X$ be a linearly ordered set with more than one element. The \emph{order topology} on $X$ is the topology generated by the subbasis consisting of the open rays.
\end{df}

\begin{prop}
  \label{prop:order:topology}
  Let $X$ be a linearly ordered set with more than one element. The following sets together form a basis $\mathcal{B}$ for the order topology:
  \begin{itemize}
    \item
    All open intervals $(a,b)$ for $a, b \in X$, $a < b$.
    \item
    All intervals of the form $[\bot, a)$ for $a \in X$ where $\bot$ is the least element in $X$, if it exists.
    \item
    All intervals of the form $(a, \top]$ for $a \in X$ where $\top$ is the greatest element in $X$, if it exists.
  \end{itemize}
\end{prop}

\begin{proof}
  \pf
  \step{<1>1}{\pflet{$X$ be a linearly ordered set with more than one element.}}
  \step{<1>2}{$\mathcal{B}$ is a basis for a topology on $X$.}
  \begin{proof}
    \step{<2>1}{$\bigcup \mathcal{B} = X$.}
    \begin{proof}
      \step{<3>1}{\pflet{$x \in X$} \prove{There exists $B \in \mathcal{B}$ with $x \in B$}}
      \step{<3>2}{\case{$x$ is least in $X$}}
      \begin{proof}
        \step{<4>1}{\pick\ $a \in X$ with $a \neq x$}
        \begin{proof}
          \pf\ \stepref{<1>1}
        \end{proof}
        \step{<4>2}{$x \in [x, a) \in \mathcal{B}$}
      \end{proof}
      \step{<3>3}{\case{$x$ is greatest in $X$}}
      \begin{proof}
        \step{<4>1}{\pick\ $a \in X$ with $a \neq x$}
        \begin{proof}
          \pf\ \stepref{<1>1}
        \end{proof}
        \step{<4>2}{$x \in (a, x] \in \mathcal{B}$}
      \end{proof}
      \step{<3>4}{\case{$x$ is neither least nor greatest in $X$}}
      \begin{proof}
        \step{<4>1}{\pick\ $a, b \in X$ with $a < x < b$}
        \step{<4>2}{$x \in (a,b) \in \mathcal{B}$}
      \end{proof}
    \end{proof}
    \step{<2>2}{For all $B_1, B_2 \in \mathcal{B}$ and $x \in B_1 \cap B_2$, there exists $B_3 \in \mathcal{B}$ such that $x \in B_3 \subseteq B_1 \cap B_2$.}
    \begin{proof}
      \pf\ Take $B_3 = B_1 \cap B_2$.
    \end{proof}
    \qedstep
    \begin{proof}
      \pf\ Proposition \ref{prop:basis}
    \end{proof}
  \end{proof}
  \step{<1>3}{Every element of $\mathcal{B}$ is open in the order topology.}
  \begin{proof}
    \pf\ We have $(a,b) = (a, + \infty) \cap (- \infty, b)$, and $[\bot, b) = (- \infty, b)$, and $(a, \top] = (a, +\infty)$.
  \end{proof}
  \step{<1>4}{Every open ray is open in the topology generated by $\mathcal{B}$.}
  \begin{proof}
    \step{<2>1}{If $x \in (a, + \infty)$ then there exists $B \in \mathcal{B}$ such that $x \in B \subseteq (a, +\infty)$.}
    \begin{proof}
      \step{<3>1}{\case{$x$ is greatest in $X$}}
      \begin{proof}
        \pf\ Take $B = (a, x]$
      \end{proof}
      \step{<3>2}{\case{$x$ is not greatest in $X$}}
      \begin{proof}
        \pf\ Pick $b$ such that $x < b$ and take $B = (a, b)$.
      \end{proof}
    \end{proof}
    \step{<2>2}{If $x \in (- \infty, b)$ then there exists $B \in \mathcal{B}$ such that $x \in B \subseteq (- \infty, b)$.}
    \begin{proof}
      \pf\ Similar.
    \end{proof}
  \end{proof}
  \qedstep
  \begin{proof}
    \pf\ Using Proposition \ref{prop:basis:coarsest} and \ref{prop:subbasis:coarsest}.
  \end{proof}
  \qed
\end{proof}

\begin{prop}
  \label{prop:order:open_up}
  Let $X$ be a linearly ordered set under the order topology and $U \subseteq X$ be open. If $a \in U$ then either $a$ is greatest in $X$ or there exists $b > a$ such that $[a,b) \subseteq U$.
\end{prop}

\begin{proof}
  \pf\ From definitions.
\end{proof}

\begin{prop}
  \label{prop:order:open_down}
  Let $X$ be a linearly ordered set under the order topology and $U \subseteq X$ be open. If $a \in U$ then either $a$ is least in $X$ or there exists $b < a$ such that $(b,a] \subseteq U$.
\end{prop}

\begin{proof}
  \pf\ From definitions.
\end{proof}

\begin{prop}
  \label{prop:order:supremum_closure}
  Let $X$ be a linearly ordered set in the order topology and $A \subseteq X$ be nonempty. If $A$ has a supremum $s$ then $s \in \overline{A}$.
\end{prop}

\begin{proof}
  \pf
  \step{1}{\pflet{$X$ be a linearly ordered set in the order topology.}}
  \step{2}{\pflet{$A \subseteq X$ be nonempty}}
  \step{3}{\pflet{$s$ be the supremum of $A$.}}
  \step{<2>1}{\pflet{$U$ be an open set containing $s$} \prove{$U$ intersects $A$}}
  \step{<2>2}{\case{$s$ is least in $X$}}
  \begin{proof}
    \step{<3>1}{\pick\ $a \in A$}
    \begin{proof}
      \pf\ $A$ is nonempty (\stepref{2}).
    \end{proof}
    \step{<3>2}{$a \leq s$}
    \begin{proof}
      \pf\ \stepref{3}, \stepref{<3>1}.
    \end{proof}
    \step{<3>3}{$s = a$}
    \begin{proof}
      \pf\ \stepref{<2>2}, \stepref{<3>2}.
    \end{proof}
    \step{<3>4}{$s \in A \cap U$}
    \begin{proof}
      \pf\ \stepref{<2>1}, \stepref{<3>1}, \stepref{<3>3}.
    \end{proof}
  \end{proof}
  \step{<2>3}{\case{$s$ is not least in $X$}}
  \begin{proof}
    \step{<3>1}{\pick\ $l < s$ such that $(l, s] \subseteq U$}
    \begin{proof}
      \pf\ Proposition \ref{prop:order:open_down}, \stepref{1}, \stepref{<2>1},\stepref{<2>3}.
    \end{proof}
    \step{<3>2}{\pick\ $a \in A$ such that $l < a$}
    \begin{proof}
      \pf\ \stepref{3}, \stepref{<3>1}.
    \end{proof}
    \step{<3>3}{$a \in A \cap U$}
    \begin{proof}
      \step{<4>1}{$l < a \leq s$}
      \begin{proof}
        \pf\ \stepref{<3>1}, \stepref{<3>2}.
      \end{proof}
      \step{<4>2}{$l \in U$}
      \begin{proof}
        \pf\ \stepref{<3>1}, \stepref{<4>1}.
      \end{proof}
      \qedstep
      \begin{proof}
        \pf\ With \stepref{<3>2}
      \end{proof}
    \end{proof}
  \end{proof}
\end{proof}

\begin{prop}
  \label{prop:order:closed_greatest}
  Let $X$ be a complete linearly ordered set in the order topology and $A \subseteq X$. If $A$ is nonempty, closed and bounded above then $A$ has a greatest element.
\end{prop}

\begin{proof}
  \pf
  \step{<1>1}{\pflet{$X$ be a complete linearly ordered set in the order topology.}}
  \step{<1>2}{\pflet{$A \subseteq X$}}
  \step{<1>3}{\assume{$A$ is nonempty, closed and bounded above.}}
  \step{<1>4}{\pflet{$s = \sup A$}}
  \begin{proof}
    \pf\ $X$ is complete (\stepref{<1>1}) and $A$ is nonempty and bounded above (\stepref{<1>3}).
  \end{proof}
  \step{<1>5}{$s \in \overline{A}$}
  \begin{proof}
    \pf\ Proposition \ref{prop:order:supremum_closure}, \stepref{<1>1}, \stepref{<1>2}, \stepref{<1>3}, \stepref{<1>4}.
  \end{proof}
  \step{<1>6}{$s \in A$}
  \begin{proof}
    \pf\ Proposition \ref{prop:closure}, \stepref{<1>3}, \stepref{<1>5}.
  \end{proof}
  \qed
\end{proof}

\subsection{The Standard Topology on the Real Line}

\begin{df}[Standard Topology on the Real Line]
  The \emph{standard topology} on the real line $\mathbb{R}$ is the order topology.

  We write $\mathbb{R}$ for the topological space consisting of the real numbers under the standard topology.
\end{df}

\begin{prop}
  The lower limit topology is strictly finer than the standard topology.
\end{prop}

\begin{proof}
  \pf
  \step{<1>1}{The lower limit topology is finer than the standard topology.}
  \begin{proof}
    \step{<2>1}{\pflet{$x \in (a,b)$]}}
    \step{<2>2}{$x \in [x,b) \subseteq (a,b)$]}
    \qedstep
    \begin{proof}
      \pf\ By Proposition \ref{prop:basis:finer}.
    \end{proof}
  \end{proof}
  \step{<1>2}{There exists a set that is open in the lower limit topology but not in the standard topology.}
  \begin{proof}
    \pf\ $[0,1)$ is not open in the standard topology because there is no open interval $(a,b)$ such that $0 \in (a,b) \subseteq [0,1)$.
  \end{proof}
  \qed
\end{proof}

\begin{prop}
  The $K$-topology is strictly finer than the standard topology.
\end{prop}

\begin{proof}
  \pf
  \step{<1>1}{The $K$-topology is finer than the standard topology.}
  \begin{proof}
    \pf\ Every basic open set for the standard topology is a basic open set in the $K$-topology.
  \end{proof}
  \step{<1>2}{There is a set open in the $K$-topology that is not open in the standard topology.}
  \begin{proof}
    \pf\ The set $(-1,1) \setminus K$ is not open in the standard topology because there is no open interval $(a,b)$ such that $0 \in (a,b) \subseteq (-1,1) \setminus K$.
  \end{proof}
  \qed
\end{proof}

\subsection{The Ordered Square}

\begin{df}[Ordered Square]
  The \emph{ordered square} $I_o^2$ is $[0,1]^2$ under the dictionary order topology.
\end{df}

\section{The Product Topology}

\begin{df}[Product Topology]
  Let $\{ X_\alpha \}_{\alpha \in J}$ be a family of topological spaces. The \emph{product topology} on $\prod_{\alpha \in J} X_\alpha$ is the topology generated by the subbasis consisting of all sets $\inv{\pi_\alpha}(U)$ where $\alpha \in J$ and $U$ is open in $X_\alpha$.
\end{df}

\begin{prop}
  \label{prop:product:neighbourhood}
  Let $\{ X_\alpha \}_{\alpha \in J}$ be a family of topological spaces and $(x_\alpha) \in \prod_{\alpha \in J} X_\alpha$. Let $\alpha \in J$. If $M$ is a neighbourhood of $x_\alpha$ in $X_\alpha$ then $\inv{\pi_\alpha}(M)$ is a neighbourhood of $(x_\alpha)$.
\end{prop}

\begin{proof}
  \pf
  \step{<1>1}{\pick\ $U$ open in $X_\alpha$ such that $x_\alpha \in U \subseteq M$}
  \step{<1>2}{$(x_\alpha) \in \inv{\pi_\alpha}(U) \subseteq \inv{\pi_\alpha}(M)$}
  \qed
\end{proof}

\begin{prop}
  Let $\{ X_\alpha \}_{\alpha \in J}$ be a family of topological spaces. Then the product topology is the topology generated by the basis consisting of all sets of the form $\prod_{\alpha \in J} U_\alpha$, where each $U_\alpha$ is open in $X_\alpha$, and $U_\alpha = X_\alpha$ for all but finitely many $\alpha$.
\end{prop}

\begin{proof}
  \pf\ From Proposition \ref{prop:subbasis}. \qed
\end{proof}

\begin{prop}
  \label{prop:product:basis}
  Let $\{ X_\alpha \}_{\alpha \in J}$ be a family of topological spaces and $\mathcal{B}_\alpha$ be a basis for $X_\alpha$ for all $\alpha \in J$. Let $\mathcal{B}$ consisting of all sets of the form $\prod_{\alpha \in J} U_\alpha$ where $U_\alpha \in \mathcal{B}_\alpha$ for finitely many $\alpha$, and
  $U_\alpha = X_\alpha$ for all other $\alpha$. Then $\mathcal{B}$ is a basis for the product topology on $\prod_{\alpha \in J} X_\alpha$.
\end{prop}

\begin{proof}
  \pf
  \step{<1>1}{Every element of $\mathcal{B}$ is open.}
  \begin{proof}
    \pf\ From definitions.
  \end{proof}
  \step{<1>2}{$\bigcup \mathcal{B} = \prod_{\alpha \in J} X_\alpha$}
  \begin{proof}
    \pf\ This holds because $\prod_{\alpha \in J} X_\alpha \in \mathcal{B}$.
  \end{proof}
  \step{<1>3}{For every open $U$ and point $(x_\alpha) \in U$, there exists $B \in \mathcal{B}$ such that $x \in B \subseteq U$.}
  \begin{proof}
    \step{<2>1}{\pick\ $U_\alpha$ for each $\alpha$ such that each $U_\alpha$ is open in $X_\alpha$, $U_\alpha = X_\alpha$ except for $\alpha = \alpha_1, \ldots, \alpha_n$, and $(x_\alpha) \in \prod_{\alpha \in J} U_\alpha \subseteq U$}
    \step{<2>2}{For $1 \leq i \leq n$ \pick\ $B_i \in \mathcal{B}_{\alpha_i}$ such that $x_{\alpha_i} \in B_i \subseteq U_{\alpha_i}$}
    \step{<2>3}{$(x_\alpha) \in \prod_{\alpha \in J} V_\alpha \subseteq U$, where $V_{\alpha_i} = B_i$ for $1 \leq i \leq n$, and $V_\alpha = X_\alpha$ for all other $\alpha$.}
  \end{proof}
\end{proof}

\begin{prop}
  \label{prop:product:closed}
  Let $\{ X_\alpha \}_{\alpha \in J}$ be a family of topological spaces and $C_\alpha$ be closed in $X_\alpha$ for all $\alpha \in J$. Then $\prod_{\alpha \in J} C_\alpha$ is closed in $\prod_{\alpha \in J} X_\alpha$.
\end{prop}

\begin{proof}
  \pf\ This holds because $\prod_{\alpha \in J} X_\alpha \setminus \prod_{\alpha \in J} C_\alpha = \bigcup_{\alpha \in J} \inv{\pi_\alpha} (X_\alpha \setminus C_\alpha)$. \qed
\end{proof}

\begin{prop}[AC]
  \label{prop:product:closure}
  Let $\{ X_\alpha \}_{\alpha \in J}$ be a family of topological spaces and $A_\alpha \subseteq X_\alpha$ for all $\alpha \in J$. Then
  \[ \prod_{\alpha \in J} \overline{A_\alpha} = \overline{\prod_{\alpha \in J} A_\alpha} \enspace . \]

  (This requires the Axiom of Countable Choice if $J$ is countably infinite, and the Axiom of Choice if $J$ is uncountable.)
\end{prop}

\begin{proof}
  \pf
  \step{<1>1}{$\prod_{\alpha \in J} \overline{A_\alpha} \subseteq \overline{\prod_{\alpha \in J} A_\alpha}$}
  \begin{proof}
    \step{<2>1}{\pflet{$(x_\alpha) \in \prod_{\alpha \in J} \overline{A_\alpha}$}}
    \step{<2>2}{\pflet{$U_\alpha$ be open in $X_\alpha$ for all $\alpha$ with $U_\alpha = X_\alpha$ for all $\alpha$ except $\alpha_1$, \ldots, $\alpha_n$ and $(x_\alpha) \in \prod_{\alpha \in J} U_\alpha$}}
    \step{<2>3}{For all $\alpha$ we have $x_\alpha \in \overline{A_\alpha}$}
    \begin{proof}
      \pf\ From \stepref{<2>1}
    \end{proof}
    \step{<2>4}{For all $\alpha$ we have $x_\alpha \in U_\alpha$}
    \begin{proof}
      \pf\ From \stepref{<2>2}
    \end{proof}
    \step{<2>5}{For all $\alpha$ we have $U_\alpha$ intersects $A_\alpha$}
    \begin{proof}
      \pf\ Proposition \ref{prop:closure:membership}, \stepref{<2>3}, \stepref{<2>4}
    \end{proof}
    \step{<2>6}{$\prod_{\alpha \in J} U_\alpha$ intersects $\prod_{\alpha \in J} A_\alpha$}
    \begin{proof}
      \pf\ From \stepref{<2>5} using the Axiom of Choice.
    \end{proof}
    \qedstep
    \begin{proof}
      \pf\ Proposition \ref{prop:basis:closure_membership}.
    \end{proof}
  \end{proof}
  \step{<1>2}{$\overline{\prod_{\alpha \in J} A_\alpha} \subseteq \prod_{\alpha \in J} \overline{A_\alpha}$}
  \begin{proof}
    \step{<2>1}{\pflet{$(x_\alpha) \in \overline{\prod_{\alpha \in J} A_\alpha}$}}
    \step{<2>2}{\pflet{$\alpha \in J$} \prove{$x_\alpha \in \overline{A_\alpha}$}}
    \step{<2>3}{\pflet{$N$ be a neighbourhood of $x_\alpha$ in $X_\alpha$} \prove{$N$ intersects $A_\alpha$}}
    \step{<2>4}{\pflet{$M = \inv{\pi_\alpha}(N)$}}
    \step{<2>5}{$M$ is a neighbourhood of $(x_\alpha)$}
    \begin{proof}
      \pf\ Proposition \ref{prop:product:neighbourhood}, \stepref{<2>3}, \stepref{<2>4}
    \end{proof}
    \step{<2>6}{$M$ intersects $\prod_{\alpha \in J} A_\alpha$}
    \begin{proof}
      \pf\ Proposition \ref{prop:closure:membership}, \stepref{<2>1}, \stepref{<2>5}.
    \end{proof}
    \step{<2>7}{\pick\ $(y_\alpha) \in M \cap \prod_{\alpha \in J} A_\alpha$}
    \begin{proof}
      \pf\ \stepref{<2>6}
    \end{proof}
    \step{<2>8}{$y_\alpha \in N \cap A_\alpha$}
    \begin{proof}
      \pf\ \stepref{<2>4}, \stepref{<2>7}
    \end{proof}
    \qedstep
    \begin{proof}
      \pf\ Proposition \ref{prop:closure:membership}
    \end{proof}
  \end{proof}
  \qed
\end{proof}

\section{The Subspace Topology}

\begin{df}[Subspace Topology]
  Let $(X, \mathcal{T})$ be a topological space and $Y \subseteq X$. The \emph{subspace topology} on $Y$ is $\mathcal{T}' = \{ U \cap Y : U \in \mathcal{T} \}$.

  A \emph{subspace} of $(X, \mathcal{T})$ is a topological space consisting of a subset of $X$ under the subspace topology.

  We prove this is a topology.
\end{df}

\begin{proof}
  \pf
  \step{<1>1}{$\emptyset \in \mathcal{T}'$}
  \begin{proof}
    \pf\ This holds because $\emptyset = \emptyset \cap Y$.
  \end{proof}
  \step{<1>2}{$Y \in \mathcal{T}'$}
  \begin{proof}
    \pf\ This holds because $Y = X \cap Y$.
  \end{proof}
  \step{<1>3}{For all $\mathcal{U} \subseteq \mathcal{T}'$ we have $\bigcup \mathcal{U} \in \mathcal{T}'$}
  \begin{proof}
    \pf\ This holds because $\bigcup \mathcal{U} = \bigcup \{ U \in \mathcal{T} : U \cap Y \in \mathcal{U} \} \cap Y$.
  \end{proof}
  \step{<1>4}{For all $U, V \in \mathcal{T}'$ we have $U \cap V \in \mathcal{T}'$}
  \begin{proof}
    \step{<2>1}{\pick\ $U' \in \mathcal{T}$ and $V' \in \mathcal{T}$ such that $U = U' \cap Y$ and $V = V' \cap Y$}
    \step{<2>2}{$U \cap V = U' \cap V' \cap Y$}
  \end{proof}
  \qed
\end{proof}

\begin{prop}
  Let $X$ be a topological space, $Y \subseteq X$ and $Z \subseteq Y$. The subspace topology on $Z$ as a subspace of $Y$ is the same as the subspace topology on $Z$ as a subspace of $X$.
\end{prop}

\begin{proof}
  \pf
  The subspace topology on $Z$ as a subspace of $Y$ is
  \begin{align*}
    \{ V \cap Z : V \text{ open in } Y \} & = \{ (U \cap Y) \cap Z : U \text{ open in } X \} \\
    & = \{ U \cap Z : U \text{ open in } X \} \enspace .
  \end{align*}
  \qed
\end{proof}

\begin{prop}
  \label{prop:subspace:open}
  Let $Y$ be a subspace of $X$. If $U$ is open in $Y$ and $Y$ is open in $X$ then $U$ is open in $X$.
\end{prop}

\begin{proof}
  \pf\ There exists $V$ open in $X$ such that $U = V \cap Y$. \qed
\end{proof}

\begin{prop}
  \label{prop:subspace:closed}
  Let $X$ be a topological space, $Y$ a subspace of $X$, and $A \subseteq Y$. Then $A$ is closed in $Y$ if and only if there exists a closed $C$ in $X$ such that $A = C \cap Y$.
\end{prop}

\begin{proof}
  \pf
  \step{<1>1}{If $A$ is closed in $Y$ then there exists $C$ closed in $X$ such that $A = C \cap Y$}
  \begin{proof}
    \step{<2>1}{\assume{$A$ is closed in $Y$}}
    \step{<2>2}{\pick\ $U$ open in $X$ such that $Y \setminus A = U \cap Y$}
    \step{<2>3}{\pflet{$C = X \setminus U$}}
    \step{<2>4}{$A = C \cap Y$}
  \end{proof}
  \step{<1>2}{For all $C$ closed in $X$, we have $C \cap Y$ is closed in $Y$.}
  \begin{proof}
    \step{<2>1}{\pflet{$C$ be closed in $X$}}
    \step{<2>2}{$Y \setminus (C \cap Y)$ is open in $Y$}
    \begin{proof}
      \pf\ $Y \setminus (C \cap Y) = (X \setminus C) \cap Y$
    \end{proof}
  \end{proof}
  \qed
\end{proof}

\begin{cor}
  Let $X$ be a topological space, $Y$ a subspace of $X$ and $C \subseteq Y$. If $C$ is closed in $Y$ and $Y$ is closed in $X$ then $C$ is closed in $X$.
\end{cor}

\begin{proof}
  \pf\ There exists $D$ closed in $X$ such that $C = D \cap Y$. \qed
\end{proof}

\begin{prop}
  Let $X$ be a topological space, $Y$ a subspace of $X$ and $A \subseteq Y$. Let $\overline{A}$ be the closure of $A$ in $X$. Then the closure of $A$ in $Y$ is $\overline{A} \cap Y$.
\end{prop}

\begin{proof}
  \pf
  \step{<1>1}{$\overline{A} \cap Y$ is closed in $Y$}
  \begin{proof}
    \pf\ Proposition \ref{prop:subspace:closed}.
  \end{proof}
  \step{<1>2}{$\overline{A} \cap Y \subseteq A$}
  \step{<1>3}{If $C$ is a closed set in $Y$ and $C \subseteq A$ then $C \subseteq \overline{A} \cap Y$.}
  \begin{proof}
    \step{<2>1}{\pick\ $D$ closed in $X$ such that $C = D \cap Y$}
    \begin{proof}
      \pf\ Proposition \ref{prop:subspace:closed}.
    \end{proof}
    \step{<2>2}{$D \subseteq \overline{A}$}
    \step{<2>3}{$C \subseteq \overline{A} \cap Y$}
  \end{proof}
  \qed
\end{proof}

\begin{prop}
  \label{prop:subspace:neighbourhood}
  Let $X$ be a topological space, $Y$ a subspace of $X$, $N \subseteq Y$ and $y \in Y$. Then $N$ is a neighbourhood of $y$ in $Y$ iff there exists a neighbourhood $M$ of $y$ in $X$ such that $N = M \cap Y$.
\end{prop}

\begin{proof}
  \pf
  \step{<1>1}{If $N$ is a neighbourhood of $y$ in $Y$ then there exists a neighbourhood $M$ of $y$ in $X$ such that $N = M \cap Y$}
  \begin{proof}
    \step{<2>1}{\assume{$N$ is a neighbourhood of $y$ in $Y$}}
    \step{<2>2}{\pick\ a set $U$ open in $Y$ such that $y \in U \subseteq N$}
    \step{<2>3}{\pick\ a set $V$ open in $X$ such that $U = V \cap Y$}
    \step{<2>4}{Take $M = V \cup N$}
  \end{proof}
  \step{<1>2}{For any neighbourhood $M$ of $y$ in $X$, we have $M \cap Y$ is a neighbourhood of $y$ in $Y$.}
  \begin{proof}
    \step{<2>1}{\pick\ $U$ open in $X$ such that $y \in U \subseteq M$}
    \step{<2>2}{$y \in U \cap Y \subseteq M \cap Y$}
  \end{proof}
  \qed
\end{proof}

\begin{prop}
  \label{prop:subspace:basis}
  Let $X$ be a topological space and $Y$ a subspace of $X$. If $\mathcal{B}$ is a basis for the topology on $X$ then $\mathcal{B}' = \{ B \cap Y : B \in \mathcal{B} \}$ is a basis for the subspace topology on $Y$.
\end{prop}

\begin{proof}
  \pf
  \step{<1>1}{Every element of $\mathcal{B}'$ is open.}
  \begin{proof}
    \pf\ Immediate from definitions.
  \end{proof}
  \step{<1>2}{For every open $U$ in $Y$ and $y \in U$ there exists $B' \in \mathcal{B}'$ such that $y \in B' \subseteq U$.}
  \begin{proof}
    \step{<2>1}{\pick\ $V$ open in $X$ such that $U = V \cap Y$}
    \step{<2>2}{\pick\ $B \in \mathcal{B}$ such that $y \in B \subseteq V$}
    \step{<2>3}{Take $B' = B \cap Y$}
  \end{proof}
  \qed
\end{proof}

\begin{prop}
  \label{prop:subspace:subbasis}
  Let $X$ be a topological space and $Y$ a subspace of $X$. If $\mathcal{S}$ is a subbasis for the topology on $X$ then $\mathcal{S}' = \{ S \cap Y : S \in \mathcal{S} \}$ is a subbasis for the subspace topology on $Y$.
\end{prop}

\begin{proof}
  \pf
  \step{<1>1}{$\{ S_1 \cap \cdots \cap S_n \cap Y : S_1, \ldots, S_n \in \mathcal{S} \}$ is a basis for the subspace topology.}
  \begin{proof}
    \pf\ Proposition \ref{prop:subspace:basis}.
  \end{proof}
  \step{<1>2}{Every element of $\mathcal{S}'$ is open in the subspace topology.}
  \begin{proof}
    \pf\ Immediate from definitions.
  \end{proof}
  \step{<1>3}{For all $S_1, \ldots, S_n \in \mathcal{S}$ we have $S_1 \cap \cdots \cap S_n \cap Y$ is a union of finite intersections of elements of $\mathcal{S}'$}
  \begin{proof}
    \pf\ $S_1 \cap \cdots \cap S_n \cap Y = (S_1 \cap Y) \cap \cdots \cap (S_n \cap Y)$.
  \end{proof}
  \qedstep
  \begin{proof}
    \pf\ Proposition \ref{prop:subbasis:from_basis}.
  \end{proof}
  \qed
\end{proof}

\begin{prop}
  Let $\{ X_\alpha \}_{\alpha \in J}$ be a family of topological spaces and $A_\alpha \subseteq X_\alpha$ for all $\alpha \in J$. Then the product topology on $\prod_{\alpha \in J} A_\alpha$ is the same as the topology that $\prod_{\alpha \in J} A_\alpha$ inherits as a subspace of $\prod_{\alpha \in J} X_\alpha$.
\end{prop}

\begin{proof}
  \pf
  \step{<1>1}{\pflet{$\{ X_\alpha \}_{\alpha \in J}$ be a family of topological spaces and $A_\alpha \subseteq X_\alpha$ for all $\alpha \in J$}}
  \step{<1>2}{\pflet{$\pi_\alpha : \prod_{\alpha \in J} X_\alpha \rightarrow X_\alpha$ be the $\alpha$th projection on $\prod_{\alpha \in J} X_\alpha$}}
  \step{<1>3}{\pflet{$p_\alpha : \prod_{\alpha \in J} A_\alpha \rightarrow A_\alpha$ be the $\alpha$th projection on $\prod_{\alpha \in J} A_\alpha$}}
  \step{<1>4}{the product topology on $\prod_{\alpha \in J} A_\alpha$ is the same as the topology that $\prod_{\alpha \in J} A_\alpha$ inherits as a subspace of $\prod_{\alpha \in J} X_\alpha$.}
  \begin{proof}
    \pf\ Each is the topology generated by the subbasis
    \begin{align*}
      \{ \inv{p_\alpha}(U) : \alpha \in J, U \text{ open in } A_\alpha \}
      & = \{ \inv{p_\alpha}(U \cap A_\alpha) : \alpha \in J, U \text{ open in } X_\alpha \}
      & = \{ \inv{\pi_\alpha}(U) \cap \prod_{\alpha \in J} A_\alpha : \alpha \in J, U \text{ open in } X_\alpha \}
    \end{align*}
    using Proposition \ref{prop:subspace:subbasis}.
  \end{proof}
  \qed
\end{proof}

\begin{prop}
  Let $X$ be a linearly ordered set under the order topology. Let $Y \subseteq X$ be convex. Then the order topology on $Y$ is the same as the subspace topology.
\end{prop}

\begin{proof}
  \pf
  \step{<1>1}{\pflet{$\mathcal{T}_o$ be the order topology on $Y$ and $\mathcal{T}_s$ the subspace topology.}}
  \step{<1>2}{$\mathcal{T}_o \subseteq \mathcal{T}_s$}
  \begin{proof}
    \step{<2>1}{For all $a \in Y$ we have $\{ x \in Y : x < a \} \in \mathcal{T}_s$}
    \begin{proof}
      \pf\ The set is $(- \infty, a) \cap Y$.
    \end{proof}
    \step{<2>2}{For all $a \in Y$ we have $\{ x \in Y : x > a \} \in \mathcal{T}_s$}
    \begin{proof}
      \pf\ The set is $(a, + \infty) \cap Y$.
    \end{proof}
    \qedstep
    \begin{proof}
      \pf\ Proposition \ref{prop:subbasis:coarsest}.
    \end{proof}
  \end{proof}
  \step{<1>3}{$\mathcal{T}_s \subseteq \mathcal{T}_o$}
  \begin{proof}
    \step{<2>1}{For all $a \in X$ we have $(- \infty, a) \cap Y \in \mathcal{T}_o$}
    \begin{proof}
      \step{<3>1}{\case{$a \in Y$}}
      \begin{proof}
        \pf\ In this case, $(- \infty, a) \cap Y = \{ x \in Y : x < a \}$.
      \end{proof}
      \step{<3>2}{\case{$a$ is less than every element of $Y$}}
      \begin{proof}
        \pf\ In this case, $(- \infty, a) \cap Y = \emptyset$.
      \end{proof}
      \step{<3>3}{\case{$a$ is greater than every element of $Y$}}
      \begin{proof}
        \pf\ In this case, $(- \infty, a) \cap Y = Y$.
      \end{proof}
      \qedstep
      \begin{proof}
        \pf\ These are the only three possibilities because $Y$ is convex.
      \end{proof}
    \end{proof}
    \step{<2>2}{For all $a \in X$ we have $(a, +\infty) \cap Y \in \mathcal{T}_o$}
    \begin{proof}
      \pf\ Similar.
    \end{proof}
    \qedstep
    \begin{proof}
      \pf\ Proposition \ref{prop:subbasis:coarsest}.
    \end{proof}
  \end{proof}
  \qed
\end{proof}

\subsection{Unit Sphere}

\begin{df}[Unit $n$-sphere]
  For $n \geq 1$, the \emph{unit $n$-sphere} $S^n$ is the space $\{ (x_1, \ldots, x_{n+1}) : x_1^2 + \cdots + x_{n+1}^2 = 1 \}$ as a subspace of $\mathbb{R}^{n+1}$.
\end{df}

\subsection{Unit Ball}

\begin{df}[Unit $n$-ball]
  For $n \geq 1$, the \emph{unit $n$-ball} $B^n$ is the space $\{ (x_1, \ldots, x_{n+1}) : x_1^2 + \cdots + x_{n+1}^2 \leq 1 \}$ as a subspace of $\mathbb{R}^{n+1}$.
\end{df}

\subsection{Punctured Euclidean Space}

\begin{df}[Punctured Euclidean Space]
  The space \emph{$n$-dimensional punctured Euclidean space} is $\mathbb{R}^n \setminus \{ 0 \}$.
\end{df}

\subsection{Topologist's Sine Curve}

\begin{df}[Topologist's Sine Curve]
  The \emph{topologist's sine curve} is the closure of $\{ (x, \sin 1/x) : 0 < x \leq 1 \}$ in $\mathbb{R}^2$.
\end{df}

\section{The Box Topology}

\begin{df}[Box Topology]
  Let $\{ X_\alpha \}_{\alpha \in J}$ be a family of topological spaces. The \emph{box topology} on $\prod_{\alpha \in J} X_\alpha$ is the topology generated by the basis $\mathcal{B}$ consisting of all sets of the form $\prod_{\alpha \in J} U_\alpha$ where each $U_\alpha$ is open in $X_\alpha$.

  We prove $\mathcal{B}$ is a basis for a topology.
\end{df}

\begin{proof}
  \pf
  \step{<1>1}{$\bigcup \mathcal{B} = \prod_{\alpha \in J} X_\alpha$}
  \begin{proof}
    \pf\ This holds because $\prod_{\alpha \in J} X_\alpha \in \mathcal{B}$.
  \end{proof}
  \step{<1>2}{For all $B_1, B_2 \in \mathcal{B}$ and $(x_\alpha) \in B_1 \cap B_2$ there exists $B_3 \in \mathcal{B}$ such that $(x_\alpha) \in B_3 \subseteq B_1 \cap B_2$}
  \begin{proof}
    \pf\ If $B_1 = \prod_{\alpha \in J} U_\alpha$ and $B_2 = \prod_{\alpha \in J} V_\alpha$ take $B_3 = \prod_{\alpha \in J} (U_\alpha \cap V_\alpha)$.
  \end{proof}
  \qedstep
  \begin{proof}
    \pf\ Proposition \ref{prop:basis}
  \end{proof}
  \qed
\end{proof}

\begin{prop}
  \label{prop:box:neighbourhood}
  Let $\{ X_\alpha \}_{\alpha \in J}$ be a family of topological spaces and $(x_\alpha) \in \prod_{\alpha \in J} X_\alpha$ under the box topology. Let $\alpha \in J$. If $M$ is a neighbourhood of $x_\alpha$ in $X_\alpha$ then $\inv{\pi_\alpha}(M)$ is a neighbourhood of $(x_\alpha)$.
\end{prop}

\begin{proof}
  \pf
  \step{<1>1}{\pick\ $U$ open in $X_\alpha$ such that $x_\alpha \in U \subseteq M$}
  \step{<1>2}{$(x_\alpha) \in \inv{\pi_\alpha}(U) \subseteq \inv{\pi_\alpha}(M)$}
  \qed
\end{proof}

\begin{prop}[AC]
  Let $\{ X_\alpha \}_{\alpha \in J}$ be a family of topological spaces. Suppose $\mathcal{B}_\alpha$ is a basis for the topology on $X_\alpha$ for all $\alpha \in J$. Then $\{ \prod_{\alpha \in J} B_\alpha : \forall \alpha \in J. B_\alpha \in \mathcal{B}_\alpha \}$ is a basis for the box topology on $\prod_{\alpha \in J} X_\alpha$.
\end{prop}

\begin{proof}
  \pf
  \step{<1>1}{\pflet{$\{ X_\alpha \}_{\alpha \in J}$ be a family of topological spaces.}}
  \step{<1>2}{\pflet{$\mathcal{B}_\alpha$ be a basis for the topology on $X_\alpha$ for all $\alpha \in J$.}}
  \step{<1>3}{\pflet{$\mathcal{B} = \{ \prod_{\alpha \in J} B_\alpha : \forall \alpha \in J. B_\alpha \in \mathcal{B}_\alpha \}$}}
  \step{<1>4}{Every element of $\mathcal{B}$ is open.}
  \begin{proof}
    \pf\ From \stepref{<1>2}, \stepref{<1>3}
  \end{proof}
  \step{<1>5}{For every open set $U$ and point $(x_\alpha) \in U$, there exists $B \in \mathcal{B}$ such that $(x_\alpha) \in B \subseteq U$.}
  \begin{proof}
    \step{<2>1}{\pflet{$U$ be open and $(x_\alpha) \in U$}}
    \step{<2>2}{\pick\ $U_\alpha$ open in $X_\alpha$ for all $\alpha$ such that $(x_\alpha) \in \prod_{\alpha \in J} U_\alpha \subseteq U$}
    \begin{proof}
      \pf\ \stepref{<2>2}
    \end{proof}
    \step{<2>3}{For $\alpha \in J$, \pick\ $B_\alpha \in \mathcal{B}_\alpha$ such that $x_\alpha \in B_\alpha \subseteq U_\alpha$}
    \begin{proof}
      \pf\ \stepref{<1>2}, \stepref{<2>2}
    \end{proof}
    \step{<2>4}{$(x_\alpha) \in \prod_{\alpha \in J} B_\alpha \subseteq U$}
    \begin{proof}
      \pf\ \stepref{<2>2}, \stepref{<2>3}
    \end{proof}
  \end{proof}
  \qed
\end{proof}

\begin{prop}
  Let $\{ X_\alpha \}_{\alpha \in J}$ be a family of topological spaces and $A_\alpha \subseteq X_\alpha$ for all $\alpha \in J$. Then the box topology on $\prod_{\alpha \in J} A_\alpha$ is the same as the topology that $\prod_{\alpha \in J} A_\alpha$ inherits as a subspace of $\prod_{\alpha \in J} X_\alpha$ under the box topology.
\end{prop}

\begin{proof}
  \pf
  Each is the topology generated by the basis
  \begin{align*}
    & \{ \prod_{\alpha \in J} U_\alpha : \forall \alpha \in J. U_\alpha \text{ open in } A_\alpha \}  \\
    = & \{ \prod_{\alpha \in J} (V_\alpha \cap A_\alpha) : \forall \alpha \in J. V_\alpha \text{ open in } X_\alpha \} \\
    = & \{ \prod_{\alpha \in J} V_\alpha \cap \prod_{\alpha \in J} A_\alpha : \forall \alpha \in J. V_\alpha \text{ open in } X_\alpha \}
  \end{align*}
  using Proposition \ref{prop:subspace:basis}. \qed
\end{proof}

\begin{prop}[AC]
  Let $\{ X_\alpha \}_{\alpha \in J}$ be a family of topological spaces and $A_\alpha \subseteq X_\alpha$ for all $\alpha \in J$. Then
  \[ \prod_{\alpha \in J} \overline{A_\alpha} = \overline{\prod_{\alpha \in J} A_\alpha} \]
  under the box topology.
\end{prop}

\begin{proof}
  \pf
  \step{<1>1}{$\prod_{\alpha \in J} \overline{A_\alpha} \subseteq \overline{\prod_{\alpha \in J} A_\alpha}$}
  \begin{proof}
    \step{<2>1}{\pflet{$(x_\alpha) \in \prod_{\alpha \in J} \overline{A_\alpha}$}}
    \step{<2>2}{\pflet{$U_\alpha$ be open in $X_\alpha$ for all $\alpha$ with $(x_\alpha) \in \prod_{\alpha \in J} U_\alpha$}}
    \step{<2>3}{For all $\alpha$ we have $x_\alpha \in \overline{A_\alpha}$}
    \begin{proof}
      \pf\ From \stepref{<2>1}
    \end{proof}
    \step{<2>4}{For all $\alpha$ we have $x_\alpha \in U_\alpha$}
    \begin{proof}
      \pf\ From \stepref{<2>2}
    \end{proof}
    \step{<2>5}{For all $\alpha$ we have $U_\alpha$ intersects $A_\alpha$}
    \begin{proof}
      \pf\ Proposition \ref{prop:closure:membership}, \stepref{<2>3}, \stepref{<2>4}
    \end{proof}
    \step{<2>6}{$\prod_{\alpha \in J} U_\alpha$ intersects $\prod_{\alpha \in J} A_\alpha$}
    \begin{proof}
      \pf\ From \stepref{<2>5} using the Axiom of Choice.
    \end{proof}
    \qedstep
    \begin{proof}
      \pf\ Proposition \ref{prop:basis:closure_membership}.
    \end{proof}
  \end{proof}
  \step{<1>2}{$\overline{\prod_{\alpha \in J} A_\alpha} \subseteq \prod_{\alpha \in J} \overline{A_\alpha}$}
  \begin{proof}
    \step{<2>1}{\pflet{$(x_\alpha) \in \overline{\prod_{\alpha \in J} A_\alpha}$}}
    \step{<2>2}{\pflet{$\alpha \in J$} \prove{$x_\alpha \in \overline{A_\alpha}$}}
    \step{<2>3}{\pflet{$N$ be a neighbourhood of $x_\alpha$ in $X_\alpha$} \prove{$N$ intersects $A_\alpha$}}
    \step{<2>4}{\pflet{$M = \inv{\pi_\alpha}(N)$}}
    \step{<2>5}{$M$ is a neighbourhood of $(x_\alpha)$}
    \begin{proof}
      \pf\ Proposition \ref{prop:box:neighbourhood}, \stepref{<2>3}, \stepref{<2>4}
    \end{proof}
    \step{<2>6}{$M$ intersects $\prod_{\alpha \in J} A_\alpha$}
    \begin{proof}
      \pf\ Proposition \ref{prop:closure:membership}, \stepref{<2>1}, \stepref{<2>5}.
    \end{proof}
    \step{<2>7}{\pick\ $(y_\alpha) \in M \cap \prod_{\alpha \in J} A_\alpha$}
    \begin{proof}
      \pf\ \stepref{<2>6}
    \end{proof}
    \step{<2>8}{$y_\alpha \in N \cap A_\alpha$}
    \begin{proof}
      \pf\ \stepref{<2>4}, \stepref{<2>7}
    \end{proof}
    \qedstep
    \begin{proof}
      \pf\ Proposition \ref{prop:closure:membership}
    \end{proof}
  \end{proof}
  \qed
\end{proof}
