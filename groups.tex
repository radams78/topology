\chapter{Topological Groups}

\begin{df}[Topological Group]
  A \emph{topological group} $G$ is a group with a $T_1$-topology such that the multiplication map $\cdot : G^2 \rightarrow G$ and inverse map $\inv{( )} : G \rightarrow G$ are continuous.
\end{df}

\begin{prop}
  Every subgroup of a topological group is a topological group under the subspace topology.
\end{prop}

\begin{proof}
  \pf\ Easy.
\end{proof}

\begin{prop}
  Every topological group is homogeneous.
\end{prop}

\begin{proof}
  \pf\ Let $G$ be a topological group and $x, y \in G$. Then $\lambda z. y \inv{x} z$ is a homeomorphism that maps $x$ to $y$. \qed
\end{proof}

\begin{prop}
  Let $G$ be a topological group and $H$ be a subgroup of $G$. Then $\overline{H}$ is a topological group under the subspace topology.
\end{prop}

\begin{proof}
  \pf
  \step{<1>1}{$\overline{H}$ is a subgroup of $G$.}
  \begin{proof}
    \step{<2>1}{\pflet{$f : G^2 \rightarrow G$, $f(x,y) = x\inv{y}$}}
    \step{<2>2}{$f(\overline{H} \times \overline{H}) \subseteq \overline{H}$}
    \begin{proof}
      \pf
      \begin{align*}
        f(\overline{H} \times \overline{H}) & = f(\overline{H \times H}) & (\text{Proposition \ref{prop:product:closure}})\\
& \subseteq \overline{f(H \times H)} & (\text{Proposition \ref{prop:continuous:closure}})\\
& \subseteq \overline{H} & (\text{$H$ is a subgroup of $G$})
\end{align*}
\end{proof}
  \end{proof}
  \qed
\end{proof}

\begin{prop}
  Let $G$ be a topological group and $H \leq G$. Give $G/H$ the quotient topology induced by the canonical map $G \rightarrow G / H$. Then $G / H$ is homogeneous.
\end{prop}

\begin{proof}
  \pf
  \step{<1>1}{\pflet{$aH, bH \in G/H$}}
  \step{<1>2}{\pflet{$\phi : G/H \rightarrow G/H$ be the function $\phi(xH) = (b\inv{a}xH)$}}
  \begin{proof}
    \step{<2>1}{\pflet{$xH = yH$} \prove{$b\inv{a}xH = b\inv{a}yH$}}
    \step{<2>2}{$\inv{x}y \in H$}
    \step{<2>3}{$\inv{(b \inv{a} x)} b \inv{a} y \in H$}
    \begin{proof}
      \pf\
      \begin{align*}
        \inv{(b \inv{a} x)} b \inv{a} y & = \inv{x} a \inv{b} b \inv{a} y \\
        & = \inv{x} y
      \end{align*}
    \end{proof}
  \end{proof}
  \step{<1>3}{$\phi$ is a homeomorphism.}
  \qed
\end{proof}

\begin{prop}
  \label{prop:group:quotient_T1}
  Let $G$ be a topological group and $H$ a closed subgroup of $G$. Give $G/H$ the quotient topology induced by the canonical map $\pi : G \rightarrow G / H$. Then $G / H$ is $T_1$.
\end{prop}

\begin{proof}
  \pf\ Let $a \in G$. Then $\inv{\pi}(\{aH\}) = aH = f_a(H)$ where $f_a(x) = ax$ for $x \in G$. This set is closed, because $f_a$ is a homeomorphism. Hence $\{ aH \}$ is closed in $G / H$.
\end{proof}

\begin{prop}
  \label{prop:group:quotient_open}
  Let $G$ be a topological group and $H$ a subgroup of $G$. Give $G / H$ the quotient topology induced by the canonical map $\pi : G \rightarrow G / H$. Then $\pi$ is an open map.
\end{prop}

\begin{proof}
  \pf\ This holds because, for $U$ open in $G$,
  \begin{align*}
    \inv{\pi}(\pi(U)) & = \{ xh : x \in U, h \in H \} \\
    & = \bigcup_{h \in H} \phi_h(U)
  \end{align*}
  where $\phi_h$ is the automorphism that maps $x$ to $xh$. \qed
\end{proof}

\begin{prop}
  Let $G$ be a topological group and $H$ a closed normal subgroup of $G$.
  Give $G / H$ the quotient topology induced by the canonical map $\pi : G \rightarrow G / H$. Then $G / H$ is a topological group.
\end{prop}

\begin{proof}
  \pf
  \step{<1>1}{$G / H$ is $T_1$.}
  \begin{proof}
    \pf\ Proposition \ref{prop:group:quotient_T1}.
  \end{proof}
  \step{<1>2}{Multiplication in $G / H$ is continuous.}
  \begin{proof}
    \step{<2>1}{\pflet{$m : G^2 \rightarrow G$ be multiplication in $G$}}
    \step{<2>2}{\pflet{$n : (G / H)^2 \rightarrow G / H$ be multiplication in $G / H$}}
    \step{<2>3}{$n \circ (\pi \times \pi) = \pi \circ m : G^2 \rightarrow G / H$}
    \step{<2>4}{$n \circ (\pi \times \pi)$ is continuous.}
    \begin{proof}
      \pf\ $n \circ (\pi \times \pi) = \pi \circ m$ and $\pi$ and $m$ are both continuous.
    \end{proof}
    \step{<2>5}{$\pi$ is an open map}
    \begin{proof}
      \pf\ Proposition \ref{prop:group:quotient_open}.
    \end{proof}
    \step{<2>6}{$\pi \times \pi$ is strongly continuous.}
    \begin{proof}
      \pf\ Proposition \ref{prop:strongly_continuous:product}.
    \end{proof}
    \step{<2>7}{$n$ is continuous.}
    \begin{proof}
      \pf\ Proposition \ref{prop:strongly_continuous:continuous}.
    \end{proof}
  \end{proof}
  \step{<1>3}{The inverse map in $G / H$ is continuous.}
  \begin{proof}
    \pf\ Similar.
  \end{proof}
  \qed
\end{proof}

\begin{df}[Symmetric Neighbourhood of $e$]
  A neighbourhood $U$ of $e$ is \emph{symmetric} iff $U = \inv{U}$.
\end{df}

\begin{prop}
  \label{prop:group:symmetric_neighbourhood}
  For every neighbourhood $U$ of $e$, there exists a symmetric neighbourhood $V$ of $e$ such that $V \cdot V \subseteq U$.
\end{prop}

\begin{proof}
  \pf
  \step{<1>1}{\pick\ a neighbourhood $V'$ of $e$ such that $V' \cdot V' \subseteq U$}
  \begin{proof}
    \step{<2>1}{\pick\ a neighbourhood $N$ of $(e, e)$ in $G \times G$ such that, for all $x, y \in N$ we have $xy \in U$}
    \begin{proof}
      \pf\ Since multiplication is continuous.
    \end{proof}
    \step{<2>2}{\pick\ neighbourhoods $V_1$, $V_2$ of $e$ such that $V_1 \times V_2 \subseteq N$}
    \step{<2>3}{\pflet{$V' = V_1 \cap V_2$}}
  \end{proof}
  \step{<1>2}{\pick\ a neighbourhood $W$ of $e$ such that $W \cdot \inv{W} \subseteq V'$}
  \begin{proof}
    \pf\ Similar, since $\lambda x,y.x \inv{y}$ is continuous.
  \end{proof}
  \step{<1>3}{\pflet{$V = W \cdot \inv{W}$}}
  \step{<1>4}{$V$ is a neighbourhood of $e$}
  \begin{proof}
    \step{<2>1}{$e \in V$}
    \begin{proof}
      \pf\ This holds because $e \in W$ and $e = e \inv{e}$.
    \end{proof}
    \step{<2>2}{\pick\ an open set $U_1$ such that $e \in U_1 \subseteq W$}
    \step{<2>3}{$U_1 \cdot \inv{U_1}$ is open}
    \begin{proof}
      \pf\ This holds because $U_1 \cdot \inv{U_1} = \bigcup_{x \in U_1} U_1 \inv{x}$, and each $U_1 \inv{x}$ is open since it is an automorphic image of $U_1$.
    \end{proof}
    \step{<2>4}{$e \in U_1 \cdot \inv{U_1} \subseteq V$}
  \end{proof}
  \step{<1>5}{$V$ is symmetric}
  \begin{proof}
    \step{<2>1}{\pflet{$x \in V$}}
    \step{<2>2}{\pick\ $y, z \in W$ such that $x = y \inv{z}$}
    \step{<2>3}{$\inv{x} \in V$}
    \begin{proof}
      \pf $\inv{x} = z \inv{y}$
    \end{proof}
  \end{proof}
  \step{<1>6}{$V \cdot V \subseteq U$}
  \begin{proof}
    \pf\ $V \cdot V \subseteq V' \cdot V' \subseteq U$ by \stepref{<1>1}, \stepref{<1>2}, \stepref{<1>3}.
  \end{proof}
  \qed
\end{proof}

\begin{prop}
  \label{prop:group:regular}
  Every topological group is regular.
\end{prop}

\begin{proof}
  \pf
  \step{<1>1}{\pflet{$G$ be a topological group.}}
  \step{<1>2}{\pflet{$A \subseteq G$ be closed and $a \in G \setminus A$}}
  \step{<1>3}{$A \inv{a}$ is closed}
  \begin{proof}
    \pf\ Since the map $\lambda x. x \inv{a}$ is an automorphism.
  \end{proof}
  \step{<1>4}{$G \setminus A \inv{a}$ is a neighbourhood of $e$}
  \begin{proof}
    \pf\ If $e \in A \inv{a}$ then $a \in A$.
  \end{proof}
  \step{<1>5}{\pick\ a symmetric neighbourhood $V$ of $e$ such that $V \cdot V \subseteq G \setminus A \inv{a}$}
  \begin{proof}
    \pf\ Proposition \ref{prop:group:symmetric_neighbourhood}.
  \end{proof}
  \step{<1>6}{$V \cdot A$ and $Va$ are disjoint neighbourhoods of $A$ and $a$ respectively.}
  \begin{proof}
    \step{<2>1}{$V \cdot A \cap Va = \emptyset$}
    \begin{proof}
      \step{<3>1}{\assume{for a contradiction $xy = za$ where $x, z \in V$ and $y \in A$}}
      \step{<3>2}{$y \inv{a} = \inv{x} z$}
      \step{<3>3}{$\inv{x} \in V$}
      \step{<3>4}{$\inv{x} z \in G \setminus A \inv{a}$}
    \end{proof}
    \step{<2>2}{$V \cdot A$ is a neighbourhood of $A$}
    \begin{proof}
      \step{<3>1}{\pick\ an open $U$ such that $e \in U \subseteq V$}
      \step{<3>2}{$e \in U \cdot A \subseteq V \cdot A$}
      \step{<3>3}{$U \cdot A$ is open}
      \begin{proof}
        \pf\ $U \cdot A = \bigcup_{x \in A} Ux$
      \end{proof}
    \end{proof}
    \step{<2>3}{$Va$ is a neighbourhood of $a$}
    \begin{proof}
      \pf\ Similar.
    \end{proof}
  \end{proof}
  \qed
\end{proof}

\begin{prop}
  Let $G$ be a topological group and $H$ a closed subgroup of $G$. Give $G / H$ the quotient topology induced by the canonical map $\pi : G \rightarrow G / H$. Then $G / H$ is regular.
\end{prop}

\begin{proof}
  \pf
  \step{<1>1}{$G/H$ is $T_1$}
  \begin{proof}
    \pf\ Proposition \ref{prop:group:quotient_T1}.
  \end{proof}
  \step{<1>2}{The closed subsets of $G / H$ are the sets $\pi(A)$ where $A$ is a saturated closed set in $G$.}
  \begin{proof}
    \step{<2>1}{For every closed $C \subseteq G / H$ we have $C = \pi(\inv{\pi}(C))$}
    \step{<2>2}{If $A$ is a saturated closed set in $G$ then $\pi(A)$ is closed in $G / H$}
    \begin{proof}
      \pf\ Proposition \ref{prop:strongly_continuous}.
    \end{proof}
  \end{proof}
  \step{<1>3}{\pflet{$\pi(A)$ be a closed set in $G / H$ and $aH \in G / H \setminus \pi(A)$, where $A$ is a saturated closed set in $G$.}}
  \step{<1>4}{$A \inv{a}$ is closed}
  \begin{proof}
    \pf\ Since $\lambda t . t \inv{a}$ is an automorphism of $G$.
  \end{proof}
  \step{<1>5}{\pflet{$U = G \setminus A \inv{a}$}}
  \step{<1>6}{$U$ is a neighbourhood of $e$}
  \begin{proof}
    \pf\ If $e \in A \inv{a}$ then $a \in A$.
  \end{proof}
  \step{<1>7}{\pick\ a symmetric neighbourhood $V$ of $e$ such that $V \cdot V \subseteq U$}
  \begin{proof}
    \pf\ Proposition \ref{prop:group:symmetric_neighbourhood}.
  \end{proof}
  \step{<1>8}{$\pi(V \cdot A)$ and $\pi(Va)$ are disjoint neighbourhoods of $\pi(A)$ and $aH$}
  \begin{proof}
    \step{<2>1}{$\pi(V \cdot A) \cap \pi(Va) = \emptyset$}
    \begin{proof}
      \step{<3>1}{\assume{for a contradiction $xyH = zaH$ where $x, z \in V$ and $y \in A$}}
      \step{<3>2}{$yH = \inv{x} z a H$}
      \step{<3>3}{$\inv{x} z a \in A$}
      \begin{proof}
        \pf\ Since $A$ is saturated.
      \end{proof}
      \step{<3>4}{$\inv{x} \in V$}
      \begin{proof}
        \pf\ Since $V$ is symmetric.
      \end{proof}
      \step{<3>5}{$\inv{x} z \in U$}
      \begin{proof}
        \pf\ From \stepref{<1>7}
      \end{proof}
      \step{<3>6}{$\inv{x} z \notin A \inv{a}$}
      \begin{proof}
        \pf\ From \stepref{<1>5}
      \end{proof}
      \qedstep
      \begin{proof}
        \pf\ \stepref{<3>3} and \stepref{<3>6} form a contradiction.
      \end{proof}
    \end{proof}
    \step{<2>2}{$\pi(V \cdot A)$ is a neighbourhood of $\pi(A)$}
    \begin{proof}
      \pf\ Using Proposition \ref{prop:group:quotient_open}
    \end{proof}
    \step{<2>3}{$\pi(Va)$ is a neighbourhood of $aH$}
    \begin{proof}
      \pf\ Using Proposition \ref{prop:group:quotient_open}
    \end{proof}
  \end{proof}
  \qed
\end{proof}
