\chapter{Metric Spaces}

\section{The Metric Topology}

\begin{df}[Metric]
  A \emph{metric} on a set $X$ is a function $d : X^2 \rightarrow \mathbb{R}$ such that, for all $x,y,z \in X$:
  \begin{enumerate}
    \item
    $d(x,y) \geq 0$
    \item
    $d(x,y) = 0$ if and only if $x = y$
    \item
    $d(x,y) = d(y,x)$
    \item
    \textbf{Triangle Inequality}

    $d(x,z) \leq d(x,y) + d(y,z)$
  \end{enumerate}

  A \emph{metric space} consists of a set $X$ and a metric on $X$. We call the elements of $X$ \emph{points} and $d(x,y)$ the \emph{distance} between $x$ and $y$.
\end{df}

\begin{df}[Open Ball]
  Let $a \in X$ and $\epsilon > 0$. The \emph{open ball} with \emph{center} $a$ and \emph{radius} $\epsilon$ is
  \[ B(a, \epsilon) = \{ x \in X : d(x,a) < \epsilon \} \enspace . \]
\end{df}

\begin{df}[Metric Topology]
  The \emph{metric topology} induced by $d$ is the topology generated by the basis consisting of the open balls.

  A topological space is \emph{metrizable} iff there exists a metric that induces its topology.

  We prove this is a basis.
\end{df}

\begin{proof}
  \pf
  \step{<1>1}{\pflet{$\mathcal{B}$ be the set of all open balls.}}
  \step{<1>2}{$\bigcup \mathcal{B} = X$}
  \begin{proof}
    \pf\ This holds because $a \in B(a,1)$ for all $a \in X$.
  \end{proof}
  \step{<1>3}{For all $B_1, B_2 \in \mathcal{B}$ and $x \in B_1 \cap B_2$, there exists $B_3 \in \mathcal{B}$ such that $x \in B_3 \subseteq B_1 \cap B_2$.}
  \begin{proof}
    \step{<2>1}{\pflet{$B_1 = B(a,\epsilon_1)$ and $B_2 = B(b, \epsilon_2)$}}
    \step{<2>2}{\pflet{$\delta = \min(\epsilon_1 - d(a,x), \epsilon_2 - d(b,x))$}}
    \step{<2>3}{\pflet{$B_3 = B(x, \delta)$}}
    \step{<2>4}{$x \in B_3 \subseteq B_1 \cap B_2$}
    \begin{proof}
      \step{<3>1}{$B_3 \subseteq B_1$}
      \begin{proof}
        \step{<4>1}{\pflet{$y \in B_3$}}
        \step{<4>2}{$d(y, a) < \epsilon_1$}
        \begin{proof}
          \pf
          \begin{align*}
            d(y, a) & \leq d(y, x) + d(x, a) & (\text{Triangle inequality})\\
            & < \delta + d(x, a) & (\text{\stepref{<2>3}, \stepref{<4>1}})\\
            & \leq \epsilon_1 & (\text{\stepref{<2>2}})
          \end{align*}
        \end{proof}
      \end{proof}
      \step{<3>2}{$B_3 \subseteq B_2$}
      \begin{proof}
        \pf\ Similar.
      \end{proof}
    \end{proof}
  \end{proof}
  \qedstep
  \begin{proof}
    \pf\ Proposition \ref{prop:basis}
  \end{proof}
  \qed
\end{proof}

\begin{prop}
  \label{prop:metric:open}
  A set $U$ is open in the metric topology if and only if, for all $x \in U$, there exists $\epsilon > 0$ such that $B(x, \epsilon) \subseteq U$.
\end{prop}

\begin{proof}
  \pf
  \step{<1>1}{If $U$ is open then, for all $x \in U$, there exists $\epsilon > 0$ such that $B(x, \epsilon) \subseteq U$}
  \begin{proof}
    \step{<2>1}{\assume{$U$ is open}}
    \step{<2>2}{\pflet{$x \in U$}}
    \step{<2>3}{\pick\ an open ball $B(a, \delta)$ such that $x \in B(a, \delta) \subseteq U$}
    \step{<2>4}{\pflet{$\epsilon = \delta - d(a,x)$}}
    \step{<2>5}{$B(x, \epsilon) \subseteq U$}
    \begin{proof}
      \step{<3>1}{\pflet{$y \in B(x, \epsilon)$}}
      \step{<3>2}{$y \in B(a, \delta)$}
      \begin{proof}
        \pf
        \begin{align*}
          d(a, y) & \leq d(a, x) + d(x, y) & (\text{Triangle Inequality})\\
          & < d(a, x) + \epsilon & (\text{\stepref{<3>1}})\\
          & = \delta & (\text{\stepref{<2>4}})
        \end{align*}
      \end{proof}
      \step{<3>3}{$y \in U$}
      \begin{proof}
        \pf\ \stepref{<2>3}, \stepref{<3>2}
      \end{proof}
    \end{proof}
  \end{proof}
  \step{<1>2}{If, for all $x \in U$, there exists $\epsilon > 0$ such that $B(x, \epsilon) \subseteq U$, then $U$ is open.}
  \begin{proof}
    \pf\ Immediate from definitions.
  \end{proof}
  \qed
\end{proof}

\begin{ex}$ $
  \begin{enumerate}
    \item
    The discrete topology on $X$ is induced by the \emph{discrete metric}
    \[ d(x,y) = \begin{cases}
    0 & \text{if } x \neq y \\
    1 & \text{if } x = y
  \end{cases} \]
  \item
  The standard topology on $\mathbb{R}$ is induced by the \emph{Euclidean metric} $d(x, y) = |x-y|$.
\end{enumerate}
\end{ex}

\begin{prop}
  \label{prop:metric:finer}
  Let $d$ and $d'$ be two metrics on the set $X$. Then the topology induced by $d$ is finer than the topology induced by $d'$ if and only if, for all $x \in X$ and $\epsilon > 0$, there exists $\delta > 0$ such that
  \[ B_d(x, \delta) \subseteq B_{d'}(x, \epsilon) \]
\end{prop}

\begin{proof}
  \pf
  \step{<1>1}{If the topology induced by $d$ is finer than the topology induced by $d'$ if and only if, for all $x \in X$ and $\epsilon > 0$, there exists $\delta > 0$ such that $B_d(x, \delta) \subseteq B_{d'}(x, \epsilon)$.}
  \begin{proof}
    \pf\ From Proposition \ref{prop:metric:open}.
  \end{proof}
  \step{<1>2}{If, for all $x \in X$ and $\epsilon > 0$, there exists $\delta > 0$ such that $B_d(x, \delta) \subseteq B_{d'}(x, \epsilon)$, then the topology induced by $d$ is finer than the topology induced by $d'$}
  \begin{proof}
    \step{<2>1}{\assume{for all $x \in X$ and $\epsilon > 0$, there exists $\delta > 0$ such that $B_d(x, \delta) \subseteq B_{d'}(x, \epsilon)$}}
    \step{<2>2}{\pflet{$x \in X$ and $\epsilon > 0$} \prove{$B_{d'}(x, \epsilon)$ is open under $d$}}
    \step{<2>3}{\pflet{$y \in B_{d'}(x, \epsilon)$}}
    \step{<2>4}{\pick\ $\epsilon' > 0$ such that $B_{d'}(y, \epsilon') \subseteq B_{d'}(x, \epsilon)$}
    \begin{proof}
      \pf\ Proposition \ref{prop:metric:open}, \stepref{<2>3}
    \end{proof}
    \step{<2>5}{\pick\ $\delta > 0$ such that $B_d(y, \delta) \subseteq B_{d'}(y, \epsilon')$}
    \begin{proof}
      \pf\ \stepref{<2>1}, \stepref{<2>4}
    \end{proof}
    \step{<2>6}{$B_d(y, \delta) \subseteq B_{d'}(x, \epsilon)$}
    \begin{proof}
      \pf\ \stepref{<2>4}, \stepref{<2>5}
    \end{proof}
    \qedstep
    \begin{proof}
      \pf\ Proposition \ref{prop:basis:coarsest}, Proposition \ref{prop:metric:open}.
    \end{proof}
  \end{proof}
  \qed
\end{proof}

\begin{prop}
  The metric topology is the coarsest topology such that $d : X^2 \rightarrow \mathbb{R}$ is continuous.
\end{prop}

\begin{proof}
  \pf
  \step{<1>1}{$d$ is continuous with respect to the metric topology.}
  \begin{proof}
    \step{<2>1}{\pflet{$(x,y) \in X^2$ and $N$ be a neighbourhood of $d(x,y)$}}
    \step{<2>2}{\pick\ $\epsilon > 0$ such that $(d(x,y) - \epsilon, d(x,y) + \epsilon) \subseteq N$}
    \step{<2>3}{\pflet{$M = B(x, \epsilon/2) \times B(y, \epsilon/2)$} \prove{$d(M) \subseteq N$}}
    \step{<2>4}{\pflet{$(a, b) \in M$}}
    \step{<2>5}{$d(a,b) - d(x,y) < \epsilon$}
    \begin{proof}
      \pf
      \begin{align*}
        d(a,b) & \leq d(a,x) + d(x,y) + d(y,b) \\
        & < \epsilon / 2 + d(x,y) + \epsilon / 2
      \end{align*}
    \end{proof}
    \step{<2>6}{$d(x,y) - d(a,b) < \epsilon$}
    \begin{proof}
      \pf\ Similar.
    \end{proof}
  \end{proof}
  \step{<1>2}{If $\mathcal{T}$ is a topology with respect to which $d$ is continuous then $\mathcal{T}$ is finer than the metric topology.}
  \begin{proof}
    \step{<2>1}{\pflet{$a \in X$ and $\epsilon > 0$} \prove{$B(a, \epsilon)$ is open in $\mathcal{T}$}}
    \step{<2>2}{The function $d_a = \lambda x. d(a,x)$ is continuous.}
    \step{<2>3}{$B(a, \epsilon) = \inv{d_a}((- \epsilon, \epsilon))$}
  \end{proof}
  \qed
\end{proof}

\begin{prop}
  Every metric space is Hausdorff.
\end{prop}

\begin{proof}
  \pf\ Let $X$ be a metric space and $a, b \in X$ with $a \neq b$. Let $\epsilon = d(a, b)$. Then $B(a, \epsilon / 2)$ and $B(b, \epsilon / 2)$ are disjoint neighbourhoods of $a$ and $b$. \qed
\end{proof}

\begin{prop}
  Every metric space is first countable.
\end{prop}

\begin{proof}
  \pf\ For any point $a$, the open balls $B(a, q)$ for $q$ a positive rational form a basis at $a$. \qed
\end{proof}

\begin{ex}
  The space $\mathbb{R}^\omega$ under the box topology is not metrizable, because it is not first countable.
\end{ex}

\begin{df}[Bounded]
  Let $X$ be a metric space and $A \subseteq X$. Then $A$ is \emph{bounded} iff $\{ d(x,y) : x,y \in A \}$ is bounded above, in which case its \emph{diameter} is
  \[ \diam A = \sup \{ d(x,y) : x, y \in A \} \]
\end{df}

\section{Subspaces}

\begin{prop}
  Let $(X, d)$ be a metric space and $A \subseteq X$. Then the restriction of $d$ to $A^2$ is a metric on $A$ that induces the subspace topology on $A$.
\end{prop}

\begin{proof}
  \pf
  \step{<1>1}{\pflet{$d'$ be the restriction of $d$ to $A^2$.}}
  \step{<1>2}{$d'$ is a metric on $A$.}
  \begin{proof}
    \pf\ Easy.
  \end{proof}
  \step{<1>3}{$d'$ induces the subspace topology on $A$.}
  \begin{proof}
    \pf\ The topology induced by $d'$ and the subspace topology are each the topology with basis $B_{d'}(x, \epsilon) = B_d(x, \epsilon) \cap A$ for $x \in A$ and $\epsilon > 0$.
  \end{proof}
  \qed
\end{proof}

\section{Continuous Functions}

\begin{prop}
  Let $X$ and $Y$ be metric spaces, $f : X \rightarrow Y$ and $x \in X$. Then $f$ is continuous at $x$ if and only if, for all $\epsilon > 0$, there exists $\delta > 0$ such that, for all $y \in X$, if $d_X(x,y) < \delta$ then $d_Y(f(x),f(y)) < \epsilon$.
\end{prop}

\begin{proof}
  \pf
  \step{<1>1}{If $f$ is continuous at $x$ then the right-hand side holds.}
  \begin{proof}
    \step{<2>1}{\assume{$f$ is continuous at $x$}}
    \step{<2>2}{\pflet{$\epsilon > 0$}}
    \step{<2>3}{\pick\ a neighbourhood $M$ of $x$ such that $f(M) \subseteq B_{d_Y}(f(x), \epsilon)$}
    \step{<2>4}{\pick\ $\delta > 0$ such that $B_{d_X}(x, \delta) \subseteq M$}
    \step{<2>5}{\pflet{$y \in X$ with $d_X(x, y) < \delta$}}
    \step{<2>6}{$y \in M$}
    \step{<2>7}{$f(y) \in B_{d_Y}(f(x), \epsilon)$}
  \end{proof}
  \step{<1>2}{If the right-hand side holds then $f$ is continuous at $x$.}
  \begin{proof}
    \step{<2>1}{\assume{the right-hand side holds}}
    \step{<2>2}{\pflet{$N$ be a neighbourhood of $f(x)$}}
    \step{<2>3}{\pick\ $\epsilon > 0$ such that $B_{d_Y}(f(x), \epsilon) \subseteq N$}
    \step{<2>4}{\pick\ $\delta > 0$ such that, for all $y \in X$, if $d_X(x,y) < \delta$ then $d_Y(f(x), f(y)) < \epsilon$}
    \step{<2>5}{\pflet{$M = B_{d_X}(x, \delta)$}}
    \step{<2>6}{$f(M) \subseteq N$}
  \end{proof}
  \qed
\end{proof}

\section{Uniform Convergence}

\begin{df}[Uniform Convergence]
  Let $X$ be a set and $Y$ a metric space. Let $(f_n)$ be a sequence of functions $X \rightarrow Y$ and $f : X \rightarrow Y$. Then $f_n$ \emph{converges uniformly} to $f$ as $n \rightarrow \infty$ iff, for all $\epsilon > 0$, there exists $N$ such that, for all $n \geq N$ and $x \in X$,
  \[ d(f_n(x), f(x)) < \epsilon \enspace . \]
\end{df}

\begin{thm}[Uniform Limit Theorem]
  Let $X$ be a topological space and $Y$ a metric space. Let $(f_n)$ be a sequence of continuous functions $X \rightarrow Y$ and $f : X \rightarrow Y$.
  If $f_n$ converges uniformly to $f$ as $n \rightarrow \infty$ then $f$ is continuous.
\end{thm}

\begin{proof}
  \pf
  \step{<1>1}{\pflet{$x \in X$ and $V$ be a neighbourhood of $f(x)$}}
  \step{<1>2}{\pick\ $\epsilon > 0$ such that $B(f(x), \epsilon) \subseteq V$}
  \step{<1>3}{\pick\ $N$ such that, for all $n \geq N$ and $y \in X$, we have $d(f_n(y), f(y)) < \epsilon / 3$}
  \step{<1>4}{\pick\ a neighbourhood $M$ of $x$ such that $f_N(M) \subseteq B(f_N(x), \epsilon / 3)$}
  \step{<1>5}{$f(M) \subseteq V$}
  \begin{proof}
    \pf\ For $y \in M$ we have
    \begin{align*}
      d(f(x), f(y)) & \leq d(f(x), f_N(x)) + d(f_N(x), f_N(y)) + d(f_N(y), f(y)) \\
      & < \epsilon / 3 + \epsilon / 3 + \epsilon / 3\\
      & = \epsilon
    \end{align*}
  \end{proof}
  \qed
\end{proof}

\section{Isometric Imbeddings}

\begin{df}[Isometric Imbedding]
  Let $X$ and $Y$ be metric spaces and $f : X \rightarrow Y$. Then $f$ is an \emph{isometric imbedding} iff, for all $x, y \in X$, we have $d_Y(f(x), f(y)) = d_X(x,y)$.
\end{df}

\begin{prop}
  Every isometric imbedding is a topological imbedding.
\end{prop}

\begin{proof}
  \pf
  \step{<1>1}{\pflet{$f : X \rightarrow Y$ be an isometric imbedding.}}
  \step{<1>2}{$f$ is continuous}
  \begin{proof}
    \pf\ If $d_X(x,y) < \epsilon$ then $d_Y(f(x), f(y)) < \epsilon$.
  \end{proof}
  \step{<1>3}{$f$ is injective}
  \begin{proof}
    \pf\ $f(x) = f(y) \Leftrightarrow d_Y(f(x),f(y)) = 0 \Leftrightarrow d_X(x,y) = 0 \Leftrightarrow x = y$.
  \end{proof}
  \step{<1>4}{For all $U$ open in $X$ we have $f(U)$ is open in $f(X)$.}
  \begin{proof}
    \step{<2>1}{\pflet{$U$ be open in $X$}}
    \step{<2>2}{\pflet{$y \in f(U)$}}
    \step{<2>3}{\pick\ $x \in U$ such that $f(x) = y$}
    \step{<2>4}{\pick\ $\epsilon > 0$ such that $B(x, \epsilon) \subseteq U$}
    \step{<2>5}{$B(y, \epsilon) \cap f(X) \subseteq f(U)$}
    \begin{proof}
      \step{<3>1}{\pflet{$y' \in B(y, \epsilon) \cap f(X)$}}
      \step{<3>2}{\pick\ $x' \in X$ such that $f(x') = y'$}
      \step{<3>3}{$d(y, y') < \epsilon$}
      \step{<3>4}{$d(x, x') < \epsilon$}
      \step{<3>5}{$x' \in U$}
    \end{proof}
  \end{proof}
  \qed
\end{proof}

\section{The Square Metric}

\begin{df}[Square Metric]
  The \emph{square metric} $\rho$ on $\mathbb{R}^n$ is defined by
  \[ \rho((x_1, \ldots, x_n), (y_1, \ldots, y_n)) = \max(|x_1 - y_1|, \ldots, |x_n - y_n|) \]

  We prove this is a metric.
\end{df}

\begin{proof}
  \pf
  \step{<1>1}{$\rho(\vec{x}, \vec{y}) \geq 0$}
  \begin{proof}
    \pf\ Immediate from definitions.
  \end{proof}
  \step{<1>2}{$\rho(\vec{x}, \vec{y}) = 0$ iff $\vec{x} = \vec{y}$}
  \begin{proof}
    \pf\ Immediate from definitions.
  \end{proof}
  \step{<1>3}{$\rho(\vec{x}, \vec{y}) = \rho(\vec{y}, \vec{x})$}
  \begin{proof}
    \pf\ Immediate from definitions.
  \end{proof}
  \step{<1>4}{$\rho(\vec{x}, \vec{z}) \leq \rho(\vec{x}, \vec{y}) + \rho(\vec{y}, \vec{z})$}
  \begin{proof}
    \step{<2>1}{For $1 \leq i \leq n$ we have $|x_i - z_i| \leq |x_i - y_i| + |y_i - z_i|$}
    \step{<2>2}{For $1 \leq i \leq n$ we have $|x_1 - z_i| \leq \rho(\vec{x}, \vec{y}) + \rho(\vec{y}, \vec{z})$}
    \step{<2>3}{$\rho(\vec{x}, \vec{z}) \leq \rho(\vec{x}, \vec{y}) + \rho(\vec{y}, \vec{z})$}
  \end{proof}
  \qed
\end{proof}

\begin{prop}
  \label{prop:metric:square}
  The square metric induces the product topology on $\mathbb{R}^n$.
\end{prop}

\begin{proof}
  \pf
  \step{<1>1}{For all $\vec{a} \in \mathbb{R}^n$ and $\epsilon > 0$ we have that $B_\rho(\vec{a}, \epsilon)$ is open in the product topology.}
  \begin{proof}
    \pf\ This holds because $B_\rho(\vec{a}, \epsilon) = (a_1 - \epsilon, a_1 + \epsilon) \times \cdots \times (a_n - \epsilon, a_n + \epsilon)$.
  \end{proof}
  \step{<1>2}{The set of all sets of the form $(a_1, b_1) \times \cdots \times (a_n, b_n)$ is a basis for the product topology on $\mathbb{R}^n$}
  \begin{proof}
    \pf\ Propositions \ref{prop:order:topology} and \ref{prop:product:basis}.
  \end{proof}
  \step{<1>3}{Every set of the form $(a_1, b_1) \times \cdots \times (a_n, b_n)$ is open under $\rho$.}
  \begin{proof}
    \step{<2>1}{\pflet{$\vec{x} \in (a_1, b_1) \times \cdots \times (a_n, b_n)$}}
    \step{<2>2}{\pflet{$\epsilon = \min(x_1 - a_1, b_1 - x_1, \ldots, x_n - a_n, b_n - x_n)$}}
    \step{<2>3}{$B_\rho(\vec{x}, \epsilon) \subseteq (a_1, b_1) \times \cdots \times (a_n, b_n)$}
    \qedstep
    \begin{proof}
      \pf\ From Proposition \ref{prop:metric:open}.
    \end{proof}
  \end{proof}
  \qed
\end{proof}

\begin{prop}
  Addition is continuous $+ : \mathbb{R}^2 \rightarrow \mathbb{R}$.
\end{prop}

\begin{proof}
  \pf
  \step{<1>1}{\pflet{$(a, b) \in \mathbb{R}^2$ and $\epsilon > 0$} \prove{There exists $\delta > 0$, such that, for all $(x,y) \in \mathbb{R}^2$, if $\rho((a,b), (x,y)) < \delta$ then $|(a + b) - (x + y)| < \epsilon$}}
  \step{<1>2}{\pflet{$\delta = \epsilon / 2$}}
  \step{<1>3}{\pflet{$(x,y) \in \mathbb{R}^2$ with $\rho((a,b),(x,y)) < \delta$}}
  \step{<1>4}{$|(a + b) - (x + y)| < \epsilon$}
  \begin{proof}
    \pf
    \begin{align*}
      |(a + b) - (x + y)| & = |a - x| + |b - y| \\
      & < \delta + \delta \\
      & = \epsilon
    \end{align*}
  \end{proof}
  \qed
\end{proof}

\begin{prop}
  Multiplication is continuous $\cdot : \mathbb{R}^2 \rightarrow \mathbb{R}$.
\end{prop}

\begin{proof}
  \pf
  \step{<1>1}{\pflet{$(a, b) \in \mathbb{R}^2$ and $\epsilon > 0$} \prove{There exists $\delta > 0$, such that, for all $(x,y) \in \mathbb{R}^2$, if $\rho((a,b), (x,y)) < \delta$ then $|ab - xy| < \epsilon$}}
  \step{<1>2}{\pflet{$\delta = \min(\epsilon / (|a| + |b| + 1), 1/2)$}}
  \step{<1>3}{\pflet{$(x,y) \in \mathbb{R}^2$ with $\rho((a,b),(x,y)) < \delta$}}
  \step{<1>4}{$|ab - xy| < \epsilon$}
  \begin{proof}
    \pf
    \begin{align*}
      |xy-ab| & \leq |a||y-b| + |b||x-a| + |x-a||y-b| \\
      & < |a| \delta + |b| \delta + \delta^2 \\
      & < |a| \delta + |b| \delta + \delta \\
      & = (|a| + |b| + 1) \delta \\
      & \leq \epsilon
    \end{align*}
  \end{proof}
  \qed
\end{proof}

\begin{cor}
  Additive inverse is continuous $- : \mathbb{R} \rightarrow \mathbb{R}$.
\end{cor}

\begin{prop}
  Multiplicative inverse is continuous $m = \inv{(\ )} : \mathbb{R} \setminus \{ 0 \} \rightarrow \mathbb{R}$.
\end{prop}

\begin{proof}
  \pf
  \step{<1>1}{\pflet{$a,b \in \mathbb{R}$ with $a < b$} \prove{$\inv{m}((a,b) \setminus \{0\})$ is open.}}
  \step{<1>2}{\case{$b < 0$}}
  \begin{proof}
    \pf\ In this case $\inv{m}((a,b)) = (1/b, 1/a)$.
  \end{proof}
  \step{<1>3}{\case{$b = 0$}}
  \begin{proof}
    \pf\ In this case $\inv{m}((a,b)) = (- \infty, 1/a)$.
  \end{proof}
  \step{<1>4}{\case{$a < 0 < b$}}
  \begin{proof}
    \pf\ In this case $\inv{m}((a,b)) = (- \infty, 1/a) \cup (1/b, + \infty)$.
  \end{proof}
  \step{<1>5}{\case{$a = 0$}}
  \begin{proof}
    \pf\ In this case $\inv{m}((a,b)) = (1/b, + \infty)$.
  \end{proof}
  \step{<1>6}{\case{$a > 0$}}
  \begin{proof}
    \pf\ In this case $\inv{m}((a,b)) = (1/b, 1/a)$.
  \end{proof}
  \qed
\end{proof}

\section{The Standard Bounded Metric}

\begin{df}[Standard Bounded Metric]
  Let $d$ be a metric on $X$. The \emph{standard bounded metric} corresponding to $d$ is the function $\overline{d} : X^2 \rightarrow \mathbb{R}$ given by
  \[ \overline{d}(x,y) = \min(d(x,y), 1) \]

  We prove this is a metric.
\end{df}

\begin{proof}
  \pf
  \step{<1>1}{$\overline{d}(x,y) \geq 0$}
  \begin{proof}
    \pf\ Immediate from definition.
  \end{proof}
  \step{<1>2}{$\overline{d}(x,y) = 0$ iff $x = y$}
  \begin{proof}
    \pf\ This holds because $\overline{d}(x,y) = 0$ iff $d(x,y) = 0$.
  \end{proof}
  \step{<1>3}{$\overline{d}(x,y) = \overline{d}(y,x)$}
  \begin{proof}
    \pf\ Immediate from definition.
  \end{proof}
  \step{<1>4}{$\overline{d}(x,z) \leq \overline{d}(x,y) + \overline{d}(y,z)$}
  \begin{proof}
    \step{<2>1}{\case{$d(x,y) \geq 1$}}
    \begin{proof}
      \pf
      \begin{align*}
        \overline{d}(x,z) & \leq 1 \\
        & \leq 1 + \overline{d}(y, z) \\
        & = \overline{d}(x,y) + \overline{d}(y, z)
      \end{align*}
    \end{proof}
    \step{<2>2}{\case{$d(y,z) \geq 1$}}
    \begin{proof}
      \pf\ Similar.
    \end{proof}
    \step{<2>3}{\case{$d(x,y) < 1$ and $d(y,z) < 1$}}
    \begin{proof}
      \pf
      \begin{align*}
        \overline{d}(x,z) & \leq d(x,z) \\
        & \leq d(x,y) + d(y,z) \\
        & = \overline{d}(x,y) + \overline{d}(y,z)
      \end{align*}
    \end{proof}
  \end{proof}
  \qed
\end{proof}

\begin{prop}
  The standard bounded metric corresponding to $d$ induces the same topology as $d$.
\end{prop}

\begin{proof}
  \pf
  \step{<1>1}{Every $d$-ball is open under $\overline{d}$}
  \begin{proof}
    \step{<2>1}{\pflet{$x \in B_d(a, \epsilon)$} \prove{There exists $\delta > 0$ such that $B_{\overline{d}}(x, \delta) \subseteq B_d(a, \epsilon)$}}
    \step{<2>2}{\pick\ $\gamma > 0$ such that $B_d(x, \gamma) \subseteq B_d(a, \epsilon)$}
    \step{<2>3}{\pflet{$\delta = \min(\gamma, 1/2)$}}
    \step{<2>4}{$B_{\overline{d}}(x, \delta) \subseteq B_d(a, \epsilon)$}
    \begin{proof}
      \pf
      \begin{align*}
      B_{\overline{d}}(x, \delta) & = B_d(x, \delta) & (\delta < 1) \\
      & \subseteq B_d(x, \gamma) & (\delta \leq \gamma) \\
      & \subseteq B_d(a, \epsilon) & (\text{\stepref{<2>2}})
    \end{align*}
    \end{proof}
  \end{proof}
  \step{<1>2}{Every $\overline{d}$-ball is open under $d$}
  \begin{proof}
    \pf\ $B_{\overline{d}}(a, \epsilon) = B_d(a, \epsilon)$ if $\epsilon < 1$, $X$ if $\epsilon \geq 1$.
  \end{proof}
  \qedstep
  \begin{proof}
    \pf\ Using Proposition \ref{prop:basis:coarsest}
  \end{proof}
  \qed
\end{proof}

\begin{cor}
  For every metrizable space $X$, there exists a bounded metric that induces the topology on $X$.
\end{cor}

\begin{prop}
  The product of a countable family of metrizable spaces is metrizable.
\end{prop}

\begin{proof}
  \pf
  \step{<1>1}{\pflet{$\{ (X_n, d_n) \}_{n \geq 1}$ be a countable family of metric spaces.}}
  \step{<1>2}{\assume{w.l.o.g.~every $d_n$ is bounded by 1}}
  \step{<1>3}{\pflet{$D : (\prod_{n=1}^\infty X_n)^2 \rightarrow \mathbb{R}$ be defined by
  \[ D(\vec{x}, \vec{y}) = \sup_{i \geq 1} (d_i(x_i, y_i) / i) \]
  }}
  \step{<1>4}{$D$ is a metric.}
  \begin{proof}
    \step{<2>1}{$D(\vec{x}, \vec{y}) \geq 0$}
    \begin{proof}
      \pf\ Immediate from definition.
    \end{proof}
    \step{<2>2}{$D(\vec{x}, \vec{y}) = 0$ iff $\vec{x} = \vec{y}$}
    \begin{proof}
      \pf\ Immediate from definition.
    \end{proof}
    \step{<2>3}{$D(\vec{x}, \vec{y}) = D(\vec{y}, \vec{x})$}
    \begin{proof}
      \pf\ Immediate from definition.
    \end{proof}
    \step{<2>4}{$D(\vec{x}, \vec{z}) \leq D(\vec{x}, \vec{y}) + D(\vec{y}, \vec{z})$}
    \begin{proof}
      \step{<3>1}{For all $i$ we have $d_i(x_i, z_i) / i \leq d_i(x_i, y_i) / i + d_i(y_i, z_i) / i$}
      \step{<3>2}{For all $i$ we have $d_i(x_i, z_i) / i \leq
      D(\vec{x}, \vec{y}) + D(\vec{y}, \vec{z})$}
      \step{<3>3}{$D(\vec{x}, \vec{z}) \leq D(\vec{x}, \vec{y}) + D(\vec{y}, \vec{z})$}
    \end{proof}
  \end{proof}
  \step{<1>5}{$D$ induces the product topology on $\mathbb{R}^\omega$.}
  \begin{proof}
    \step{<2>1}{The product topology is finer than the topology induced by $D$.}
    \begin{proof}
      \step{<3>1}{\pflet{$U$ be open in the topology induced by $D$.}}
      \step{<3>2}{\pflet{$\vec{x} \in U$}}
      \step{<3>3}{\pick\ $\epsilon > 0$ such that $B_D(\vec{x}, \epsilon) \subseteq U$}
      \step{<3>4}{\pick\ $N$ such that $1 / N < \epsilon$}
      \step{<3>5}{\pflet{$V = B_{d_1}(x_1, \epsilon) \times \cdots \times B_{d_N}(x_N, \epsilon) \times X_{N+1} \times X_{N+2} \times \cdots$}}
      \step{<3>6}{$\vec{x} \in V \subseteq U$}
    \end{proof}
    \step{<2>2}{The topology induced by $D$ is finer than the product topology.}
    \begin{proof}
      \step{<3>1}{\pflet{$n \geq 1$ and $U$ be open in $X_n$} \prove{$\inv{\pi_n}(U)$ is open in the topology induced by $D$.}}
      \step{<3>2}{\pflet{$\vec{x} \in \inv{\pi_n}(U)$}}
      \step{<3>3}{\pick\ $\epsilon > 0$ such that $B_{d_n}(x_n, \epsilon) \subseteq U$}
      \step{<3>4}{$B_D(\vec{x}, \epsilon / n) \subseteq \inv{\pi_n}(U)$}
    \end{proof}
  \end{proof}
  \qed
\end{proof}

\begin{ex}
  For $J$ uncountable, the space $\mathbb{R}^J$ is not metrizable, because it is not first countable.
\end{ex}

\section{The Uniform Metric}

\begin{df}[Uniform Metric]
  Let $J$ be a nonempty set and $(X,d)$ a metric space. The \emph{uniform metric} $\overline{\rho}$ on $X^J$ is defined by
  \[ \overline{\rho}((x_\alpha), (y_\alpha)) = \sup_{\alpha \in J} \overline{d}(x_\alpha, y_\alpha) \]
  where $\overline{d}$ is the standard bounded metric corresponding to $X$.

  The \emph{uniform topology} on $X^J$ is the topology induced by the uniform metric.
\end{df}

It is easy to check that this is a metric.

\begin{prop}
  Let $J$ be a set, $(f_n)$ a sequence of functions $J \rightarrow X$ and $f : J \rightarrow X$. Then $f_n$ converges uniformly to $f$ as $n \rightarrow \infty$ if and only if $f_n \rightarrow f$ as $n \rightarrow \infty$ in $X^J$ under the uniform topology.
\end{prop}

\begin{proof}
  \pf\ Easy. \qed
\end{proof}

\begin{prop}[DC]
  The uniform topology is finer than the product topology and coarser than the box topology.
\end{prop}

\begin{proof}
  \pf
  \step{<1>1}{The uniform topology is finer than the product topology.}
  \begin{proof}
    \step{<2>1}{\pflet{$\alpha \in J$ and $U$ be open in $X$} \prove{$\inv{\pi_\alpha}(U)$ is open in the uniform topology.}}
    \step{<2>2}{\pflet{$(x_\alpha) \in \inv{\pi_\alpha}(U)$}}
    \step{<2>3}{\pick\ $\epsilon > 0$ such that $B_d(x_\alpha, \epsilon) \subseteq U$}
    \begin{proof}
      \pf\ Proposition \ref{prop:metric:open}, \stepref{<2>1}, \stepref{<2>2}
    \end{proof}
    \step{<2>4}{\assume{w.l.o.g.~$\epsilon < 1$}}
    \step{<2>5}{$B_{\overline{\rho}}((x_\alpha), \epsilon) \subseteq \inv{\pi_\alpha}(U)$}
  \end{proof}
  \step{<1>2}{The uniform topology is coarser than the box topology.}
  \begin{proof}
    \step{<2>1}{\pflet{$(x_\alpha) \in X^J$ and $\epsilon > 0$} \prove{$U = B_{\overline{\rho}}((x_\alpha), \epsilon)$ is open in the box topology.}}
    \step{<2>2}{\case{$\epsilon > 1$}}
    \begin{proof}
      \pf\ In this case $U = X^J$
    \end{proof}
    \step{<2>3}{\case{$\epsilon \leq 1$}}
    \begin{proof}
      \pf\ For $(y_\alpha) \in U$ we have $(y_\alpha) \in \prod_{\alpha \in J} B_d(y_\alpha, \epsilon / 2) \subseteq U$.
    \end{proof}
  \end{proof}
  \qed
\end{proof}

\begin{prop}
  Let $X$ be a topological space and $Y$ a metric space. Let $\mathcal{C}(X, Y)$ be the set of all continuous functions $X \rightarrow Y$ under the uniform topology. Then the evaluation map $\mathcal{C}(X, Y) \times X \rightarrow Y$ is continuous.
\end{prop}

\begin{proof}
  \pf
  \step{<1>1}{\pflet($(f, x) \in \mathcal{C}(X, Y)$)}
  \step{<1>2}{\pflet{$V$ be a neighbourhood of $f(x)$}}
  \step{<1>3}{\pick\ $\epsilon > 0$ such that $B(f(x), \epsilon) \subseteq V$}
  \step{<1>4}{\pick\ a neighbourhood $U$ of $x$ such that $f(U) \subseteq B(f(x), \epsilon / 2)$
  \prove{For all $(g, y) \in B_{\overline{\rho}}(f, \epsilon / 2) \times U$ we have $g(y) \in V$}}
  \step{<1>5}{\pflet{$(g, y) \in B_{\overline{\rho}}(f, \epsilon / 2) \times U$}}
  \step{<1>6}{$d(g(y), f(x)) < \epsilon$}
  \begin{proof}
    \pf
    \begin{align*}
      d(g(y), f(x)) & \leq d(g(y), f(y)) + d(f(y), f(x)) \\
      & < \overline{\rho}(g, f) + d(f(y), f(x)) \\
      & < \epsilon / 2 + \epsilon / 2 \\
      & = \epsilon
    \end{align*}
  \end{proof}
  \qed
\end{proof}
